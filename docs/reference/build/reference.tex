% Options for packages loaded elsewhere
\PassOptionsToPackage{unicode}{hyperref}
\PassOptionsToPackage{hyphens}{url}
%
\documentclass[
]{article}
\usepackage{lmodern}
\usepackage{amssymb,amsmath}
\usepackage{ifxetex,ifluatex}
\ifnum 0\ifxetex 1\fi\ifluatex 1\fi=0 % if pdftex
  \usepackage[T1]{fontenc}
  \usepackage[utf8]{inputenc}
  \usepackage{textcomp} % provide euro and other symbols
\else % if luatex or xetex
  \usepackage{unicode-math}
  \defaultfontfeatures{Scale=MatchLowercase}
  \defaultfontfeatures[\rmfamily]{Ligatures=TeX,Scale=1}
\fi
% Use upquote if available, for straight quotes in verbatim environments
\IfFileExists{upquote.sty}{\usepackage{upquote}}{}
\IfFileExists{microtype.sty}{% use microtype if available
  \usepackage[]{microtype}
  \UseMicrotypeSet[protrusion]{basicmath} % disable protrusion for tt fonts
}{}
\makeatletter
\@ifundefined{KOMAClassName}{% if non-KOMA class
  \IfFileExists{parskip.sty}{%
    \usepackage{parskip}
  }{% else
    \setlength{\parindent}{0pt}
    \setlength{\parskip}{6pt plus 2pt minus 1pt}}
}{% if KOMA class
  \KOMAoptions{parskip=half}}
\makeatother
\usepackage{xcolor}
\IfFileExists{xurl.sty}{\usepackage{xurl}}{} % add URL line breaks if available
\IfFileExists{bookmark.sty}{\usepackage{bookmark}}{\usepackage{hyperref}}
\hypersetup{
  pdftitle={Z3S5 Lisp Reference Manual},
  pdfauthor={by Erich H. Rast, Ph.D., and all other Help system contributors},
  hidelinks,
  pdfcreator={LaTeX via pandoc}}
\urlstyle{same} % disable monospaced font for URLs
\usepackage{listings}
\newcommand{\passthrough}[1]{#1}
\lstset{defaultdialect=[5.3]Lua}
\lstset{defaultdialect=[x86masm]Assembler}
\setlength{\emergencystretch}{3em} % prevent overfull lines
\providecommand{\tightlist}{%
  \setlength{\itemsep}{0pt}\setlength{\parskip}{0pt}}
\setcounter{secnumdepth}{5}
\lstset{% for listings
    basicstyle=\footnotesize\ttfamily,
    breaklines=true,
}
\usepackage{xcolor}

\title{Z3S5 Lisp Reference Manual}
\author{by Erich H. Rast, Ph.D., and all other Help system contributors}
\date{2022-7-27 10:11}

\begin{document}
\maketitle

\hypertarget{introduction}{%
\section{Introduction}\label{introduction}}

This is the reference manual for the programming language Z3S5 Lisp,
version 2.3.2+5fb70dc. This manual has been automatically generated from
the entries of the online help system. The reference manual is divided
into two large sections. Section \emph{By Topics} lists functions and
symbols organized by topics. Within each topic, entries are sorted
alphabetically. Section \emph{Complete Reference} lists all functions
and symbols alphabetically. Please consult the \emph{User Manual} and
the \emph{Readme} document for more general information about Z3S5 Lisp,
an introduction to its use, and how to embedd it into Go programs.

Incorrect documentation strings are bugs. Please report bugs using the
corresponding \href{https://github.com/rasteric/z3s5-lisp/issues}{Github
issue tracker nfor Z3S5 Lisp} and make sure to be as precise as
possible. Unneeded or missing documentation entries are misfeatures and
may also be reported.

\hypertarget{by-topics}{%
\section{By Topics}\label{by-topics}}

\hypertarget{arrays}{%
\subsection{Arrays}\label{arrays}}

This section concerns functions related to arrays, which are dynamic
indexed sequences of values.

\hypertarget{array-procedure0-or-more}{%
\subsubsection{\texorpdfstring{\texttt{array} : procedure/0 or
more}{array : procedure/0 or more}}\label{array-procedure0-or-more}}

Usage: \passthrough{\lstinline!(array [arg1] ...) => array!}

Create an array containing the arguments given to it.

See also: \passthrough{\lstinline!array?, build-array!}.

\hypertarget{array-copy-procedure1}{%
\subsubsection{\texorpdfstring{\texttt{array-copy} :
procedure/1}{array-copy : procedure/1}}\label{array-copy-procedure1}}

Usage: \passthrough{\lstinline!(array-copy arr) => array!}

Return a copy of \passthrough{\lstinline!arr.!}

See also:
\passthrough{\lstinline"array, array?, array-map!, array-pmap!"}.

\hypertarget{array-exists-procedure2}{%
\subsubsection{\texorpdfstring{\texttt{array-exists?} :
procedure/2}{array-exists? : procedure/2}}\label{array-exists-procedure2}}

Usage: \passthrough{\lstinline!(array-exists? arr pred) => bool!}

Return true if \passthrough{\lstinline!pred!} returns true for at least
one element in array \passthrough{\lstinline!arr!}, nil otherwise.

See also:
\passthrough{\lstinline!exists?, forall?, list-exists?, str-exists?, seq?!}.

\hypertarget{array-forall-procedure2}{%
\subsubsection{\texorpdfstring{\texttt{array-forall?} :
procedure/2}{array-forall? : procedure/2}}\label{array-forall-procedure2}}

Usage: \passthrough{\lstinline!(array-forall? arr pred) => bool!}

Return true if predicate \passthrough{\lstinline!pred!} returns true for
all elements of array \passthrough{\lstinline!arr!}, nil otherwise.

See also:
\passthrough{\lstinline!foreach, map, forall?, str-forall?, list-forall?, exists?!}.

\hypertarget{array-foreach-procedure2}{%
\subsubsection{\texorpdfstring{\texttt{array-foreach} :
procedure/2}{array-foreach : procedure/2}}\label{array-foreach-procedure2}}

Usage: \passthrough{\lstinline!(array-foreach arr proc)!}

Apply \passthrough{\lstinline!proc!} to each element of array
\passthrough{\lstinline!arr!} in order, for the side effects.

See also: \passthrough{\lstinline!foreach, list-foreach, map!}.

\hypertarget{array-len-procedure1}{%
\subsubsection{\texorpdfstring{\texttt{array-len} :
procedure/1}{array-len : procedure/1}}\label{array-len-procedure1}}

Usage: \passthrough{\lstinline!(array-len arr) => int!}

Return the length of array \passthrough{\lstinline!arr.!}

See also: \passthrough{\lstinline!len!}.

\hypertarget{array-map-procedure2}{%
\subsubsection{\texorpdfstring{\texttt{array-map!} :
procedure/2}{array-map! : procedure/2}}\label{array-map-procedure2}}

Usage: \passthrough{\lstinline"(array-map! arr proc)"}

Traverse array \passthrough{\lstinline!arr!} in unspecified order and
apply \passthrough{\lstinline!proc!} to each element. This mutates the
array.

See also:
\passthrough{\lstinline"array-walk, array-pmap!, array?, map, seq?"}.

\hypertarget{array-pmap-procedure2}{%
\subsubsection{\texorpdfstring{\texttt{array-pmap!} :
procedure/2}{array-pmap! : procedure/2}}\label{array-pmap-procedure2}}

Usage: \passthrough{\lstinline"(array-pmap! arr proc)"}

Apply \passthrough{\lstinline!proc!} in unspecified order in parallel to
array \passthrough{\lstinline!arr!}, mutating the array to contain the
value returned by \passthrough{\lstinline!proc!} each time. Because of
the calling overhead for parallel execution, for many workloads
array-map! might be faster if \passthrough{\lstinline!proc!} is very
fast. If \passthrough{\lstinline!proc!} is slow, then array-pmap! may be
much faster for large arrays on machines with many cores.

See also:
\passthrough{\lstinline"array-map!, array-walk, array?, map, seq?"}.

\hypertarget{array-ref-procedure1}{%
\subsubsection{\texorpdfstring{\texttt{array-ref} :
procedure/1}{array-ref : procedure/1}}\label{array-ref-procedure1}}

Usage: \passthrough{\lstinline!(array-ref arr n) => any!}

Return the element of \passthrough{\lstinline!arr!} at index
\passthrough{\lstinline!n!}. Arrays are 0-indexed.

See also: \passthrough{\lstinline!array?, array, nth, seq?!}.

\hypertarget{array-reverse-procedure1}{%
\subsubsection{\texorpdfstring{\texttt{array-reverse} :
procedure/1}{array-reverse : procedure/1}}\label{array-reverse-procedure1}}

Usage: \passthrough{\lstinline!(array-reverse arr) => array!}

Create a copy of \passthrough{\lstinline!arr!} that reverses the order
of all of its elements.

See also: \passthrough{\lstinline!reverse, list-reverse, str-reverse!}.

\hypertarget{array-set-procedure3}{%
\subsubsection{\texorpdfstring{\texttt{array-set} :
procedure/3}{array-set : procedure/3}}\label{array-set-procedure3}}

Usage: \passthrough{\lstinline!(array-set arr idx value)!}

Set the value at index \passthrough{\lstinline!idx!} in
\passthrough{\lstinline!arr!} to \passthrough{\lstinline!value!}. Arrays
are 0-indexed. This mutates the array.

See also: \passthrough{\lstinline!array?, array!}.

\hypertarget{array-slice-procedure3}{%
\subsubsection{\texorpdfstring{\texttt{array-slice} :
procedure/3}{array-slice : procedure/3}}\label{array-slice-procedure3}}

Usage: \passthrough{\lstinline!(array-slice arr low high) => array!}

Slice the array \passthrough{\lstinline!arr!} starting from
\passthrough{\lstinline!low!} (inclusive) and ending at
\passthrough{\lstinline!high!} (exclusive) and return the slice.

See also: \passthrough{\lstinline!array-ref, array-len!}.

\hypertarget{array-sort-procedure2}{%
\subsubsection{\texorpdfstring{\texttt{array-sort} :
procedure/2}{array-sort : procedure/2}}\label{array-sort-procedure2}}

Usage: \passthrough{\lstinline!(array-sort arr proc) => arr!}

Destructively sorts array \passthrough{\lstinline!arr!} by using
comparison proc \passthrough{\lstinline!proc!}, which takes two
arguments and returns true if the first argument is smaller than the
second argument, nil otherwise. The array is returned but it is not
copied and modified in place by this procedure. The sorting algorithm is
not guaranteed to be stable.

See also: \passthrough{\lstinline!sort!}.

\hypertarget{array-walk-procedure2}{%
\subsubsection{\texorpdfstring{\texttt{array-walk} :
procedure/2}{array-walk : procedure/2}}\label{array-walk-procedure2}}

Usage: \passthrough{\lstinline!(array-walk arr proc)!}

Traverse the array \passthrough{\lstinline!arr!} from first to last
element and apply \passthrough{\lstinline!proc!} to each element for
side-effects. Function \passthrough{\lstinline!proc!} takes the index
and the array element at that index as argument. If
\passthrough{\lstinline!proc!} returns nil, then the traversal stops and
the index is returned. If \passthrough{\lstinline!proc!} returns
non-nil, traversal continues. If \passthrough{\lstinline!proc!} never
returns nil, then the index returned is -1. This function does not
mutate the array.

See also:
\passthrough{\lstinline"array-map!, array-pmap!, array?, map, seq?"}.

\hypertarget{array-procedure1}{%
\subsubsection{\texorpdfstring{\texttt{array?} :
procedure/1}{array? : procedure/1}}\label{array-procedure1}}

Usage: \passthrough{\lstinline!(array? obj) => bool!}

Return true of \passthrough{\lstinline!obj!} is an array, nil otherwise.

See also: \passthrough{\lstinline!seq?, array!}.

\hypertarget{build-array-procedure2}{%
\subsubsection{\texorpdfstring{\texttt{build-array} :
procedure/2}{build-array : procedure/2}}\label{build-array-procedure2}}

Usage: \passthrough{\lstinline!(build-array n init) => array!}

Create an array containing \passthrough{\lstinline!n!} elements with
initial value \passthrough{\lstinline!init.!}

See also: \passthrough{\lstinline!array, array?!}.

\hypertarget{binary-manipulation}{%
\subsection{Binary Manipulation}\label{binary-manipulation}}

This section lists functions for manipulating binary data in memory and
on disk.

\hypertarget{bitand-procedure2}{%
\subsubsection{\texorpdfstring{\texttt{bitand} :
procedure/2}{bitand : procedure/2}}\label{bitand-procedure2}}

Usage: \passthrough{\lstinline!(bitand n m) => int!}

Return the bitwise and of integers \passthrough{\lstinline!n!} and
\passthrough{\lstinline!m.!}

See also:
\passthrough{\lstinline!bitxor, bitor, bitclear, bitshl, bitshr!}.

\hypertarget{bitclear-procedure2}{%
\subsubsection{\texorpdfstring{\texttt{bitclear} :
procedure/2}{bitclear : procedure/2}}\label{bitclear-procedure2}}

Usage: \passthrough{\lstinline!(bitclear n m) => int!}

Return the bitwise and-not of integers \passthrough{\lstinline!n!} and
\passthrough{\lstinline!m.!}

See also:
\passthrough{\lstinline!bitxor, bitand, bitor, bitshl, bitshr!}.

\hypertarget{bitor-procedure2}{%
\subsubsection{\texorpdfstring{\texttt{bitor} :
procedure/2}{bitor : procedure/2}}\label{bitor-procedure2}}

Usage: \passthrough{\lstinline!(bitor n m) => int!}

Return the bitwise or of integers \passthrough{\lstinline!n!} and
\passthrough{\lstinline!m.!}

See also:
\passthrough{\lstinline!bitxor, bitand, bitclear, bitshl, bitshr!}.

\hypertarget{bitshl-procedure2}{%
\subsubsection{\texorpdfstring{\texttt{bitshl} :
procedure/2}{bitshl : procedure/2}}\label{bitshl-procedure2}}

Usage: \passthrough{\lstinline!(bitshl n m) => int!}

Return the bitwise left shift of \passthrough{\lstinline!n!} by
\passthrough{\lstinline!m.!}

See also:
\passthrough{\lstinline!bitxor, bitor, bitand, bitclear, bitshr!}.

\hypertarget{bitshr-procedure2}{%
\subsubsection{\texorpdfstring{\texttt{bitshr} :
procedure/2}{bitshr : procedure/2}}\label{bitshr-procedure2}}

Usage: \passthrough{\lstinline!(bitshr n m) => int!}

Return the bitwise right shift of \passthrough{\lstinline!n!} by
\passthrough{\lstinline!m.!}

See also:
\passthrough{\lstinline!bitxor, bitor, bitand, bitclear, bitshl!}.

\hypertarget{bitxor-procedure2}{%
\subsubsection{\texorpdfstring{\texttt{bitxor} :
procedure/2}{bitxor : procedure/2}}\label{bitxor-procedure2}}

Usage: \passthrough{\lstinline!(bitxor n m) => int!}

Return the bitwise exclusive or value of integers
\passthrough{\lstinline!n!} and \passthrough{\lstinline!m.!}

See also:
\passthrough{\lstinline!bitand, bitor, bitclear, bitshl, bitshr!}.

\hypertarget{blob-chksum-procedure1-or-more}{%
\subsubsection{\texorpdfstring{\texttt{blob-chksum} : procedure/1 or
more}{blob-chksum : procedure/1 or more}}\label{blob-chksum-procedure1-or-more}}

Usage: \passthrough{\lstinline!(blob-chksum b [start] [end]) => blob!}

Return the checksum of the contents of blob \passthrough{\lstinline!b!}
as new blob. The checksum is cryptographically secure. If the optional
\passthrough{\lstinline!start!} and \passthrough{\lstinline!end!} are
provided, then only the bytes from \passthrough{\lstinline!start!}
(inclusive) to \passthrough{\lstinline!end!} (exclusive) are
checksummed.

See also: \passthrough{\lstinline!fchksum, blob-free!}.

\hypertarget{blob-equal-procedure2}{%
\subsubsection{\texorpdfstring{\texttt{blob-equal?} :
procedure/2}{blob-equal? : procedure/2}}\label{blob-equal-procedure2}}

Usage: \passthrough{\lstinline!(blob-equal? b1 b2) => bool!}

Return true if \passthrough{\lstinline!b1!} and
\passthrough{\lstinline!b2!} are equal, nil otherwise. Two blobs are
equal if they are either both invalid, both contain no valid data, or
their contents contain exactly the same binary data.

See also: \passthrough{\lstinline!str->blob, blob->str, blob-free!}.

\hypertarget{blob-free-procedure1}{%
\subsubsection{\texorpdfstring{\texttt{blob-free} :
procedure/1}{blob-free : procedure/1}}\label{blob-free-procedure1}}

Usage: \passthrough{\lstinline!(blob-free b)!}

Frees the binary data stored in blob \passthrough{\lstinline!b!} and
makes the blob invalid.

See also:
\passthrough{\lstinline!make-blob, valid?, str->blob, blob->str, blob-equal?!}.

\hypertarget{make-blob-procedure1}{%
\subsubsection{\texorpdfstring{\texttt{make-blob} :
procedure/1}{make-blob : procedure/1}}\label{make-blob-procedure1}}

Usage: \passthrough{\lstinline!(make-blob n) => blob!}

Make a binary blob of size \passthrough{\lstinline!n!} initialized to
zeroes.

See also: \passthrough{\lstinline!blob-free, valid?, blob-equal?!}.

\hypertarget{peek-procedure4}{%
\subsubsection{\texorpdfstring{\texttt{peek} :
procedure/4}{peek : procedure/4}}\label{peek-procedure4}}

Usage: \passthrough{\lstinline!(peek b pos end sel) => num!}

Read a numeric value determined by selector
\passthrough{\lstinline!sel!} from binary blob
\passthrough{\lstinline!b!} at position \passthrough{\lstinline!pos!}
with endianness \passthrough{\lstinline!end!}. Possible values for
endianness are `little and 'big, and possible values for
\passthrough{\lstinline!sel!} must be one of'(bool int8 uint8 int16
uint16 int32 uint32 int64 uint64 float32 float64).

See also: \passthrough{\lstinline!poke, read-binary!}.

\hypertarget{poke-procedure5}{%
\subsubsection{\texorpdfstring{\texttt{poke} :
procedure/5}{poke : procedure/5}}\label{poke-procedure5}}

Usage: \passthrough{\lstinline!(poke b pos end sel n)!}

Write numeric value \passthrough{\lstinline!n!} as type
\passthrough{\lstinline!sel!} with endianness
\passthrough{\lstinline!end!} into the binary blob
\passthrough{\lstinline!b!} at position \passthrough{\lstinline!pos!}.
Possible values for endianness are `little and 'big, and possible values
for \passthrough{\lstinline!sel!} must be one of'(bool int8 uint8 int16
uint16 int32 uint32 int64 uint64 float32 float64).

See also: \passthrough{\lstinline!peek, write-binary!}.

\hypertarget{boxed-data-structures}{%
\subsection{Boxed Data Structures}\label{boxed-data-structures}}

Boxed values are used for dealing with foreign data structures in Lisp.

\hypertarget{valid-procedure1}{%
\subsubsection{\texorpdfstring{\texttt{valid?} :
procedure/1}{valid? : procedure/1}}\label{valid-procedure1}}

Usage: \passthrough{\lstinline!(valid? obj) => bool!}

Return true if \passthrough{\lstinline!obj!} is a valid object, nil
otherwise. What exactly object validity means is undefined, but certain
kind of objects such as graphics objects may be marked invalid when they
can no longer be used because they have been disposed off by a subsystem
and cannot be automatically garbage collected. Generally, invalid
objects ought no longer be used and need to be discarded.

See also: \passthrough{\lstinline!gfx.reset!}.

\hypertarget{concurrency-and-parallel-programming}{%
\subsection{Concurrency and Parallel
Programming}\label{concurrency-and-parallel-programming}}

There are several mechanisms for doing parallel and concurrent
programming in Z3S5 Lisp. Synchronization primitives are also listed in
this section. Generally, users are advised to remain vigilant about
potential race conditions.

\hypertarget{ccmp-macro2}{%
\subsubsection{\texorpdfstring{\texttt{ccmp} :
macro/2}{ccmp : macro/2}}\label{ccmp-macro2}}

Usage: \passthrough{\lstinline!(ccmp sym value) => int!}

Compare the integer value of \passthrough{\lstinline!sym!} with the
integer \passthrough{\lstinline!value!}, return 0 if
\passthrough{\lstinline!sym!} = \passthrough{\lstinline!value!}, -1 if
\passthrough{\lstinline!sym!} \textless{}
\passthrough{\lstinline!value!}, and 1 if \passthrough{\lstinline!sym!}
\textgreater{} \passthrough{\lstinline!value!}. This operation is
synchronized between tasks and futures.

See also: \passthrough{\lstinline"cinc!, cdec!, cwait, cst!"}.

\hypertarget{cdec-macro1}{%
\subsubsection{\texorpdfstring{\texttt{cdec!} :
macro/1}{cdec! : macro/1}}\label{cdec-macro1}}

Usage: \passthrough{\lstinline"(cdec! sym) => int"}

Decrease the integer value stored in top-level symbol
\passthrough{\lstinline!sym!} by 1 and return the new value. This
operation is synchronized between tasks and futures.

See also: \passthrough{\lstinline"cinc!, cwait, ccmp, cst!"}.

\hypertarget{cinc-macro1}{%
\subsubsection{\texorpdfstring{\texttt{cinc!} :
macro/1}{cinc! : macro/1}}\label{cinc-macro1}}

Usage: \passthrough{\lstinline"(cinc! sym) => int"}

Increase the integer value stored in top-level symbol
\passthrough{\lstinline!sym!} by 1 and return the new value. This
operation is synchronized between tasks and futures.

See also: \passthrough{\lstinline"cdec!, cwait, ccmp, cst!"}.

\hypertarget{cpunum-procedure0}{%
\subsubsection{\texorpdfstring{\texttt{cpunum} :
procedure/0}{cpunum : procedure/0}}\label{cpunum-procedure0}}

Usage: \passthrough{\lstinline!(cpunum)!}

Return the number of cpu cores of this machine.

See also: \passthrough{\lstinline!sys!}.

\textbf{Warning: This function also counts virtual cores on the
emulator. The original Z3S5 machine did not have virtual cpu cores.}

\hypertarget{cst-procedure2}{%
\subsubsection{\texorpdfstring{\texttt{cst!} :
procedure/2}{cst! : procedure/2}}\label{cst-procedure2}}

Usage: \passthrough{\lstinline"(cst! sym value)"}

Set the value of \passthrough{\lstinline!sym!} to integer
\passthrough{\lstinline!value!}. This operation is synchronized between
tasks and futures.

See also: \passthrough{\lstinline"cinc!, cdec!, ccmp, cwait"}.

\hypertarget{cwait-procedure3}{%
\subsubsection{\texorpdfstring{\texttt{cwait} :
procedure/3}{cwait : procedure/3}}\label{cwait-procedure3}}

Usage: \passthrough{\lstinline!(cwait sym value timeout)!}

Wait until integer counter \passthrough{\lstinline!sym!} has
\passthrough{\lstinline!value!} or \passthrough{\lstinline!timeout!}
milliseconds have passed. If \passthrough{\lstinline!imeout!} is 0, then
this routine might wait indefinitely. This operation is synchronized
between tasks and futures.

See also: \passthrough{\lstinline"cinc!, cdec!, ccmp, cst!"}.

\hypertarget{enq-procedure1}{%
\subsubsection{\texorpdfstring{\texttt{enq} :
procedure/1}{enq : procedure/1}}\label{enq-procedure1}}

Usage: \passthrough{\lstinline!(enq proc)!}

Put \passthrough{\lstinline!proc!} on a special internal queue for
sequential execution and execute it when able.
\passthrough{\lstinline!proc!} must be a prodedure that takes no
arguments. The queue can be used to synchronizing i/o commands but
special care must be taken that \passthrough{\lstinline!proc!}
terminates, or else the system might be damaged.

See also: \passthrough{\lstinline!task, future, synout, synouty!}.

\textbf{Warning: Calls to enq can never be nested, neither explicitly or
implicitly by calling enq anywhere else in the call chain!}

\hypertarget{force-procedure1}{%
\subsubsection{\texorpdfstring{\texttt{force} :
procedure/1}{force : procedure/1}}\label{force-procedure1}}

Usage: \passthrough{\lstinline!(force fut) => any!}

Obtain the value of the computation encapsulated by future
\passthrough{\lstinline!fut!}, halting the current task until it has
been obtained. If the future never ends computation, e.g.~in an infinite
loop, the program may halt indefinitely.

See also: \passthrough{\lstinline!future, task, make-mutex!}.

\hypertarget{future-special-form}{%
\subsubsection{future : special form}\label{future-special-form}}

Usage: \passthrough{\lstinline!(future ...) => future!}

Turn the body of this form into a promise for a future value. The body
is executed in parallel and the final value can be retrieved by using
(force f) on the future returned by this macro.

See also: \passthrough{\lstinline!force, task!}.

\hypertarget{make-mutex-procedure1}{%
\subsubsection{\texorpdfstring{\texttt{make-mutex} :
procedure/1}{make-mutex : procedure/1}}\label{make-mutex-procedure1}}

Usage: \passthrough{\lstinline!(make-mutex) => mutex!}

Create a new mutex.

See also:
\passthrough{\lstinline!mutex-lock, mutex-unlock, mutex-rlock, mutex-runlock!}.

\hypertarget{mutex-lock-procedure1}{%
\subsubsection{\texorpdfstring{\texttt{mutex-lock} :
procedure/1}{mutex-lock : procedure/1}}\label{mutex-lock-procedure1}}

Usage: \passthrough{\lstinline!(mutex-lock m)!}

Lock the mutex \passthrough{\lstinline!m!} for writing. This may halt
the current task until the mutex has been unlocked by another task.

See also:
\passthrough{\lstinline!mutex-unlock, make-mutex, mutex-rlock, mutex-runlock!}.

\hypertarget{mutex-rlock-procedure1}{%
\subsubsection{\texorpdfstring{\texttt{mutex-rlock} :
procedure/1}{mutex-rlock : procedure/1}}\label{mutex-rlock-procedure1}}

Usage: \passthrough{\lstinline!(mutex-rlock m)!}

Lock the mutex \passthrough{\lstinline!m!} for reading. This will allow
other tasks to read from it, too, but may block if another task is
currently locking it for writing.

See also:
\passthrough{\lstinline!mutex-runlock, mutex-lock, mutex-unlock, make-mutex!}.

\hypertarget{mutex-runlock-procedure1}{%
\subsubsection{\texorpdfstring{\texttt{mutex-runlock} :
procedure/1}{mutex-runlock : procedure/1}}\label{mutex-runlock-procedure1}}

Usage: \passthrough{\lstinline!(mutex-runlock m)!}

Unlock the mutex \passthrough{\lstinline!m!} from reading.

See also:
\passthrough{\lstinline!mutex-lock, mutex-unlock, mutex-rlock, make-mutex!}.

\hypertarget{mutex-unlock-procedure1}{%
\subsubsection{\texorpdfstring{\texttt{mutex-unlock} :
procedure/1}{mutex-unlock : procedure/1}}\label{mutex-unlock-procedure1}}

Usage: \passthrough{\lstinline!(mutex-unlock m)!}

Unlock the mutex \passthrough{\lstinline!m!} for writing. This releases
ownership of the mutex and allows other tasks to lock it for writing.

See also:
\passthrough{\lstinline!mutex-lock, make-mutex, mutex-rlock, mutex-runlock!}.

\hypertarget{prune-task-table-procedure0}{%
\subsubsection{\texorpdfstring{\texttt{prune-task-table} :
procedure/0}{prune-task-table : procedure/0}}\label{prune-task-table-procedure0}}

Usage: \passthrough{\lstinline!(prune-task-table)!}

Remove tasks that are finished from the task table. This includes tasks
for which an error has occurred.

See also: \passthrough{\lstinline!task-remove, task, task?, task-run!}.

\hypertarget{run-at-procedure2}{%
\subsubsection{\texorpdfstring{\texttt{run-at} :
procedure/2}{run-at : procedure/2}}\label{run-at-procedure2}}

Usage: \passthrough{\lstinline!(run-at date repeater proc) => int!}

Run procedure \passthrough{\lstinline!proc!} with no arguments as task
periodically according to the specification in
\passthrough{\lstinline!spec!} and return the task ID for the periodic
task. Herbey, \passthrough{\lstinline!date!} is either a datetime
specification or one of `(now skip next-minute next-quarter
next-halfhour next-hour in-2-hours in-3-hours tomorrow next-week
next-month next-year), and \passthrough{\lstinline!repeater!} is nil or
a procedure that takes a task ID and unix-epoch-nanoseconds and yields a
new unix-epoch-nanoseconds value for the next time the procedure shall
be run. While the other names are self-explanatory, the 'skip
specification means that the task is not run immediately but rather that
it is first run at (repeater -1 (now)). Timing resolution for the
scheduler is about 1 minute. Consider using interrupts for periodic
events with smaller time resolutions. The scheduler uses relative
intervals and has 'drift'.

See also: \passthrough{\lstinline!task, task-send!}.

\textbf{Warning: Tasks scheduled by run-at are not persistent! They are
only run until the system is shutdown.}

\hypertarget{systask-special-form}{%
\subsubsection{systask : special form}\label{systask-special-form}}

Usage: \passthrough{\lstinline!(systask body ...)!}

Evaluate the expressions of \passthrough{\lstinline!body!} in parallel
in a system task, which is similar to a future but cannot be forced.

See also: \passthrough{\lstinline!future, task!}.

\hypertarget{task-procedure1}{%
\subsubsection{\texorpdfstring{\texttt{task} :
procedure/1}{task : procedure/1}}\label{task-procedure1}}

Usage: \passthrough{\lstinline!(task sel proc) => int!}

Create a new task for concurrently running
\passthrough{\lstinline!proc!}, a procedure that takes its own ID as
argument. The \passthrough{\lstinline!sel!} argument must be a symbol in
'(auto manual remove). If \passthrough{\lstinline!sel!} is 'remove, then
the task is always removed from the task table after it has finished,
even if an error has occurred. If sel is 'auto, then the task is removed
from the task table if it ends without producing an error. If
\passthrough{\lstinline!sel!} is 'manual then the task is not removed
from the task table, its state is either 'canceled, 'finished, or
'error, and it and must be removed manually with
\passthrough{\lstinline!task-remove!} or
\passthrough{\lstinline!prune-task-table!}. Broadcast messages are never
removed. Tasks are more heavy-weight than futures and allow for
message-passing.

See also:
\passthrough{\lstinline!task?, task-run, task-state, task-broadcast, task-send, task-recv, task-remove, prune-task-table!}.

\hypertarget{task-broadcast-procedure2}{%
\subsubsection{\texorpdfstring{\texttt{task-broadcast} :
procedure/2}{task-broadcast : procedure/2}}\label{task-broadcast-procedure2}}

Usage: \passthrough{\lstinline!(task-broadcast id msg)!}

Send a message from task \passthrough{\lstinline!id!} to the blackboard.
Tasks automatically send the message 'finished to the blackboard when
they are finished.

See also:
\passthrough{\lstinline!task, task?, task-run, task-state, task-send, task-recv!}.

\hypertarget{task-recv-procedure1}{%
\subsubsection{\texorpdfstring{\texttt{task-recv} :
procedure/1}{task-recv : procedure/1}}\label{task-recv-procedure1}}

Usage: \passthrough{\lstinline!(task-recv id) => any!}

Receive a message for task \passthrough{\lstinline!id!}, or nil if there
is no message. This is typically used by the task with
\passthrough{\lstinline!id!} itself to periodically check for new
messages while doing other work. By convention, if a task receives the
message 'end it ought to terminate at the next convenient occasion,
whereas upon receiving 'cancel it ought to terminate in an expedited
manner.

See also:
\passthrough{\lstinline!task-send, task, task?, task-run, task-state, task-broadcast!}.

\textbf{Warning: Busy polling for new messages in a tight loop is
inefficient and ought to be avoided.}

\hypertarget{task-remove-procedure1}{%
\subsubsection{\texorpdfstring{\texttt{task-remove} :
procedure/1}{task-remove : procedure/1}}\label{task-remove-procedure1}}

Usage: \passthrough{\lstinline!(task-remove id)!}

Remove task \passthrough{\lstinline!id!} from the task table. The task
can no longer be interacted with.

See also: \passthrough{\lstinline!task, task?, task-state!}.

\hypertarget{task-run-procedure1}{%
\subsubsection{\texorpdfstring{\texttt{task-run} :
procedure/1}{task-run : procedure/1}}\label{task-run-procedure1}}

Usage: \passthrough{\lstinline!(task-run id)!}

Run task \passthrough{\lstinline!id!}, which must have been previously
created with task. Attempting to run a task that is already running
results in an error unless \passthrough{\lstinline!silent?!} is true. If
silent? is true, the function does never produce an error.

See also:
\passthrough{\lstinline!task, task?, task-state, task-send, task-recv, task-broadcast-!}.

\hypertarget{task-schedule-procedure1}{%
\subsubsection{\texorpdfstring{\texttt{task-schedule} :
procedure/1}{task-schedule : procedure/1}}\label{task-schedule-procedure1}}

Usage: \passthrough{\lstinline!(task-schedule sel id)!}

Schedule task \passthrough{\lstinline!id!} for running, starting it as
soon as other tasks have finished. The scheduler attempts to avoid
running more than (cpunum) tasks at once.

See also: \passthrough{\lstinline!task, task-run!}.

\hypertarget{task-send-procedure2}{%
\subsubsection{\texorpdfstring{\texttt{task-send} :
procedure/2}{task-send : procedure/2}}\label{task-send-procedure2}}

Usage: \passthrough{\lstinline!(task-send id msg)!}

Send a message \passthrough{\lstinline!msg!} to task
\passthrough{\lstinline!id!}. The task needs to cooperatively use
task-recv to reply to the message. It is up to the receiving task what
to do with the message once it has been received, or how often to check
for new messages.

See also:
\passthrough{\lstinline!task-broadcast, task-recv, task, task?, task-run, task-state!}.

\hypertarget{task-state-procedure1}{%
\subsubsection{\texorpdfstring{\texttt{task-state} :
procedure/1}{task-state : procedure/1}}\label{task-state-procedure1}}

Usage: \passthrough{\lstinline!(task-state id) => sym!}

Return the state of the task, which is a symbol in '(finished error
stopped new waiting running).

See also:
\passthrough{\lstinline!task, task?, task-run, task-broadcast, task-recv, task-send!}.

\hypertarget{task-procedure1-1}{%
\subsubsection{\texorpdfstring{\texttt{task?} :
procedure/1}{task? : procedure/1}}\label{task-procedure1-1}}

Usage: \passthrough{\lstinline!(task? id) => bool!}

Check whether the given \passthrough{\lstinline!id!} is for a valid
task, return true if it is valid, nil otherwise.

See also:
\passthrough{\lstinline!task, task-run, task-state, task-broadcast, task-send, task-recv!}.

\hypertarget{wait-for-procedure2}{%
\subsubsection{\texorpdfstring{\texttt{wait-for} :
procedure/2}{wait-for : procedure/2}}\label{wait-for-procedure2}}

Usage: \passthrough{\lstinline!(wait-for dict key)!}

Block execution until the value for \passthrough{\lstinline!key!} in
\passthrough{\lstinline!dict!} is not-nil. This function may wait
indefinitely if no other thread sets the value for
\passthrough{\lstinline!key!} to not-nil.

See also:
\passthrough{\lstinline!wait-for*, future, force, wait-until, wait-until*!}.

\textbf{Warning: This cannot be used for synchronization of multiple
tasks due to potential race-conditions.}

\hypertarget{wait-for-procedure3}{%
\subsubsection{\texorpdfstring{\texttt{wait-for*} :
procedure/3}{wait-for* : procedure/3}}\label{wait-for-procedure3}}

Usage: \passthrough{\lstinline!(wait-for* dict key timeout)!}

Blocks execution until the value for \passthrough{\lstinline!key!} in
\passthrough{\lstinline!dict!} is not-nil or
\passthrough{\lstinline!timeout!} nanoseconds have passed, and returns
that value or nil if waiting timed out. If
\passthrough{\lstinline!timeout!} is negative, then the function waits
potentially indefinitely without any timeout. If a non-nil key is not
found, the function sleeps at least \emph{sync-wait-lower-bound}
nanoseconds and up to \emph{sync-wait-upper-bound} nanoseconds until it
looks for the key again.

See also:
\passthrough{\lstinline!future, force, wait-for, wait-until, wait-until*!}.

\textbf{Warning: This cannot be used for synchronization of multiple
tasks due to potential race-conditions.}

\hypertarget{wait-for-empty-procedure3}{%
\subsubsection{\texorpdfstring{\texttt{wait-for-empty*} :
procedure/3}{wait-for-empty* : procedure/3}}\label{wait-for-empty-procedure3}}

Usage: \passthrough{\lstinline!(wait-for-empty* dict key timeout)!}

Blocks execution until the \passthrough{\lstinline!key!} is no longer
present in \passthrough{\lstinline!dict!} or
\passthrough{\lstinline!timeout!} nanoseconds have passed. If
\passthrough{\lstinline!timeout!} is negative, then the function waits
potentially indefinitely without any timeout.

See also:
\passthrough{\lstinline!future, force, wait-for, wait-until, wait-until*!}.

\textbf{Warning: This cannot be used for synchronization of multiple
tasks due to potential race-conditions.}

\hypertarget{wait-until-procedure2}{%
\subsubsection{\texorpdfstring{\texttt{wait-until} :
procedure/2}{wait-until : procedure/2}}\label{wait-until-procedure2}}

Usage: \passthrough{\lstinline!(wait-until dict key pred)!}

Blocks execution until the unary predicate
\passthrough{\lstinline!pred!} returns true for the value at
\passthrough{\lstinline!key!} in \passthrough{\lstinline!dict!}. This
function may wait indefinitely if no other thread sets the value in such
a way that \passthrough{\lstinline!pred!} returns true when applied to
it.

See also:
\passthrough{\lstinline!wait-for, future, force, wait-until*!}.

\textbf{Warning: This cannot be used for synchronization of multiple
tasks due to potential race-conditions.}

\hypertarget{wait-until-procedure4}{%
\subsubsection{\texorpdfstring{\texttt{wait-until*} :
procedure/4}{wait-until* : procedure/4}}\label{wait-until-procedure4}}

Usage: \passthrough{\lstinline!(wait-until* dict key pred timeout)!}

Blocks execution until the unary predicate
\passthrough{\lstinline!pred!} returns true for the value at
\passthrough{\lstinline!key!} in \passthrough{\lstinline!dict!}, or
\passthrough{\lstinline!timeout!} nanoseconds have passed, and returns
the value or nil if waiting timed out. If
\passthrough{\lstinline!timeout!} is negative, then the function waits
potentially indefinitely without any timeout. If a non-nil key is not
found, the function sleeps at least \emph{sync-wait-lower-bound}
nanoseconds and up to \emph{sync-wait-upper-bound} nanoseconds until it
looks for the key again.

See also:
\passthrough{\lstinline!future, force, wait-for, wait-until*, wait-until!}.

\textbf{Warning: This cannot be used for synchronization of multiple
tasks due to potential race-conditions.}

\hypertarget{with-mutex-rlock-macro1-or-more}{%
\subsubsection{\texorpdfstring{\texttt{with-mutex-rlock} : macro/1 or
more}{with-mutex-rlock : macro/1 or more}}\label{with-mutex-rlock-macro1-or-more}}

Usage: \passthrough{\lstinline!(with-mutex-rlock m ...) => any!}

Execute the body with mutex \passthrough{\lstinline!m!} locked for
reading and unlock the mutex afterwards.

See also:
\passthrough{\lstinline!with-mutex-lock, make-mutex, mutex-lock, mutex-rlock, mutex-unlock, mutex-runlock!}.

\hypertarget{console-input-output}{%
\subsection{Console Input \& Output}\label{console-input-output}}

These functions access the operating system console (terminal) mostly
for string output.

\hypertarget{nl-procedure0}{%
\subsubsection{\texorpdfstring{\texttt{nl} :
procedure/0}{nl : procedure/0}}\label{nl-procedure0}}

Usage: \passthrough{\lstinline!(nl)!}

Display a newline, advancing the cursor to the next line.

See also: \passthrough{\lstinline!out, outy, output-at!}.

\hypertarget{prin1-procedure1}{%
\subsubsection{\texorpdfstring{\texttt{prin1} :
procedure/1}{prin1 : procedure/1}}\label{prin1-procedure1}}

Usage: \passthrough{\lstinline!(prin1 s)!}

Print \passthrough{\lstinline!s!} to the host OS terminal, where strings
are quoted.

See also: \passthrough{\lstinline!princ, terpri, out, outy!}.

\hypertarget{princ-procedure1}{%
\subsubsection{\texorpdfstring{\texttt{princ} :
procedure/1}{princ : procedure/1}}\label{princ-procedure1}}

Usage: \passthrough{\lstinline!(princ s)!}

Print \passthrough{\lstinline!s!} to the host OS terminal without
quoting strings.

See also: \passthrough{\lstinline!prin1, terpri, out, outy!}.

\hypertarget{print-procedure1}{%
\subsubsection{\texorpdfstring{\texttt{print} :
procedure/1}{print : procedure/1}}\label{print-procedure1}}

Usage: \passthrough{\lstinline!(print x)!}

Output \passthrough{\lstinline!x!} on the host OS console and end it
with a newline.

See also: \passthrough{\lstinline!prin1, princ!}.

\hypertarget{terpri-procedure0}{%
\subsubsection{\texorpdfstring{\texttt{terpri} :
procedure/0}{terpri : procedure/0}}\label{terpri-procedure0}}

Usage: \passthrough{\lstinline!(terpri)!}

Advance the host OS terminal to the next line.

See also: \passthrough{\lstinline!princ, out, outy!}.

\hypertarget{data-type-conversion}{%
\subsection{Data Type Conversion}\label{data-type-conversion}}

This section lists various ways in which one data type can be converted
to another.

\hypertarget{alist-dict-procedure1}{%
\subsubsection{\texorpdfstring{\texttt{alist-\textgreater{}dict} :
procedure/1}{alist-\textgreater dict : procedure/1}}\label{alist-dict-procedure1}}

Usage: \passthrough{\lstinline!(alist->dict li) => dict!}

Convert an association list \passthrough{\lstinline!li!} into a
dictionary. Note that the value will be the cdr of each list element,
not the second element, so you need to use an alist with proper pairs
'(a . b) if you want b to be a single value.

See also:
\passthrough{\lstinline!dict->alist, dict, dict->list, list->dict!}.

\hypertarget{array-list-procedure1}{%
\subsubsection{\texorpdfstring{\texttt{array-\textgreater{}list} :
procedure/1}{array-\textgreater list : procedure/1}}\label{array-list-procedure1}}

Usage: \passthrough{\lstinline!(array->list arr) => li!}

Convert array \passthrough{\lstinline!arr!} into a list.

See also: \passthrough{\lstinline!list->array, array!}.

\hypertarget{array-str-procedure1}{%
\subsubsection{\texorpdfstring{\texttt{array-\textgreater{}str} :
procedure/1}{array-\textgreater str : procedure/1}}\label{array-str-procedure1}}

Usage: \passthrough{\lstinline!(array-str arr) => s!}

Convert an array of unicode glyphs as integer values into a string. If
the given sequence is not a valid UTF-8 sequence, an error is thrown.

See also: \passthrough{\lstinline!str->array!}.

\hypertarget{ascii85-blob-procedure1}{%
\subsubsection{\texorpdfstring{\texttt{ascii85-\textgreater{}blob} :
procedure/1}{ascii85-\textgreater blob : procedure/1}}\label{ascii85-blob-procedure1}}

Usage: \passthrough{\lstinline!(ascii85->blob str) => blob!}

Convert the ascii85 encoded string \passthrough{\lstinline!str!} to a
binary blob. This will raise an error if \passthrough{\lstinline!str!}
is not a valid ascii85 encoded string.

See also:
\passthrough{\lstinline!blob->ascii85, base64->blob, str->blob, hex->blob!}.

\hypertarget{base64-blob-procedure1}{%
\subsubsection{\texorpdfstring{\texttt{base64-\textgreater{}blob} :
procedure/1}{base64-\textgreater blob : procedure/1}}\label{base64-blob-procedure1}}

Usage: \passthrough{\lstinline!(base64->blob str) => blob!}

Convert the base64 encoded string \passthrough{\lstinline!str!} to a
binary blob. This will raise an error if \passthrough{\lstinline!str!}
is not a valid base64 encoded string.

See also:
\passthrough{\lstinline!blob->base64, hex->blob, ascii85->blob, str->blob!}.

\hypertarget{blob-ascii85-procedure1-or-more}{%
\subsubsection{\texorpdfstring{\texttt{blob-\textgreater{}ascii85} :
procedure/1 or
more}{blob-\textgreater ascii85 : procedure/1 or more}}\label{blob-ascii85-procedure1-or-more}}

Usage: \passthrough{\lstinline!(blob->ascii85 b [start] [end]) => str!}

Convert the blob \passthrough{\lstinline!b!} to an ascii85 encoded
string. If the optional \passthrough{\lstinline!start!} and
\passthrough{\lstinline!end!} are provided, then only bytes from
\passthrough{\lstinline!start!} (inclusive) to
\passthrough{\lstinline!end!} (exclusive) are converted.

See also:
\passthrough{\lstinline!blob->hex, blob->str, blob->base64, valid?, blob?!}.

\hypertarget{blob-base64-procedure1-or-more}{%
\subsubsection{\texorpdfstring{\texttt{blob-\textgreater{}base64} :
procedure/1 or
more}{blob-\textgreater base64 : procedure/1 or more}}\label{blob-base64-procedure1-or-more}}

Usage: \passthrough{\lstinline!(blob->base64 b [start] [end]) => str!}

Convert the blob \passthrough{\lstinline!b!} to a base64 encoded string.
If the optional \passthrough{\lstinline!start!} and
\passthrough{\lstinline!end!} are provided, then only bytes from
\passthrough{\lstinline!start!} (inclusive) to
\passthrough{\lstinline!end!} (exclusive) are converted.

See also:
\passthrough{\lstinline!base64->blob, valid?, blob?, blob->str, blob->hex, blob->ascii85!}.

\hypertarget{blob-hex-procedure1-or-more}{%
\subsubsection{\texorpdfstring{\texttt{blob-\textgreater{}hex} :
procedure/1 or
more}{blob-\textgreater hex : procedure/1 or more}}\label{blob-hex-procedure1-or-more}}

Usage: \passthrough{\lstinline!(blob->hex b [start] [end]) => str!}

Convert the blob \passthrough{\lstinline!b!} to a hexadecimal string of
byte values. If the optional \passthrough{\lstinline!start!} and
\passthrough{\lstinline!end!} are provided, then only bytes from
\passthrough{\lstinline!start!} (inclusive) to
\passthrough{\lstinline!end!} (exclusive) are converted.

See also:
\passthrough{\lstinline!hex->blob, str->blob, valid?, blob?, blob->base64, blob->ascii85!}.

\hypertarget{blob-str-procedure1-or-more}{%
\subsubsection{\texorpdfstring{\texttt{blob-\textgreater{}str} :
procedure/1 or
more}{blob-\textgreater str : procedure/1 or more}}\label{blob-str-procedure1-or-more}}

Usage: \passthrough{\lstinline!(blob->str b [start] [end]) => str!}

Convert blob \passthrough{\lstinline!b!} into a string. Notice that the
string may contain binary data that is not suitable for displaying and
does not represent valid UTF-8 glyphs. If the optional
\passthrough{\lstinline!start!} and \passthrough{\lstinline!end!} are
provided, then only bytes from \passthrough{\lstinline!start!}
(inclusive) to \passthrough{\lstinline!end!} (exclusive) are converted.

See also: \passthrough{\lstinline!str->blob, valid?, blob?!}.

\hypertarget{char-str-procedure1}{%
\subsubsection{\texorpdfstring{\texttt{char-\textgreater{}str} :
procedure/1}{char-\textgreater str : procedure/1}}\label{char-str-procedure1}}

Usage: \passthrough{\lstinline!(char->str n) => str!}

Return a string containing the unicode char based on integer
\passthrough{\lstinline!n.!}

See also: \passthrough{\lstinline!str->char!}.

\hypertarget{chars-str-procedure1}{%
\subsubsection{\texorpdfstring{\texttt{chars-\textgreater{}str} :
procedure/1}{chars-\textgreater str : procedure/1}}\label{chars-str-procedure1}}

Usage: \passthrough{\lstinline!(chars->str a) => str!}

Convert an array of UTF-8 rune integers \passthrough{\lstinline!a!} into
a UTF-8 encoded string.

See also: \passthrough{\lstinline!str->runes, str->char, char->str!}.

\hypertarget{dict-alist-procedure1}{%
\subsubsection{\texorpdfstring{\texttt{dict-\textgreater{}alist} :
procedure/1}{dict-\textgreater alist : procedure/1}}\label{dict-alist-procedure1}}

Usage: \passthrough{\lstinline!(dict->alist d) => li!}

Convert a dictionary into an association list. Note that the resulting
alist will be a set of proper pairs of the form '(a . b) if the values
in the dictionary are not lists.

See also: \passthrough{\lstinline!dict, dict-map, dict->list!}.

\hypertarget{dict-array-procedure1}{%
\subsubsection{\texorpdfstring{\texttt{dict-\textgreater{}array} :
procedure/1}{dict-\textgreater array : procedure/1}}\label{dict-array-procedure1}}

Usage: \passthrough{\lstinline!(dict-array d) => array!}

Return an array that contains all key, value pairs of
\passthrough{\lstinline!d!}. A key comes directly before its value, but
otherwise the order is unspecified.

See also: \passthrough{\lstinline!dict->list, dict!}.

\hypertarget{dict-keys-procedure1}{%
\subsubsection{\texorpdfstring{\texttt{dict-\textgreater{}keys} :
procedure/1}{dict-\textgreater keys : procedure/1}}\label{dict-keys-procedure1}}

Usage: \passthrough{\lstinline!(dict->keys d) => li!}

Return the keys of dictionary \passthrough{\lstinline!d!} in arbitrary
order.

See also:
\passthrough{\lstinline!dict, dict->values, dict->alist, dict->list!}.

\hypertarget{dict-list-procedure1}{%
\subsubsection{\texorpdfstring{\texttt{dict-\textgreater{}list} :
procedure/1}{dict-\textgreater list : procedure/1}}\label{dict-list-procedure1}}

Usage: \passthrough{\lstinline!(dict->list d) => li!}

Return a list of the form '(key1 value1 key2 value2 \ldots), where the
order of key, value pairs is unspecified.

See also: \passthrough{\lstinline!dict->array, dict!}.

\hypertarget{dict-values-procedure1}{%
\subsubsection{\texorpdfstring{\texttt{dict-\textgreater{}values} :
procedure/1}{dict-\textgreater values : procedure/1}}\label{dict-values-procedure1}}

Usage: \passthrough{\lstinline!(dict->values d) => li!}

Return the values of dictionary \passthrough{\lstinline!d!} in arbitrary
order.

See also:
\passthrough{\lstinline!dict, dict->keys, dict->alist, dict->list!}.

\hypertarget{expr-str-procedure1}{%
\subsubsection{\texorpdfstring{\texttt{expr-\textgreater{}str} :
procedure/1}{expr-\textgreater str : procedure/1}}\label{expr-str-procedure1}}

Usage: \passthrough{\lstinline!(expr->str expr) => str!}

Convert a Lisp expression \passthrough{\lstinline!expr!} into a string.
Does not use a stream port.

See also:
\passthrough{\lstinline!str->expr, str->expr*, openstr, internalize, externalize!}.

\hypertarget{hex-blob-procedure1}{%
\subsubsection{\texorpdfstring{\texttt{hex-\textgreater{}blob} :
procedure/1}{hex-\textgreater blob : procedure/1}}\label{hex-blob-procedure1}}

Usage: \passthrough{\lstinline!(hex->blob str) => blob!}

Convert hex string \passthrough{\lstinline!str!} to a blob. This will
raise an error if \passthrough{\lstinline!str!} is not a valid hex
string.

See also:
\passthrough{\lstinline!blob->hex, base64->blob, ascii85->blob, str->blob!}.

\hypertarget{list-array-procedure1}{%
\subsubsection{\texorpdfstring{\texttt{list-\textgreater{}array} :
procedure/1}{list-\textgreater array : procedure/1}}\label{list-array-procedure1}}

Usage: \passthrough{\lstinline!(list->array li) => array!}

Convert the list \passthrough{\lstinline!li!} to an array.

See also: \passthrough{\lstinline!list, array, string, nth, seq?!}.

\hypertarget{list-set-procedure1}{%
\subsubsection{\texorpdfstring{\texttt{list-\textgreater{}set} :
procedure/1}{list-\textgreater set : procedure/1}}\label{list-set-procedure1}}

Usage: \passthrough{\lstinline!(list->set li) => dict!}

Create a dict containing true for each element of list
\passthrough{\lstinline!li.!}

See also:
\passthrough{\lstinline!make-set, set-element?, set-union, set-intersection, set-complement, set-difference, set?, set-empty!}.

\hypertarget{list-str-procedure1}{%
\subsubsection{\texorpdfstring{\texttt{list-\textgreater{}str} :
procedure/1}{list-\textgreater str : procedure/1}}\label{list-str-procedure1}}

Usage: \passthrough{\lstinline!(list->str li) => string!}

Return the string that is composed out of the chars in list
\passthrough{\lstinline!li.!}

See also: \passthrough{\lstinline!array->str, str->list, chars!}.

\hypertarget{set-list-procedure1}{%
\subsubsection{\texorpdfstring{\texttt{set-\textgreater{}list} :
procedure/1}{set-\textgreater list : procedure/1}}\label{set-list-procedure1}}

Usage: \passthrough{\lstinline!(set->list s) => li!}

Convert set \passthrough{\lstinline!s!} to a list of set elements.

See also:
\passthrough{\lstinline!list->set, make-set, set-element?, set-union, set-intersection, set-complement, set-difference, set?, set-empty!}.

\hypertarget{str-array-procedure1}{%
\subsubsection{\texorpdfstring{\texttt{str-\textgreater{}array} :
procedure/1}{str-\textgreater array : procedure/1}}\label{str-array-procedure1}}

Usage: \passthrough{\lstinline!(str->array s) => array!}

Return the string \passthrough{\lstinline!s!} as an array of unicode
glyph integer values.

See also: \passthrough{\lstinline!array->str!}.

\hypertarget{str-blob-procedure1}{%
\subsubsection{\texorpdfstring{\texttt{str-\textgreater{}blob} :
procedure/1}{str-\textgreater blob : procedure/1}}\label{str-blob-procedure1}}

Usage: \passthrough{\lstinline!(str->blob s) => blob!}

Convert string \passthrough{\lstinline!s!} into a blob.

See also: \passthrough{\lstinline!blob->str!}.

\hypertarget{str-char-procedure1}{%
\subsubsection{\texorpdfstring{\texttt{str-\textgreater{}char} :
procedure/1}{str-\textgreater char : procedure/1}}\label{str-char-procedure1}}

Usage: \passthrough{\lstinline!(str->char s)!}

Return the first character of \passthrough{\lstinline!s!} as unicode
integer.

See also: \passthrough{\lstinline!char->str!}.

\hypertarget{str-chars-procedure1}{%
\subsubsection{\texorpdfstring{\texttt{str-\textgreater{}chars} :
procedure/1}{str-\textgreater chars : procedure/1}}\label{str-chars-procedure1}}

Usage: \passthrough{\lstinline!(str->chars s) => array!}

Convert the UTF-8 string \passthrough{\lstinline!s!} into an array of
UTF-8 rune integers. An error may occur if the string is not a valid
UTF-8 string.

See also: \passthrough{\lstinline!runes->str, str->char, char->str!}.

\hypertarget{str-expr-procedure0-or-more}{%
\subsubsection{\texorpdfstring{\texttt{str-\textgreater{}expr} :
procedure/0 or
more}{str-\textgreater expr : procedure/0 or more}}\label{str-expr-procedure0-or-more}}

Usage: \passthrough{\lstinline!(str->expr s [default]) => any!}

Convert a string \passthrough{\lstinline!s!} into a Lisp expression. If
\passthrough{\lstinline!default!} is provided, it is returned if an
error occurs, otherwise an error is raised.

See also:
\passthrough{\lstinline!expr->str, str->expr*, openstr, externalize, internalize!}.

\hypertarget{str-expr-procedure0-or-more-1}{%
\subsubsection{\texorpdfstring{\texttt{str-\textgreater{}expr*} :
procedure/0 or
more}{str-\textgreater expr* : procedure/0 or more}}\label{str-expr-procedure0-or-more-1}}

Usage: \passthrough{\lstinline!(str->expr* s [default]) => li!}

Convert a string \passthrough{\lstinline!s!} into a list consisting of
the Lisp expressions in \passthrough{\lstinline!s!}. If
\passthrough{\lstinline!default!} is provided, then this value is put in
the result list whenever an error occurs. Otherwise an error is raised.
Notice that it might not always be obvious what expression in
\passthrough{\lstinline!s!} triggers an error, since this hinges on the
way the internal expession parser works.

See also:
\passthrough{\lstinline!str->expr, expr->str, openstr, internalize, externalize!}.

\hypertarget{str-list-procedure1}{%
\subsubsection{\texorpdfstring{\texttt{str-\textgreater{}list} :
procedure/1}{str-\textgreater list : procedure/1}}\label{str-list-procedure1}}

Usage: \passthrough{\lstinline!(str->list s) => list!}

Return the sequence of numeric chars that make up string
\passthrough{\lstinline!s.!}

See also:
\passthrough{\lstinline!str->array, list->str, array->str, chars!}.

\hypertarget{str-sym-procedure1}{%
\subsubsection{\texorpdfstring{\texttt{str-\textgreater{}sym} :
procedure/1}{str-\textgreater sym : procedure/1}}\label{str-sym-procedure1}}

Usage: \passthrough{\lstinline!(str->sym s) => sym!}

Convert a string into a symbol.

See also: \passthrough{\lstinline!sym->str, intern, make-symbol!}.

\hypertarget{sym-str-procedure1}{%
\subsubsection{\texorpdfstring{\texttt{sym-\textgreater{}str} :
procedure/1}{sym-\textgreater str : procedure/1}}\label{sym-str-procedure1}}

Usage: \passthrough{\lstinline!(sym->str sym) => str!}

Convert a symbol into a string.

See also: \passthrough{\lstinline!str->sym, intern, make-symbol!}.

\hypertarget{special-data-structures}{%
\subsection{Special Data Structures}\label{special-data-structures}}

This section lists some more specialized data structures and helper
functions for them.

\hypertarget{chars-procedure1}{%
\subsubsection{\texorpdfstring{\texttt{chars} :
procedure/1}{chars : procedure/1}}\label{chars-procedure1}}

Usage: \passthrough{\lstinline!(chars str) => dict!}

Return a charset based on \passthrough{\lstinline!str!}, i.e., dict with
the chars of \passthrough{\lstinline!str!} as keys and true as value.

See also: \passthrough{\lstinline!dict, get, set, contains!}.

\hypertarget{dequeue-macro1-or-more}{%
\subsubsection{\texorpdfstring{\texttt{dequeue!} : macro/1 or
more}{dequeue! : macro/1 or more}}\label{dequeue-macro1-or-more}}

Usage: \passthrough{\lstinline"(dequeue! sym [def]) => any"}

Get the next element from queue \passthrough{\lstinline!sym!}, which
must be the unquoted name of a variable, and return it. If a default
\passthrough{\lstinline!def!} is given, then this is returned if the
queue is empty, otherwise nil is returned.

See also:
\passthrough{\lstinline"make-queue, queue?, enqueue!, glance, queue-empty?, queue-len"}.

\hypertarget{enqueue-macro2}{%
\subsubsection{\texorpdfstring{\texttt{enqueue!} :
macro/2}{enqueue! : macro/2}}\label{enqueue-macro2}}

Usage: \passthrough{\lstinline"(enqueue! sym elem)"}

Put \passthrough{\lstinline!elem!} in queue
\passthrough{\lstinline!sym!}, where \passthrough{\lstinline!sym!} is
the unquoted name of a variable.

See also:
\passthrough{\lstinline"make-queue, queue?, dequeue!, glance, queue-empty?, queue-len"}.

\hypertarget{glance-procedure1}{%
\subsubsection{\texorpdfstring{\texttt{glance} :
procedure/1}{glance : procedure/1}}\label{glance-procedure1}}

Usage: \passthrough{\lstinline!(glance s [def]) => any!}

Peek the next element in a stack or queue without changing the data
structure. If default \passthrough{\lstinline!def!} is provided, this is
returned in case the stack or queue is empty; otherwise nil is returned.

See also:
\passthrough{\lstinline"make-queue, make-stack, queue?, enqueue?, dequeue?, queue-len, stack-len, pop!, push!"}.

\hypertarget{inchars-procedure2}{%
\subsubsection{\texorpdfstring{\texttt{inchars} :
procedure/2}{inchars : procedure/2}}\label{inchars-procedure2}}

Usage: \passthrough{\lstinline!(inchars char chars) => bool!}

Return true if char is in the charset chars, nil otherwise.

See also: \passthrough{\lstinline!chars, dict, get, set, has!}.

\hypertarget{make-queue-procedure0}{%
\subsubsection{\texorpdfstring{\texttt{make-queue} :
procedure/0}{make-queue : procedure/0}}\label{make-queue-procedure0}}

Usage: \passthrough{\lstinline!(make-queue) => array!}

Make a synchronized queue.

See also:
\passthrough{\lstinline"queue?, enqueue!, dequeue!, glance, queue-empty?, queue-len"}.

\textbf{Warning: Never change the array of a synchronized data structure
directly, or your warranty is void!}

\hypertarget{make-set-procedure0-or-more}{%
\subsubsection{\texorpdfstring{\texttt{make-set} : procedure/0 or
more}{make-set : procedure/0 or more}}\label{make-set-procedure0-or-more}}

Usage: \passthrough{\lstinline!(make-set [arg1] ... [argn]) => dict!}

Create a dictionary out of arguments \passthrough{\lstinline!arg1!} to
\passthrough{\lstinline!argn!} that stores true for very argument.

See also:
\passthrough{\lstinline!list->set, set->list, set-element?, set-union, set-intersection, set-complement, set-difference, set?, set-empty?!}.

\hypertarget{make-stack-procedure0}{%
\subsubsection{\texorpdfstring{\texttt{make-stack} :
procedure/0}{make-stack : procedure/0}}\label{make-stack-procedure0}}

Usage: \passthrough{\lstinline!(make-stack) => array!}

Make a synchronized stack.

See also:
\passthrough{\lstinline"stack?, push!, pop!, stack-empty?, stack-len, glance"}.

\textbf{Warning: Never change the array of a synchronized data structure
directly, or your warranty is void!}

\hypertarget{pop-macro1-or-more}{%
\subsubsection{\texorpdfstring{\texttt{pop!} : macro/1 or
more}{pop! : macro/1 or more}}\label{pop-macro1-or-more}}

Usage: \passthrough{\lstinline"(pop! sym [def]) => any"}

Get the next element from stack \passthrough{\lstinline!sym!}, which
must be the unquoted name of a variable, and return it. If a default
\passthrough{\lstinline!def!} is given, then this is returned if the
queue is empty, otherwise nil is returned.

See also:
\passthrough{\lstinline"make-stack, stack?, push!, stack-len, stack-empty?, glance"}.

\hypertarget{push-macro2}{%
\subsubsection{\texorpdfstring{\texttt{push!} :
macro/2}{push! : macro/2}}\label{push-macro2}}

Usage: \passthrough{\lstinline"(push! sym elem)"}

Put \passthrough{\lstinline!elem!} in stack
\passthrough{\lstinline!sym!}, where \passthrough{\lstinline!sym!} is
the unquoted name of a variable.

See also:
\passthrough{\lstinline"make-stack, stack?, pop!, stack-len, stack-empty?, glance"}.

\hypertarget{queue-empty-procedure1}{%
\subsubsection{\texorpdfstring{\texttt{queue-empty?} :
procedure/1}{queue-empty? : procedure/1}}\label{queue-empty-procedure1}}

Usage: \passthrough{\lstinline!(queue-empty? q) => bool!}

Return true if the queue \passthrough{\lstinline!q!} is empty, nil
otherwise.

See also:
\passthrough{\lstinline"make-queue, queue?, enqueue!, dequeue!, glance, queue-len"}.

\hypertarget{queue-len-procedure1}{%
\subsubsection{\texorpdfstring{\texttt{queue-len} :
procedure/1}{queue-len : procedure/1}}\label{queue-len-procedure1}}

Usage: \passthrough{\lstinline!(queue-len q) => int!}

Return the length of the queue \passthrough{\lstinline!q.!}

See also:
\passthrough{\lstinline"make-queue, queue?, enqueue!, dequeue!, glance, queue-len"}.

\textbf{Warning: Be advised that this is of limited use in some
concurrent contexts, since the length of the queue might have changed
already once you've obtained it!}

\hypertarget{queue-procedure1}{%
\subsubsection{\texorpdfstring{\texttt{queue?} :
procedure/1}{queue? : procedure/1}}\label{queue-procedure1}}

Usage: \passthrough{\lstinline!(queue? q) => bool!}

Return true if \passthrough{\lstinline!q!} is a queue, nil otherwise.

See also:
\passthrough{\lstinline"make-queue, enqueue!, dequeue, glance, queue-empty?, queue-len"}.

\hypertarget{set-complement-procedure2}{%
\subsubsection{\texorpdfstring{\texttt{set-complement} :
procedure/2}{set-complement : procedure/2}}\label{set-complement-procedure2}}

Usage: \passthrough{\lstinline!(set-complement a domain) => set!}

Return all elements in \passthrough{\lstinline!domain!} that are not
elements of \passthrough{\lstinline!a.!}

See also:
\passthrough{\lstinline!list->set, set->list, make-set, set-element?, set-union, set-difference, set-intersection, set?, set-empty?, set-subset?, set-equal?!}.

\hypertarget{set-difference-procedure2}{%
\subsubsection{\texorpdfstring{\texttt{set-difference} :
procedure/2}{set-difference : procedure/2}}\label{set-difference-procedure2}}

Usage: \passthrough{\lstinline!(set-difference a b) => set!}

Return the set-theoretic difference of set \passthrough{\lstinline!a!}
minus set \passthrough{\lstinline!b!}, i.e., all elements in
\passthrough{\lstinline!a!} that are not in \passthrough{\lstinline!b.!}

See also:
\passthrough{\lstinline!list->set, set->list, make-set, set-element?, set-union, set-intersection, set-complement, set?, set-empty?, set-subset?, set-equal?!}.

\hypertarget{set-element-procedure2}{%
\subsubsection{\texorpdfstring{\texttt{set-element?} :
procedure/2}{set-element? : procedure/2}}\label{set-element-procedure2}}

Usage: \passthrough{\lstinline!(set-element? s elem) => bool!}

Return true if set \passthrough{\lstinline!s!} has element
\passthrough{\lstinline!elem!}, nil otherwise.

See also:
\passthrough{\lstinline!make-set, list->set, set->list, set-union, set-intersection, set-complement, set-difference, set?, set-empty?!}.

\hypertarget{set-empty-procedure1}{%
\subsubsection{\texorpdfstring{\texttt{set-empty?} :
procedure/1}{set-empty? : procedure/1}}\label{set-empty-procedure1}}

Usage: \passthrough{\lstinline!(set-empty? s) => bool!}

Return true if set \passthrough{\lstinline!s!} is empty, nil otherwise.

See also:
\passthrough{\lstinline!make-set, list->set, set->list, set-union, set-intersection, set-complement, set-difference, set?!}.

\hypertarget{set-equal-procedure2}{%
\subsubsection{\texorpdfstring{\texttt{set-equal?} :
procedure/2}{set-equal? : procedure/2}}\label{set-equal-procedure2}}

Usage: \passthrough{\lstinline!(set-equal? a b) => bool!}

Return true if \passthrough{\lstinline!a!} and
\passthrough{\lstinline!b!} contain the same elements.

See also:
\passthrough{\lstinline!set-subset?, list->set, set-element?, set->list, set-union, set-difference, set-intersection, set-complement, set?, set-empty?!}.

\hypertarget{set-intersection-procedure2}{%
\subsubsection{\texorpdfstring{\texttt{set-intersection} :
procedure/2}{set-intersection : procedure/2}}\label{set-intersection-procedure2}}

Usage: \passthrough{\lstinline!(set-intersection a b) => set!}

Return the intersection of sets \passthrough{\lstinline!a!} and
\passthrough{\lstinline!b!}, i.e., the set of elements that are both in
\passthrough{\lstinline!a!} and in \passthrough{\lstinline!b.!}

See also:
\passthrough{\lstinline!list->set, set->list, make-set, set-element?, set-union, set-complement, set-difference, set?, set-empty?, set-subset?, set-equal?!}.

\hypertarget{set-subset-procedure2}{%
\subsubsection{\texorpdfstring{\texttt{set-subset?} :
procedure/2}{set-subset? : procedure/2}}\label{set-subset-procedure2}}

Usage: \passthrough{\lstinline!(set-subset? a b) => bool!}

Return true if \passthrough{\lstinline!a!} is a subset of
\passthrough{\lstinline!b!}, nil otherwise.

See also:
\passthrough{\lstinline!set-equal?, list->set, set->list, make-set, set-element?, set-union, set-difference, set-intersection, set-complement, set?, set-empty?!}.

\hypertarget{set-union-procedure2}{%
\subsubsection{\texorpdfstring{\texttt{set-union} :
procedure/2}{set-union : procedure/2}}\label{set-union-procedure2}}

Usage: \passthrough{\lstinline!(set-union a b) => set!}

Return the union of sets \passthrough{\lstinline!a!} and
\passthrough{\lstinline!b!} containing all elements that are in
\passthrough{\lstinline!a!} or in \passthrough{\lstinline!b!} (or both).

See also:
\passthrough{\lstinline!list->set, set->list, make-set, set-element?, set-intersection, set-complement, set-difference, set?, set-empty?!}.

\hypertarget{set-procedure1}{%
\subsubsection{\texorpdfstring{\texttt{set?} :
procedure/1}{set? : procedure/1}}\label{set-procedure1}}

Usage: \passthrough{\lstinline!(set? x) => bool!}

Return true if \passthrough{\lstinline!x!} can be used as a set, nil
otherwise.

See also:
\passthrough{\lstinline!list->set, make-set, set->list, set-element?, set-union, set-intersection, set-complement, set-difference, set-empty?!}.

\hypertarget{stack-empty-procedure1}{%
\subsubsection{\texorpdfstring{\texttt{stack-empty?} :
procedure/1}{stack-empty? : procedure/1}}\label{stack-empty-procedure1}}

Usage: \passthrough{\lstinline!(queue-empty? s) => bool!}

Return true if the stack \passthrough{\lstinline!s!} is empty, nil
otherwise.

See also:
\passthrough{\lstinline"make-stack, stack?, push!, pop!, stack-len, glance"}.

\hypertarget{stack-len-procedure1}{%
\subsubsection{\texorpdfstring{\texttt{stack-len} :
procedure/1}{stack-len : procedure/1}}\label{stack-len-procedure1}}

Usage: \passthrough{\lstinline!(stack-len s) => int!}

Return the length of the stack \passthrough{\lstinline!s.!}

See also:
\passthrough{\lstinline"make-queue, queue?, enqueue!, dequeue!, glance, queue-len"}.

\textbf{Warning: Be advised that this is of limited use in some
concurrent contexts, since the length of the queue might have changed
already once you've obtained it!}

\hypertarget{stack-procedure1}{%
\subsubsection{\texorpdfstring{\texttt{stack?} :
procedure/1}{stack? : procedure/1}}\label{stack-procedure1}}

Usage: \passthrough{\lstinline!(stack? q) => bool!}

Return true if \passthrough{\lstinline!q!} is a stack, nil otherwise.

See also:
\passthrough{\lstinline"make-stack, push!, pop!, stack-empty?, stack-len, glance"}.

\hypertarget{dictionaries}{%
\subsection{Dictionaries}\label{dictionaries}}

Dictionaries are thread-safe key-value repositories held in memory. They
are internally based on hash tables and have fast access.

\hypertarget{delete-procedure2}{%
\subsubsection{\texorpdfstring{\texttt{delete} :
procedure/2}{delete : procedure/2}}\label{delete-procedure2}}

Usage: \passthrough{\lstinline!(delete d key)!}

Remove the value for \passthrough{\lstinline!key!} in dict
\passthrough{\lstinline!d!}. This also removes the key.

See also: \passthrough{\lstinline!dict?, get, set!}.

\hypertarget{dict-procedure0-or-more}{%
\subsubsection{\texorpdfstring{\texttt{dict} : procedure/0 or
more}{dict : procedure/0 or more}}\label{dict-procedure0-or-more}}

Usage: \passthrough{\lstinline!(dict [li]) => dict!}

Create a dictionary. The option \passthrough{\lstinline!li!} must be a
list of the form '(key1 value1 key2 value2 \ldots). Dictionaries are
unordered, hence also not sequences. Dictionaries are safe for
concurrent access.

See also: \passthrough{\lstinline!array, list!}.

\hypertarget{dict-copy-procedure1}{%
\subsubsection{\texorpdfstring{\texttt{dict-copy} :
procedure/1}{dict-copy : procedure/1}}\label{dict-copy-procedure1}}

Usage: \passthrough{\lstinline!(dict-copy d) => dict!}

Return a copy of dict \passthrough{\lstinline!d.!}

See also: \passthrough{\lstinline!dict, dict?!}.

\hypertarget{dict-empty-procedure1}{%
\subsubsection{\texorpdfstring{\texttt{dict-empty?} :
procedure/1}{dict-empty? : procedure/1}}\label{dict-empty-procedure1}}

Usage: \passthrough{\lstinline!(dict-empty? d) => bool!}

Return true if dict \passthrough{\lstinline!d!} is empty, nil otherwise.
As crazy as this may sound, this can have O(n) complexity if the dict is
not empty, but it is still going to be more efficient than any other
method.

See also: \passthrough{\lstinline!dict!}.

\hypertarget{dict-foreach-procedure2}{%
\subsubsection{\texorpdfstring{\texttt{dict-foreach} :
procedure/2}{dict-foreach : procedure/2}}\label{dict-foreach-procedure2}}

Usage: \passthrough{\lstinline!(dict-foreach d proc)!}

Call \passthrough{\lstinline!proc!} for side-effects with the key and
value for each key, value pair in dict \passthrough{\lstinline!d.!}

See also: \passthrough{\lstinline"dict-map!, dict?, dict"}.

\hypertarget{dict-map-procedure2}{%
\subsubsection{\texorpdfstring{\texttt{dict-map} :
procedure/2}{dict-map : procedure/2}}\label{dict-map-procedure2}}

Usage: \passthrough{\lstinline!(dict-map dict proc) => dict!}

Returns a copy of \passthrough{\lstinline!dict!} with
\passthrough{\lstinline!proc!} applies to each key value pair as
aruments. Keys are immutable, so \passthrough{\lstinline!proc!} must
take two arguments and return the new value.

See also: \passthrough{\lstinline"dict-map!, map"}.

\hypertarget{dict-map-procedure2-1}{%
\subsubsection{\texorpdfstring{\texttt{dict-map!} :
procedure/2}{dict-map! : procedure/2}}\label{dict-map-procedure2-1}}

Usage: \passthrough{\lstinline"(dict-map! d proc)"}

Apply procedure \passthrough{\lstinline!proc!} which takes the key and
value as arguments to each key, value pair in dict
\passthrough{\lstinline!d!} and set the respective value in
\passthrough{\lstinline!d!} to the result of
\passthrough{\lstinline!proc!}. Keys are not changed.

See also: \passthrough{\lstinline!dict, dict?, dict-foreach!}.

\hypertarget{dict-merge-procedure2}{%
\subsubsection{\texorpdfstring{\texttt{dict-merge} :
procedure/2}{dict-merge : procedure/2}}\label{dict-merge-procedure2}}

Usage: \passthrough{\lstinline!(dict-merge a b) => dict!}

Create a new dict that contains all key-value pairs from dicts
\passthrough{\lstinline!a!} and \passthrough{\lstinline!b!}. Note that
this function is not symmetric. If a key is in both
\passthrough{\lstinline!a!} and \passthrough{\lstinline!b!}, then the
key value pair in \passthrough{\lstinline!a!} is retained for this key.

See also:
\passthrough{\lstinline"dict, dict-map, dict-map!, dict-foreach"}.

\hypertarget{dict-procedure1}{%
\subsubsection{\texorpdfstring{\texttt{dict?} :
procedure/1}{dict? : procedure/1}}\label{dict-procedure1}}

Usage: \passthrough{\lstinline!(dict? obj) => bool!}

Return true if \passthrough{\lstinline!obj!} is a dict, nil otherwise.

See also: \passthrough{\lstinline!dict!}.

\hypertarget{get-procedure2-or-more}{%
\subsubsection{\texorpdfstring{\texttt{get} : procedure/2 or
more}{get : procedure/2 or more}}\label{get-procedure2-or-more}}

Usage: \passthrough{\lstinline!(get dict key [default]) => any!}

Get the value for \passthrough{\lstinline!key!} in
\passthrough{\lstinline!dict!}, return \passthrough{\lstinline!default!}
if there is no value for \passthrough{\lstinline!key!}. If
\passthrough{\lstinline!default!} is omitted, then nil is returned.
Provide your own default if you want to store nil.

See also: \passthrough{\lstinline!dict, dict?, set!}.

\hypertarget{get-or-set-procedure3}{%
\subsubsection{\texorpdfstring{\texttt{get-or-set} :
procedure/3}{get-or-set : procedure/3}}\label{get-or-set-procedure3}}

Usage: \passthrough{\lstinline!(get-or-set d key value)!}

Get the value for \passthrough{\lstinline!key!} in dict
\passthrough{\lstinline!d!} if it already exists, otherwise set it to
\passthrough{\lstinline!value.!}

See also: \passthrough{\lstinline!dict?, get, set!}.

\hypertarget{getstacked-procedure3}{%
\subsubsection{\texorpdfstring{\texttt{getstacked} :
procedure/3}{getstacked : procedure/3}}\label{getstacked-procedure3}}

Usage: \passthrough{\lstinline!(getstacked dict key default)!}

Get the topmost element from the stack stored at
\passthrough{\lstinline!key!} in \passthrough{\lstinline!dict!}. If the
stack is empty or no stack is stored at key, then
\passthrough{\lstinline!default!} is returned.

See also: \passthrough{\lstinline!pushstacked, popstacked!}.

\hypertarget{has-procedure2}{%
\subsubsection{\texorpdfstring{\texttt{has} :
procedure/2}{has : procedure/2}}\label{has-procedure2}}

Usage: \passthrough{\lstinline!(has dict key) => bool!}

Return true if the dict \passthrough{\lstinline!dict!} contains an entry
for \passthrough{\lstinline!key!}, nil otherwise.

See also: \passthrough{\lstinline!dict, get, set!}.

\hypertarget{has-key-procedure2}{%
\subsubsection{\texorpdfstring{\texttt{has-key?} :
procedure/2}{has-key? : procedure/2}}\label{has-key-procedure2}}

Usage: \passthrough{\lstinline!(has-key? d key) => bool!}

Return true if \passthrough{\lstinline!d!} has key
\passthrough{\lstinline!key!}, nil otherwise.

See also: \passthrough{\lstinline!dict?, get, set, delete!}.

\hypertarget{popstacked-procedure3}{%
\subsubsection{\texorpdfstring{\texttt{popstacked} :
procedure/3}{popstacked : procedure/3}}\label{popstacked-procedure3}}

Usage: \passthrough{\lstinline!(popstacked dict key default)!}

Get the topmost element from the stack stored at
\passthrough{\lstinline!key!} in \passthrough{\lstinline!dict!} and
remove it from the stack. If the stack is empty or no stack is stored at
key, then \passthrough{\lstinline!default!} is returned.

See also: \passthrough{\lstinline!pushstacked, getstacked!}.

\hypertarget{pushstacked-procedure3}{%
\subsubsection{\texorpdfstring{\texttt{pushstacked} :
procedure/3}{pushstacked : procedure/3}}\label{pushstacked-procedure3}}

Usage: \passthrough{\lstinline!(pushstacked dict key datum)!}

Push \passthrough{\lstinline!datum!} onto the stack maintained under
\passthrough{\lstinline!key!} in the \passthrough{\lstinline!dict.!}

See also: \passthrough{\lstinline!getstacked, popstacked!}.

\hypertarget{set-procedure3}{%
\subsubsection{\texorpdfstring{\texttt{set} :
procedure/3}{set : procedure/3}}\label{set-procedure3}}

Usage: \passthrough{\lstinline!(set d key value)!}

Set \passthrough{\lstinline!value!} for \passthrough{\lstinline!key!} in
dict \passthrough{\lstinline!d.!}

See also: \passthrough{\lstinline!dict, get, get-or-set!}.

\hypertarget{set-procedure2}{%
\subsubsection{\texorpdfstring{\texttt{set*} :
procedure/2}{set* : procedure/2}}\label{set-procedure2}}

Usage: \passthrough{\lstinline!(set* d li)!}

Set in dict \passthrough{\lstinline!d!} the keys and values in list
\passthrough{\lstinline!li!}. The list \passthrough{\lstinline!li!} must
be of the form (key-1 value-1 key-2 value-2 \ldots{} key-n value-n).
This function may be slightly faster than using individual
\passthrough{\lstinline!set!} operations.

See also: \passthrough{\lstinline!dict, set!}.

\hypertarget{equality-predicates}{%
\subsection{Equality Predicates}\label{equality-predicates}}

Equality predicates are used to test whether two values are equal in
some sense.

\hypertarget{eq-procedure2}{%
\subsubsection{\texorpdfstring{\texttt{eq?} :
procedure/2}{eq? : procedure/2}}\label{eq-procedure2}}

Usage: \passthrough{\lstinline!(eq? x y) => bool!}

Return true if \passthrough{\lstinline!x!} and
\passthrough{\lstinline!y!} are equal, nil otherwise. In contrast to
other LISPs, eq? checks for deep equality of arrays and dicts. However,
lists are compared by checking whether they are the same cell in memory.
Use \passthrough{\lstinline!equal?!} to check for deep equality of lists
and other objects.

See also: \passthrough{\lstinline!equal?!}.

\hypertarget{eql-procedure2}{%
\subsubsection{\texorpdfstring{\texttt{eql?} :
procedure/2}{eql? : procedure/2}}\label{eql-procedure2}}

Usage: \passthrough{\lstinline!(eql? x y) => bool!}

Returns true if \passthrough{\lstinline!x!} is equal to
\passthrough{\lstinline!y!}, nil otherwise. This is currently the same
as equal? but the behavior might change.

See also: \passthrough{\lstinline!equal?!}.

\textbf{Warning: Deprecated.}

\hypertarget{file-input-output}{%
\subsection{File Input \& Output}\label{file-input-output}}

These functions allow direct access for reading and writing to files.
This module requires the \passthrough{\lstinline!fileio!} build tag.

\hypertarget{close-procedure1}{%
\subsubsection{\texorpdfstring{\texttt{close} :
procedure/1}{close : procedure/1}}\label{close-procedure1}}

Usage: \passthrough{\lstinline!(close p)!}

Close the port \passthrough{\lstinline!p!}. Calling close twice on the
same port should be avoided.

See also: \passthrough{\lstinline!open, stropen!}.

\hypertarget{dir-procedure1}{%
\subsubsection{\texorpdfstring{\texttt{dir} :
procedure/1}{dir : procedure/1}}\label{dir-procedure1}}

Usage: \passthrough{\lstinline!(dir [path]) => li!}

Obtain a directory list for \passthrough{\lstinline!path!}. If
\passthrough{\lstinline!path!} is not specified, the current working
directory is listed.

See also: \passthrough{\lstinline!dir?, open, close, read, write!}.

\hypertarget{dir-procedure1-1}{%
\subsubsection{\texorpdfstring{\texttt{dir?} :
procedure/1}{dir? : procedure/1}}\label{dir-procedure1-1}}

Usage: \passthrough{\lstinline!(dir? path) => bool!}

Check if the file at \passthrough{\lstinline!path!} is a directory and
return true, nil if the file does not exist or is not a directory.

See also:
\passthrough{\lstinline!file-exists?, dir, open, close, read, write!}.

\hypertarget{fdelete-procedure1}{%
\subsubsection{\texorpdfstring{\texttt{fdelete} :
procedure/1}{fdelete : procedure/1}}\label{fdelete-procedure1}}

Usage: \passthrough{\lstinline!(fdelete path)!}

Removes the file or directory at \passthrough{\lstinline!path.!}

See also: \passthrough{\lstinline!file-exists?, dir?, dir!}.

\textbf{Warning: This function also deletes directories containing files
and all of their subdirectories!}

\hypertarget{file-port-procedure1}{%
\subsubsection{\texorpdfstring{\texttt{file-port?} :
procedure/1}{file-port? : procedure/1}}\label{file-port-procedure1}}

Usage: \passthrough{\lstinline!(file-port? p) => bool!}

Return true if \passthrough{\lstinline!p!} is a file port, nil
otherwise.

See also: \passthrough{\lstinline!port?, str-port?, open, stropen!}.

\hypertarget{open-procedure1-or-more}{%
\subsubsection{\texorpdfstring{\texttt{open} : procedure/1 or
more}{open : procedure/1 or more}}\label{open-procedure1-or-more}}

Usage:
\passthrough{\lstinline!(open file-path [modes] [permissions]) => int!}

Open the file at \passthrough{\lstinline!file-path!} for reading and
writing, and return the stream ID. The optional
\passthrough{\lstinline!modes!} argument must be a list containing one
of `(read write read-write) for read, write, or read-write access
respectively, and may contain any of the following symbols: 'append to
append to an existing file, 'create for creating the file if it doesn't
exist, 'exclusive for exclusive file access, 'truncate for truncating
the file if it exists, and 'sync for attempting to sync file access. The
optional \passthrough{\lstinline!permissions!} argument must be a
numeric value specifying the Unix file permissions of the file. If these
are omitted, then default values'(read-write append create) and 0640 are
used.

See also: \passthrough{\lstinline!stropen, close, read, write!}.

\hypertarget{read-procedure1}{%
\subsubsection{\texorpdfstring{\texttt{read} :
procedure/1}{read : procedure/1}}\label{read-procedure1}}

Usage: \passthrough{\lstinline!(read p) => any!}

Read an expression from input port \passthrough{\lstinline!p.!}

See also: \passthrough{\lstinline!input, write!}.

\hypertarget{read-binary-procedure3}{%
\subsubsection{\texorpdfstring{\texttt{read-binary} :
procedure/3}{read-binary : procedure/3}}\label{read-binary-procedure3}}

Usage: \passthrough{\lstinline!(read-binary p buff n) => int!}

Read \passthrough{\lstinline!n!} or less bytes from input port
\passthrough{\lstinline!p!} into binary blob
\passthrough{\lstinline!buff!}. If \passthrough{\lstinline!buff!} is
smaller than \passthrough{\lstinline!n!}, then an error is raised. If
less than \passthrough{\lstinline!n!} bytes are available before the end
of file is reached, then the amount k of bytes is read into
\passthrough{\lstinline!buff!} and k is returned. If the end of file is
reached and no byte has been read, then 0 is returned. So to loop
through this, read into the buffer and do something with it while the
amount of bytes returned is larger than 0.

See also: \passthrough{\lstinline!write-binary, read, close, open!}.

\hypertarget{read-string-procedure2}{%
\subsubsection{\texorpdfstring{\texttt{read-string} :
procedure/2}{read-string : procedure/2}}\label{read-string-procedure2}}

Usage: \passthrough{\lstinline!(read-string p delstr) => str!}

Reads a string from port \passthrough{\lstinline!p!} until the
single-byte delimiter character in \passthrough{\lstinline!delstr!} is
encountered, and returns the string including the delimiter. If the
input ends before the delimiter is encountered, it returns the string up
until EOF. Notice that if the empty string is returned then the end of
file must have been encountered, since otherwise the string would
contain the delimiter.

See also:
\passthrough{\lstinline!read, read-binary, write-string, write, read, close, open!}.

\hypertarget{str-port-procedure1}{%
\subsubsection{\texorpdfstring{\texttt{str-port?} :
procedure/1}{str-port? : procedure/1}}\label{str-port-procedure1}}

Usage: \passthrough{\lstinline!(str-port? p) => bool!}

Return true if \passthrough{\lstinline!p!} is a string port, nil
otherwise.

See also: \passthrough{\lstinline!port?, file-port?, stropen, open!}.

\hypertarget{write-procedure2}{%
\subsubsection{\texorpdfstring{\texttt{write} :
procedure/2}{write : procedure/2}}\label{write-procedure2}}

Usage: \passthrough{\lstinline!(write p datum) => int!}

Write \passthrough{\lstinline!datum!} to output port
\passthrough{\lstinline!p!} and return the number of bytes written.

See also:
\passthrough{\lstinline!write-binary, write-binary-at, read, close, open!}.

\hypertarget{write-binary-procedure4}{%
\subsubsection{\texorpdfstring{\texttt{write-binary} :
procedure/4}{write-binary : procedure/4}}\label{write-binary-procedure4}}

Usage: \passthrough{\lstinline!(write-binary p buff n offset) => int!}

Write \passthrough{\lstinline!n!} bytes starting at
\passthrough{\lstinline!offset!} in binary blob
\passthrough{\lstinline!buff!} to the stream port
\passthrough{\lstinline!p!}. This function returns the number of bytes
actually written.

See also:
\passthrough{\lstinline!write-binary-at, read-binary, write, close, open!}.

\hypertarget{write-binary-at-procedure5}{%
\subsubsection{\texorpdfstring{\texttt{write-binary-at} :
procedure/5}{write-binary-at : procedure/5}}\label{write-binary-at-procedure5}}

Usage:
\passthrough{\lstinline!(write-binary-at p buff n offset fpos) => int!}

Write \passthrough{\lstinline!n!} bytes starting at
\passthrough{\lstinline!offset!} in binary blob
\passthrough{\lstinline!buff!} to the seekable stream port
\passthrough{\lstinline!p!} at the stream position
\passthrough{\lstinline!fpos!}. If there is not enough data in
\passthrough{\lstinline!p!} to overwrite at position
\passthrough{\lstinline!fpos!}, then an error is caused and only part of
the data might be written. The function returns the number of bytes
actually written.

See also:
\passthrough{\lstinline!read-binary, write-binary, write, close, open!}.

\hypertarget{write-string-procedure2}{%
\subsubsection{\texorpdfstring{\texttt{write-string} :
procedure/2}{write-string : procedure/2}}\label{write-string-procedure2}}

Usage: \passthrough{\lstinline!(write-string p s) => int!}

Write string \passthrough{\lstinline!s!} to output port
\passthrough{\lstinline!p!} and return the number of bytes written. LF
are \emph{not} automatically converted to CR LF sequences on windows.

See also:
\passthrough{\lstinline!write, write-binary, write-binary-at, read, close, open!}.

\hypertarget{floating-point-arithmetics-package}{%
\subsection{Floating Point Arithmetics
Package}\label{floating-point-arithmetics-package}}

The package \passthrough{\lstinline!fl!} provides floating point
arithmetics functions. They require the given number not to exceed a
value that can be held by a 64 bit float in the range 2.2E-308 to
1.7E+308.

\hypertarget{fl.abs-procedure1}{%
\subsubsection{\texorpdfstring{\texttt{fl.abs} :
procedure/1}{fl.abs : procedure/1}}\label{fl.abs-procedure1}}

Usage: \passthrough{\lstinline!(fl.abs x) => fl!}

Return the absolute value of \passthrough{\lstinline!x.!}

See also: \passthrough{\lstinline!float, *!}.

\hypertarget{fl.acos-procedure1}{%
\subsubsection{\texorpdfstring{\texttt{fl.acos} :
procedure/1}{fl.acos : procedure/1}}\label{fl.acos-procedure1}}

Usage: \passthrough{\lstinline!(fl.acos x) => fl!}

Return the arc cosine of \passthrough{\lstinline!x.!}

See also: \passthrough{\lstinline!fl.cos!}.

\hypertarget{fl.asin-procedure1}{%
\subsubsection{\texorpdfstring{\texttt{fl.asin} :
procedure/1}{fl.asin : procedure/1}}\label{fl.asin-procedure1}}

Usage: \passthrough{\lstinline!(fl.asin x) => fl!}

Return the arc sine of \passthrough{\lstinline!x.!}

See also: \passthrough{\lstinline!fl.acos!}.

\hypertarget{fl.asinh-procedure1}{%
\subsubsection{\texorpdfstring{\texttt{fl.asinh} :
procedure/1}{fl.asinh : procedure/1}}\label{fl.asinh-procedure1}}

Usage: \passthrough{\lstinline!(fl.asinh x) => fl!}

Return the inverse hyperbolic sine of \passthrough{\lstinline!x.!}

See also: \passthrough{\lstinline!fl.cosh!}.

\hypertarget{fl.atan-procedure1}{%
\subsubsection{\texorpdfstring{\texttt{fl.atan} :
procedure/1}{fl.atan : procedure/1}}\label{fl.atan-procedure1}}

Usage: \passthrough{\lstinline!(fl.atan x) => fl!}

Return the arctangent of \passthrough{\lstinline!x!} in radians.

See also: \passthrough{\lstinline!fl.atanh, fl.tan!}.

\hypertarget{fl.atan2-procedure2}{%
\subsubsection{\texorpdfstring{\texttt{fl.atan2} :
procedure/2}{fl.atan2 : procedure/2}}\label{fl.atan2-procedure2}}

Usage: \passthrough{\lstinline!(fl.atan2 x y) => fl!}

Atan2 returns the arc tangent of \passthrough{\lstinline!y!} /
\passthrough{\lstinline!x!}, using the signs of the two to determine the
quadrant of the return value.

See also: \passthrough{\lstinline!fl.atan!}.

\hypertarget{fl.atanh-procedure1}{%
\subsubsection{\texorpdfstring{\texttt{fl.atanh} :
procedure/1}{fl.atanh : procedure/1}}\label{fl.atanh-procedure1}}

Usage: \passthrough{\lstinline!(fl.atanh x) => fl!}

Return the inverse hyperbolic tangent of \passthrough{\lstinline!x.!}

See also: \passthrough{\lstinline!fl.atan!}.

\hypertarget{fl.cbrt-procedure1}{%
\subsubsection{\texorpdfstring{\texttt{fl.cbrt} :
procedure/1}{fl.cbrt : procedure/1}}\label{fl.cbrt-procedure1}}

Usage: \passthrough{\lstinline!(fl.cbrt x) => fl!}

Return the cube root of \passthrough{\lstinline!x.!}

See also: \passthrough{\lstinline!fl.sqrt!}.

\hypertarget{fl.ceil-procedure1}{%
\subsubsection{\texorpdfstring{\texttt{fl.ceil} :
procedure/1}{fl.ceil : procedure/1}}\label{fl.ceil-procedure1}}

Usage: \passthrough{\lstinline!(fl.ceil x) => fl!}

Round \passthrough{\lstinline!x!} up to the nearest integer, return it
as a floating point number.

See also:
\passthrough{\lstinline!fl.floor, truncate, int, fl.round, fl.trunc!}.

\hypertarget{fl.cos-procedure1}{%
\subsubsection{\texorpdfstring{\texttt{fl.cos} :
procedure/1}{fl.cos : procedure/1}}\label{fl.cos-procedure1}}

Usage: \passthrough{\lstinline!(fl.cos x) => fl!}

Return the cosine of \passthrough{\lstinline!x.!}

See also: \passthrough{\lstinline!fl.sin!}.

\hypertarget{fl.cosh-procedure1}{%
\subsubsection{\texorpdfstring{\texttt{fl.cosh} :
procedure/1}{fl.cosh : procedure/1}}\label{fl.cosh-procedure1}}

Usage: \passthrough{\lstinline!(fl.cosh x) => fl!}

Return the hyperbolic cosine of \passthrough{\lstinline!x.!}

See also: \passthrough{\lstinline!fl.cos!}.

\hypertarget{fl.dim-procedure2}{%
\subsubsection{\texorpdfstring{\texttt{fl.dim} :
procedure/2}{fl.dim : procedure/2}}\label{fl.dim-procedure2}}

Usage: \passthrough{\lstinline!(fl.dim x y) => fl!}

Return the maximum of x, y or 0.

See also: \passthrough{\lstinline!max!}.

\hypertarget{fl.erf-procedure1}{%
\subsubsection{\texorpdfstring{\texttt{fl.erf} :
procedure/1}{fl.erf : procedure/1}}\label{fl.erf-procedure1}}

Usage: \passthrough{\lstinline!(fl.erf x) => fl!}

Return the result of the error function of \passthrough{\lstinline!x.!}

See also: \passthrough{\lstinline!fl.erfc, fl.dim!}.

\hypertarget{fl.erfc-procedure1}{%
\subsubsection{\texorpdfstring{\texttt{fl.erfc} :
procedure/1}{fl.erfc : procedure/1}}\label{fl.erfc-procedure1}}

Usage: \passthrough{\lstinline!(fl.erfc x) => fl!}

Return the result of the complementary error function of
\passthrough{\lstinline!x.!}

See also: \passthrough{\lstinline!fl.erfcinv, fl.erf!}.

\hypertarget{fl.erfcinv-procedure1}{%
\subsubsection{\texorpdfstring{\texttt{fl.erfcinv} :
procedure/1}{fl.erfcinv : procedure/1}}\label{fl.erfcinv-procedure1}}

Usage: \passthrough{\lstinline!(fl.erfcinv x) => fl!}

Return the inverse of (fl.erfc \passthrough{\lstinline!x!}).

See also: \passthrough{\lstinline!fl.erfc!}.

\hypertarget{fl.erfinv-procedure1}{%
\subsubsection{\texorpdfstring{\texttt{fl.erfinv} :
procedure/1}{fl.erfinv : procedure/1}}\label{fl.erfinv-procedure1}}

Usage: \passthrough{\lstinline!(fl.erfinv x) => fl!}

Return the inverse of (fl.erf \passthrough{\lstinline!x!}).

See also: \passthrough{\lstinline!fl.erf!}.

\hypertarget{fl.exp-procedure1}{%
\subsubsection{\texorpdfstring{\texttt{fl.exp} :
procedure/1}{fl.exp : procedure/1}}\label{fl.exp-procedure1}}

Usage: \passthrough{\lstinline!(fl.exp x) => fl!}

Return e\^{}\passthrough{\lstinline!x!}, the base-e exponential of
\passthrough{\lstinline!x.!}

See also: \passthrough{\lstinline!fl.exp!}.

\hypertarget{fl.exp2-procedure2}{%
\subsubsection{\texorpdfstring{\texttt{fl.exp2} :
procedure/2}{fl.exp2 : procedure/2}}\label{fl.exp2-procedure2}}

Usage: \passthrough{\lstinline!(fl.exp2 x) => fl!}

Return 2\^{}\passthrough{\lstinline!x!}, the base-2 exponential of
\passthrough{\lstinline!x.!}

See also: \passthrough{\lstinline!fl.exp!}.

\hypertarget{fl.expm1-procedure1}{%
\subsubsection{\texorpdfstring{\texttt{fl.expm1} :
procedure/1}{fl.expm1 : procedure/1}}\label{fl.expm1-procedure1}}

Usage: \passthrough{\lstinline!(fl.expm1 x) => fl!}

Return e\^{}\passthrough{\lstinline!x-1!}, the base-e exponential of
(sub1 \passthrough{\lstinline!x!}). This is more accurate than (sub1
(fl.exp \passthrough{\lstinline!x!})) when \passthrough{\lstinline!x!}
is very small.

See also: \passthrough{\lstinline!fl.exp!}.

\hypertarget{fl.floor-procedure1}{%
\subsubsection{\texorpdfstring{\texttt{fl.floor} :
procedure/1}{fl.floor : procedure/1}}\label{fl.floor-procedure1}}

Usage: \passthrough{\lstinline!(fl.floor x) => fl!}

Return \passthrough{\lstinline!x!} rounded to the nearest integer below
as floating point number.

See also: \passthrough{\lstinline!fl.ceil, truncate, int!}.

\hypertarget{fl.fma-procedure3}{%
\subsubsection{\texorpdfstring{\texttt{fl.fma} :
procedure/3}{fl.fma : procedure/3}}\label{fl.fma-procedure3}}

Usage: \passthrough{\lstinline!(fl.fma x y z) => fl!}

Return the fused multiply-add of \passthrough{\lstinline!x!},
\passthrough{\lstinline!y!}, \passthrough{\lstinline!z!}, which is
\passthrough{\lstinline!x!} * \passthrough{\lstinline!y!} +
\passthrough{\lstinline!z.!}

See also: \passthrough{\lstinline!*, +!}.

\hypertarget{fl.frexp-procedure1}{%
\subsubsection{\texorpdfstring{\texttt{fl.frexp} :
procedure/1}{fl.frexp : procedure/1}}\label{fl.frexp-procedure1}}

Usage: \passthrough{\lstinline!(fl.frexp x) => li!}

Break \passthrough{\lstinline!x!} into a normalized fraction and an
integral power of two. It returns a list of (frac exp) containing a
float and an integer satisfying \passthrough{\lstinline!x!} ==
\passthrough{\lstinline!frac!} × 2\^{}\passthrough{\lstinline!exp!}
where the absolute value of \passthrough{\lstinline!frac!} is in the
interval {[}0.5, 1).

See also: \passthrough{\lstinline!fl.exp!}.

\hypertarget{fl.gamma-procedure1}{%
\subsubsection{\texorpdfstring{\texttt{fl.gamma} :
procedure/1}{fl.gamma : procedure/1}}\label{fl.gamma-procedure1}}

Usage: \passthrough{\lstinline!(fl.gamma x) => fl!}

Compute the Gamma function of \passthrough{\lstinline!x.!}

See also: \passthrough{\lstinline!fl.lgamma!}.

\hypertarget{fl.hypot-procedure2}{%
\subsubsection{\texorpdfstring{\texttt{fl.hypot} :
procedure/2}{fl.hypot : procedure/2}}\label{fl.hypot-procedure2}}

Usage: \passthrough{\lstinline!(fl.hypot x y) => fl!}

Compute the square root of x\^{}2 and y\^{}2.

See also: \passthrough{\lstinline!fl.sqrt!}.

\hypertarget{fl.ilogb-procedure1}{%
\subsubsection{\texorpdfstring{\texttt{fl.ilogb} :
procedure/1}{fl.ilogb : procedure/1}}\label{fl.ilogb-procedure1}}

Usage: \passthrough{\lstinline!(fl.ilogb x) => fl!}

Return the binary exponent of \passthrough{\lstinline!x!} as a floating
point number.

See also: \passthrough{\lstinline!fl.exp2!}.

\hypertarget{fl.inf-procedure1}{%
\subsubsection{\texorpdfstring{\texttt{fl.inf} :
procedure/1}{fl.inf : procedure/1}}\label{fl.inf-procedure1}}

Usage: \passthrough{\lstinline!(fl.inf x) => fl!}

Return positive 64 bit floating point infinity +INF if
\passthrough{\lstinline!x!} \textgreater= 0 and negative 64 bit floating
point finfinity -INF if \passthrough{\lstinline!x!} \textless{} 0.

See also: \passthrough{\lstinline!fl.is-nan?!}.

\hypertarget{fl.is-nan-procedure1}{%
\subsubsection{\texorpdfstring{\texttt{fl.is-nan?} :
procedure/1}{fl.is-nan? : procedure/1}}\label{fl.is-nan-procedure1}}

Usage: \passthrough{\lstinline!(fl.is-nan? x) => bool!}

Return true if \passthrough{\lstinline!x!} is not a number according to
IEEE 754 floating point arithmetics, nil otherwise.

See also: \passthrough{\lstinline!fl.inf!}.

\hypertarget{fl.j0-procedure1}{%
\subsubsection{\texorpdfstring{\texttt{fl.j0} :
procedure/1}{fl.j0 : procedure/1}}\label{fl.j0-procedure1}}

Usage: \passthrough{\lstinline!(fl.j0 x) => fl!}

Apply the order-zero Bessel function of the first kind to
\passthrough{\lstinline!x.!}

See also: \passthrough{\lstinline!fl.j1, fl.jn, fl.y0, fl.y1, fl.yn!}.

\hypertarget{fl.j1-procedure1}{%
\subsubsection{\texorpdfstring{\texttt{fl.j1} :
procedure/1}{fl.j1 : procedure/1}}\label{fl.j1-procedure1}}

Usage: \passthrough{\lstinline!(fl.j1 x) => fl!}

Apply the the order-one Bessel function of the first kind
\passthrough{\lstinline!x.!}

See also: \passthrough{\lstinline!fl.j0, fl.jn, fl.y0, fl.y1, fl.yn!}.

\hypertarget{fl.jn-procedure1}{%
\subsubsection{\texorpdfstring{\texttt{fl.jn} :
procedure/1}{fl.jn : procedure/1}}\label{fl.jn-procedure1}}

Usage: \passthrough{\lstinline!(fl.jn n x) => fl!}

Apply the Bessel function of order \passthrough{\lstinline!n!} to
\passthrough{\lstinline!x!}. The number \passthrough{\lstinline!n!} must
be an integer.

See also: \passthrough{\lstinline!fl.j1, fl.j0, fl.y0, fl.y1, fl.yn!}.

\hypertarget{fl.ldexp-procedure2}{%
\subsubsection{\texorpdfstring{\texttt{fl.ldexp} :
procedure/2}{fl.ldexp : procedure/2}}\label{fl.ldexp-procedure2}}

Usage: \passthrough{\lstinline!(fl.ldexp x n) => fl!}

Return the inverse of fl.frexp, \passthrough{\lstinline!x!} *
2\^{}\passthrough{\lstinline!n.!}

See also: \passthrough{\lstinline!fl.frexp!}.

\hypertarget{fl.lgamma-procedure1}{%
\subsubsection{\texorpdfstring{\texttt{fl.lgamma} :
procedure/1}{fl.lgamma : procedure/1}}\label{fl.lgamma-procedure1}}

Usage: \passthrough{\lstinline!(fl.lgamma x) => li!}

Return a list containing the natural logarithm and sign (-1 or +1) of
the Gamma function applied to \passthrough{\lstinline!x.!}

See also: \passthrough{\lstinline!fl.gamma!}.

\hypertarget{fl.log-procedure1}{%
\subsubsection{\texorpdfstring{\texttt{fl.log} :
procedure/1}{fl.log : procedure/1}}\label{fl.log-procedure1}}

Usage: \passthrough{\lstinline!(fl.log x) => fl!}

Return the natural logarithm of \passthrough{\lstinline!x.!}

See also:
\passthrough{\lstinline!fl.log10, fl.log2, fl.logb, fl.log1p!}.

\hypertarget{fl.log10-procedure1}{%
\subsubsection{\texorpdfstring{\texttt{fl.log10} :
procedure/1}{fl.log10 : procedure/1}}\label{fl.log10-procedure1}}

Usage: \passthrough{\lstinline!(fl.log10 x) => fl!}

Return the decimal logarithm of \passthrough{\lstinline!x.!}

See also: \passthrough{\lstinline!fl.log, fl.log2, fl.logb, fl.log1p!}.

\hypertarget{fl.log1p-procedure1}{%
\subsubsection{\texorpdfstring{\texttt{fl.log1p} :
procedure/1}{fl.log1p : procedure/1}}\label{fl.log1p-procedure1}}

Usage: \passthrough{\lstinline!(fl.log1p x) => fl!}

Return the natural logarithm of \passthrough{\lstinline!x!} + 1. This
function is more accurate than (fl.log (add1 x)) if
\passthrough{\lstinline!x!} is close to 0.

See also: \passthrough{\lstinline!fl.log, fl.log2, fl.logb, fl.log10!}.

\hypertarget{fl.log2-procedure1}{%
\subsubsection{\texorpdfstring{\texttt{fl.log2} :
procedure/1}{fl.log2 : procedure/1}}\label{fl.log2-procedure1}}

Usage: \passthrough{\lstinline!(fl.log2 x) => fl!}

Return the binary logarithm of \passthrough{\lstinline!x!}. This is
important for calculating entropy, for example.

See also: \passthrough{\lstinline!fl.log, fl.log10, fl.log1p, fl.logb!}.

\hypertarget{fl.logb-procedure1}{%
\subsubsection{\texorpdfstring{\texttt{fl.logb} :
procedure/1}{fl.logb : procedure/1}}\label{fl.logb-procedure1}}

Usage: \passthrough{\lstinline!(fl.logb x) => fl!}

Return the binary exponent of \passthrough{\lstinline!x.!}

See also:
\passthrough{\lstinline!fl.log, fl.log10, fl.log1p, fl.logb, fl.log2!}.

\hypertarget{fl.max-procedure2}{%
\subsubsection{\texorpdfstring{\texttt{fl.max} :
procedure/2}{fl.max : procedure/2}}\label{fl.max-procedure2}}

Usage: \passthrough{\lstinline!(fl.max x y) => fl!}

Return the larger value of two floating point arguments
\passthrough{\lstinline!x!} and \passthrough{\lstinline!y.!}

See also: \passthrough{\lstinline!fl.min, max, min!}.

\hypertarget{fl.min-procedure2}{%
\subsubsection{\texorpdfstring{\texttt{fl.min} :
procedure/2}{fl.min : procedure/2}}\label{fl.min-procedure2}}

Usage: \passthrough{\lstinline!(fl.min x y) => fl!}

Return the smaller value of two floating point arguments
\passthrough{\lstinline!x!} and \passthrough{\lstinline!y.!}

See also: \passthrough{\lstinline!fl.min, max, min!}.

\hypertarget{fl.mod-procedure2}{%
\subsubsection{\texorpdfstring{\texttt{fl.mod} :
procedure/2}{fl.mod : procedure/2}}\label{fl.mod-procedure2}}

Usage: \passthrough{\lstinline!(fl.mod x y) => fl!}

Return the floating point remainder of \passthrough{\lstinline!x!} /
\passthrough{\lstinline!y.!}

See also: \passthrough{\lstinline!fl.remainder!}.

\hypertarget{fl.modf-procedure1}{%
\subsubsection{\texorpdfstring{\texttt{fl.modf} :
procedure/1}{fl.modf : procedure/1}}\label{fl.modf-procedure1}}

Usage: \passthrough{\lstinline!(fl.modf x) => li!}

Return integer and fractional floating-point numbers that sum to
\passthrough{\lstinline!x!}. Both values have the same sign as
\passthrough{\lstinline!x.!}

See also: \passthrough{\lstinline!fl.mod!}.

\hypertarget{fl.nan-procedure1}{%
\subsubsection{\texorpdfstring{\texttt{fl.nan} :
procedure/1}{fl.nan : procedure/1}}\label{fl.nan-procedure1}}

Usage: \passthrough{\lstinline!(fl.nan) => fl!}

Return the IEEE 754 not-a-number value.

See also: \passthrough{\lstinline!fl.is-nan?, fl.inf!}.

\hypertarget{fl.next-after-procedure1}{%
\subsubsection{\texorpdfstring{\texttt{fl.next-after} :
procedure/1}{fl.next-after : procedure/1}}\label{fl.next-after-procedure1}}

Usage: \passthrough{\lstinline!(fl.next-after x) => fl!}

Return the next representable floating point number after
\passthrough{\lstinline!x.!}

See also: \passthrough{\lstinline!fl.is-nan?, fl.nan, fl.inf!}.

\hypertarget{fl.pow-procedure2}{%
\subsubsection{\texorpdfstring{\texttt{fl.pow} :
procedure/2}{fl.pow : procedure/2}}\label{fl.pow-procedure2}}

Usage: \passthrough{\lstinline!(fl.pow x y) => fl!}

Return \passthrough{\lstinline!x!} to the power of
\passthrough{\lstinline!y!} according to 64 bit floating point
arithmetics.

See also: \passthrough{\lstinline!fl.pow10!}.

\hypertarget{fl.pow10-procedure1}{%
\subsubsection{\texorpdfstring{\texttt{fl.pow10} :
procedure/1}{fl.pow10 : procedure/1}}\label{fl.pow10-procedure1}}

Usage: \passthrough{\lstinline!(fl.pow10 n) => fl!}

Return 10 to the power of integer \passthrough{\lstinline!n!} as a 64
bit floating point number.

See also: \passthrough{\lstinline!fl.pow!}.

\hypertarget{fl.remainder-procedure2}{%
\subsubsection{\texorpdfstring{\texttt{fl.remainder} :
procedure/2}{fl.remainder : procedure/2}}\label{fl.remainder-procedure2}}

Usage: \passthrough{\lstinline!(fl.remainder x y) => fl!}

Return the IEEE 754 floating-point remainder of
\passthrough{\lstinline!x!} / \passthrough{\lstinline!y.!}

See also: \passthrough{\lstinline!fl.mod!}.

\hypertarget{fl.round-procedure1}{%
\subsubsection{\texorpdfstring{\texttt{fl.round} :
procedure/1}{fl.round : procedure/1}}\label{fl.round-procedure1}}

Usage: \passthrough{\lstinline!(fl.round x) => fl!}

Round \passthrough{\lstinline!x!} to the nearest integer floating point
number according to floating point arithmetics.

See also:
\passthrough{\lstinline!fl.round-to-even, fl.truncate, int, float!}.

\hypertarget{fl.round-to-even-procedure1}{%
\subsubsection{\texorpdfstring{\texttt{fl.round-to-even} :
procedure/1}{fl.round-to-even : procedure/1}}\label{fl.round-to-even-procedure1}}

Usage: \passthrough{\lstinline!(fl.round-to-even x) => fl!}

Round \passthrough{\lstinline!x!} to the nearest even integer floating
point number according to floating point arithmetics.

See also: \passthrough{\lstinline!fl.round, fl.truncate, int, float!}.

\hypertarget{fl.signbit-procedure1}{%
\subsubsection{\texorpdfstring{\texttt{fl.signbit} :
procedure/1}{fl.signbit : procedure/1}}\label{fl.signbit-procedure1}}

Usage: \passthrough{\lstinline!(fl.signbit x) => bool!}

Return true if \passthrough{\lstinline!x!} is negative, nil otherwise.

See also: \passthrough{\lstinline!fl.abs!}.

\hypertarget{fl.sin-procedure1}{%
\subsubsection{\texorpdfstring{\texttt{fl.sin} :
procedure/1}{fl.sin : procedure/1}}\label{fl.sin-procedure1}}

Usage: \passthrough{\lstinline!(fl.sin x) => fl!}

Return the sine of \passthrough{\lstinline!x.!}

See also: \passthrough{\lstinline!fl.cos!}.

\hypertarget{fl.sinh-procedure1}{%
\subsubsection{\texorpdfstring{\texttt{fl.sinh} :
procedure/1}{fl.sinh : procedure/1}}\label{fl.sinh-procedure1}}

Usage: \passthrough{\lstinline!(fl.sinh x) => fl!}

Return the hyperbolic sine of \passthrough{\lstinline!x.!}

See also: \passthrough{\lstinline!fl.sin!}.

\hypertarget{fl.sqrt-procedure1}{%
\subsubsection{\texorpdfstring{\texttt{fl.sqrt} :
procedure/1}{fl.sqrt : procedure/1}}\label{fl.sqrt-procedure1}}

Usage: \passthrough{\lstinline!(fl.sqrt x) => fl!}

Return the square root of \passthrough{\lstinline!x.!}

See also: \passthrough{\lstinline!fl.pow!}.

\hypertarget{fl.tan-procedure1}{%
\subsubsection{\texorpdfstring{\texttt{fl.tan} :
procedure/1}{fl.tan : procedure/1}}\label{fl.tan-procedure1}}

Usage: \passthrough{\lstinline!(fl.tan x) => fl!}

Return the tangent of \passthrough{\lstinline!x!} in radian.

See also: \passthrough{\lstinline!fl.tanh, fl.sin, fl.cos!}.

\hypertarget{fl.tanh-procedure1}{%
\subsubsection{\texorpdfstring{\texttt{fl.tanh} :
procedure/1}{fl.tanh : procedure/1}}\label{fl.tanh-procedure1}}

Usage: \passthrough{\lstinline!(fl.tanh x) => fl!}

Return the hyperbolic tangent of \passthrough{\lstinline!x.!}

See also: \passthrough{\lstinline!fl.tan, flsinh, fl.cosh!}.

\hypertarget{fl.trunc-procedure1}{%
\subsubsection{\texorpdfstring{\texttt{fl.trunc} :
procedure/1}{fl.trunc : procedure/1}}\label{fl.trunc-procedure1}}

Usage: \passthrough{\lstinline!(fl.trunc x) => fl!}

Return the integer value of \passthrough{\lstinline!x!} as floating
point number.

See also: \passthrough{\lstinline!truncate, int, fl.floor!}.

\hypertarget{fl.y0-procedure1}{%
\subsubsection{\texorpdfstring{\texttt{fl.y0} :
procedure/1}{fl.y0 : procedure/1}}\label{fl.y0-procedure1}}

Usage: \passthrough{\lstinline!(fl.y0 x) => fl!}

Return the order-zero Bessel function of the second kind applied to
\passthrough{\lstinline!x.!}

See also: \passthrough{\lstinline!fl.y1, fl.yn, fl.j0, fl.j1, fl.jn!}.

\hypertarget{fl.y1-procedure1}{%
\subsubsection{\texorpdfstring{\texttt{fl.y1} :
procedure/1}{fl.y1 : procedure/1}}\label{fl.y1-procedure1}}

Usage: \passthrough{\lstinline!(fl.y1 x) => fl!}

Return the order-one Bessel function of the second kind applied to
\passthrough{\lstinline!x.!}

See also: \passthrough{\lstinline!fl.y0, fl.yn, fl.j0, fl.j1, fl.jn!}.

\hypertarget{fl.yn-procedure1}{%
\subsubsection{\texorpdfstring{\texttt{fl.yn} :
procedure/1}{fl.yn : procedure/1}}\label{fl.yn-procedure1}}

Usage: \passthrough{\lstinline!(fl.yn n x) => fl!}

Return the Bessel function of the second kind of order
\passthrough{\lstinline!n!} applied to \passthrough{\lstinline!x!}.
Argument \passthrough{\lstinline!n!} must be an integer value.

See also: \passthrough{\lstinline!fl.y0, fl.y1, fl.j0, fl.j1, fl.jn!}.

\hypertarget{help-system}{%
\subsection{Help System}\label{help-system}}

This section lists functions related to the built-in help system.

\hypertarget{help-dict}{%
\subsubsection{\texorpdfstring{\emph{help} :
dict}{help : dict}}\label{help-dict}}

Usage: \passthrough{\lstinline!*help*!}

Dict containing all help information for symbols.

See also: \passthrough{\lstinline!help, defhelp, apropos!}.

\hypertarget{apropos-procedure1}{%
\subsubsection{\texorpdfstring{\texttt{apropos} :
procedure/1}{apropos : procedure/1}}\label{apropos-procedure1}}

Usage: \passthrough{\lstinline!(apropos sym) => \#li!}

Get a list of procedures and symbols related to
\passthrough{\lstinline!sym!} from the help system.

See also: \passthrough{\lstinline!defhelp, help-entry, help, *help*!}.

\hypertarget{help-macro1}{%
\subsubsection{\texorpdfstring{\texttt{help} :
macro/1}{help : macro/1}}\label{help-macro1}}

Usage: \passthrough{\lstinline!(help sym)!}

Display help information about \passthrough{\lstinline!sym!} (unquoted).

See also:
\passthrough{\lstinline!defhelp, help-entry, *help*, apropos!}.

\hypertarget{help-manual-entry-nil}{%
\subsubsection{help-\textgreater manual-entry :
nil}\label{help-manual-entry-nil}}

Usage: \passthrough{\lstinline!(help->manual-entry key [level]) => str!}

Looks up help for \passthrough{\lstinline!key!} and converts it to a
manual section as markdown string. If there is no entry for
\passthrough{\lstinline!key!}, then nil is returned. The optional
\passthrough{\lstinline!level!} integer indicates the heading nesting.

See also: \passthrough{\lstinline!help!}.

\hypertarget{help-about-procedure1-or-more}{%
\subsubsection{\texorpdfstring{\texttt{help-about} : procedure/1 or
more}{help-about : procedure/1 or more}}\label{help-about-procedure1-or-more}}

Usage: \passthrough{\lstinline!(help-about topic [sel]) => li!}

Obtain a list of symbols for which help about
\passthrough{\lstinline!topic!} is available. If optional
\passthrough{\lstinline!sel!} argument is left out or
\passthrough{\lstinline!any!}, then any symbols with which the topic is
associated are listed. If the optional \passthrough{\lstinline!sel!}
argument is \passthrough{\lstinline!first!}, then a symbol is only
listed if it has \passthrough{\lstinline!topic!} as first topic entry.
This restricts the number of entries returned to a more essential
selection.

See also: \passthrough{\lstinline!help-topics, help, apropos!}.

\hypertarget{help-entry-procedure1}{%
\subsubsection{\texorpdfstring{\texttt{help-entry} :
procedure/1}{help-entry : procedure/1}}\label{help-entry-procedure1}}

Usage: \passthrough{\lstinline!(help-entry sym) => list!}

Get usage and help information for \passthrough{\lstinline!sym.!}

See also: \passthrough{\lstinline!defhelp, help, apropos, *help*!}.

\hypertarget{help-topic-info-procedure1}{%
\subsubsection{\texorpdfstring{\texttt{help-topic-info} :
procedure/1}{help-topic-info : procedure/1}}\label{help-topic-info-procedure1}}

Usage: \passthrough{\lstinline!(help-topic-info topic) => li!}

Return a list containing a heading and an info string for help
\passthrough{\lstinline!topic!}, or nil if no info is available.

See also: \passthrough{\lstinline!set-help-topic-info, defhelp, help!}.

\hypertarget{help-topics-procedure0}{%
\subsubsection{\texorpdfstring{\texttt{help-topics} :
procedure/0}{help-topics : procedure/0}}\label{help-topics-procedure0}}

Usage: \passthrough{\lstinline!(help-topics) => li!}

Obtain a list of help topics for commands.

See also: \passthrough{\lstinline!help, help-topic, apropos!}.

\hypertarget{set-help-topic-info-procedure3}{%
\subsubsection{\texorpdfstring{\texttt{set-help-topic-info} :
procedure/3}{set-help-topic-info : procedure/3}}\label{set-help-topic-info-procedure3}}

Usage: \passthrough{\lstinline!(set-help-topic-info topic header info)!}

Set a human-readable information entry for help
\passthrough{\lstinline!topic!} with human-readable
\passthrough{\lstinline!header!} and \passthrough{\lstinline!info!}
strings.

See also: \passthrough{\lstinline!defhelp, help-topic-info!}.

\hypertarget{soundex-metaphone-etc.}{%
\subsection{Soundex, Metaphone, etc.}\label{soundex-metaphone-etc.}}

The package \passthrough{\lstinline!ling!} provides various phonemic
transcription functions like Soundex and Metaphone that are commonly
used for fuzzy search and similarity comparisons between strings.

\hypertarget{ling.damerau-levenshtein-procedure2}{%
\subsubsection{\texorpdfstring{\texttt{ling.damerau-levenshtein} :
procedure/2}{ling.damerau-levenshtein : procedure/2}}\label{ling.damerau-levenshtein-procedure2}}

Usage: \passthrough{\lstinline!(ling.damerau-levenshtein s1 s2) => num!}

Compute the Damerau-Levenshtein distance between
\passthrough{\lstinline!s1!} and \passthrough{\lstinline!s2.!}

See also:
\passthrough{\lstinline!ling.match-rating-compare, ling.levenshtein, ling.jaro-winkler, ling.jaro, ling.hamming, ling.match-rating-codex, ling.porter, ling.nysiis, ling.metaphone, ling.soundex!}.

\hypertarget{ling.hamming-procedure2}{%
\subsubsection{\texorpdfstring{\texttt{ling.hamming} :
procedure/2}{ling.hamming : procedure/2}}\label{ling.hamming-procedure2}}

Usage: \passthrough{\lstinline!(ling-hamming s1 s2) => num!}

Compute the Hamming distance between \passthrough{\lstinline!s1!} and
\passthrough{\lstinline!s2.!}

See also:
\passthrough{\lstinline!ling.match-rating-compare, ling.levenshtein, ling.jaro-winkler, ling.jaro, ling.damerau-levenshtein, ling.match-rating-codex, ling.porter, ling.nysiis, ling.metaphone, ling.soundex!}.

\hypertarget{ling.jaro-procedure2}{%
\subsubsection{\texorpdfstring{\texttt{ling.jaro} :
procedure/2}{ling.jaro : procedure/2}}\label{ling.jaro-procedure2}}

Usage: \passthrough{\lstinline!(ling.jaro s1 s2) => num!}

Compute the Jaro distance between \passthrough{\lstinline!s1!} and
\passthrough{\lstinline!s2.!}

See also:
\passthrough{\lstinline!ling.match-rating-compare, ling.levenshtein, ling.jaro-winkler, ling.hamming, ling.damerau-levenshtein, ling.match-rating-codex, ling.porter, ling.nysiis, ling.metaphone, ling.soundex!}.

\hypertarget{ling.jaro-winkler-procedure2}{%
\subsubsection{\texorpdfstring{\texttt{ling.jaro-winkler} :
procedure/2}{ling.jaro-winkler : procedure/2}}\label{ling.jaro-winkler-procedure2}}

Usage: \passthrough{\lstinline!(ling.jaro-winkler s1 s2) => num!}

Compute the Jaro-Winkler distance between \passthrough{\lstinline!s1!}
and \passthrough{\lstinline!s2.!}

See also:
\passthrough{\lstinline!ling.match-rating-compare, ling.levenshtein, ling.jaro, ling.hamming, ling.damerau-levenshtein, ling.match-rating-codex, ling.porter, ling.nysiis, ling.metaphone, ling.soundex!}.

\hypertarget{ling.levenshtein-procedure2}{%
\subsubsection{\texorpdfstring{\texttt{ling.levenshtein} :
procedure/2}{ling.levenshtein : procedure/2}}\label{ling.levenshtein-procedure2}}

Usage: \passthrough{\lstinline!(ling.levenshtein s1 s2) => num!}

Compute the Levenshtein distance between \passthrough{\lstinline!s1!}
and \passthrough{\lstinline!s2.!}

See also:
\passthrough{\lstinline!ling.match-rating-compare, ling.jaro-winkler, ling.jaro, ling.hamming, ling.damerau-levenshtein, ling.match-rating-codex, ling.porter, ling.nysiis, ling.metaphone, ling.soundex!}.

\hypertarget{ling.match-rating-codex-procedure1}{%
\subsubsection{\texorpdfstring{\texttt{ling.match-rating-codex} :
procedure/1}{ling.match-rating-codex : procedure/1}}\label{ling.match-rating-codex-procedure1}}

Usage: \passthrough{\lstinline!(ling.match-rating-codex s) => str!}

Compute the Match-Rating-Codex of string \passthrough{\lstinline!s.!}

See also:
\passthrough{\lstinline!ling.match-rating-compare, ling.levenshtein, ling.jaro-winkler, ling.jaro, ling.hamming, ling.damerau-levenshtein, ling.porter, ling.nysiis, ling.metaphone, ling.soundex!}.

\hypertarget{ling.match-rating-compare-procedure2}{%
\subsubsection{\texorpdfstring{\texttt{ling.match-rating-compare} :
procedure/2}{ling.match-rating-compare : procedure/2}}\label{ling.match-rating-compare-procedure2}}

Usage:
\passthrough{\lstinline!(ling.match-rating-compare s1 s2) => bool!}

Returns true if \passthrough{\lstinline!s1!} and
\passthrough{\lstinline!s2!} are equal according to the Match-rating
Comparison algorithm, nil otherwise.

See also:
\passthrough{\lstinline!ling.match-rating-compare, ling.levenshtein, ling.jaro-winkler, ling.jaro, ling.hamming, ling.damerau-levenshtein, ling.match-rating-codex, ling.porter, ling.nysiis, ling.metaphone, ling.soundex!}.

\hypertarget{ling.metaphone-procedure1}{%
\subsubsection{\texorpdfstring{\texttt{ling.metaphone} :
procedure/1}{ling.metaphone : procedure/1}}\label{ling.metaphone-procedure1}}

Usage: \passthrough{\lstinline!(ling.metaphone s) => str!}

Compute the Metaphone representation of string
\passthrough{\lstinline!s.!}

See also:
\passthrough{\lstinline!ling.match-rating-compare, ling.levenshtein, ling.jaro-winkler, ling.jaro, ling.hamming, ling.damerau-levenshtein, ling.match-rating-codex, ling.porter, ling.nysiis, ling.soundex!}.

\hypertarget{ling.nysiis-procedure1}{%
\subsubsection{\texorpdfstring{\texttt{ling.nysiis} :
procedure/1}{ling.nysiis : procedure/1}}\label{ling.nysiis-procedure1}}

Usage: \passthrough{\lstinline!(ling.nysiis s) => str!}

Compute the Nysiis representation of string \passthrough{\lstinline!s.!}

See also:
\passthrough{\lstinline!ling.match-rating-compare, ling.levenshtein, ling.jaro-winkler, ling.jaro, ling.hamming, ling.damerau-levenshtein, ling.match-rating-codex, ling.porter, ling.metaphone, ling.soundex!}.

\hypertarget{ling.porter-procedure1}{%
\subsubsection{\texorpdfstring{\texttt{ling.porter} :
procedure/1}{ling.porter : procedure/1}}\label{ling.porter-procedure1}}

Usage: \passthrough{\lstinline!(ling.porter s) => str!}

Compute the stem of word string \passthrough{\lstinline!s!} using the
Porter stemming algorithm.

See also:
\passthrough{\lstinline!ling.match-rating-compare, ling.levenshtein, ling.jaro-winkler, ling.jaro, ling.hamming, ling.damerau-levenshtein, ling.match-rating-codex, ling.nysiis, ling.metaphone, ling.soundex!}.

\hypertarget{ling.soundex-procedure1}{%
\subsubsection{\texorpdfstring{\texttt{ling.soundex} :
procedure/1}{ling.soundex : procedure/1}}\label{ling.soundex-procedure1}}

Usage: \passthrough{\lstinline!(ling.soundex s) => str!}

Compute the Soundex representation of string
\passthrough{\lstinline!s.!}

See also:
\passthrough{\lstinline!ling.match-rating-compare, ling.levenshtein, ling.jaro-winkler, ling.jaro, ling.hamming, ling.damerau-levenshtein, ling.match-rating-codex, ling.porter, ling.nysiis, ling.metaphone, ling.soundex!}.

\hypertarget{lisp---traditional-lisp-functions}{%
\subsection{Lisp - Traditional Lisp
Functions}\label{lisp---traditional-lisp-functions}}

This section comprises a large number of list processing functions as
well the standard control flow macros and functions you'd expect in a
Lisp system.

\hypertarget{alist-procedure1}{%
\subsubsection{\texorpdfstring{\texttt{alist?} :
procedure/1}{alist? : procedure/1}}\label{alist-procedure1}}

Usage: \passthrough{\lstinline!(alist? li) => bool!}

Return true if \passthrough{\lstinline!li!} is an association list, nil
otherwise. This also works for a-lists where each element is a pair
rather than a full list.

See also: \passthrough{\lstinline!assoc!}.

\hypertarget{and-macro0-or-more}{%
\subsubsection{\texorpdfstring{\texttt{and} : macro/0 or
more}{and : macro/0 or more}}\label{and-macro0-or-more}}

Usage: \passthrough{\lstinline!(and expr1 expr2 ...) => any!}

Evaluate \passthrough{\lstinline!expr1!} and if it is not nil, then
evaluate \passthrough{\lstinline!expr2!} and if it is not nil, evaluate
the next expression, until all expressions have been evaluated. This is
a shortcut logical and.

See also: \passthrough{\lstinline!or!}.

\hypertarget{append-procedure1-or-more}{%
\subsubsection{\texorpdfstring{\texttt{append} : procedure/1 or
more}{append : procedure/1 or more}}\label{append-procedure1-or-more}}

Usage: \passthrough{\lstinline!(append li1 li2 ...) => li!}

Concatenate the lists given as arguments.

See also: \passthrough{\lstinline!cons!}.

\hypertarget{apply-procedure2}{%
\subsubsection{\texorpdfstring{\texttt{apply} :
procedure/2}{apply : procedure/2}}\label{apply-procedure2}}

Usage: \passthrough{\lstinline!(apply proc arg) => any!}

Apply function \passthrough{\lstinline!proc!} to argument list
\passthrough{\lstinline!arg.!}

See also: \passthrough{\lstinline!functional?!}.

\hypertarget{assoc-procedure2}{%
\subsubsection{\texorpdfstring{\texttt{assoc} :
procedure/2}{assoc : procedure/2}}\label{assoc-procedure2}}

Usage: \passthrough{\lstinline!(assoc key alist) => li!}

Return the sublist of \passthrough{\lstinline!alist!} that starts with
\passthrough{\lstinline!key!} if there is any, nil otherwise. Testing is
done with equal?. An association list may be of the form ((key1
value1)(key2 value2)\ldots) or ((key1 . value1) (key2 . value2) \ldots)

See also: \passthrough{\lstinline!assoc, assoc1, alist?, eq?, equal?!}.

\hypertarget{assoc1-procedure2}{%
\subsubsection{\texorpdfstring{\texttt{assoc1} :
procedure/2}{assoc1 : procedure/2}}\label{assoc1-procedure2}}

Usage: \passthrough{\lstinline!(assoc1 sym li) => any!}

Get the second element in the first sublist in
\passthrough{\lstinline!li!} that starts with
\passthrough{\lstinline!sym!}. This is equivalent to (cadr (assoc sym
li)).

See also: \passthrough{\lstinline!assoc, alist?!}.

\hypertarget{assq-procedure2}{%
\subsubsection{\texorpdfstring{\texttt{assq} :
procedure/2}{assq : procedure/2}}\label{assq-procedure2}}

Usage: \passthrough{\lstinline!(assq key alist) => li!}

Return the sublist of \passthrough{\lstinline!alist!} that starts with
\passthrough{\lstinline!key!} if there is any, nil otherwise. Testing is
done with eq?. An association list may be of the form ((key1
value1)(key2 value2)\ldots) or ((key1 . value1) (key2 . value2) \ldots)

See also: \passthrough{\lstinline!assoc, assoc1, eq?, alist?, equal?!}.

\hypertarget{atom-procedure1}{%
\subsubsection{\texorpdfstring{\texttt{atom?} :
procedure/1}{atom? : procedure/1}}\label{atom-procedure1}}

Usage: \passthrough{\lstinline!(atom? x) => bool!}

Return true if \passthrough{\lstinline!x!} is an atomic value, nil
otherwise. Atomic values are numbers and symbols.

See also: \passthrough{\lstinline!sym?!}.

\hypertarget{build-list-procedure2}{%
\subsubsection{\texorpdfstring{\texttt{build-list} :
procedure/2}{build-list : procedure/2}}\label{build-list-procedure2}}

Usage: \passthrough{\lstinline!(build-list n proc) => list!}

Build a list with \passthrough{\lstinline!n!} elements by applying
\passthrough{\lstinline!proc!} to the counter
\passthrough{\lstinline!n!} each time.

See also: \passthrough{\lstinline!list, list?, map, foreach!}.

\hypertarget{caaar-procedure1}{%
\subsubsection{\texorpdfstring{\texttt{caaar} :
procedure/1}{caaar : procedure/1}}\label{caaar-procedure1}}

Usage: \passthrough{\lstinline!(caaar x) => any!}

Equivalent to (car (car (car \passthrough{\lstinline!x!}))).

See also:
\passthrough{\lstinline!car, cdr, caar, cadr, cdar, cddr, caadr, cadar, caddr, cdaar, cdadr, cddar, cdddr, nth, 1st, 2nd, 3rd!}.

\hypertarget{caadr-procedure1}{%
\subsubsection{\texorpdfstring{\texttt{caadr} :
procedure/1}{caadr : procedure/1}}\label{caadr-procedure1}}

Usage: \passthrough{\lstinline!(caadr x) => any!}

Equivalent to (car (car (cdr \passthrough{\lstinline!x!}))).

See also:
\passthrough{\lstinline!car, cdr, caar, cadr, cdar, cddr, caaar, cadar, caddr, cdaar, cdadr, cddar, cdddr, nth, 1st, 2nd, 3rd!}.

\hypertarget{caar-procedure1}{%
\subsubsection{\texorpdfstring{\texttt{caar} :
procedure/1}{caar : procedure/1}}\label{caar-procedure1}}

Usage: \passthrough{\lstinline!(caar x) => any!}

Equivalent to (car (car \passthrough{\lstinline!x!})).

See also:
\passthrough{\lstinline!car, cdr, cadr, cdar, cddr, caaar, caadr, cadar, caddr, cdaar, cdadr, cddar, cdddr, nth, 1st, 2nd, 3rd!}.

\hypertarget{cadar-procedure1}{%
\subsubsection{\texorpdfstring{\texttt{cadar} :
procedure/1}{cadar : procedure/1}}\label{cadar-procedure1}}

Usage: \passthrough{\lstinline!(cadar x) => any!}

Equivalent to (car (cdr (car \passthrough{\lstinline!x!}))).

See also:
\passthrough{\lstinline!car, cdr, caar, cadr, cdar, cddr, caaar, caadr, caddr, cdaar, cdadr, cddar, cdddr, nth, 1st, 2nd, 3rd!}.

\hypertarget{caddr-procedure1}{%
\subsubsection{\texorpdfstring{\texttt{caddr} :
procedure/1}{caddr : procedure/1}}\label{caddr-procedure1}}

Usage: \passthrough{\lstinline!(caddr x) => any!}

Equivalent to (car (cdr (cdr \passthrough{\lstinline!x!}))).

See also:
\passthrough{\lstinline!car, cdr, caar, cadr, cdar, cddr, caaar, caadr, cadar, cdaar, cdadr, cddar, cdddr, nth, 1st, 2nd, 3rd!}.

\hypertarget{cadr-procedure1}{%
\subsubsection{\texorpdfstring{\texttt{cadr} :
procedure/1}{cadr : procedure/1}}\label{cadr-procedure1}}

Usage: \passthrough{\lstinline!(cadr x) => any!}

Equivalent to (car (cdr \passthrough{\lstinline!x!})).

See also:
\passthrough{\lstinline!car, cdr, caar, cdar, cddr, caaar, caadr, cadar, caddr, cdaar, cdadr, cddar, cdddr, nth, 1st, 2nd, 3rd!}.

\hypertarget{car-procedure1}{%
\subsubsection{\texorpdfstring{\texttt{car} :
procedure/1}{car : procedure/1}}\label{car-procedure1}}

Usage: \passthrough{\lstinline!(car li) => any!}

Get the first element of a list or pair \passthrough{\lstinline!li!}, an
error if there is not first element.

See also: \passthrough{\lstinline!list, list?, pair?!}.

\hypertarget{case-macro2-or-more}{%
\subsubsection{\texorpdfstring{\texttt{case} : macro/2 or
more}{case : macro/2 or more}}\label{case-macro2-or-more}}

Usage:
\passthrough{\lstinline!(case expr (clause1 ... clausen)) => any!}

Standard case macro, where you should use t for the remaining
alternative. Example: (case (get dict 'key) ((a b) (out ``a or b''))(t
(out ``something else!''))).

See also: \passthrough{\lstinline!cond!}.

\hypertarget{cdaar-procedure1}{%
\subsubsection{\texorpdfstring{\texttt{cdaar} :
procedure/1}{cdaar : procedure/1}}\label{cdaar-procedure1}}

Usage: \passthrough{\lstinline!(cdaar x) => any!}

Equivalent to (cdr (car (car \passthrough{\lstinline!x!}))).

See also:
\passthrough{\lstinline!car, cdr, caar, cadr, cdar, cddr, caaar, caadr, cadar, caddr, cdadr, cddar, cdddr, nth, 1st, 2nd, 3rd!}.

\hypertarget{cdadr-procedure1}{%
\subsubsection{\texorpdfstring{\texttt{cdadr} :
procedure/1}{cdadr : procedure/1}}\label{cdadr-procedure1}}

Usage: \passthrough{\lstinline!(cdadr x) => any!}

Equivalent to (cdr (car (cdr \passthrough{\lstinline!x!}))).

See also:
\passthrough{\lstinline!car, cdr, caar, cadr, cdar, cddr, caaar, caadr, cadar, caddr, cdaar, cddar, cdddr, nth, 1st, 2nd, 3rd!}.

\hypertarget{cdar-procedure1}{%
\subsubsection{\texorpdfstring{\texttt{cdar} :
procedure/1}{cdar : procedure/1}}\label{cdar-procedure1}}

Usage: \passthrough{\lstinline!(cdar x) => any!}

Equivalent to (cdr (car \passthrough{\lstinline!x!})).

See also:
\passthrough{\lstinline!car, cdr, caar, cadr, cddr, caaar, caadr, cadar, caddr, cdaar, cdadr, cddar, cdddr, nth, 1st, 2nd, 3rd!}.

\hypertarget{cddar-procedure1}{%
\subsubsection{\texorpdfstring{\texttt{cddar} :
procedure/1}{cddar : procedure/1}}\label{cddar-procedure1}}

Usage: \passthrough{\lstinline!(cddar x) => any!}

Equivalent to (cdr (cdr (car \passthrough{\lstinline!x!}))).

See also:
\passthrough{\lstinline!car, cdr, caar, cadr, cdar, cddr, caaar, caadr, cadar, caddr, cdaar, cdadr, cdddr, nth, 1st, 2nd, 3rd!}.

\hypertarget{cdddr-procedure1}{%
\subsubsection{\texorpdfstring{\texttt{cdddr} :
procedure/1}{cdddr : procedure/1}}\label{cdddr-procedure1}}

Usage: \passthrough{\lstinline!(cdddr x) => any!}

Equivalent to (cdr (cdr (cdr \passthrough{\lstinline!x!}))).

See also:
\passthrough{\lstinline!car, cdr, caar, cadr, cdar, cddr, caaar, caadr, cadar, caddr, cdaar, cdadr, cddar, nth, 1st, 2nd, 3rd!}.

\hypertarget{cddr-procedure1}{%
\subsubsection{\texorpdfstring{\texttt{cddr} :
procedure/1}{cddr : procedure/1}}\label{cddr-procedure1}}

Usage: \passthrough{\lstinline!(cddr x) => any!}

Equivalent to (cdr (cdr \passthrough{\lstinline!x!})).

See also:
\passthrough{\lstinline!car, cdr, caar, cadr, cdar, caaar, caadr, cadar, caddr, cdaar, cdadr, cddar, cdddr, nth, 1st, 2nd, 3rd!}.

\hypertarget{cdr-procedure1}{%
\subsubsection{\texorpdfstring{\texttt{cdr} :
procedure/1}{cdr : procedure/1}}\label{cdr-procedure1}}

Usage: \passthrough{\lstinline!(cdr li) => any!}

Get the rest of a list \passthrough{\lstinline!li!}. If the list is
proper, the cdr is a list. If it is a pair, then it may be an element.
If the list is empty, nil is returned.

See also: \passthrough{\lstinline!car, list, list?, pair?!}.

\hypertarget{cond-special-form}{%
\subsubsection{cond : special form}\label{cond-special-form}}

Usage:
\passthrough{\lstinline!(cond ((test1 expr1 ...) (test2 expr2 ...) ...) => any!}

Evaluate the tests sequentially and execute the expressions after the
test when a test is true. To express the else case, use (t exprn \ldots)
at the end of the cond-clauses to execute
\passthrough{\lstinline!exprn!}\ldots{}

See also: \passthrough{\lstinline!if, when, unless!}.

\hypertarget{cons-procedure2}{%
\subsubsection{\texorpdfstring{\texttt{cons} :
procedure/2}{cons : procedure/2}}\label{cons-procedure2}}

Usage: \passthrough{\lstinline!(cons a b) => pair!}

Cons two values into a pair. If \passthrough{\lstinline!b!} is a list,
the result is a list. Otherwise the result is a pair.

See also: \passthrough{\lstinline!cdr, car, list?, pair?!}.

\hypertarget{cons-procedure1}{%
\subsubsection{\texorpdfstring{\texttt{cons?} :
procedure/1}{cons? : procedure/1}}\label{cons-procedure1}}

Usage: \passthrough{\lstinline!(cons? x) => bool!}

return true if \passthrough{\lstinline!x!} is not an atom, nil
otherwise.

See also: \passthrough{\lstinline!atom?!}.

\hypertarget{count-partitions-procedure2}{%
\subsubsection{\texorpdfstring{\texttt{count-partitions} :
procedure/2}{count-partitions : procedure/2}}\label{count-partitions-procedure2}}

Usage: \passthrough{\lstinline!(count-partitions m k) => int!}

Return the number of partitions for divding \passthrough{\lstinline!m!}
items into parts of size \passthrough{\lstinline!k!} or less, where the
size of the last partition may be less than \passthrough{\lstinline!k!}
but the remaining ones have size \passthrough{\lstinline!k.!}

See also: \passthrough{\lstinline!nth-partition, get-partitions!}.

\hypertarget{defmacro-macro2-or-more}{%
\subsubsection{\texorpdfstring{\texttt{defmacro} : macro/2 or
more}{defmacro : macro/2 or more}}\label{defmacro-macro2-or-more}}

Usage: \passthrough{\lstinline!(defmacro name args body ...)!}

Define a macro \passthrough{\lstinline!name!} with argument list
\passthrough{\lstinline!args!} and \passthrough{\lstinline!body!}.
Macros are expanded at compile-time.

See also: \passthrough{\lstinline!macro!}.

\hypertarget{dolist-macro1-or-more}{%
\subsubsection{\texorpdfstring{\texttt{dolist} : macro/1 or
more}{dolist : macro/1 or more}}\label{dolist-macro1-or-more}}

Usage:
\passthrough{\lstinline!(dolist (name list [result]) body ...) => li!}

Traverse the list \passthrough{\lstinline!list!} in order, binding
\passthrough{\lstinline!name!} to each element subsequently and evaluate
the \passthrough{\lstinline!body!} expressions with this binding. The
optional \passthrough{\lstinline!result!} is the result of the
traversal, nil if it is not provided.

See also: \passthrough{\lstinline!letrec, foreach, map!}.

\hypertarget{dotimes-macro1-or-more}{%
\subsubsection{\texorpdfstring{\texttt{dotimes} : macro/1 or
more}{dotimes : macro/1 or more}}\label{dotimes-macro1-or-more}}

Usage:
\passthrough{\lstinline!(dotimes (name count [result]) body ...) => any!}

Iterate \passthrough{\lstinline!count!} times, binding
\passthrough{\lstinline!name!} to the counter starting from 0 until the
counter has reached count-1, and evaluate the
\passthrough{\lstinline!body!} expressions each time with this binding.
The optional \passthrough{\lstinline!result!} is the result of the
iteration, nil if it is not provided.

See also: \passthrough{\lstinline!letrec, dolist, while!}.

\hypertarget{equal-procedure2}{%
\subsubsection{\texorpdfstring{\texttt{equal?} :
procedure/2}{equal? : procedure/2}}\label{equal-procedure2}}

Usage: \passthrough{\lstinline!(equal? x y) => bool!}

Return true if \passthrough{\lstinline!x!} and
\passthrough{\lstinline!y!} are equal, nil otherwise. The equality is
tested recursively for containers like lists and arrays.

See also: \passthrough{\lstinline!eq?, eql?!}.

\hypertarget{filter-procedure2}{%
\subsubsection{\texorpdfstring{\texttt{filter} :
procedure/2}{filter : procedure/2}}\label{filter-procedure2}}

Usage: \passthrough{\lstinline!(filter li pred) => li!}

Return the list based on \passthrough{\lstinline!li!} with each element
removed for which \passthrough{\lstinline!pred!} returns nil.

See also: \passthrough{\lstinline!list!}.

\hypertarget{flatten-procedure1}{%
\subsubsection{\texorpdfstring{\texttt{flatten} :
procedure/1}{flatten : procedure/1}}\label{flatten-procedure1}}

Usage: \passthrough{\lstinline!(flatten lst) => list!}

Flatten \passthrough{\lstinline!lst!}, making all elements of sublists
elements of the flattened list.

See also: \passthrough{\lstinline!car, cdr, remove-duplicates!}.

\hypertarget{get-partitions-procedure2}{%
\subsubsection{\texorpdfstring{\texttt{get-partitions} :
procedure/2}{get-partitions : procedure/2}}\label{get-partitions-procedure2}}

Usage: \passthrough{\lstinline!(get-partitions x n) => proc/1*!}

Return an iterator procedure that returns lists of the form
(start-offset end-offset bytes) with 0-index offsets for a given index
\passthrough{\lstinline!k!}, or nil if there is no corresponding part,
such that the sizes of the partitions returned in
\passthrough{\lstinline!bytes!} summed up are
\passthrough{\lstinline!x!} and and each partition is
\passthrough{\lstinline!n!} or lower in size. The last partition will be
the smallest partition with a \passthrough{\lstinline!bytes!} value
smaller than \passthrough{\lstinline!n!} if \passthrough{\lstinline!x!}
is not dividable without rest by \passthrough{\lstinline!n!}. If no
argument is provided for the returned iterator, then it returns the
number of partitions.

See also:
\passthrough{\lstinline!nth-partition, count-partitions, get-file-partitions, iterate!}.

\hypertarget{identity-procedure1}{%
\subsubsection{\texorpdfstring{\texttt{identity} :
procedure/1}{identity : procedure/1}}\label{identity-procedure1}}

Usage: \passthrough{\lstinline!(identity x)!}

Return \passthrough{\lstinline!x.!}

See also: \passthrough{\lstinline!apply, equal?!}.

\hypertarget{if-macro3}{%
\subsubsection{\texorpdfstring{\texttt{if} :
macro/3}{if : macro/3}}\label{if-macro3}}

Usage: \passthrough{\lstinline!(if cond expr1 expr2) => any!}

Evaluate \passthrough{\lstinline!expr1!} if
\passthrough{\lstinline!cond!} is true, otherwise evaluate
\passthrough{\lstinline!expr2.!}

See also: \passthrough{\lstinline!cond, when, unless!}.

\hypertarget{iterate-procedure2}{%
\subsubsection{\texorpdfstring{\texttt{iterate} :
procedure/2}{iterate : procedure/2}}\label{iterate-procedure2}}

Usage: \passthrough{\lstinline!(iterate it proc)!}

Apply \passthrough{\lstinline!proc!} to each argument returned by
iterator \passthrough{\lstinline!it!} in sequence, similar to the way
foreach works. An iterator is a procedure that takes one integer as
argument or no argument at all. If no argument is provided, the iterator
returns the number of iterations. If an integer is provided, the
iterator returns a non-nil value for the given index.

See also: \passthrough{\lstinline!foreach, get-partitions!}.

\hypertarget{lambda-special-form}{%
\subsubsection{lambda : special form}\label{lambda-special-form}}

Usage: \passthrough{\lstinline!(lambda args body ...) => closure!}

Form a function closure (lambda term) with argument list in
\passthrough{\lstinline!args!} and body expressions
\passthrough{\lstinline!body.!}

See also:
\passthrough{\lstinline!defun, functional?, macro?, closure?!}.

\hypertarget{lcons-procedure2}{%
\subsubsection{\texorpdfstring{\texttt{lcons} :
procedure/2}{lcons : procedure/2}}\label{lcons-procedure2}}

Usage: \passthrough{\lstinline!(lcons datum li) => list!}

Insert \passthrough{\lstinline!datum!} at the end of the list
\passthrough{\lstinline!li!}. There may be a more efficient
implementation of this in the future. Or, maybe not. Who knows?

See also: \passthrough{\lstinline!cons, list, append, nreverse!}.

\hypertarget{let-macro1-or-more}{%
\subsubsection{\texorpdfstring{\texttt{let} : macro/1 or
more}{let : macro/1 or more}}\label{let-macro1-or-more}}

Usage: \passthrough{\lstinline!(let args body ...) => any!}

Bind each pair of symbol and expression in
\passthrough{\lstinline!args!} and evaluate the expressions in
\passthrough{\lstinline!body!} with these local bindings. Return the
value of the last expression in \passthrough{\lstinline!body.!}

See also: \passthrough{\lstinline!letrec!}.

\hypertarget{letrec-macro1-or-more}{%
\subsubsection{\texorpdfstring{\texttt{letrec} : macro/1 or
more}{letrec : macro/1 or more}}\label{letrec-macro1-or-more}}

Usage: \passthrough{\lstinline!(letrec args body ...) => any!}

Recursive let binds the symbol, expression pairs in
\passthrough{\lstinline!args!} in a way that makes prior bindings
available to later bindings and allows for recursive definitions in
\passthrough{\lstinline!args!}, then evaluates the
\passthrough{\lstinline!body!} expressions with these bindings.

See also: \passthrough{\lstinline!let!}.

\hypertarget{list-procedure0-or-more}{%
\subsubsection{\texorpdfstring{\texttt{list} : procedure/0 or
more}{list : procedure/0 or more}}\label{list-procedure0-or-more}}

Usage: \passthrough{\lstinline!(list [args] ...) => li!}

Create a list from all \passthrough{\lstinline!args!}. The arguments
must be quoted.

See also: \passthrough{\lstinline!cons!}.

\hypertarget{list-exists-procedure2}{%
\subsubsection{\texorpdfstring{\texttt{list-exists?} :
procedure/2}{list-exists? : procedure/2}}\label{list-exists-procedure2}}

Usage: \passthrough{\lstinline!(list-exists? li pred) => bool!}

Return true if \passthrough{\lstinline!pred!} returns true for at least
one element in list \passthrough{\lstinline!li!}, nil otherwise.

See also:
\passthrough{\lstinline!exists?, forall?, array-exists?, str-exists?, seq?!}.

\hypertarget{list-forall-procedure2}{%
\subsubsection{\texorpdfstring{\texttt{list-forall?} :
procedure/2}{list-forall? : procedure/2}}\label{list-forall-procedure2}}

Usage: \passthrough{\lstinline!(list-all? li pred) => bool!}

Return true if predicate \passthrough{\lstinline!pred!} returns true for
all elements of list \passthrough{\lstinline!li!}, nil otherwise.

See also:
\passthrough{\lstinline!foreach, map, forall?, array-forall?, str-forall?, exists?!}.

\hypertarget{list-foreach-procedure2}{%
\subsubsection{\texorpdfstring{\texttt{list-foreach} :
procedure/2}{list-foreach : procedure/2}}\label{list-foreach-procedure2}}

Usage: \passthrough{\lstinline!(list-foreach li proc)!}

Apply \passthrough{\lstinline!proc!} to each element of list
\passthrough{\lstinline!li!} in order, for the side effects.

See also: \passthrough{\lstinline!mapcar, map, foreach!}.

\hypertarget{list-last-procedure1}{%
\subsubsection{\texorpdfstring{\texttt{list-last} :
procedure/1}{list-last : procedure/1}}\label{list-last-procedure1}}

Usage: \passthrough{\lstinline!(list-last li) => any!}

Return the last element of \passthrough{\lstinline!li.!}

See also: \passthrough{\lstinline!reverse, nreverse, car, 1st, last!}.

\hypertarget{list-ref-procedure2}{%
\subsubsection{\texorpdfstring{\texttt{list-ref} :
procedure/2}{list-ref : procedure/2}}\label{list-ref-procedure2}}

Usage: \passthrough{\lstinline!(list-ref li n) => any!}

Return the element with index \passthrough{\lstinline!n!} of list
\passthrough{\lstinline!li!}. Lists are 0-indexed.

See also: \passthrough{\lstinline!array-ref, nth!}.

\hypertarget{list-reverse-procedure1}{%
\subsubsection{\texorpdfstring{\texttt{list-reverse} :
procedure/1}{list-reverse : procedure/1}}\label{list-reverse-procedure1}}

Usage: \passthrough{\lstinline!(list-reverse li) => li!}

Create a reversed copy of \passthrough{\lstinline!li.!}

See also: \passthrough{\lstinline!reverse, array-reverse, str-reverse!}.

\hypertarget{list-slice-procedure3}{%
\subsubsection{\texorpdfstring{\texttt{list-slice} :
procedure/3}{list-slice : procedure/3}}\label{list-slice-procedure3}}

Usage: \passthrough{\lstinline!(list-slice li low high) => li!}

Return the slice of the list \passthrough{\lstinline!li!} starting at
index \passthrough{\lstinline!low!} (inclusive) and ending at index
\passthrough{\lstinline!high!} (exclusive).

See also: \passthrough{\lstinline!slice, array-slice!}.

\hypertarget{list-procedure1}{%
\subsubsection{\texorpdfstring{\texttt{list?} :
procedure/1}{list? : procedure/1}}\label{list-procedure1}}

Usage: \passthrough{\lstinline!(list? obj) => bool!}

Return true if \passthrough{\lstinline!obj!} is a list, nil otherwise.

See also: \passthrough{\lstinline!cons?, atom?, null?!}.

\hypertarget{macro-special-form}{%
\subsubsection{macro : special form}\label{macro-special-form}}

Usage: \passthrough{\lstinline!(macro args body ...) => macro!}

Like a lambda term but the \passthrough{\lstinline!body!} expressions
are macro-expanded at compile time instead of runtime.

See also:
\passthrough{\lstinline!defun, lambda, funcional?, macro?, closure?!}.

\hypertarget{mapcar-procedure2}{%
\subsubsection{\texorpdfstring{\texttt{mapcar} :
procedure/2}{mapcar : procedure/2}}\label{mapcar-procedure2}}

Usage: \passthrough{\lstinline!(mapcar li proc) => li!}

Return the list obtained from applying \passthrough{\lstinline!proc!} to
each elements in \passthrough{\lstinline!li.!}

See also: \passthrough{\lstinline!map, foreach!}.

\hypertarget{member-procedure2}{%
\subsubsection{\texorpdfstring{\texttt{member} :
procedure/2}{member : procedure/2}}\label{member-procedure2}}

Usage: \passthrough{\lstinline!(member key li) => li!}

Return the cdr of \passthrough{\lstinline!li!} starting with
\passthrough{\lstinline!key!} if \passthrough{\lstinline!li!} contains
an element equal? to \passthrough{\lstinline!key!}, nil otherwise.

See also: \passthrough{\lstinline!assoc, equal?!}.

\hypertarget{memq-procedure2}{%
\subsubsection{\texorpdfstring{\texttt{memq} :
procedure/2}{memq : procedure/2}}\label{memq-procedure2}}

Usage: \passthrough{\lstinline!(memq key li)!}

Return the cdr of \passthrough{\lstinline!li!} starting with
\passthrough{\lstinline!key!} if \passthrough{\lstinline!li!} contains
an element eq? to \passthrough{\lstinline!key!}, nil otherwise.

See also: \passthrough{\lstinline!member, eq?!}.

\hypertarget{nconc-procedure0-or-more}{%
\subsubsection{\texorpdfstring{\texttt{nconc} : procedure/0 or
more}{nconc : procedure/0 or more}}\label{nconc-procedure0-or-more}}

Usage: \passthrough{\lstinline!(nconc li1 li2 ...) => li!}

Concatenate \passthrough{\lstinline!li1!},
\passthrough{\lstinline!li2!}, and so forth, like with append, but
destructively modifies \passthrough{\lstinline!li1.!}

See also: \passthrough{\lstinline!append!}.

\hypertarget{not-procedure1}{%
\subsubsection{\texorpdfstring{\texttt{not} :
procedure/1}{not : procedure/1}}\label{not-procedure1}}

Usage: \passthrough{\lstinline!(not x) => bool!}

Return true if \passthrough{\lstinline!x!} is nil, nil otherwise.

See also: \passthrough{\lstinline!and, or!}.

\hypertarget{nreverse-procedure1}{%
\subsubsection{\texorpdfstring{\texttt{nreverse} :
procedure/1}{nreverse : procedure/1}}\label{nreverse-procedure1}}

Usage: \passthrough{\lstinline!(nreverse li) => li!}

Destructively reverse \passthrough{\lstinline!li.!}

See also: \passthrough{\lstinline!reverse!}.

\hypertarget{nth-partition-procedure3}{%
\subsubsection{\texorpdfstring{\texttt{nth-partition} :
procedure/3}{nth-partition : procedure/3}}\label{nth-partition-procedure3}}

Usage: \passthrough{\lstinline!(nth-partition m k idx) => li!}

Return a list of the form (start-offset end-offset bytes) for the
partition with index \passthrough{\lstinline!idx!} of
\passthrough{\lstinline!m!} into parts of size
\passthrough{\lstinline!k!}. The index \passthrough{\lstinline!idx!} as
well as the start- and end-offsets are 0-based.

See also: \passthrough{\lstinline!count-partitions, get-partitions!}.

\hypertarget{null-procedure1}{%
\subsubsection{\texorpdfstring{\texttt{null?} :
procedure/1}{null? : procedure/1}}\label{null-procedure1}}

Usage: \passthrough{\lstinline!(null? li) => bool!}

Return true if \passthrough{\lstinline!li!} is nil, nil otherwise.

See also: \passthrough{\lstinline!not, list?, cons?!}.

\hypertarget{num-procedure1}{%
\subsubsection{\texorpdfstring{\texttt{num?} :
procedure/1}{num? : procedure/1}}\label{num-procedure1}}

Usage: \passthrough{\lstinline!(num? n) => bool!}

Return true if \passthrough{\lstinline!n!} is a number (exact or
inexact), nil otherwise.

See also:
\passthrough{\lstinline!str?, atom?, sym?, closure?, intrinsic?, macro?!}.

\hypertarget{or-macro0-or-more}{%
\subsubsection{\texorpdfstring{\texttt{or} : macro/0 or
more}{or : macro/0 or more}}\label{or-macro0-or-more}}

Usage: \passthrough{\lstinline!(or expr1 expr2 ...) => any!}

Evaluate the expressions until one of them is not nil. This is a logical
shortcut or.

See also: \passthrough{\lstinline!and!}.

\hypertarget{progn-special-form}{%
\subsubsection{progn : special form}\label{progn-special-form}}

Usage: \passthrough{\lstinline!(progn expr1 expr2 ...) => any!}

Sequentially execute the expressions \passthrough{\lstinline!expr1!},
\passthrough{\lstinline!expr2!}, and so forth, and return the value of
the last expression.

See also: \passthrough{\lstinline!defun, lambda, cond!}.

\hypertarget{quasiquote-special-form}{%
\subsubsection{quasiquote : special
form}\label{quasiquote-special-form}}

Usage: \passthrough{\lstinline!(quasiquote li)!}

Quote \passthrough{\lstinline!li!}, except that values in
\passthrough{\lstinline!li!} may be unquoted (\textasciitilde evaluated)
when prefixed with ``,'' and embedded lists can be unquote-spliced by
prefixing them with unquote-splice ``,@''. An unquoted expression's
value is inserted directly, whereas unquote-splice inserts the values of
a list in-sequence into the embedding list. Quasiquote is used in
combination with gensym to define non-hygienic macros. In Z3S5 Lisp,
``,'' and ``,@'' are syntactic markers and there are no corresponding
unquote and unquote-splice functions. The shortcut for quasiquote is
```''.

See also: \passthrough{\lstinline!quote, gensym, macro, defmacro!}.

\hypertarget{quote-special-form}{%
\subsubsection{quote : special form}\label{quote-special-form}}

Usage: \passthrough{\lstinline!(quote x)!}

Quote symbol \passthrough{\lstinline!x!}, so it evaluates to
\passthrough{\lstinline!x!} instead of the value bound to it. Syntactic
shortcut is '.

See also: \passthrough{\lstinline!quasiquote!}.

\hypertarget{replacd-procedure2}{%
\subsubsection{\texorpdfstring{\texttt{replacd} :
procedure/2}{replacd : procedure/2}}\label{replacd-procedure2}}

Usage: \passthrough{\lstinline!(rplacd li1 li2) => li!}

Destructively replace the cdr of \passthrough{\lstinline!li1!} with
\passthrough{\lstinline!li2!} and return the result afterwards.

See also: \passthrough{\lstinline!rplaca!}.

\hypertarget{rplaca-procedure2}{%
\subsubsection{\texorpdfstring{\texttt{rplaca} :
procedure/2}{rplaca : procedure/2}}\label{rplaca-procedure2}}

Usage: \passthrough{\lstinline!(rplaca li a) => li!}

Destructively mutate \passthrough{\lstinline!li!} such that its car is
\passthrough{\lstinline!a!}, return the list afterwards.

See also: \passthrough{\lstinline!rplacd!}.

\hypertarget{setcar-procedure1}{%
\subsubsection{\texorpdfstring{\texttt{setcar} :
procedure/1}{setcar : procedure/1}}\label{setcar-procedure1}}

Usage: \passthrough{\lstinline!(setcar li elem) => li!}

Mutate \passthrough{\lstinline!li!} such that its car is
\passthrough{\lstinline!elem!}. Same as rplaca.

See also: \passthrough{\lstinline!rplaca, rplacd, setcdr!}.

\hypertarget{setcdr-procedure1}{%
\subsubsection{\texorpdfstring{\texttt{setcdr} :
procedure/1}{setcdr : procedure/1}}\label{setcdr-procedure1}}

Usage: \passthrough{\lstinline!(setcdr li1 li2) => li!}

Mutate \passthrough{\lstinline!li1!} such that its cdr is
\passthrough{\lstinline!li2!}. Same as rplacd.

See also: \passthrough{\lstinline!rplacd, rplaca, setcar!}.

\hypertarget{setq-special-form}{%
\subsubsection{setq : special form}\label{setq-special-form}}

Usage: \passthrough{\lstinline!(setq sym1 value1 ...)!}

Set \passthrough{\lstinline!sym1!} (without need for quoting it) to
\passthrough{\lstinline!value!}, and so forth for any further symbol,
value pairs.

See also: \passthrough{\lstinline!bind, unbind!}.

\hypertarget{sort-procedure2}{%
\subsubsection{\texorpdfstring{\texttt{sort} :
procedure/2}{sort : procedure/2}}\label{sort-procedure2}}

Usage: \passthrough{\lstinline!(sort li proc) => li!}

Sort the list \passthrough{\lstinline!li!} by the given less-than
procedure \passthrough{\lstinline!proc!}, which takes two arguments and
returns true if the first one is less than the second, nil otheriwse.

See also: \passthrough{\lstinline!array-sort!}.

\hypertarget{sort-symbols-nil}{%
\subsubsection{sort-symbols : nil}\label{sort-symbols-nil}}

Usage: \passthrough{\lstinline!(sort-symbols li) => list!}

Sort the list of symbols \passthrough{\lstinline!li!} alphabetically.

See also: \passthrough{\lstinline!out, dp, du, dump!}.

\hypertarget{sym-procedure1}{%
\subsubsection{\texorpdfstring{\texttt{sym?} :
procedure/1}{sym? : procedure/1}}\label{sym-procedure1}}

Usage: \passthrough{\lstinline!(sym? sym) => bool!}

Return true if \passthrough{\lstinline!sym!} is a symbol, nil otherwise.

See also: \passthrough{\lstinline!str?, atom?!}.

\hypertarget{unless-macro1-or-more}{%
\subsubsection{\texorpdfstring{\texttt{unless} : macro/1 or
more}{unless : macro/1 or more}}\label{unless-macro1-or-more}}

Usage: \passthrough{\lstinline!(unless cond expr ...) => any!}

Evaluate expressions \passthrough{\lstinline!expr!} if
\passthrough{\lstinline!cond!} is not true, returns void otherwise.

See also: \passthrough{\lstinline!if, when, cond!}.

\hypertarget{void-procedure0-or-more}{%
\subsubsection{\texorpdfstring{\texttt{void} : procedure/0 or
more}{void : procedure/0 or more}}\label{void-procedure0-or-more}}

Usage: \passthrough{\lstinline!(void [any] ...)!}

Always returns void, no matter what values are given to it. Void is a
special value that is not printed in the console.

See also: \passthrough{\lstinline!void?!}.

\hypertarget{when-macro1-or-more}{%
\subsubsection{\texorpdfstring{\texttt{when} : macro/1 or
more}{when : macro/1 or more}}\label{when-macro1-or-more}}

Usage: \passthrough{\lstinline!(when cond expr ...) => any!}

Evaluate the expressions \passthrough{\lstinline!expr!} if
\passthrough{\lstinline!cond!} is true, returns void otherwise.

See also: \passthrough{\lstinline!if, cond, unless!}.

\hypertarget{while-macro1-or-more}{%
\subsubsection{\texorpdfstring{\texttt{while} : macro/1 or
more}{while : macro/1 or more}}\label{while-macro1-or-more}}

Usage: \passthrough{\lstinline!(while test body ...) => any!}

Evaluate the expressions in \passthrough{\lstinline!body!} while
\passthrough{\lstinline!test!} is not nil.

See also: \passthrough{\lstinline!letrec, dotimes, dolist!}.

\hypertarget{numeric-functions}{%
\subsection{Numeric Functions}\label{numeric-functions}}

This section describes functions that provide standard arithmetics for
non-floating point numbers such as integers. Notice that Z3S5 Lisp uses
automatic bignum support but only for select standard operations like
multiplication, addition, and subtraction.

\hypertarget{procedure2} :
procedure/2}{\% : procedure/2}}\label{procedure2}}

Usage: \passthrough{\lstinline!(\% x y) => num!}

Compute the remainder of dividing number \passthrough{\lstinline!x!} by
\passthrough{\lstinline!y.!}

See also: \passthrough{\lstinline!mod, /!}.

\hypertarget{procedure0-or-more}{%
\subsubsection{\texorpdfstring{\texttt{*} : procedure/0 or
more}{* : procedure/0 or more}}\label{procedure0-or-more}}

Usage: \passthrough{\lstinline!(* [args] ...) => num!}

Multiply all \passthrough{\lstinline!args!}. Special cases: (\emph{) is
1 and (} x) is x.

See also: \passthrough{\lstinline!+, -, /!}.

\hypertarget{procedure0-or-more-1}{%
\subsubsection{\texorpdfstring{\texttt{+} : procedure/0 or
more}{+ : procedure/0 or more}}\label{procedure0-or-more-1}}

Usage: \passthrough{\lstinline!(+ [args] ...) => num!}

Sum up all \passthrough{\lstinline!args!}. Special cases: (+) is 0 and
(+ x) is x.

See also: \passthrough{\lstinline!-, *, /!}.

\hypertarget{procedure1-or-more}{%
\subsubsection{\texorpdfstring{\texttt{-} : procedure/1 or
more}{- : procedure/1 or more}}\label{procedure1-or-more}}

Usage: \passthrough{\lstinline!(- x [y1] [y2] ...) => num!}

Subtract \passthrough{\lstinline!y1!}, \passthrough{\lstinline!y2!},
\ldots, from \passthrough{\lstinline!x!}. Special case: (- x) is -x.

See also: \passthrough{\lstinline!+, *, /!}.

\hypertarget{procedure1-or-more-1}{%
\subsubsection{\texorpdfstring{\texttt{/} : procedure/1 or
more}{/ : procedure/1 or more}}\label{procedure1-or-more-1}}

Usage: \passthrough{\lstinline!(/ x y1 [y2] ...) => float!}

Divide \passthrough{\lstinline!x!} by \passthrough{\lstinline!y1!}, then
by \passthrough{\lstinline!y2!}, and so forth. The result is a float.

See also: \passthrough{\lstinline!+, *, -!}.

\hypertarget{procedure2-1}{%
\subsubsection{\texorpdfstring{\texttt{/=} :
procedure/2}{/= : procedure/2}}\label{procedure2-1}}

Usage: \passthrough{\lstinline!(/= x y) => bool!}

Return true if number \passthrough{\lstinline!x!} is not equal to
\passthrough{\lstinline!y!}, nil otherwise.

See also: \passthrough{\lstinline!>, >=, <, <=!}.

\hypertarget{procedure2-2}{%
\subsubsection{\texorpdfstring{\texttt{\textless{}} :
procedure/2}{\textless{} : procedure/2}}\label{procedure2-2}}

Usage: \passthrough{\lstinline!(< x y) => bool!}

Return true if \passthrough{\lstinline!x!} is smaller than
\passthrough{\lstinline!y.!}

See also: \passthrough{\lstinline!<=, >=, >!}.

\hypertarget{procedure2-3}{%
\subsubsection{\texorpdfstring{\texttt{\textless{}=} :
procedure/2}{\textless= : procedure/2}}\label{procedure2-3}}

Usage: \passthrough{\lstinline!(<= x y) => bool!}

Return true if \passthrough{\lstinline!x!} is smaller than or equal to
\passthrough{\lstinline!y!}, nil otherwise.

See also: \passthrough{\lstinline!>, <, >=, /=!}.

\hypertarget{procedure2-4}{%
\subsubsection{\texorpdfstring{\texttt{=} :
procedure/2}{= : procedure/2}}\label{procedure2-4}}

Usage: \passthrough{\lstinline!(= x y) => bool!}

Return true if number \passthrough{\lstinline!x!} equals number
\passthrough{\lstinline!y!}, nil otherwise.

See also: \passthrough{\lstinline!eql?, equal?!}.

\hypertarget{procedure2-5}{%
\subsubsection{\texorpdfstring{\texttt{\textgreater{}} :
procedure/2}{\textgreater{} : procedure/2}}\label{procedure2-5}}

Usage: \passthrough{\lstinline!(> x y) => bool!}

Return true if \passthrough{\lstinline!x!} is larger than
\passthrough{\lstinline!y!}, nil otherwise.

See also: \passthrough{\lstinline!<, >=, <=, /=!}.

\hypertarget{procedure2-6}{%
\subsubsection{\texorpdfstring{\texttt{\textgreater{}=} :
procedure/2}{\textgreater= : procedure/2}}\label{procedure2-6}}

Usage: \passthrough{\lstinline!(>= x y) => bool!}

Return true if \passthrough{\lstinline!x!} is larger than or equal to
\passthrough{\lstinline!y!}, nil otherwise.

See also: \passthrough{\lstinline!>, <, <=, /=!}.

\hypertarget{abs-procedure1}{%
\subsubsection{\texorpdfstring{\texttt{abs} :
procedure/1}{abs : procedure/1}}\label{abs-procedure1}}

Usage: \passthrough{\lstinline!(abs x) => num!}

Returns the absolute value of number \passthrough{\lstinline!x.!}

See also: \passthrough{\lstinline!*, -, +, /!}.

\hypertarget{add1-procedure1}{%
\subsubsection{\texorpdfstring{\texttt{add1} :
procedure/1}{add1 : procedure/1}}\label{add1-procedure1}}

Usage: \passthrough{\lstinline!(add1 n) => num!}

Add 1 to number \passthrough{\lstinline!n.!}

See also: \passthrough{\lstinline!sub1, +, -!}.

\hypertarget{div-procedure2}{%
\subsubsection{\texorpdfstring{\texttt{div} :
procedure/2}{div : procedure/2}}\label{div-procedure2}}

Usage: \passthrough{\lstinline!(div n k) => int!}

Integer division of \passthrough{\lstinline!n!} by
\passthrough{\lstinline!k.!}

See also: \passthrough{\lstinline!truncate, /, int!}.

\hypertarget{even-procedure1}{%
\subsubsection{\texorpdfstring{\texttt{even?} :
procedure/1}{even? : procedure/1}}\label{even-procedure1}}

Usage: \passthrough{\lstinline!(even? n) => bool!}

Returns true if the integer \passthrough{\lstinline!n!} is even, nil if
it is not even.

See also: \passthrough{\lstinline!odd?!}.

\hypertarget{float-procedure1}{%
\subsubsection{\texorpdfstring{\texttt{float} :
procedure/1}{float : procedure/1}}\label{float-procedure1}}

Usage: \passthrough{\lstinline!(float n) => float!}

Convert \passthrough{\lstinline!n!} to a floating point value.

See also: \passthrough{\lstinline!int!}.

\hypertarget{int-procedure1}{%
\subsubsection{\texorpdfstring{\texttt{int} :
procedure/1}{int : procedure/1}}\label{int-procedure1}}

Usage: \passthrough{\lstinline!(int n) => int!}

Return \passthrough{\lstinline!n!} as an integer, rounding down to the
nearest integer if necessary.

See also: \passthrough{\lstinline!float!}.

\textbf{Warning: If the number is very large this may result in
returning the maximum supported integer number rather than the number as
integer.}

\hypertarget{max-procedure1-or-more}{%
\subsubsection{\texorpdfstring{\texttt{max} : procedure/1 or
more}{max : procedure/1 or more}}\label{max-procedure1-or-more}}

Usage: \passthrough{\lstinline!(max x1 x2 ...) => num!}

Return the maximum of the given numbers.

See also: \passthrough{\lstinline!min, minmax!}.

\hypertarget{min-procedure1-or-more}{%
\subsubsection{\texorpdfstring{\texttt{min} : procedure/1 or
more}{min : procedure/1 or more}}\label{min-procedure1-or-more}}

Usage: \passthrough{\lstinline!(min x1 x2 ...) => num!}

Return the minimum of the given numbers.

See also: \passthrough{\lstinline!max, minmax!}.

\hypertarget{minmax-procedure3}{%
\subsubsection{\texorpdfstring{\texttt{minmax} :
procedure/3}{minmax : procedure/3}}\label{minmax-procedure3}}

Usage: \passthrough{\lstinline!(minmax pred li acc) => any!}

Go through \passthrough{\lstinline!li!} and test whether for each
\passthrough{\lstinline!elem!} the comparison (pred elem acc) is true.
If so, \passthrough{\lstinline!elem!} becomes
\passthrough{\lstinline!acc!}. Once all elements of the list have been
compared, \passthrough{\lstinline!acc!} is returned. This procedure can
be used to implement generalized minimum or maximum procedures.

See also: \passthrough{\lstinline!min, max!}.

\hypertarget{mod-procedure2}{%
\subsubsection{\texorpdfstring{\texttt{mod} :
procedure/2}{mod : procedure/2}}\label{mod-procedure2}}

Usage: \passthrough{\lstinline!(mod x y) => num!}

Compute \passthrough{\lstinline!x!} modulo \passthrough{\lstinline!y.!}

See also: \passthrough{\lstinline!\%, /!}.

\hypertarget{odd-procedure1}{%
\subsubsection{\texorpdfstring{\texttt{odd?} :
procedure/1}{odd? : procedure/1}}\label{odd-procedure1}}

Usage: \passthrough{\lstinline!(odd? n) => bool!}

Returns true if the integer \passthrough{\lstinline!n!} is odd, nil
otherwise.

See also: \passthrough{\lstinline!even?!}.

\hypertarget{rand-procedure2}{%
\subsubsection{\texorpdfstring{\texttt{rand} :
procedure/2}{rand : procedure/2}}\label{rand-procedure2}}

Usage: \passthrough{\lstinline!(rand prng lower upper) => int!}

Return a random integer in the interval
{[}\passthrough{\lstinline!lower`` upper!}{]}, both inclusive, from
pseudo-random number generator \passthrough{\lstinline!prng!}. The
\passthrough{\lstinline!prng!} argument must be an integer from 0 to 9
(inclusive).

See also: \passthrough{\lstinline!rnd, rndseed!}.

\hypertarget{rnd-procedure0}{%
\subsubsection{\texorpdfstring{\texttt{rnd} :
procedure/0}{rnd : procedure/0}}\label{rnd-procedure0}}

Usage: \passthrough{\lstinline!(rnd prng) => num!}

Return a random value in the interval {[}0, 1{]} from pseudo-random
number generator \passthrough{\lstinline!prng!}. The
\passthrough{\lstinline!prng!} argument must be an integer from 0 to 9
(inclusive).

See also: \passthrough{\lstinline!rand, rndseed!}.

\hypertarget{rndseed-procedure1}{%
\subsubsection{\texorpdfstring{\texttt{rndseed} :
procedure/1}{rndseed : procedure/1}}\label{rndseed-procedure1}}

Usage: \passthrough{\lstinline!(rndseed prng n)!}

Seed the pseudo-random number generator \passthrough{\lstinline!prng!}
(0 to 9) with 64 bit integer value \passthrough{\lstinline!n!}. Larger
values will be truncated. Seeding affects both the rnd and the rand
function for the given \passthrough{\lstinline!prng.!}

See also: \passthrough{\lstinline!rnd, rand!}.

\hypertarget{sub1-procedure1}{%
\subsubsection{\texorpdfstring{\texttt{sub1} :
procedure/1}{sub1 : procedure/1}}\label{sub1-procedure1}}

Usage: \passthrough{\lstinline!(sub1 n) => num!}

Subtract 1 from \passthrough{\lstinline!n.!}

See also: \passthrough{\lstinline!add1, +, -!}.

\hypertarget{truncate-procedure1-or-more}{%
\subsubsection{\texorpdfstring{\texttt{truncate} : procedure/1 or
more}{truncate : procedure/1 or more}}\label{truncate-procedure1-or-more}}

Usage: \passthrough{\lstinline!(truncate x [y]) => int!}

Round down to nearest integer of \passthrough{\lstinline!x!}. If
\passthrough{\lstinline!y!} is present, divide
\passthrough{\lstinline!x!} by \passthrough{\lstinline!y!} and round
down to the nearest integer.

See also: \passthrough{\lstinline!div, /, int!}.

\hypertarget{semver-semantic-versioning}{%
\subsection{Semver Semantic
Versioning}\label{semver-semantic-versioning}}

The \passthrough{\lstinline!semver!} package provides functions to deal
with the validation and parsing of semantic versioning strings.

\hypertarget{semver.build-procedure1}{%
\subsubsection{\texorpdfstring{\texttt{semver.build} :
procedure/1}{semver.build : procedure/1}}\label{semver.build-procedure1}}

Usage: \passthrough{\lstinline!(semver.build s) => str!}

Return the build part of a semantic versioning string.

See also:
\passthrough{\lstinline!semver.canonical, semver.major, semver.major-minor!}.

\hypertarget{semver.canonical-procedure1}{%
\subsubsection{\texorpdfstring{\texttt{semver.canonical} :
procedure/1}{semver.canonical : procedure/1}}\label{semver.canonical-procedure1}}

Usage: \passthrough{\lstinline!(semver.canonical s) => str!}

Return a canonical semver string based on a valid, yet possibly not
canonical version string \passthrough{\lstinline!s.!}

See also: \passthrough{\lstinline!semver.major!}.

\hypertarget{semver.compare-procedure2}{%
\subsubsection{\texorpdfstring{\texttt{semver.compare} :
procedure/2}{semver.compare : procedure/2}}\label{semver.compare-procedure2}}

Usage: \passthrough{\lstinline!(semver.compare s1 s2) => int!}

Compare two semantic version strings \passthrough{\lstinline!s1!} and
\passthrough{\lstinline!s2!}. The result is 0 if
\passthrough{\lstinline!s1!} and \passthrough{\lstinline!s2!} are the
same version, -1 if \passthrough{\lstinline!s1!} \textless{}
\passthrough{\lstinline!s2!} and 1 if \passthrough{\lstinline!s1!}
\textgreater{} \passthrough{\lstinline!s2.!}

See also: \passthrough{\lstinline!semver.major, semver.major-minor!}.

\hypertarget{semver.is-valid-procedure1}{%
\subsubsection{\texorpdfstring{\texttt{semver.is-valid?} :
procedure/1}{semver.is-valid? : procedure/1}}\label{semver.is-valid-procedure1}}

Usage: \passthrough{\lstinline!(semver.is-valid? s) => bool!}

Return true if \passthrough{\lstinline!s!} is a valid semantic
versioning string, nil otherwise.

See also:
\passthrough{\lstinline!semver.major, semver.major-minor, semver.compare!}.

\hypertarget{semver.major-procedure1}{%
\subsubsection{\texorpdfstring{\texttt{semver.major} :
procedure/1}{semver.major : procedure/1}}\label{semver.major-procedure1}}

Usage: \passthrough{\lstinline!(semver.major s) => str!}

Return the major part of the semantic versioning string.

See also: \passthrough{\lstinline!semver.major-minor, semver.build!}.

\hypertarget{semver.major-minor-procedure1}{%
\subsubsection{\texorpdfstring{\texttt{semver.major-minor} :
procedure/1}{semver.major-minor : procedure/1}}\label{semver.major-minor-procedure1}}

Usage: \passthrough{\lstinline!(semver.major-minor s) => str!}

Return the major.minor prefix of a semantic versioning string. For
example, (semver.major-minor ``v2.1.4'') returns ``v2.1''.

See also: \passthrough{\lstinline!semver.major, semver.build!}.

\hypertarget{semver.max-procedure2}{%
\subsubsection{\texorpdfstring{\texttt{semver.max} :
procedure/2}{semver.max : procedure/2}}\label{semver.max-procedure2}}

Usage: \passthrough{\lstinline!(semver.max s1 s2) => str!}

Canonicalize \passthrough{\lstinline!s1!} and
\passthrough{\lstinline!s2!} and return the larger version of them.

See also: \passthrough{\lstinline!semver.compare!}.

\hypertarget{semver.prerelease-procedure1}{%
\subsubsection{\texorpdfstring{\texttt{semver.prerelease} :
procedure/1}{semver.prerelease : procedure/1}}\label{semver.prerelease-procedure1}}

Usage: \passthrough{\lstinline!(semver.prerelease s) => str!}

Return the prerelease part of a version string, or the empty string if
there is none. For example, (semver.prerelease ``v2.1.0-pre+build'')
returns ``-pre''.

See also:
\passthrough{\lstinline!semver.build, semver.major, semver.major-minor!}.

\hypertarget{sequence-functions}{%
\subsection{Sequence Functions}\label{sequence-functions}}

Sequences are either strings, lists, or arrays. Sequences functions are
generally abstractions for more specific functions of these data types,
and therefore may be a bit slower than their native counterparts. It is
still recommended to use them liberally, since they make programs more
readable.

\hypertarget{th-procedure1-or-more}{%
\subsubsection{\texorpdfstring{\texttt{10th} : procedure/1 or
more}{10th : procedure/1 or more}}\label{th-procedure1-or-more}}

Usage: \passthrough{\lstinline!(10th seq [default]) => any!}

Get the tenth element of a sequence or the optional
\passthrough{\lstinline!default!}. If there is no such element and no
default is provided, then an error is raised.

See also:
\passthrough{\lstinline!nth, nthdef, car, list-ref, array-ref, string-ref, 1st, 2nd, 3rd, 4th, 5th, 6th, 7th, 8th, 9th!}.

\hypertarget{st-procedure1-or-more}{%
\subsubsection{\texorpdfstring{\texttt{1st} : procedure/1 or
more}{1st : procedure/1 or more}}\label{st-procedure1-or-more}}

Usage: \passthrough{\lstinline!(1st seq [default]) => any!}

Get the first element of a sequence or the optional
\passthrough{\lstinline!default!}. If there is no such element and no
default is provided, then an error is raised.

See also:
\passthrough{\lstinline!nth, nthdef, car, list-ref, array-ref, string-ref, 2nd, 3rd, 4th, 5th, 6th, 7th, 8th, 9th, 10th!}.

\hypertarget{nd-procedure1-or-more}{%
\subsubsection{\texorpdfstring{\texttt{2nd} : procedure/1 or
more}{2nd : procedure/1 or more}}\label{nd-procedure1-or-more}}

Usage: \passthrough{\lstinline!(2nd seq [default]) => any!}

Get the second element of a sequence or the optional
\passthrough{\lstinline!default!}. If there is no such element and no
default is provided, then an error is raised.

See also:
\passthrough{\lstinline!nth, nthdef, car, list-ref, array-ref, string-ref, 1st, 3rd, 4th, 5th, 6th, 7th, 8th, 9th, 10th!}.

\hypertarget{rd-procedure1-or-more}{%
\subsubsection{\texorpdfstring{\texttt{3rd} : procedure/1 or
more}{3rd : procedure/1 or more}}\label{rd-procedure1-or-more}}

Usage: \passthrough{\lstinline!(3rd seq [default]) => any!}

Get the third element of a sequence or the optional
\passthrough{\lstinline!default!}. If there is no such element and no
default is provided, then an error is raised.

See also:
\passthrough{\lstinline!nth, nthdef, car, list-ref, array-ref, string-ref, 1st, 2nd, 4th, 5th, 6th, 7th, 8th, 9th, 10th!}.

\hypertarget{th-procedure1-or-more-1}{%
\subsubsection{\texorpdfstring{\texttt{4th} : procedure/1 or
more}{4th : procedure/1 or more}}\label{th-procedure1-or-more-1}}

Usage: \passthrough{\lstinline!(4th seq [default]) => any!}

Get the fourth element of a sequence or the optional
\passthrough{\lstinline!default!}. If there is no such element and no
default is provided, then an error is raised.

See also:
\passthrough{\lstinline!nth, nthdef, car, list-ref, array-ref, string-ref, 1st, 2nd, 3rd, 5th, 6th, 7th, 8th, 9th, 10th!}.

\hypertarget{th-procedure1-or-more-2}{%
\subsubsection{\texorpdfstring{\texttt{5th} : procedure/1 or
more}{5th : procedure/1 or more}}\label{th-procedure1-or-more-2}}

Usage: \passthrough{\lstinline!(5th seq [default]) => any!}

Get the fifth element of a sequence or the optional
\passthrough{\lstinline!default!}. If there is no such element and no
default is provided, then an error is raised.

See also:
\passthrough{\lstinline!nth, nthdef, car, list-ref, array-ref, string-ref, 1st, 2nd, 3rd, 4th, 6th, 7th, 8th, 9th, 10th!}.

\hypertarget{th-procedure1-or-more-3}{%
\subsubsection{\texorpdfstring{\texttt{6th} : procedure/1 or
more}{6th : procedure/1 or more}}\label{th-procedure1-or-more-3}}

Usage: \passthrough{\lstinline!(6th seq [default]) => any!}

Get the sixth element of a sequence or the optional
\passthrough{\lstinline!default!}. If there is no such element and no
default is provided, then an error is raised.

See also:
\passthrough{\lstinline!nth, nthdef, car, list-ref, array-ref, string-ref, 1st, 2nd, 3rd, 4th, 5th, 7th, 8th, 9th, 10th!}.

\hypertarget{th-procedure1-or-more-4}{%
\subsubsection{\texorpdfstring{\texttt{7th} : procedure/1 or
more}{7th : procedure/1 or more}}\label{th-procedure1-or-more-4}}

Usage: \passthrough{\lstinline!(7th seq [default]) => any!}

Get the seventh element of a sequence or the optional
\passthrough{\lstinline!default!}. If there is no such element and no
default is provided, then an error is raised.

See also:
\passthrough{\lstinline!nth, nthdef, car, list-ref, array-ref, string-ref, 1st, 2nd, 3rd, 4th, 5th, 6th, 8th, 9th, 10th!}.

\hypertarget{th-procedure1-or-more-5}{%
\subsubsection{\texorpdfstring{\texttt{8th} : procedure/1 or
more}{8th : procedure/1 or more}}\label{th-procedure1-or-more-5}}

Usage: \passthrough{\lstinline!(8th seq [default]) => any!}

Get the eighth element of a sequence or the optional
\passthrough{\lstinline!default!}. If there is no such element and no
default is provided, then an error is raised.

See also:
\passthrough{\lstinline!nth, nthdef, car, list-ref, array-ref, string-ref, 1st, 2nd, 3rd, 4th, 5th, 6th, 7th, 9th, 10th!}.

\hypertarget{th-procedure1-or-more-6}{%
\subsubsection{\texorpdfstring{\texttt{9th} : procedure/1 or
more}{9th : procedure/1 or more}}\label{th-procedure1-or-more-6}}

Usage: \passthrough{\lstinline!(9th seq [default]) => any!}

Get the nineth element of a sequence or the optional
\passthrough{\lstinline!default!}. If there is no such element and no
default is provided, then an error is raised.

See also:
\passthrough{\lstinline!nth, nthdef, car, list-ref, array-ref, string-ref, 1st, 2nd, 3rd, 4th, 5th, 6th, 7th, 8th, 10th!}.

\hypertarget{exists-procedure2}{%
\subsubsection{\texorpdfstring{\texttt{exists?} :
procedure/2}{exists? : procedure/2}}\label{exists-procedure2}}

Usage: \passthrough{\lstinline!(exists? seq pred) => bool!}

Return true if \passthrough{\lstinline!pred!} returns true for at least
one element in sequence \passthrough{\lstinline!seq!}, nil otherwise.

See also:
\passthrough{\lstinline!forall?, list-exists?, array-exists?, str-exists?, seq?!}.

\hypertarget{forall-procedure2}{%
\subsubsection{\texorpdfstring{\texttt{forall?} :
procedure/2}{forall? : procedure/2}}\label{forall-procedure2}}

Usage: \passthrough{\lstinline!(forall? seq pred) => bool!}

Return true if predicate \passthrough{\lstinline!pred!} returns true for
all elements of sequence \passthrough{\lstinline!seq!}, nil otherwise.

See also:
\passthrough{\lstinline!foreach, map, list-forall?, array-forall?, str-forall?, exists?, str-exists?, array-exists?, list-exists?!}.

\hypertarget{foreach-procedure2}{%
\subsubsection{\texorpdfstring{\texttt{foreach} :
procedure/2}{foreach : procedure/2}}\label{foreach-procedure2}}

Usage: \passthrough{\lstinline!(foreach seq proc)!}

Apply \passthrough{\lstinline!proc!} to each element of sequence
\passthrough{\lstinline!seq!} in order, for the side effects.

See also: \passthrough{\lstinline!seq?, map!}.

\hypertarget{index-procedure2-or-more}{%
\subsubsection{\texorpdfstring{\texttt{index} : procedure/2 or
more}{index : procedure/2 or more}}\label{index-procedure2-or-more}}

Usage: \passthrough{\lstinline!(index seq elem [pred]) => int!}

Return the first index of \passthrough{\lstinline!elem!} in
\passthrough{\lstinline!seq!} going from left to right, using equality
predicate \passthrough{\lstinline!pred!} for comparisons (default is
eq?). If \passthrough{\lstinline!elem!} is not in
\passthrough{\lstinline!seq!}, -1 is returned.

See also: \passthrough{\lstinline!nth, seq?!}.

\hypertarget{last-procedure1-or-more}{%
\subsubsection{\texorpdfstring{\texttt{last} : procedure/1 or
more}{last : procedure/1 or more}}\label{last-procedure1-or-more}}

Usage: \passthrough{\lstinline!(last seq [default]) => any!}

Get the last element of sequence \passthrough{\lstinline!seq!} or return
\passthrough{\lstinline!default!} if the sequence is empty. If
\passthrough{\lstinline!default!} is not given and the sequence is
empty, an error is raised.

See also:
\passthrough{\lstinline!nth, nthdef, car, list-ref, array-ref, string, ref, 1st, 2nd, 3rd, 4th, 5th, 6th, 7th, 8th, 9th, 10th!}.

\hypertarget{len-procedure1}{%
\subsubsection{\texorpdfstring{\texttt{len} :
procedure/1}{len : procedure/1}}\label{len-procedure1}}

Usage: \passthrough{\lstinline!(len seq) => int!}

Return the length of \passthrough{\lstinline!seq!}. Works for lists,
strings, arrays, and dicts.

See also: \passthrough{\lstinline!seq?!}.

\hypertarget{map-procedure2}{%
\subsubsection{\texorpdfstring{\texttt{map} :
procedure/2}{map : procedure/2}}\label{map-procedure2}}

Usage: \passthrough{\lstinline!(map seq proc) => seq!}

Return the copy of \passthrough{\lstinline!seq!} that is the result of
applying \passthrough{\lstinline!proc!} to each element of
\passthrough{\lstinline!seq.!}

See also: \passthrough{\lstinline!seq?, mapcar, strmap!}.

\hypertarget{map-pairwise-procedure2}{%
\subsubsection{\texorpdfstring{\texttt{map-pairwise} :
procedure/2}{map-pairwise : procedure/2}}\label{map-pairwise-procedure2}}

Usage: \passthrough{\lstinline!(map-pairwise seq proc) => seq!}

Applies \passthrough{\lstinline!proc!} in order to subsequent pairs in
\passthrough{\lstinline!seq!}, assembling the sequence that results from
the results of \passthrough{\lstinline!proc!}. Function
\passthrough{\lstinline!proc!} takes two arguments and must return a
proper list containing two elements. If the number of elements in
\passthrough{\lstinline!seq!} is odd, an error is raised.

See also: \passthrough{\lstinline!map!}.

\hypertarget{nth-procedure2}{%
\subsubsection{\texorpdfstring{\texttt{nth} :
procedure/2}{nth : procedure/2}}\label{nth-procedure2}}

Usage: \passthrough{\lstinline!(nth seq n) => any!}

Get the \passthrough{\lstinline!n-th!} element of sequence
\passthrough{\lstinline!seq!}. Sequences are 0-indexed.

See also:
\passthrough{\lstinline!nthdef, list, array, string, 1st, 2nd, 3rd, 4th, 5th, 6th, 7th, 8th, 9th, 10th!}.

\hypertarget{nthdef-procedure3}{%
\subsubsection{\texorpdfstring{\texttt{nthdef} :
procedure/3}{nthdef : procedure/3}}\label{nthdef-procedure3}}

Usage: \passthrough{\lstinline!(nthdef seq n default) => any!}

Return the \passthrough{\lstinline!n-th!} element of sequence
\passthrough{\lstinline!seq!} (0-indexed) if
\passthrough{\lstinline!seq!} is a sequence and has at least
\passthrough{\lstinline!n+1!} elements, default otherwise.

See also:
\passthrough{\lstinline!nth, seq?, 1st, 2nd, 3rd, 4th, 5th, 6th, 7th, 8th, 9th, 10th!}.

\hypertarget{remove-duplicates-procedure1}{%
\subsubsection{\texorpdfstring{\texttt{remove-duplicates} :
procedure/1}{remove-duplicates : procedure/1}}\label{remove-duplicates-procedure1}}

Usage: \passthrough{\lstinline!(remove-duplicates seq) => seq!}

Remove all duplicates in sequence \passthrough{\lstinline!seq!}, return
a new sequence with the duplicates removed.

See also: \passthrough{\lstinline!seq?, map, foreach, nth!}.

\hypertarget{reverse-procedure1}{%
\subsubsection{\texorpdfstring{\texttt{reverse} :
procedure/1}{reverse : procedure/1}}\label{reverse-procedure1}}

Usage: \passthrough{\lstinline!(reverse seq) => sequence!}

Reverse a sequence non-destructively, i.e., return a copy of the
reversed sequence.

See also:
\passthrough{\lstinline!nth, seq?, 1st, 2nd, 3rd, 4th, 6th, 7th, 8th, 9th, 10th, last!}.

\hypertarget{seq-procedure1}{%
\subsubsection{\texorpdfstring{\texttt{seq?} :
procedure/1}{seq? : procedure/1}}\label{seq-procedure1}}

Usage: \passthrough{\lstinline!(seq? seq) => bool!}

Return true if \passthrough{\lstinline!seq!} is a sequence, nil
otherwise.

See also: \passthrough{\lstinline!list, array, string, slice, nth!}.

\hypertarget{slice-procedure3}{%
\subsubsection{\texorpdfstring{\texttt{slice} :
procedure/3}{slice : procedure/3}}\label{slice-procedure3}}

Usage: \passthrough{\lstinline!(slice seq low high) => seq!}

Return the subsequence of \passthrough{\lstinline!seq!} starting from
\passthrough{\lstinline!low!} inclusive and ending at
\passthrough{\lstinline!high!} exclusive. Sequences are 0-indexed.

See also: \passthrough{\lstinline!list, array, string, nth, seq?!}.

\hypertarget{take-procedure3}{%
\subsubsection{\texorpdfstring{\texttt{take} :
procedure/3}{take : procedure/3}}\label{take-procedure3}}

Usage: \passthrough{\lstinline!(take seq n) => seq!}

Return the sequence consisting of the \passthrough{\lstinline!n!} first
elements of \passthrough{\lstinline!seq.!}

See also: \passthrough{\lstinline!list, array, string, nth, seq?!}.

\hypertarget{sound-support}{%
\subsection{Sound Support}\label{sound-support}}

Only a few functions are provided for sound support.

\hypertarget{beep-procedure1}{%
\subsubsection{\texorpdfstring{\texttt{beep} :
procedure/1}{beep : procedure/1}}\label{beep-procedure1}}

Usage: \passthrough{\lstinline!(beep sel)!}

Play a built-in system sound. The argument \passthrough{\lstinline!sel!}
may be one of '(error start ready click okay confirm info).

See also: \passthrough{\lstinline!play-sound, load-sound!}.

\hypertarget{set-volume-procedure1}{%
\subsubsection{\texorpdfstring{\texttt{set-volume} :
procedure/1}{set-volume : procedure/1}}\label{set-volume-procedure1}}

Usage: \passthrough{\lstinline!(set-volume fl)!}

Set the master volume for all sound to \passthrough{\lstinline!fl!}, a
value between 0.0 and 1.0.

See also: \passthrough{\lstinline!play-sound, play-music!}.

\hypertarget{string-manipulation}{%
\subsection{String Manipulation}\label{string-manipulation}}

These functions all manipulate strings in one way or another.

\hypertarget{fmt-procedure1-or-more}{%
\subsubsection{\texorpdfstring{\texttt{fmt} : procedure/1 or
more}{fmt : procedure/1 or more}}\label{fmt-procedure1-or-more}}

Usage: \passthrough{\lstinline!(fmt s [args] ...) => str!}

Format string \passthrough{\lstinline!s!} that contains format
directives with arbitrary many \passthrough{\lstinline!args!} as
arguments. The number of format directives must match the number of
arguments. The format directives are the same as those for the esoteric
and arcane programming language ``Go'', which was used on Earth for some
time.

See also: \passthrough{\lstinline!out!}.

\hypertarget{instr-procedure2}{%
\subsubsection{\texorpdfstring{\texttt{instr} :
procedure/2}{instr : procedure/2}}\label{instr-procedure2}}

Usage: \passthrough{\lstinline!(instr s1 s2) => int!}

Return the index of the first occurrence of \passthrough{\lstinline!s2!}
in \passthrough{\lstinline!s1!} (from left), or -1 if
\passthrough{\lstinline!s1!} does not contain
\passthrough{\lstinline!s2.!}

See also: \passthrough{\lstinline!str?, index!}.

\hypertarget{shorten-procedure2}{%
\subsubsection{\texorpdfstring{\texttt{shorten} :
procedure/2}{shorten : procedure/2}}\label{shorten-procedure2}}

Usage: \passthrough{\lstinline!(shorten s n) => str!}

Shorten string \passthrough{\lstinline!s!} to length
\passthrough{\lstinline!n!} in a smart way if possible, leave it
untouched if the length of \passthrough{\lstinline!s!} is smaller than
\passthrough{\lstinline!n.!}

See also: \passthrough{\lstinline!substr!}.

\hypertarget{spaces-procedure1}{%
\subsubsection{\texorpdfstring{\texttt{spaces} :
procedure/1}{spaces : procedure/1}}\label{spaces-procedure1}}

Usage: \passthrough{\lstinline!(spaces n) => str!}

Create a string consisting of \passthrough{\lstinline!n!} spaces.

See also: \passthrough{\lstinline!strbuild, strleft, strright!}.

\hypertarget{str-procedure0-or-more}{%
\subsubsection{\texorpdfstring{\texttt{str+} : procedure/0 or
more}{str+ : procedure/0 or more}}\label{str-procedure0-or-more}}

Usage: \passthrough{\lstinline!(str+ [s] ...) => str!}

Append all strings given to the function.

See also: \passthrough{\lstinline!str?!}.

\hypertarget{str-count-substr-procedure2}{%
\subsubsection{\texorpdfstring{\texttt{str-count-substr} :
procedure/2}{str-count-substr : procedure/2}}\label{str-count-substr-procedure2}}

Usage: \passthrough{\lstinline!(str-count-substr s1 s2) => int!}

Count the number of non-overlapping occurrences of substring
\passthrough{\lstinline!s2!} in string \passthrough{\lstinline!s1.!}

See also: \passthrough{\lstinline!str-replace, str-replace*, instr!}.

\hypertarget{str-empty-procedure1}{%
\subsubsection{\texorpdfstring{\texttt{str-empty?} :
procedure/1}{str-empty? : procedure/1}}\label{str-empty-procedure1}}

Usage: \passthrough{\lstinline!(str-empty? s) => bool!}

Return true if the string \passthrough{\lstinline!s!} is empty, nil
otherwise.

See also: \passthrough{\lstinline!strlen!}.

\hypertarget{str-exists-procedure2}{%
\subsubsection{\texorpdfstring{\texttt{str-exists?} :
procedure/2}{str-exists? : procedure/2}}\label{str-exists-procedure2}}

Usage: \passthrough{\lstinline!(str-exists? s pred) => bool!}

Return true if \passthrough{\lstinline!pred!} returns true for at least
one character in string \passthrough{\lstinline!s!}, nil otherwise.

See also:
\passthrough{\lstinline!exists?, forall?, list-exists?, array-exists?, seq?!}.

\hypertarget{str-forall-procedure2}{%
\subsubsection{\texorpdfstring{\texttt{str-forall?} :
procedure/2}{str-forall? : procedure/2}}\label{str-forall-procedure2}}

Usage: \passthrough{\lstinline!(str-forall? s pred) => bool!}

Return true if predicate \passthrough{\lstinline!pred!} returns true for
all characters in string \passthrough{\lstinline!s!}, nil otherwise.

See also:
\passthrough{\lstinline!foreach, map, forall?, array-forall?, list-forall, exists?!}.

\hypertarget{str-foreach-procedure2}{%
\subsubsection{\texorpdfstring{\texttt{str-foreach} :
procedure/2}{str-foreach : procedure/2}}\label{str-foreach-procedure2}}

Usage: \passthrough{\lstinline!(str-foreach s proc)!}

Apply \passthrough{\lstinline!proc!} to each element of string
\passthrough{\lstinline!s!} in order, for the side effects.

See also:
\passthrough{\lstinline!foreach, list-foreach, array-foreach, map!}.

\hypertarget{str-index-procedure2-or-more}{%
\subsubsection{\texorpdfstring{\texttt{str-index} : procedure/2 or
more}{str-index : procedure/2 or more}}\label{str-index-procedure2-or-more}}

Usage: \passthrough{\lstinline!(str-index s chars [pos]) => int!}

Find the first char in \passthrough{\lstinline!s!} that is in the
charset \passthrough{\lstinline!chars!}, starting from the optional
\passthrough{\lstinline!pos!} in \passthrough{\lstinline!s!}, and return
its index in the string. If no macthing char is found, nil is returned.

See also: \passthrough{\lstinline!strsplit, chars, inchars!}.

\hypertarget{str-join-procedure2}{%
\subsubsection{\texorpdfstring{\texttt{str-join} :
procedure/2}{str-join : procedure/2}}\label{str-join-procedure2}}

Usage: \passthrough{\lstinline!(str-join li del) => str!}

Join a list of strings \passthrough{\lstinline!li!} where each of the
strings is separated by string \passthrough{\lstinline!del!}, and return
the result string.

See also: \passthrough{\lstinline!strlen, strsplit, str-slice!}.

\hypertarget{str-ref-procedure2}{%
\subsubsection{\texorpdfstring{\texttt{str-ref} :
procedure/2}{str-ref : procedure/2}}\label{str-ref-procedure2}}

Usage: \passthrough{\lstinline!(str-ref s n) => n!}

Return the unicode char as integer at position
\passthrough{\lstinline!n!} in \passthrough{\lstinline!s!}. Strings are
0-indexed.

See also: \passthrough{\lstinline!nth!}.

\hypertarget{str-remove-number-procedure1}{%
\subsubsection{\texorpdfstring{\texttt{str-remove-number} :
procedure/1}{str-remove-number : procedure/1}}\label{str-remove-number-procedure1}}

Usage: \passthrough{\lstinline!(str-remove-number s [del]) => str!}

Remove the suffix number in \passthrough{\lstinline!s!}, provided there
is one and it is separated from the rest of the string by
\passthrough{\lstinline!del!}, where the default is a space character.
For instance, ``Test 29'' will be converted to ``Test'',
``User-Name1-23-99'' with delimiter ``-'' will be converted to
``User-Name1-23''. This function will remove intermediate delimiters in
the middle of the string, since it disassembles and reassembles the
string, so be aware that this is not preserving inputs in that respect.

See also: \passthrough{\lstinline!strsplit!}.

\hypertarget{str-remove-prefix-procedure1}{%
\subsubsection{\texorpdfstring{\texttt{str-remove-prefix} :
procedure/1}{str-remove-prefix : procedure/1}}\label{str-remove-prefix-procedure1}}

Usage: \passthrough{\lstinline!(str-remove-prefix s prefix) => str!}

Remove the prefix \passthrough{\lstinline!prefix!} from string
\passthrough{\lstinline!s!}, return the string without the prefix. If
the prefix does not match, \passthrough{\lstinline!s!} is returned. If
\passthrough{\lstinline!prefix!} is longer than
\passthrough{\lstinline!s!} and matches, the empty string is returned.

See also: \passthrough{\lstinline!str-remove-suffix!}.

\hypertarget{str-remove-suffix-procedure1}{%
\subsubsection{\texorpdfstring{\texttt{str-remove-suffix} :
procedure/1}{str-remove-suffix : procedure/1}}\label{str-remove-suffix-procedure1}}

Usage: \passthrough{\lstinline!(str-remove-suffix s suffix) => str!}

remove the suffix \passthrough{\lstinline!suffix!} from string
\passthrough{\lstinline!s!}, return the string without the suffix. If
the suffix does not match, \passthrough{\lstinline!s!} is returned. If
\passthrough{\lstinline!suffix!} is longer than
\passthrough{\lstinline!s!} and matches, the empty string is returned.

See also: \passthrough{\lstinline!str-remove-prefix!}.

\hypertarget{str-replace-procedure4}{%
\subsubsection{\texorpdfstring{\texttt{str-replace} :
procedure/4}{str-replace : procedure/4}}\label{str-replace-procedure4}}

Usage: \passthrough{\lstinline!(str-replace s t1 t2 n) => str!}

Replace the first \passthrough{\lstinline!n!} instances of substring
\passthrough{\lstinline!t1!} in \passthrough{\lstinline!s!} by
\passthrough{\lstinline!t2.!}

See also: \passthrough{\lstinline!str-replace*, str-count-substr!}.

\hypertarget{str-replace-procedure3}{%
\subsubsection{\texorpdfstring{\texttt{str-replace*} :
procedure/3}{str-replace* : procedure/3}}\label{str-replace-procedure3}}

Usage: \passthrough{\lstinline!(str-replace* s t1 t2) => str!}

Replace all non-overlapping substrings \passthrough{\lstinline!t1!} in
\passthrough{\lstinline!s!} by \passthrough{\lstinline!t2.!}

See also: \passthrough{\lstinline!str-replace, str-count-substr!}.

\hypertarget{str-reverse-procedure1}{%
\subsubsection{\texorpdfstring{\texttt{str-reverse} :
procedure/1}{str-reverse : procedure/1}}\label{str-reverse-procedure1}}

Usage: \passthrough{\lstinline!(str-reverse s) => str!}

Reverse string \passthrough{\lstinline!s.!}

See also:
\passthrough{\lstinline!reverse, array-reverse, list-reverse!}.

\hypertarget{str-segment-procedure3}{%
\subsubsection{\texorpdfstring{\texttt{str-segment} :
procedure/3}{str-segment : procedure/3}}\label{str-segment-procedure3}}

Usage: \passthrough{\lstinline!(str-segment str start end) => list!}

Parse a string \passthrough{\lstinline!str!} into words that start with
one of the characters in string \passthrough{\lstinline!start!} and end
in one of the characters in string \passthrough{\lstinline!end!} and
return a list consisting of lists of the form (bool s) where bool is
true if the string starts with a character in
\passthrough{\lstinline!start!}, nil otherwise, and
\passthrough{\lstinline!s!} is the extracted string including start and
end characters.

See also: \passthrough{\lstinline!str+, strsplit, fmt, strbuild!}.

\hypertarget{str-slice-procedure3}{%
\subsubsection{\texorpdfstring{\texttt{str-slice} :
procedure/3}{str-slice : procedure/3}}\label{str-slice-procedure3}}

Usage: \passthrough{\lstinline!(str-slice s low high) => s!}

Return a slice of string \passthrough{\lstinline!s!} starting at
character with index \passthrough{\lstinline!low!} (inclusive) and
ending at character with index \passthrough{\lstinline!high!}
(exclusive).

See also: \passthrough{\lstinline!slice!}.

\hypertarget{strbuild-procedure2}{%
\subsubsection{\texorpdfstring{\texttt{strbuild} :
procedure/2}{strbuild : procedure/2}}\label{strbuild-procedure2}}

Usage: \passthrough{\lstinline!(strbuild s n) => str!}

Build a string by repeating string \passthrough{\lstinline!s`` n!}
times.

See also: \passthrough{\lstinline!str+!}.

\hypertarget{strcase-procedure2}{%
\subsubsection{\texorpdfstring{\texttt{strcase} :
procedure/2}{strcase : procedure/2}}\label{strcase-procedure2}}

Usage: \passthrough{\lstinline!(strcase s sel) => str!}

Change the case of the string \passthrough{\lstinline!s!} according to
selector \passthrough{\lstinline!sel!} and return a copy. Valid values
for \passthrough{\lstinline!sel!} are 'lower for conversion to
lower-case, 'upper for uppercase, 'title for title case and 'utf-8 for
utf-8 normalization (which replaces unprintable characters with ``?'').

See also: \passthrough{\lstinline!strmap!}.

\hypertarget{strcenter-procedure2}{%
\subsubsection{\texorpdfstring{\texttt{strcenter} :
procedure/2}{strcenter : procedure/2}}\label{strcenter-procedure2}}

Usage: \passthrough{\lstinline!(strcenter s n) => str!}

Center string \passthrough{\lstinline!s!} by wrapping space characters
around it, such that the total length the result string is
\passthrough{\lstinline!n.!}

See also: \passthrough{\lstinline!strleft, strright, strlimit!}.

\hypertarget{strcnt-procedure2}{%
\subsubsection{\texorpdfstring{\texttt{strcnt} :
procedure/2}{strcnt : procedure/2}}\label{strcnt-procedure2}}

Usage: \passthrough{\lstinline!(strcnt s del) => int!}

Returnt the number of non-overlapping substrings
\passthrough{\lstinline!del!} in \passthrough{\lstinline!s.!}

See also: \passthrough{\lstinline!strsplit, str-index!}.

\hypertarget{strleft-procedure2}{%
\subsubsection{\texorpdfstring{\texttt{strleft} :
procedure/2}{strleft : procedure/2}}\label{strleft-procedure2}}

Usage: \passthrough{\lstinline!(strleft s n) => str!}

Align string \passthrough{\lstinline!s!} left by adding space characters
to the right of it, such that the total length the result string is
\passthrough{\lstinline!n.!}

See also: \passthrough{\lstinline!strcenter, strright, strlimit!}.

\hypertarget{strlen-procedure1}{%
\subsubsection{\texorpdfstring{\texttt{strlen} :
procedure/1}{strlen : procedure/1}}\label{strlen-procedure1}}

Usage: \passthrough{\lstinline!(strlen s) => int!}

Return the length of \passthrough{\lstinline!s.!}

See also: \passthrough{\lstinline!len, seq?, str?!}.

\hypertarget{strless-procedure2}{%
\subsubsection{\texorpdfstring{\texttt{strless} :
procedure/2}{strless : procedure/2}}\label{strless-procedure2}}

Usage: \passthrough{\lstinline!(strless s1 s2) => bool!}

Return true if string \passthrough{\lstinline!s1!} \textless{}
\passthrough{\lstinline!s2!} in lexicographic comparison, nil otherwise.

See also: \passthrough{\lstinline!sort, array-sort, strcase!}.

\hypertarget{strlimit-procedure2}{%
\subsubsection{\texorpdfstring{\texttt{strlimit} :
procedure/2}{strlimit : procedure/2}}\label{strlimit-procedure2}}

Usage: \passthrough{\lstinline!(strlimit s n) => str!}

Return a string based on \passthrough{\lstinline!s!} cropped to a
maximal length of \passthrough{\lstinline!n!} (or less if
\passthrough{\lstinline!s!} is shorter).

See also: \passthrough{\lstinline!strcenter, strleft, strright!}.

\hypertarget{strmap-procedure2}{%
\subsubsection{\texorpdfstring{\texttt{strmap} :
procedure/2}{strmap : procedure/2}}\label{strmap-procedure2}}

Usage: \passthrough{\lstinline!(strmap s proc) => str!}

Map function \passthrough{\lstinline!proc!}, which takes a number and
returns a number, over all unicode characters in
\passthrough{\lstinline!s!} and return the result as new string.

See also: \passthrough{\lstinline!map!}.

\hypertarget{stropen-procedure1}{%
\subsubsection{\texorpdfstring{\texttt{stropen} :
procedure/1}{stropen : procedure/1}}\label{stropen-procedure1}}

Usage: \passthrough{\lstinline!(stropen s) => streamport!}

Open the string \passthrough{\lstinline!s!} as input stream.

See also: \passthrough{\lstinline!open, close!}.

\hypertarget{strright-procedure2}{%
\subsubsection{\texorpdfstring{\texttt{strright} :
procedure/2}{strright : procedure/2}}\label{strright-procedure2}}

Usage: \passthrough{\lstinline!(strright s n) => str!}

Align string \passthrough{\lstinline!s!} right by adding space
characters in front of it, such that the total length the result string
is \passthrough{\lstinline!n.!}

See also: \passthrough{\lstinline!strcenter, strleft, strlimit!}.

\hypertarget{strsplit-procedure2}{%
\subsubsection{\texorpdfstring{\texttt{strsplit} :
procedure/2}{strsplit : procedure/2}}\label{strsplit-procedure2}}

Usage: \passthrough{\lstinline!(strsplit s del) => array!}

Return an array of strings obtained from \passthrough{\lstinline!s!} by
splitting \passthrough{\lstinline!s!} at each occurrence of string
\passthrough{\lstinline!del.!}

See also: \passthrough{\lstinline!str?!}.

\hypertarget{system-functions}{%
\subsection{System Functions}\label{system-functions}}

These functions concern the inner workings of the Lisp interpreter. Your
warranty might be void if you abuse them!

\hypertarget{error-handler-dict}{%
\subsubsection{\texorpdfstring{\emph{error-handler} :
dict}{error-handler : dict}}\label{error-handler-dict}}

Usage: \passthrough{\lstinline!(*error-handler* err)!}

The global error handler dict that contains procedures which take an
error and handle it. If an entry is nil, the default handler is used,
which outputs the error using \emph{error-printer}. The dict contains
handlers based on concurrent thread IDs and ought not be manipulated
directly.

See also: \passthrough{\lstinline!*error-printer*!}.

\hypertarget{error-printer-procedure1}{%
\subsubsection{\texorpdfstring{\texttt{*error-printer*} :
procedure/1}{*error-printer* : procedure/1}}\label{error-printer-procedure1}}

Usage: \passthrough{\lstinline!(*error-printer* err)!}

The global printer procedure which takes an error and prints it.

See also: \passthrough{\lstinline!error!}.

\hypertarget{last-error-sym}{%
\subsubsection{\texorpdfstring{\emph{last-error} :
sym}{last-error : sym}}\label{last-error-sym}}

Usage: \passthrough{\lstinline!*last-error* => str!}

Contains the last error that has occurred.

See also: \passthrough{\lstinline!*error-printer*, *error-handler*!}.

\textbf{Warning: This may only be used for debugging! Do \emph{not} use
this for error handling, it will surely fail!}

\hypertarget{reflect-symbol}{%
\subsubsection{\texorpdfstring{\emph{reflect} :
symbol}{reflect : symbol}}\label{reflect-symbol}}

Usage: \passthrough{\lstinline!*reflect* => li!}

The list of feature identifiers as symbols that this Lisp implementation
supports.

See also: \passthrough{\lstinline!feature?, on-feature!}.

\hypertarget{add-hook-procedure2}{%
\subsubsection{\texorpdfstring{\texttt{add-hook} :
procedure/2}{add-hook : procedure/2}}\label{add-hook-procedure2}}

Usage: \passthrough{\lstinline!(add-hook hook proc) => id!}

Add hook procedure \passthrough{\lstinline!proc!} which takes a list of
arguments as argument under symbolic or numeric
\passthrough{\lstinline!hook!} and return an integer hook
\passthrough{\lstinline!id!} for this hook. If
\passthrough{\lstinline!hook!} is not known, nil is returned.

See also:
\passthrough{\lstinline!remove-hook, remove-hooks, replace-hook!}.

\hypertarget{add-hook-internal-procedure2}{%
\subsubsection{\texorpdfstring{\texttt{add-hook-internal} :
procedure/2}{add-hook-internal : procedure/2}}\label{add-hook-internal-procedure2}}

Usage: \passthrough{\lstinline!(add-hook-internal hook proc) => int!}

Add a procedure \passthrough{\lstinline!proc!} to hook with numeric ID
\passthrough{\lstinline!hook!} and return this procedures hook ID. The
function does not check whether the hook exists.

See also: \passthrough{\lstinline!add-hook!}.

\textbf{Warning: Internal use only.}

\hypertarget{add-hook-once-procedure2}{%
\subsubsection{\texorpdfstring{\texttt{add-hook-once} :
procedure/2}{add-hook-once : procedure/2}}\label{add-hook-once-procedure2}}

Usage: \passthrough{\lstinline!(add-hook-once hook proc) => id!}

Add a hook procedure \passthrough{\lstinline!proc!} which takes a list
of arguments under symbolic or numeric \passthrough{\lstinline!hook!}
and return an integer hook \passthrough{\lstinline!id!}. If
\passthrough{\lstinline!hook!} is not known, nil is returned.

See also: \passthrough{\lstinline!add-hook, remove-hook, replace-hook!}.

\hypertarget{bind-procedure2}{%
\subsubsection{\texorpdfstring{\texttt{bind} :
procedure/2}{bind : procedure/2}}\label{bind-procedure2}}

Usage: \passthrough{\lstinline!(bind sym value)!}

Bind \passthrough{\lstinline!value!} to the global symbol
\passthrough{\lstinline!sym!}. In contrast to setq both values need
quoting.

See also: \passthrough{\lstinline!setq!}.

\hypertarget{bound-macro1}{%
\subsubsection{\texorpdfstring{\texttt{bound?} :
macro/1}{bound? : macro/1}}\label{bound-macro1}}

Usage: \passthrough{\lstinline!(bound? sym) => bool!}

Return true if a value is bound to the symbol
\passthrough{\lstinline!sym!}, nil otherwise.

See also: \passthrough{\lstinline!bind, setq!}.

\hypertarget{closure-procedure1}{%
\subsubsection{\texorpdfstring{\texttt{closure?} :
procedure/1}{closure? : procedure/1}}\label{closure-procedure1}}

Usage: \passthrough{\lstinline!(closure? x) => bool!}

Return true if \passthrough{\lstinline!x!} is a closure, nil otherwise.
Use \passthrough{\lstinline!function?!} for texting whether
\passthrough{\lstinline!x!} can be executed.

See also:
\passthrough{\lstinline!functional?, macro?, intrinsic?, functional-arity, functional-has-rest?!}.

\hypertarget{collect-garbage-procedure0-or-more}{%
\subsubsection{\texorpdfstring{\texttt{collect-garbage} : procedure/0 or
more}{collect-garbage : procedure/0 or more}}\label{collect-garbage-procedure0-or-more}}

Usage: \passthrough{\lstinline!(collect-garbage [sort])!}

Force a garbage-collection of the system's memory. If
\passthrough{\lstinline!sort!} is 'normal, then only a normal
incremental garbage colllection is performed. If
\passthrough{\lstinline!sort!} is 'total, then the garbage collection is
more thorough and the system attempts to return unused memory to the
host OS. Default is 'normal.

See also: \passthrough{\lstinline!memstats!}.

\textbf{Warning: There should rarely be a use for this. Try to use less
memory-consuming data structures instead.}

\hypertarget{current-error-handler-procedure0}{%
\subsubsection{\texorpdfstring{\texttt{current-error-handler} :
procedure/0}{current-error-handler : procedure/0}}\label{current-error-handler-procedure0}}

Usage: \passthrough{\lstinline!(current-error-handler) => proc!}

Return the current error handler, a default if there is none.

See also:
\passthrough{\lstinline!default-error-handler, push-error-handler, pop-error-handler, *current-error-handler*, *current-error-continuation*!}.

\hypertarget{def-custom-hook-procedure2}{%
\subsubsection{\texorpdfstring{\texttt{def-custom-hook} :
procedure/2}{def-custom-hook : procedure/2}}\label{def-custom-hook-procedure2}}

Usage: \passthrough{\lstinline!(def-custom-hook sym proc)!}

Define a custom hook point, to be called manually from Lisp. These have
IDs starting from 65636.

See also: \passthrough{\lstinline!add-hook!}.

\hypertarget{default-error-handler-procedure0}{%
\subsubsection{\texorpdfstring{\texttt{default-error-handler} :
procedure/0}{default-error-handler : procedure/0}}\label{default-error-handler-procedure0}}

Usage: \passthrough{\lstinline!(default-error-handler) => proc!}

Return the default error handler, irrespectively of the
current-error-handler.

See also:
\passthrough{\lstinline!current-error-handler, push-error-handler, pop-error-handler, *current-error-handler*, *current-error-continuation*!}.

\hypertarget{dict-protect-procedure1}{%
\subsubsection{\texorpdfstring{\texttt{dict-protect} :
procedure/1}{dict-protect : procedure/1}}\label{dict-protect-procedure1}}

Usage: \passthrough{\lstinline!(dict-protect d)!}

Protect dict \passthrough{\lstinline!d!} against changes. Attempting to
set values in a protected dict will cause an error, but all values can
be read and the dict can be copied. This function requires permission
'allow-protect.

See also:
\passthrough{\lstinline!dict-unprotect, dict-protected?, protect, unprotect, protected?, permissions, permission?!}.

\textbf{Warning: Protected dicts are full readable and can be copied, so
you may need to use protect to also prevent changes to the toplevel
symbol storing the dict!}

\hypertarget{dict-protected-procedure1}{%
\subsubsection{\texorpdfstring{\texttt{dict-protected?} :
procedure/1}{dict-protected? : procedure/1}}\label{dict-protected-procedure1}}

Usage: \passthrough{\lstinline!(dict-protected? d)!}

Return true if the dict \passthrough{\lstinline!d!} is protected against
mutation, nil otherwise.

See also:
\passthrough{\lstinline!dict-protect, dict-unprotect, protect, unprotect, protected?, permissions, permission?!}.

\hypertarget{dict-unprotect-procedure1}{%
\subsubsection{\texorpdfstring{\texttt{dict-unprotect} :
procedure/1}{dict-unprotect : procedure/1}}\label{dict-unprotect-procedure1}}

Usage: \passthrough{\lstinline!(dict-unprotect d)!}

Unprotect the dict \passthrough{\lstinline!d!} so it can be mutated
again. This function requires permission 'allow-unprotect.

See also:
\passthrough{\lstinline!dict-protect, dict-protected?, protect, unprotect, protected?, permissions, permission?!}.

\hypertarget{dump-procedure0-or-more}{%
\subsubsection{\texorpdfstring{\texttt{dump} : procedure/0 or
more}{dump : procedure/0 or more}}\label{dump-procedure0-or-more}}

Usage: \passthrough{\lstinline!(dump [sym] [all?]) => li!}

Return a list of symbols starting with the characters of
\passthrough{\lstinline!sym!} or starting with any characters if
\passthrough{\lstinline!sym!} is omitted, sorted alphabetically. When
\passthrough{\lstinline!all?!} is true, then all symbols are listed,
otherwise only symbols that do not contain "\_" are listed. By
convention, the underscore is used for auxiliary functions.

See also:
\passthrough{\lstinline!dump-bindings, save-zimage, load-zimage!}.

\hypertarget{dump-bindings-procedure0}{%
\subsubsection{\texorpdfstring{\texttt{dump-bindings} :
procedure/0}{dump-bindings : procedure/0}}\label{dump-bindings-procedure0}}

Usage: \passthrough{\lstinline!(dump-bindings) => li!}

Return a list of all top-level symbols with bound values, including
those intended for internal use.

See also: \passthrough{\lstinline!dump!}.

\hypertarget{error-procedure0-or-more}{%
\subsubsection{\texorpdfstring{\texttt{error} : procedure/0 or
more}{error : procedure/0 or more}}\label{error-procedure0-or-more}}

Usage: \passthrough{\lstinline!(error [msgstr] [expr] ...)!}

Raise an error, where \passthrough{\lstinline!msgstr!} and the optional
expressions \passthrough{\lstinline!expr!}\ldots{} work as in a call to
fmt.

See also: \passthrough{\lstinline!fmt, with-final!}.

\hypertarget{eval-procedure1}{%
\subsubsection{\texorpdfstring{\texttt{eval} :
procedure/1}{eval : procedure/1}}\label{eval-procedure1}}

Usage: \passthrough{\lstinline!(eval expr) => any!}

Evaluate the expression \passthrough{\lstinline!expr!} in the Z3S5
Machine Lisp interpreter and return the result. The evaluation
environment is the system's environment at the time of the call.

See also: \passthrough{\lstinline!break, apply!}.

\hypertarget{exit-procedure0-or-more}{%
\subsubsection{\texorpdfstring{\texttt{exit} : procedure/0 or
more}{exit : procedure/0 or more}}\label{exit-procedure0-or-more}}

Usage: \passthrough{\lstinline!(exit [n])!}

Immediately shut down the system and return OS host error code
\passthrough{\lstinline!n!}. The shutdown is performed gracefully and
exit hooks are executed.

See also: \passthrough{\lstinline!n/a!}.

\hypertarget{expand-macros-procedure1}{%
\subsubsection{\texorpdfstring{\texttt{expand-macros} :
procedure/1}{expand-macros : procedure/1}}\label{expand-macros-procedure1}}

Usage: \passthrough{\lstinline!(expand-macros expr) => expr!}

Expands the macros in \passthrough{\lstinline!expr!}. This is an
ordinary function and will not work on already compiled expressions such
as a function bound to a symbol. However, it can be used to expand
macros in expressions obtained by \passthrough{\lstinline!read.!}

See also:
\passthrough{\lstinline!internalize, externalize, load-library!}.

\hypertarget{expect-macro2}{%
\subsubsection{\texorpdfstring{\texttt{expect} :
macro/2}{expect : macro/2}}\label{expect-macro2}}

Usage: \passthrough{\lstinline!(expect value given)!}

Registers a test under the current test name that checks that
\passthrough{\lstinline!value!} is returned by
\passthrough{\lstinline!given!}. The test is only executed when
(run-selftest) is executed.

See also:
\passthrough{\lstinline!expect-err, expect-ok, run-selftest, testing!}.

\hypertarget{expect-err-macro1-or-more}{%
\subsubsection{\texorpdfstring{\texttt{expect-err} : macro/1 or
more}{expect-err : macro/1 or more}}\label{expect-err-macro1-or-more}}

Usage: \passthrough{\lstinline!(expect-err expr ...)!}

Registers a test under the current test name that checks that
\passthrough{\lstinline!expr!} produces an error.

See also:
\passthrough{\lstinline!expect, expect-ok, run-selftest, testing!}.

\hypertarget{expect-false-macro1-or-more}{%
\subsubsection{\texorpdfstring{\texttt{expect-false} : macro/1 or
more}{expect-false : macro/1 or more}}\label{expect-false-macro1-or-more}}

Usage: \passthrough{\lstinline!(expect-false expr ...)!}

Registers a test under the current test name that checks that
\passthrough{\lstinline!expr!} is nil.

See also:
\passthrough{\lstinline!expect, expect-ok, run-selftest, testing!}.

\hypertarget{expect-ok-macro1-or-more}{%
\subsubsection{\texorpdfstring{\texttt{expect-ok} : macro/1 or
more}{expect-ok : macro/1 or more}}\label{expect-ok-macro1-or-more}}

Usage: \passthrough{\lstinline!(expect-err expr ...)!}

Registers a test under the current test name that checks that
\passthrough{\lstinline!expr!} does not produce an error.

See also:
\passthrough{\lstinline!expect, expect-ok, run-selftest, testing!}.

\hypertarget{expect-true-macro1-or-more}{%
\subsubsection{\texorpdfstring{\texttt{expect-true} : macro/1 or
more}{expect-true : macro/1 or more}}\label{expect-true-macro1-or-more}}

Usage: \passthrough{\lstinline!(expect-true expr ...)!}

Registers a test under the current test name that checks that
\passthrough{\lstinline!expr!} is true (not nil).

See also:
\passthrough{\lstinline!expect, expect-ok, run-selftest, testing!}.

\hypertarget{feature-procedure1}{%
\subsubsection{\texorpdfstring{\texttt{feature?} :
procedure/1}{feature? : procedure/1}}\label{feature-procedure1}}

Usage: \passthrough{\lstinline!(feature? sym) => bool!}

Return true if the Lisp feature identified by symbol
\passthrough{\lstinline!sym!} is available, nil otherwise.

See also: \passthrough{\lstinline!*reflect*, on-feature!}.

\hypertarget{find-missing-help-entries-procedure0}{%
\subsubsection{\texorpdfstring{\texttt{find-missing-help-entries} :
procedure/0}{find-missing-help-entries : procedure/0}}\label{find-missing-help-entries-procedure0}}

Usage: \passthrough{\lstinline!(find-missing-help-entries) => li!}

Return a list of global symbols for which help entries are missing.

See also:
\passthrough{\lstinline!dump, dump-bindings, find-unneeded-help-entries!}.

\hypertarget{find-unneeded-help-entries-procedure0}{%
\subsubsection{\texorpdfstring{\texttt{find-unneeded-help-entries} :
procedure/0}{find-unneeded-help-entries : procedure/0}}\label{find-unneeded-help-entries-procedure0}}

Usage: \passthrough{\lstinline!(find-unneeded-help-entries) => li!}

Return a list of help entries for which no symbols are defined.

See also:
\passthrough{\lstinline!dump, dump-bindings, find-missing-help-entries!}.

\hypertarget{functional-arity-procedure1}{%
\subsubsection{\texorpdfstring{\texttt{functional-arity} :
procedure/1}{functional-arity : procedure/1}}\label{functional-arity-procedure1}}

Usage: \passthrough{\lstinline!(functional-arity proc) => int!}

Return the arity of a functional \passthrough{\lstinline!proc.!}

See also: \passthrough{\lstinline!functional?, functional-has-rest?!}.

\hypertarget{functional-has-rest-procedure1}{%
\subsubsection{\texorpdfstring{\texttt{functional-has-rest?} :
procedure/1}{functional-has-rest? : procedure/1}}\label{functional-has-rest-procedure1}}

Usage: \passthrough{\lstinline!(functional-has-rest? proc) => bool!}

Return true if the functional \passthrough{\lstinline!proc!} has a
\&rest argument, nil otherwise.

See also: \passthrough{\lstinline!functional?, functional-arity!}.

\hypertarget{functional-macro1}{%
\subsubsection{\texorpdfstring{\texttt{functional?} :
macro/1}{functional? : macro/1}}\label{functional-macro1}}

Usage: \passthrough{\lstinline!(functional? arg) => bool!}

Return true if \passthrough{\lstinline!arg!} is either a builtin
function, a closure, or a macro, nil otherwise. This is the right
predicate for testing whether the argument is applicable and has an
arity.

See also:
\passthrough{\lstinline!closure?, proc?, functional-arity, functional-has-rest?!}.

\hypertarget{gensym-procedure0}{%
\subsubsection{\texorpdfstring{\texttt{gensym} :
procedure/0}{gensym : procedure/0}}\label{gensym-procedure0}}

Usage: \passthrough{\lstinline!(gensym) => sym!}

Return a new symbol guaranteed to be unique during runtime.

See also: \passthrough{\lstinline!nonce!}.

\hypertarget{hook-procedure1}{%
\subsubsection{\texorpdfstring{\texttt{hook} :
procedure/1}{hook : procedure/1}}\label{hook-procedure1}}

Usage: \passthrough{\lstinline!(hook symbol)!}

Lookup the internal hook number from a symbolic name.

See also:
\passthrough{\lstinline!*hooks*, add-hook, remove-hook, remove-hooks!}.

\hypertarget{include-procedure1}{%
\subsubsection{\texorpdfstring{\texttt{include} :
procedure/1}{include : procedure/1}}\label{include-procedure1}}

Usage: \passthrough{\lstinline!(include fi) => any!}

Evaluate the lisp file \passthrough{\lstinline!fi!} one expression after
the other in the current environment.

See also: \passthrough{\lstinline!read, write, open, close!}.

\hypertarget{intern-procedure1}{%
\subsubsection{\texorpdfstring{\texttt{intern} :
procedure/1}{intern : procedure/1}}\label{intern-procedure1}}

Usage: \passthrough{\lstinline!(intern s) => sym!}

Create a new interned symbol based on string
\passthrough{\lstinline!s.!}

See also: \passthrough{\lstinline!gensym, str->sym, make-symbol!}.

\hypertarget{intrinsic-procedure1}{%
\subsubsection{\texorpdfstring{\texttt{intrinsic} :
procedure/1}{intrinsic : procedure/1}}\label{intrinsic-procedure1}}

Usage: \passthrough{\lstinline!(intrinsic sym) => any!}

Attempt to obtain the value that is intrinsically bound to
\passthrough{\lstinline!sym!}. Use this function to express the
intention to use the pre-defined builtin value of a symbol in the base
language.

See also: \passthrough{\lstinline!bind, unbind!}.

\textbf{Warning: This function currently only returns the binding but
this behavior might change in future.}

\hypertarget{intrinsic-procedure1-1}{%
\subsubsection{\texorpdfstring{\texttt{intrinsic?} :
procedure/1}{intrinsic? : procedure/1}}\label{intrinsic-procedure1-1}}

Usage: \passthrough{\lstinline!(intrinsic? x) => bool!}

Return true if \passthrough{\lstinline!x!} is an intrinsic built-in
function, nil otherwise. Notice that this function tests the value and
not that a symbol has been bound to the intrinsic.

See also: \passthrough{\lstinline!functional?, macro?, closure?!}.

\textbf{Warning: What counts as an intrinsic or not may change from
version to version. This is for internal use only.}

\hypertarget{macro-procedure1}{%
\subsubsection{\texorpdfstring{\texttt{macro?} :
procedure/1}{macro? : procedure/1}}\label{macro-procedure1}}

Usage: \passthrough{\lstinline!(macro? x) => bool!}

Return true if \passthrough{\lstinline!x!} is a macro, nil otherwise.

See also:
\passthrough{\lstinline!functional?, intrinsic?, closure?, functional-arity, functional-has-rest?!}.

\hypertarget{make-symbol-procedure1}{%
\subsubsection{\texorpdfstring{\texttt{make-symbol} :
procedure/1}{make-symbol : procedure/1}}\label{make-symbol-procedure1}}

Usage: \passthrough{\lstinline!(make-symbol s) => sym!}

Create a new symbol based on string \passthrough{\lstinline!s.!}

See also: \passthrough{\lstinline!str->sym!}.

\hypertarget{memstats-procedure0}{%
\subsubsection{\texorpdfstring{\texttt{memstats} :
procedure/0}{memstats : procedure/0}}\label{memstats-procedure0}}

Usage: \passthrough{\lstinline!(memstats) => dict!}

Return a dict with detailed memory statistics for the system.

See also: \passthrough{\lstinline!collect-garbage!}.

\hypertarget{nonce-procedure0}{%
\subsubsection{\texorpdfstring{\texttt{nonce} :
procedure/0}{nonce : procedure/0}}\label{nonce-procedure0}}

Usage: \passthrough{\lstinline!(nonce) => str!}

Return a unique random string. This is not cryptographically secure but
the string satisfies reasonable GUID requirements.

See also: \passthrough{\lstinline!externalize, internalize!}.

\hypertarget{on-feature-macro1-or-more}{%
\subsubsection{\texorpdfstring{\texttt{on-feature} : macro/1 or
more}{on-feature : macro/1 or more}}\label{on-feature-macro1-or-more}}

Usage: \passthrough{\lstinline!(on-feature sym body ...) => any!}

Evaluate the expressions of \passthrough{\lstinline!body!} if the Lisp
feature \passthrough{\lstinline!sym!} is supported by this
implementation, do nothing otherwise.

See also: \passthrough{\lstinline!feature?, *reflect*!}.

\hypertarget{permission-procedure1}{%
\subsubsection{\texorpdfstring{\texttt{permission?} :
procedure/1}{permission? : procedure/1}}\label{permission-procedure1}}

Usage: \passthrough{\lstinline!(permission? sym [default]) => bool!}

Return true if the permission for \passthrough{\lstinline!sym!} is set,
nil otherwise. If the permission flag is unknown, then
\passthrough{\lstinline!default!} is returned. The default for
\passthrough{\lstinline!default!} is nil.

See also:
\passthrough{\lstinline!permissions, set-permissions, when-permission, sys!}.

\hypertarget{permissions-procedure0}{%
\subsubsection{\texorpdfstring{\texttt{permissions} :
procedure/0}{permissions : procedure/0}}\label{permissions-procedure0}}

Usage: \passthrough{\lstinline!(permissions)!}

Return a list of all active permissions of the current interpreter.
Permissions are: \passthrough{\lstinline!load-prelude!} - load the init
file on start; \passthrough{\lstinline!load-user-init!} - load the local
user init on startup, file if present;
\passthrough{\lstinline!allow-unprotect!} - allow the user to unprotect
protected symbols (for redefining them);
\passthrough{\lstinline!allow-protect!} - allow the user to protect
symbols from redefinition or unbinding;
\passthrough{\lstinline!interactive!} - make the session interactive,
this is particularly used during startup to determine whether hooks are
installed and feedback is given. Permissions have to generally be set or
removed in careful combination with
\passthrough{\lstinline!revoke-permissions!}, which redefines symbols
and functions.

See also:
\passthrough{\lstinline!set-permissions, permission?, when-permission, sys!}.

\hypertarget{pop-error-handler-procedure0}{%
\subsubsection{\texorpdfstring{\texttt{pop-error-handler} :
procedure/0}{pop-error-handler : procedure/0}}\label{pop-error-handler-procedure0}}

Usage: \passthrough{\lstinline!(pop-error-handler) => proc!}

Remove the topmost error handler from the error handler stack and return
it. For internal use only.

See also: \passthrough{\lstinline!with-error-handler!}.

\hypertarget{pop-finalizer-procedure0}{%
\subsubsection{\texorpdfstring{\texttt{pop-finalizer} :
procedure/0}{pop-finalizer : procedure/0}}\label{pop-finalizer-procedure0}}

Usage: \passthrough{\lstinline!(pop-finalizer) => proc!}

Remove a finalizer from the finalizer stack and return it. For internal
use only.

See also: \passthrough{\lstinline!push-finalizer, with-final!}.

\hypertarget{proc-macro1}{%
\subsubsection{\texorpdfstring{\texttt{proc?} :
macro/1}{proc? : macro/1}}\label{proc-macro1}}

Usage: \passthrough{\lstinline!(proc? arg) => bool!}

Return true if \passthrough{\lstinline!arg!} is a procedure, nil
otherwise.

See also:
\passthrough{\lstinline!functional?, closure?, functional-arity, functional-has-rest?!}.

\hypertarget{protect-procedure0-or-more}{%
\subsubsection{\texorpdfstring{\texttt{protect} : procedure/0 or
more}{protect : procedure/0 or more}}\label{protect-procedure0-or-more}}

Usage: \passthrough{\lstinline!(protect [sym] ...)!}

Protect symbols \passthrough{\lstinline!sym!} \ldots{} against changes
or rebinding. The symbols need to be quoted. This operation requires the
permission 'allow-protect to be set.

See also:
\passthrough{\lstinline!protected?, unprotect, dict-protect, dict-unprotect, dict-protected?, permissions, permission?, setq, bind, interpret!}.

\hypertarget{protect-toplevel-symbols-procedure0}{%
\subsubsection{\texorpdfstring{\texttt{protect-toplevel-symbols} :
procedure/0}{protect-toplevel-symbols : procedure/0}}\label{protect-toplevel-symbols-procedure0}}

Usage: \passthrough{\lstinline!(protect-toplevel-symbols)!}

Protect all toplevel symbols that are not yet protected and aren't in
the \emph{mutable-toplevel-symbols} dict.

See also:
\passthrough{\lstinline!protected?, protect, unprotect, declare-unprotected, when-permission?, dict-protect, dict-protected?, dict-unprotect!}.

\hypertarget{protected-procedure1}{%
\subsubsection{\texorpdfstring{\texttt{protected?} :
procedure/1}{protected? : procedure/1}}\label{protected-procedure1}}

Usage: \passthrough{\lstinline!(protected? sym)!}

Return true if \passthrough{\lstinline!sym!} is protected, nil
otherwise.

See also:
\passthrough{\lstinline!protect, unprotect, dict-unprotect, dict-protected?, permission, permission?, setq, bind, interpret!}.

\hypertarget{push-error-handler-procedure1}{%
\subsubsection{\texorpdfstring{\texttt{push-error-handler} :
procedure/1}{push-error-handler : procedure/1}}\label{push-error-handler-procedure1}}

Usage: \passthrough{\lstinline!(push-error-handler proc)!}

Push an error handler \passthrough{\lstinline!proc!} on the error
handler stack. For internal use only.

See also: \passthrough{\lstinline!with-error-handler!}.

\hypertarget{push-finalizer-procedure1}{%
\subsubsection{\texorpdfstring{\texttt{push-finalizer} :
procedure/1}{push-finalizer : procedure/1}}\label{push-finalizer-procedure1}}

Usage: \passthrough{\lstinline!(push-finalizer proc)!}

Push a finalizer procedure \passthrough{\lstinline!proc!} on the
finalizer stack. For internal use only.

See also: \passthrough{\lstinline!with-final, pop-finalizer!}.

\hypertarget{read-eval-reply-procedure0}{%
\subsubsection{\texorpdfstring{\texttt{read-eval-reply} :
procedure/0}{read-eval-reply : procedure/0}}\label{read-eval-reply-procedure0}}

Usage: \passthrough{\lstinline!(read-eval-reply)!}

Start a new read-eval-reply loop.

See also: \passthrough{\lstinline!end-input, sys!}.

\textbf{Warning: Internal use only. This function might not do what you
expect it to do.}

\hypertarget{remove-hook-procedure2}{%
\subsubsection{\texorpdfstring{\texttt{remove-hook} :
procedure/2}{remove-hook : procedure/2}}\label{remove-hook-procedure2}}

Usage: \passthrough{\lstinline!(remove-hook hook id) => bool!}

Remove the symbolic or numberic \passthrough{\lstinline!hook!} with
\passthrough{\lstinline!id!} and return true if the hook was removed,
nil otherwise.

See also:
\passthrough{\lstinline!add-hook, remove-hooks, replace-hook!}.

\hypertarget{remove-hook-internal-procedure2}{%
\subsubsection{\texorpdfstring{\texttt{remove-hook-internal} :
procedure/2}{remove-hook-internal : procedure/2}}\label{remove-hook-internal-procedure2}}

Usage: \passthrough{\lstinline!(remove-hook-internal hook id)!}

Remove the hook with ID \passthrough{\lstinline!id!} from numeric
\passthrough{\lstinline!hook.!}

See also: \passthrough{\lstinline!remove-hook!}.

\textbf{Warning: Internal use only.}

\hypertarget{remove-hooks-procedure1}{%
\subsubsection{\texorpdfstring{\texttt{remove-hooks} :
procedure/1}{remove-hooks : procedure/1}}\label{remove-hooks-procedure1}}

Usage: \passthrough{\lstinline!(remove-hooks hook) => bool!}

Remove all hooks for symbolic or numeric \passthrough{\lstinline!hook!},
return true if the hook exists and the associated procedures were
removed, nil otherwise.

See also: \passthrough{\lstinline!add-hook, remove-hook, replace-hook!}.

\hypertarget{replace-hook-procedure2}{%
\subsubsection{\texorpdfstring{\texttt{replace-hook} :
procedure/2}{replace-hook : procedure/2}}\label{replace-hook-procedure2}}

Usage: \passthrough{\lstinline!(replace-hook hook proc)!}

Remove all hooks for symbolic or numeric \passthrough{\lstinline!hook!}
and install the given \passthrough{\lstinline!proc!} as the only hook
procedure.

See also: \passthrough{\lstinline!add-hook, remove-hook, remove-hooks!}.

\hypertarget{run-hook-procedure1}{%
\subsubsection{\texorpdfstring{\texttt{run-hook} :
procedure/1}{run-hook : procedure/1}}\label{run-hook-procedure1}}

Usage: \passthrough{\lstinline!(run-hook hook)!}

Manually run the hook, executing all procedures for the hook.

See also: \passthrough{\lstinline!add-hook, remove-hook!}.

\hypertarget{run-hook-internal-procedure1-or-more}{%
\subsubsection{\texorpdfstring{\texttt{run-hook-internal} : procedure/1
or
more}{run-hook-internal : procedure/1 or more}}\label{run-hook-internal-procedure1-or-more}}

Usage: \passthrough{\lstinline!(run-hook-internal hook [args] ...)!}

Run all hooks for numeric hook ID \passthrough{\lstinline!hook!} with
\passthrough{\lstinline!args!}\ldots{} as arguments.

See also: \passthrough{\lstinline!run-hook!}.

\textbf{Warning: Internal use only.}

\hypertarget{run-selftest-procedure1-or-more}{%
\subsubsection{\texorpdfstring{\texttt{run-selftest} : procedure/1 or
more}{run-selftest : procedure/1 or more}}\label{run-selftest-procedure1-or-more}}

Usage: \passthrough{\lstinline!(run-selftest [silent?]) => any!}

Run a diagnostic self-test of the Z3S5 Machine. If
\passthrough{\lstinline!silent?!} is true, then the self-test returns a
list containing a boolean for success, the number of tests performed,
the number of successes, the number of errors, and the number of
failures. If \passthrough{\lstinline!silent?!} is not provided or nil,
then the test progress and results are displayed. An error indicates a
problem with the testing, whereas a failure means that an expected value
was not returned.

See also: \passthrough{\lstinline!expect, testing!}.

\hypertarget{set-permissions-nil}{%
\subsubsection{set-permissions : nil}\label{set-permissions-nil}}

Usage: \passthrough{\lstinline!(set-permissions li)!}

Set the permissions for the current interpreter. This will trigger an
error when the permission cannot be set due to a security violation.
Generally, permissions can only be downgraded (made more stringent) and
never relaxed. See the information for
\passthrough{\lstinline!permissions!} for an overview of symbolic flags.

See also:
\passthrough{\lstinline!permissions, permission?, when-permission, sys!}.

\hypertarget{sleep-procedure1}{%
\subsubsection{\texorpdfstring{\texttt{sleep} :
procedure/1}{sleep : procedure/1}}\label{sleep-procedure1}}

Usage: \passthrough{\lstinline!(sleep ms)!}

Halt the current task execution for \passthrough{\lstinline!ms!}
milliseconds.

See also: \passthrough{\lstinline!sleep-ns, time, now, now-ns!}.

\hypertarget{sleep-ns-procedure1}{%
\subsubsection{\texorpdfstring{\texttt{sleep-ns} :
procedure/1}{sleep-ns : procedure/1}}\label{sleep-ns-procedure1}}

Usage: \passthrough{\lstinline!(sleep-ns n!}

Halt the current task execution for \passthrough{\lstinline!n!}
nanoseconds.

See also: \passthrough{\lstinline!sleep, time, now, now-ns!}.

\hypertarget{sys-key-procedure1}{%
\subsubsection{\texorpdfstring{\texttt{sys-key?} :
procedure/1}{sys-key? : procedure/1}}\label{sys-key-procedure1}}

Usage: \passthrough{\lstinline!(sys-key? key) => bool!}

Return true if the given sys key \passthrough{\lstinline!key!} exists,
nil otherwise.

See also: \passthrough{\lstinline!sys, setsys!}.

\hypertarget{sysmsg-procedure1}{%
\subsubsection{\texorpdfstring{\texttt{sysmsg} :
procedure/1}{sysmsg : procedure/1}}\label{sysmsg-procedure1}}

Usage: \passthrough{\lstinline!(sysmsg msg)!}

Asynchronously display a system message string
\passthrough{\lstinline!msg!} if in console or page mode, otherwise the
message is logged.

See also: \passthrough{\lstinline!sysmsg*, synout, synouty, out, outy!}.

\hypertarget{sysmsg-procedure1-1}{%
\subsubsection{\texorpdfstring{\texttt{sysmsg*} :
procedure/1}{sysmsg* : procedure/1}}\label{sysmsg-procedure1-1}}

Usage: \passthrough{\lstinline!(sysmsg* msg)!}

Display a system message string \passthrough{\lstinline!msg!} if in
console or page mode, otherwise the message is logged.

See also: \passthrough{\lstinline!sysmsg, synout, synouty, out, outy!}.

\hypertarget{testing-macro1}{%
\subsubsection{\texorpdfstring{\texttt{testing} :
macro/1}{testing : macro/1}}\label{testing-macro1}}

Usage: \passthrough{\lstinline!(testing name)!}

Registers the string \passthrough{\lstinline!name!} as the name of the
tests that are next registered with expect.

See also:
\passthrough{\lstinline!expect, expect-err, expect-ok, run-selftest!}.

\hypertarget{try-macro2-or-more}{%
\subsubsection{\texorpdfstring{\texttt{try} : macro/2 or
more}{try : macro/2 or more}}\label{try-macro2-or-more}}

Usage: \passthrough{\lstinline!(try (finals ...) body ...)!}

Evaluate the forms of the \passthrough{\lstinline!body!} and afterwards
the forms in \passthrough{\lstinline!finals!}. If during the execution
of \passthrough{\lstinline!body!} an error occurs, first all
\passthrough{\lstinline!finals!} are executed and then the error is
printed by the default error printer.

See also: \passthrough{\lstinline!with-final, with-error-handler!}.

\hypertarget{unprotect-procedure0-or-more}{%
\subsubsection{\texorpdfstring{\texttt{unprotect} : procedure/0 or
more}{unprotect : procedure/0 or more}}\label{unprotect-procedure0-or-more}}

Usage: \passthrough{\lstinline!(unprotect [sym] ...)!}

Unprotect symbols \passthrough{\lstinline!sym!} \ldots, allowing
mutation or rebinding them. The symbols need to be quoted. This
operation requires the permission 'allow-unprotect to be set, or else an
error is caused.

See also:
\passthrough{\lstinline!protect, protected?, dict-unprotect, dict-protected?, permissions, permission?, setq, bind, interpret!}.

\hypertarget{warn-procedure1-or-more}{%
\subsubsection{\texorpdfstring{\texttt{warn} : procedure/1 or
more}{warn : procedure/1 or more}}\label{warn-procedure1-or-more}}

Usage: \passthrough{\lstinline!(warn msg [args...])!}

Output the warning message \passthrough{\lstinline!msg!} in error
colors. The optional \passthrough{\lstinline!args!} are applied to the
message as in fmt. The message should not end with a newline.

See also: \passthrough{\lstinline!error!}.

\hypertarget{when-permission-macro1-or-more}{%
\subsubsection{\texorpdfstring{\texttt{when-permission} : macro/1 or
more}{when-permission : macro/1 or more}}\label{when-permission-macro1-or-more}}

Usage: \passthrough{\lstinline!(when-permission perm body ...) => any!}

Execute the expressions in \passthrough{\lstinline!body!} if and only if
the symbolic permission \passthrough{\lstinline!perm!} is available.

See also: \passthrough{\lstinline!permission?!}.

\hypertarget{with-colors-procedure3}{%
\subsubsection{\texorpdfstring{\texttt{with-colors} :
procedure/3}{with-colors : procedure/3}}\label{with-colors-procedure3}}

Usage: \passthrough{\lstinline!(with-colors textcolor backcolor proc)!}

Execute \passthrough{\lstinline!proc!} for display side effects, where
the default colors are set to \passthrough{\lstinline!textcolor!} and
\passthrough{\lstinline!backcolor!}. These are color specifications like
in the-color. After \passthrough{\lstinline!proc!} has finished or if an
error occurs, the default colors are restored to their original state.

See also:
\passthrough{\lstinline!the-color, color, set-color, with-final!}.

\hypertarget{with-error-handler-macro2-or-more}{%
\subsubsection{\texorpdfstring{\texttt{with-error-handler} : macro/2 or
more}{with-error-handler : macro/2 or more}}\label{with-error-handler-macro2-or-more}}

Usage: \passthrough{\lstinline!(with-error-handler handler body ...)!}

Evaluate the forms of the \passthrough{\lstinline!body!} with error
handler \passthrough{\lstinline!handler!} in place. The handler is a
procedure that takes the error as argument and handles it. If an error
occurs in \passthrough{\lstinline!handler!}, a default error handler is
used. Handlers are only active within the same thread.

See also: \passthrough{\lstinline!with-final!}.

\hypertarget{with-final-macro2-or-more}{%
\subsubsection{\texorpdfstring{\texttt{with-final} : macro/2 or
more}{with-final : macro/2 or more}}\label{with-final-macro2-or-more}}

Usage: \passthrough{\lstinline!(with-final finalizer body ...)!}

Evaluate the forms of the \passthrough{\lstinline!body!} with the given
finalizer as error handler. If an error occurs, then
\passthrough{\lstinline!finalizer!} is called with that error and nil.
If no error occurs, \passthrough{\lstinline!finalizer!} is called with
nil as first argument and the result of evaluating all forms of
\passthrough{\lstinline!body!} as second argument.

See also: \passthrough{\lstinline!with-error-handler!}.

\hypertarget{time-date}{%
\subsection{Time \& Date}\label{time-date}}

This section lists functions that are time and date-related. Most of
them use \passthrough{\lstinline!(now)!} and turn it into more
human-readable form.

\hypertarget{date-epoch-ns-procedure7}{%
\subsubsection{\texorpdfstring{\texttt{date-\textgreater{}epoch-ns} :
procedure/7}{date-\textgreater epoch-ns : procedure/7}}\label{date-epoch-ns-procedure7}}

Usage: \passthrough{\lstinline!(date->epoch-ns Y M D h m s ns) => int!}

Return the Unix epoch nanoseconds based on the given year
\passthrough{\lstinline!Y!}, month \passthrough{\lstinline!M!}, day
\passthrough{\lstinline!D!}, hour \passthrough{\lstinline!h!}, minute
\passthrough{\lstinline!m!}, seconds \passthrough{\lstinline!s!}, and
nanosecond fraction of a second \passthrough{\lstinline!ns!}, as it is
e.g.~returned in a (now) datelist.

See also:
\passthrough{\lstinline!epoch-ns->datelist, datestr->datelist, datestr, datestr*, day-of-week, week-of-date, now!}.

\hypertarget{datelist-epoch-ns-procedure1}{%
\subsubsection{\texorpdfstring{\texttt{datelist-\textgreater{}epoch-ns}
:
procedure/1}{datelist-\textgreater epoch-ns : procedure/1}}\label{datelist-epoch-ns-procedure1}}

Usage: \passthrough{\lstinline!(datelist->epoch-ns dateli) => int!}

Convert a datelist to Unix epoch nanoseconds. This function uses the
Unix nanoseconds from the 5th value of the second list in the datelist,
as it is provided by functions like (now). However, if the Unix
nanoseconds value is not specified in the list, it uses
\passthrough{\lstinline!date->epoch-ns!} to convert to Unix epoch
nanoseconds. Datelists can be incomplete. If the month is not specified,
January is assumed. If the day is not specified, the 1st is assumed. If
the hour is not specified, 12 is assumed, and corresponding defaults for
minutes, seconds, and nanoseconds are 0.

See also:
\passthrough{\lstinline!date->epoch-ns, datestr, datestr*, datestr->datelist, epoch-ns->datelist, now!}.

\hypertarget{datestr-procedure1}{%
\subsubsection{\texorpdfstring{\texttt{datestr} :
procedure/1}{datestr : procedure/1}}\label{datestr-procedure1}}

Usage: \passthrough{\lstinline!(datestr datelist) => str!}

Return datelist, as it is e.g.~returned by (now), as a string in format
YYYY-MM-DD HH:mm.

See also: \passthrough{\lstinline!now, datestr*, datestr->datelist!}.

\hypertarget{datestr-procedure1-1}{%
\subsubsection{\texorpdfstring{\texttt{datestr*} :
procedure/1}{datestr* : procedure/1}}\label{datestr-procedure1-1}}

Usage: \passthrough{\lstinline!(datestr* datelist) => str!}

Return the datelist, as it is e.g.~returned by (now), as a string in
format YYYY-MM-DD HH:mm:ss.nanoseconds.

See also: \passthrough{\lstinline!now, datestr, datestr->datelist!}.

\hypertarget{datestr-datelist-procedure1}{%
\subsubsection{\texorpdfstring{\texttt{datestr-\textgreater{}datelist} :
procedure/1}{datestr-\textgreater datelist : procedure/1}}\label{datestr-datelist-procedure1}}

Usage: \passthrough{\lstinline!(datestr->datelist s) => li!}

Convert a date string in the format of datestr and datestr* into a date
list as it is e.g.~returned by (now).

See also: \passthrough{\lstinline!datestr*, datestr, now!}.

\hypertarget{day-procedure2}{%
\subsubsection{\texorpdfstring{\texttt{day+} :
procedure/2}{day+ : procedure/2}}\label{day-procedure2}}

Usage: \passthrough{\lstinline!(day+ dateli n) => dateli!}

Adds \passthrough{\lstinline!n!} days to the given date
\passthrough{\lstinline!dateli!} in datelist format and returns the new
datelist.

See also:
\passthrough{\lstinline!sec+, minute+, hour+, week+, month+, year+, now!}.

\hypertarget{day-of-week-procedure3}{%
\subsubsection{\texorpdfstring{\texttt{day-of-week} :
procedure/3}{day-of-week : procedure/3}}\label{day-of-week-procedure3}}

Usage: \passthrough{\lstinline!(day-of-week Y M D) => int!}

Return the day of week based on the date with year
\passthrough{\lstinline!Y!}, month \passthrough{\lstinline!M!}, and day
\passthrough{\lstinline!D!}. The first day number 0 is Sunday, the last
day is Saturday with number 6.

See also:
\passthrough{\lstinline!week-of-date, datestr->datelist, date->epoch-ns, epoch-ns->datelist, datestr, datestr*, now!}.

\hypertarget{epoch-ns-datelist-procedure1}{%
\subsubsection{\texorpdfstring{\texttt{epoch-ns-\textgreater{}datelist}
:
procedure/1}{epoch-ns-\textgreater datelist : procedure/1}}\label{epoch-ns-datelist-procedure1}}

Usage: \passthrough{\lstinline!(epoch-ns->datelist ns) => li!}

Return the date list in UTC time corresponding to the Unix epoch
nanoseconds \passthrough{\lstinline!ns.!}

See also:
\passthrough{\lstinline!date->epoch-ns, datestr->datelist, datestr, datestr*, day-of-week, week-of-date, now!}.

\hypertarget{hour-procedure2}{%
\subsubsection{\texorpdfstring{\texttt{hour+} :
procedure/2}{hour+ : procedure/2}}\label{hour-procedure2}}

Usage: \passthrough{\lstinline!(hour+ dateli n) => dateli!}

Adds \passthrough{\lstinline!n!} hours to the given date
\passthrough{\lstinline!dateli!} in datelist format and returns the new
datelist.

See also:
\passthrough{\lstinline!sec+, minute+, day+, week+, month+, year+, now!}.

\hypertarget{minute-procedure2}{%
\subsubsection{\texorpdfstring{\texttt{minute+} :
procedure/2}{minute+ : procedure/2}}\label{minute-procedure2}}

Usage: \passthrough{\lstinline!(minute+ dateli n) => dateli!}

Adds \passthrough{\lstinline!n!} minutes to the given date
\passthrough{\lstinline!dateli!} in datelist format and returns the new
datelist.

See also:
\passthrough{\lstinline!sec+, hour+, day+, week+, month+, year+, now!}.

\hypertarget{month-procedure2}{%
\subsubsection{\texorpdfstring{\texttt{month+} :
procedure/2}{month+ : procedure/2}}\label{month-procedure2}}

Usage: \passthrough{\lstinline!(month+ dateli n) => dateli!}

Adds \passthrough{\lstinline!n!} months to the given date
\passthrough{\lstinline!dateli!} in datelist format and returns the new
datelist.

See also:
\passthrough{\lstinline!sec+, minute+, hour+, day+, week+, year+, now!}.

\hypertarget{now-procedure0}{%
\subsubsection{\texorpdfstring{\texttt{now} :
procedure/0}{now : procedure/0}}\label{now-procedure0}}

Usage: \passthrough{\lstinline!(now) => li!}

Return the current datetime in UTC format as a list of values in the
form '((year month day weekday iso-week) (hour minute second nanosecond
unix-nano-second)).

See also:
\passthrough{\lstinline!now-ns, datestr, time, date->epoch-ns, epoch-ns->datelist!}.

\hypertarget{now-ms-procedure0}{%
\subsubsection{\texorpdfstring{\texttt{now-ms} :
procedure/0}{now-ms : procedure/0}}\label{now-ms-procedure0}}

Usage: \passthrough{\lstinline!(now-ms) => num!}

Return the relative system time as a call to (now-ns) but in
milliseconds.

See also: \passthrough{\lstinline!now-ns, now!}.

\hypertarget{now-ns-procedure0}{%
\subsubsection{\texorpdfstring{\texttt{now-ns} :
procedure/0}{now-ns : procedure/0}}\label{now-ns-procedure0}}

Usage: \passthrough{\lstinline!(now-ns) => int!}

Return the current time in Unix nanoseconds.

See also: \passthrough{\lstinline!now, time!}.

\hypertarget{sec-procedure2}{%
\subsubsection{\texorpdfstring{\texttt{sec+} :
procedure/2}{sec+ : procedure/2}}\label{sec-procedure2}}

Usage: \passthrough{\lstinline!(sec+ dateli n) => dateli!}

Adds \passthrough{\lstinline!n!} seconds to the given date
\passthrough{\lstinline!dateli!} in datelist format and returns the new
datelist.

See also:
\passthrough{\lstinline!minute+, hour+, day+, week+, month+, year+, now!}.

\hypertarget{time-procedure1}{%
\subsubsection{\texorpdfstring{\texttt{time} :
procedure/1}{time : procedure/1}}\label{time-procedure1}}

Usage: \passthrough{\lstinline!(time proc) => int!}

Return the time in nanoseconds that it takes to execute the procedure
with no arguments \passthrough{\lstinline!proc.!}

See also: \passthrough{\lstinline!now-ns, now!}.

\hypertarget{week-procedure2}{%
\subsubsection{\texorpdfstring{\texttt{week+} :
procedure/2}{week+ : procedure/2}}\label{week-procedure2}}

Usage: \passthrough{\lstinline!(week+ dateli n) => dateli!}

Adds \passthrough{\lstinline!n!} weeks to the given date
\passthrough{\lstinline!dateli!} in datelist format and returns the new
datelist.

See also:
\passthrough{\lstinline!sec+, minute+, hour+, day+, month+, year+, now!}.

\hypertarget{week-of-date-procedure3}{%
\subsubsection{\texorpdfstring{\texttt{week-of-date} :
procedure/3}{week-of-date : procedure/3}}\label{week-of-date-procedure3}}

Usage: \passthrough{\lstinline!(week-of-date Y M D) => int!}

Return the week of the date in the year given by year
\passthrough{\lstinline!Y!}, month \passthrough{\lstinline!M!}, and day
\passthrough{\lstinline!D.!}

See also:
\passthrough{\lstinline!day-of-week, datestr->datelist, date->epoch-ns, epoch-ns->datelist, datestr, datestr*, now!}.

\hypertarget{year-procedure2}{%
\subsubsection{\texorpdfstring{\texttt{year+} :
procedure/2}{year+ : procedure/2}}\label{year-procedure2}}

Usage: \passthrough{\lstinline!(month+ dateli n) => dateli!}

Adds \passthrough{\lstinline!n!} years to the given date
\passthrough{\lstinline!dateli!} in datelist format and returns the new
datelist.

See also:
\passthrough{\lstinline!sec+, minute+, hour+, day+, week+, month+, now!}.

\hypertarget{user-interface}{%
\subsection{User Interface}\label{user-interface}}

This section lists miscellaneous user interface commands such as color
for terminals.

\hypertarget{colors-dict}{%
\subsubsection{\texorpdfstring{\emph{colors} :
dict}{colors : dict}}\label{colors-dict}}

Usage: \passthrough{\lstinline!*colors*!}

A global dict that maps default color names to color lists (r g b), (r g
b a) or selectors for (color selector). This can be used with procedure
the-color to translate symbolic names to colors.

See also: \passthrough{\lstinline!the-color!}.

\hypertarget{color-procedure1}{%
\subsubsection{\texorpdfstring{\texttt{color} :
procedure/1}{color : procedure/1}}\label{color-procedure1}}

Usage: \passthrough{\lstinline!(color sel) => (r g b a)!}

Return the color based on \passthrough{\lstinline!sel!}, which may be
'text for the text color, 'back for the background color, 'textarea for
the color of the text area, 'gfx for the current graphics foreground
color, and 'frame for the frame color.

See also: \passthrough{\lstinline!set-color, the-color, with-colors!}.

\hypertarget{darken-procedure1}{%
\subsubsection{\texorpdfstring{\texttt{darken} :
procedure/1}{darken : procedure/1}}\label{darken-procedure1}}

Usage: \passthrough{\lstinline!(darken color [amount]) => (r g b a)!}

Return a darker version of \passthrough{\lstinline!color!}. The optional
positive \passthrough{\lstinline!amount!} specifies the amount of
darkening (0-255).

See also: \passthrough{\lstinline!the-color, *colors*, lighten!}.

\hypertarget{lighten-procedure1}{%
\subsubsection{\texorpdfstring{\texttt{lighten} :
procedure/1}{lighten : procedure/1}}\label{lighten-procedure1}}

Usage: \passthrough{\lstinline!(lighten color [amount]) => (r g b a)!}

Return a lighter version of \passthrough{\lstinline!color!}. The
optional positive \passthrough{\lstinline!amount!} specifies the amount
of lightening (0-255).

See also: \passthrough{\lstinline!the-color, *colors*, darken!}.

\hypertarget{out-procedure1}{%
\subsubsection{\texorpdfstring{\texttt{out} :
procedure/1}{out : procedure/1}}\label{out-procedure1}}

Usage: \passthrough{\lstinline!(out expr)!}

Output \passthrough{\lstinline!expr!} on the console with current
default background and foreground color.

See also: \passthrough{\lstinline!outy, synout, synouty, output-at!}.

\hypertarget{outy-procedure1}{%
\subsubsection{\texorpdfstring{\texttt{outy} :
procedure/1}{outy : procedure/1}}\label{outy-procedure1}}

Usage: \passthrough{\lstinline!(outy spec)!}

Output styled text specified in \passthrough{\lstinline!spec!}. A
specification is a list of lists starting with 'fg for foreground, 'bg
for background, or 'text for unstyled text. If the list starts with 'fg
or 'bg then the next element must be a color suitable for (the-color
spec). Following may be a string to print or another color
specification. If a list starts with 'text then one or more strings may
follow.

See also:
\passthrough{\lstinline!*colors*, the-color, set-color, color, gfx.color, output-at, out!}.

\hypertarget{random-color-procedure0-or-more}{%
\subsubsection{\texorpdfstring{\texttt{random-color} : procedure/0 or
more}{random-color : procedure/0 or more}}\label{random-color-procedure0-or-more}}

Usage: \passthrough{\lstinline!(random-color [alpha])!}

Return a random color with optional \passthrough{\lstinline!alpha!}
value. If \passthrough{\lstinline!alpha!} is not specified, it is 255.

See also:
\passthrough{\lstinline!the-color, *colors*, darken, lighten!}.

\hypertarget{set-color-procedure1}{%
\subsubsection{\texorpdfstring{\texttt{set-color} :
procedure/1}{set-color : procedure/1}}\label{set-color-procedure1}}

Usage: \passthrough{\lstinline!(set-color sel colorlist)!}

Set the color according to \passthrough{\lstinline!sel!} to the color
\passthrough{\lstinline!colorlist!} of the form '(r g b a). See
\passthrough{\lstinline!color!} for information about
\passthrough{\lstinline!sel.!}

See also: \passthrough{\lstinline!color, the-color, with-colors!}.

\hypertarget{synout-procedure1}{%
\subsubsection{\texorpdfstring{\texttt{synout} :
procedure/1}{synout : procedure/1}}\label{synout-procedure1}}

Usage: \passthrough{\lstinline!(synout arg)!}

Like out, but enforcing a new input line afterwards. This needs to be
used when outputing concurrently in a future or task.

See also: \passthrough{\lstinline!out, outy, synouty!}.

\textbf{Warning: Concurrent display output can lead to unexpected visual
results and ought to be avoided.}

\hypertarget{the-color-procedure1}{%
\subsubsection{\texorpdfstring{\texttt{the-color} :
procedure/1}{the-color : procedure/1}}\label{the-color-procedure1}}

Usage: \passthrough{\lstinline!(the-color colors-spec) => (r g b a)!}

Return the color list (r g b a) based on a color specification, which
may be a color list (r g b), a color selector for (color selector) or a
color name such as 'dark-blue.

See also: \passthrough{\lstinline!*colors*, color, set-color, outy!}.

\hypertarget{the-color-names-procedure0}{%
\subsubsection{\texorpdfstring{\texttt{the-color-names} :
procedure/0}{the-color-names : procedure/0}}\label{the-color-names-procedure0}}

Usage: \passthrough{\lstinline!(the-color-names) => li!}

Return the list of color names in \emph{colors}.

See also: \passthrough{\lstinline!*colors*, the-color!}.

\hypertarget{complete-reference}{%
\section{Complete Reference}\label{complete-reference}}

\hypertarget{procedure2-7} :
procedure/2}{\% : procedure/2}}\label{procedure2-7}}

Usage: \passthrough{\lstinline!(\% x y) => num!}

Compute the remainder of dividing number \passthrough{\lstinline!x!} by
\passthrough{\lstinline!y.!}

See also: \passthrough{\lstinline!mod, /!}.

\hypertarget{procedure0-or-more-2}{%
\subsection{\texorpdfstring{\texttt{*} : procedure/0 or
more}{* : procedure/0 or more}}\label{procedure0-or-more-2}}

Usage: \passthrough{\lstinline!(* [args] ...) => num!}

Multiply all \passthrough{\lstinline!args!}. Special cases: (\emph{) is
1 and (} x) is x.

See also: \passthrough{\lstinline!+, -, /!}.

\hypertarget{colors-dict-1}{%
\subsection{\texorpdfstring{\emph{colors} :
dict}{colors : dict}}\label{colors-dict-1}}

Usage: \passthrough{\lstinline!*colors*!}

A global dict that maps default color names to color lists (r g b), (r g
b a) or selectors for (color selector). This can be used with procedure
the-color to translate symbolic names to colors.

See also: \passthrough{\lstinline!the-color!}.

\hypertarget{error-handler-dict-1}{%
\subsection{\texorpdfstring{\emph{error-handler} :
dict}{error-handler : dict}}\label{error-handler-dict-1}}

Usage: \passthrough{\lstinline!(*error-handler* err)!}

The global error handler dict that contains procedures which take an
error and handle it. If an entry is nil, the default handler is used,
which outputs the error using \emph{error-printer}. The dict contains
handlers based on concurrent thread IDs and ought not be manipulated
directly.

See also: \passthrough{\lstinline!*error-printer*!}.

\hypertarget{error-printer-procedure1-1}{%
\subsection{\texorpdfstring{\texttt{*error-printer*} :
procedure/1}{*error-printer* : procedure/1}}\label{error-printer-procedure1-1}}

Usage: \passthrough{\lstinline!(*error-printer* err)!}

The global printer procedure which takes an error and prints it.

See also: \passthrough{\lstinline!error!}.

\hypertarget{help-dict-1}{%
\subsection{\texorpdfstring{\emph{help} :
dict}{help : dict}}\label{help-dict-1}}

Usage: \passthrough{\lstinline!*help*!}

Dict containing all help information for symbols.

See also: \passthrough{\lstinline!help, defhelp, apropos!}.

\hypertarget{hooks-dict}{%
\subsection{\texorpdfstring{\emph{hooks} :
dict}{hooks : dict}}\label{hooks-dict}}

Usage: \passthrough{\lstinline!*hooks*!}

A dict containing translations from symbolic names to the internal
numeric representations of hooks.

See also:
\passthrough{\lstinline!hook, add-hook, remove-hook, remove-hooks!}.

\hypertarget{last-error-sym-1}{%
\subsection{\texorpdfstring{\emph{last-error} :
sym}{last-error : sym}}\label{last-error-sym-1}}

Usage: \passthrough{\lstinline!*last-error* => str!}

Contains the last error that has occurred.

See also: \passthrough{\lstinline!*error-printer*, *error-handler*!}.

\textbf{Warning: This may only be used for debugging! Do \emph{not} use
this for error handling, it will surely fail!}

\hypertarget{reflect-symbol-1}{%
\subsection{\texorpdfstring{\emph{reflect} :
symbol}{reflect : symbol}}\label{reflect-symbol-1}}

Usage: \passthrough{\lstinline!*reflect* => li!}

The list of feature identifiers as symbols that this Lisp implementation
supports.

See also: \passthrough{\lstinline!feature?, on-feature!}.

\hypertarget{procedure0-or-more-3}{%
\subsection{\texorpdfstring{\texttt{+} : procedure/0 or
more}{+ : procedure/0 or more}}\label{procedure0-or-more-3}}

Usage: \passthrough{\lstinline!(+ [args] ...) => num!}

Sum up all \passthrough{\lstinline!args!}. Special cases: (+) is 0 and
(+ x) is x.

See also: \passthrough{\lstinline!-, *, /!}.

\hypertarget{procedure1-or-more-2}{%
\subsection{\texorpdfstring{\texttt{-} : procedure/1 or
more}{- : procedure/1 or more}}\label{procedure1-or-more-2}}

Usage: \passthrough{\lstinline!(- x [y1] [y2] ...) => num!}

Subtract \passthrough{\lstinline!y1!}, \passthrough{\lstinline!y2!},
\ldots, from \passthrough{\lstinline!x!}. Special case: (- x) is -x.

See also: \passthrough{\lstinline!+, *, /!}.

\hypertarget{procedure1-or-more-3}{%
\subsection{\texorpdfstring{\texttt{/} : procedure/1 or
more}{/ : procedure/1 or more}}\label{procedure1-or-more-3}}

Usage: \passthrough{\lstinline!(/ x y1 [y2] ...) => float!}

Divide \passthrough{\lstinline!x!} by \passthrough{\lstinline!y1!}, then
by \passthrough{\lstinline!y2!}, and so forth. The result is a float.

See also: \passthrough{\lstinline!+, *, -!}.

\hypertarget{procedure2-8}{%
\subsection{\texorpdfstring{\texttt{/=} :
procedure/2}{/= : procedure/2}}\label{procedure2-8}}

Usage: \passthrough{\lstinline!(/= x y) => bool!}

Return true if number \passthrough{\lstinline!x!} is not equal to
\passthrough{\lstinline!y!}, nil otherwise.

See also: \passthrough{\lstinline!>, >=, <, <=!}.

\hypertarget{th-procedure1-or-more-7}{%
\subsection{\texorpdfstring{\texttt{10th} : procedure/1 or
more}{10th : procedure/1 or more}}\label{th-procedure1-or-more-7}}

Usage: \passthrough{\lstinline!(10th seq [default]) => any!}

Get the tenth element of a sequence or the optional
\passthrough{\lstinline!default!}. If there is no such element and no
default is provided, then an error is raised.

See also:
\passthrough{\lstinline!nth, nthdef, car, list-ref, array-ref, string-ref, 1st, 2nd, 3rd, 4th, 5th, 6th, 7th, 8th, 9th!}.

\hypertarget{st-procedure1-or-more-1}{%
\subsection{\texorpdfstring{\texttt{1st} : procedure/1 or
more}{1st : procedure/1 or more}}\label{st-procedure1-or-more-1}}

Usage: \passthrough{\lstinline!(1st seq [default]) => any!}

Get the first element of a sequence or the optional
\passthrough{\lstinline!default!}. If there is no such element and no
default is provided, then an error is raised.

See also:
\passthrough{\lstinline!nth, nthdef, car, list-ref, array-ref, string-ref, 2nd, 3rd, 4th, 5th, 6th, 7th, 8th, 9th, 10th!}.

\hypertarget{nd-procedure1-or-more-1}{%
\subsection{\texorpdfstring{\texttt{2nd} : procedure/1 or
more}{2nd : procedure/1 or more}}\label{nd-procedure1-or-more-1}}

Usage: \passthrough{\lstinline!(2nd seq [default]) => any!}

Get the second element of a sequence or the optional
\passthrough{\lstinline!default!}. If there is no such element and no
default is provided, then an error is raised.

See also:
\passthrough{\lstinline!nth, nthdef, car, list-ref, array-ref, string-ref, 1st, 3rd, 4th, 5th, 6th, 7th, 8th, 9th, 10th!}.

\hypertarget{rd-procedure1-or-more-1}{%
\subsection{\texorpdfstring{\texttt{3rd} : procedure/1 or
more}{3rd : procedure/1 or more}}\label{rd-procedure1-or-more-1}}

Usage: \passthrough{\lstinline!(3rd seq [default]) => any!}

Get the third element of a sequence or the optional
\passthrough{\lstinline!default!}. If there is no such element and no
default is provided, then an error is raised.

See also:
\passthrough{\lstinline!nth, nthdef, car, list-ref, array-ref, string-ref, 1st, 2nd, 4th, 5th, 6th, 7th, 8th, 9th, 10th!}.

\hypertarget{th-procedure1-or-more-8}{%
\subsection{\texorpdfstring{\texttt{4th} : procedure/1 or
more}{4th : procedure/1 or more}}\label{th-procedure1-or-more-8}}

Usage: \passthrough{\lstinline!(4th seq [default]) => any!}

Get the fourth element of a sequence or the optional
\passthrough{\lstinline!default!}. If there is no such element and no
default is provided, then an error is raised.

See also:
\passthrough{\lstinline!nth, nthdef, car, list-ref, array-ref, string-ref, 1st, 2nd, 3rd, 5th, 6th, 7th, 8th, 9th, 10th!}.

\hypertarget{th-procedure1-or-more-9}{%
\subsection{\texorpdfstring{\texttt{5th} : procedure/1 or
more}{5th : procedure/1 or more}}\label{th-procedure1-or-more-9}}

Usage: \passthrough{\lstinline!(5th seq [default]) => any!}

Get the fifth element of a sequence or the optional
\passthrough{\lstinline!default!}. If there is no such element and no
default is provided, then an error is raised.

See also:
\passthrough{\lstinline!nth, nthdef, car, list-ref, array-ref, string-ref, 1st, 2nd, 3rd, 4th, 6th, 7th, 8th, 9th, 10th!}.

\hypertarget{th-procedure1-or-more-10}{%
\subsection{\texorpdfstring{\texttt{6th} : procedure/1 or
more}{6th : procedure/1 or more}}\label{th-procedure1-or-more-10}}

Usage: \passthrough{\lstinline!(6th seq [default]) => any!}

Get the sixth element of a sequence or the optional
\passthrough{\lstinline!default!}. If there is no such element and no
default is provided, then an error is raised.

See also:
\passthrough{\lstinline!nth, nthdef, car, list-ref, array-ref, string-ref, 1st, 2nd, 3rd, 4th, 5th, 7th, 8th, 9th, 10th!}.

\hypertarget{th-procedure1-or-more-11}{%
\subsection{\texorpdfstring{\texttt{7th} : procedure/1 or
more}{7th : procedure/1 or more}}\label{th-procedure1-or-more-11}}

Usage: \passthrough{\lstinline!(7th seq [default]) => any!}

Get the seventh element of a sequence or the optional
\passthrough{\lstinline!default!}. If there is no such element and no
default is provided, then an error is raised.

See also:
\passthrough{\lstinline!nth, nthdef, car, list-ref, array-ref, string-ref, 1st, 2nd, 3rd, 4th, 5th, 6th, 8th, 9th, 10th!}.

\hypertarget{th-procedure1-or-more-12}{%
\subsection{\texorpdfstring{\texttt{8th} : procedure/1 or
more}{8th : procedure/1 or more}}\label{th-procedure1-or-more-12}}

Usage: \passthrough{\lstinline!(8th seq [default]) => any!}

Get the eighth element of a sequence or the optional
\passthrough{\lstinline!default!}. If there is no such element and no
default is provided, then an error is raised.

See also:
\passthrough{\lstinline!nth, nthdef, car, list-ref, array-ref, string-ref, 1st, 2nd, 3rd, 4th, 5th, 6th, 7th, 9th, 10th!}.

\hypertarget{th-procedure1-or-more-13}{%
\subsection{\texorpdfstring{\texttt{9th} : procedure/1 or
more}{9th : procedure/1 or more}}\label{th-procedure1-or-more-13}}

Usage: \passthrough{\lstinline!(9th seq [default]) => any!}

Get the nineth element of a sequence or the optional
\passthrough{\lstinline!default!}. If there is no such element and no
default is provided, then an error is raised.

See also:
\passthrough{\lstinline!nth, nthdef, car, list-ref, array-ref, string-ref, 1st, 2nd, 3rd, 4th, 5th, 6th, 7th, 8th, 10th!}.

\hypertarget{procedure2-9}{%
\subsection{\texorpdfstring{\texttt{\textless{}} :
procedure/2}{\textless{} : procedure/2}}\label{procedure2-9}}

Usage: \passthrough{\lstinline!(< x y) => bool!}

Return true if \passthrough{\lstinline!x!} is smaller than
\passthrough{\lstinline!y.!}

See also: \passthrough{\lstinline!<=, >=, >!}.

\hypertarget{procedure2-10}{%
\subsection{\texorpdfstring{\texttt{\textless{}=} :
procedure/2}{\textless= : procedure/2}}\label{procedure2-10}}

Usage: \passthrough{\lstinline!(<= x y) => bool!}

Return true if \passthrough{\lstinline!x!} is smaller than or equal to
\passthrough{\lstinline!y!}, nil otherwise.

See also: \passthrough{\lstinline!>, <, >=, /=!}.

\hypertarget{procedure2-11}{%
\subsection{\texorpdfstring{\texttt{=} :
procedure/2}{= : procedure/2}}\label{procedure2-11}}

Usage: \passthrough{\lstinline!(= x y) => bool!}

Return true if number \passthrough{\lstinline!x!} equals number
\passthrough{\lstinline!y!}, nil otherwise.

See also: \passthrough{\lstinline!eql?, equal?!}.

\hypertarget{procedure2-12}{%
\subsection{\texorpdfstring{\texttt{\textgreater{}} :
procedure/2}{\textgreater{} : procedure/2}}\label{procedure2-12}}

Usage: \passthrough{\lstinline!(> x y) => bool!}

Return true if \passthrough{\lstinline!x!} is larger than
\passthrough{\lstinline!y!}, nil otherwise.

See also: \passthrough{\lstinline!<, >=, <=, /=!}.

\hypertarget{procedure2-13}{%
\subsection{\texorpdfstring{\texttt{\textgreater{}=} :
procedure/2}{\textgreater= : procedure/2}}\label{procedure2-13}}

Usage: \passthrough{\lstinline!(>= x y) => bool!}

Return true if \passthrough{\lstinline!x!} is larger than or equal to
\passthrough{\lstinline!y!}, nil otherwise.

See also: \passthrough{\lstinline!>, <, <=, /=!}.

\hypertarget{abs-procedure1-1}{%
\subsection{\texorpdfstring{\texttt{abs} :
procedure/1}{abs : procedure/1}}\label{abs-procedure1-1}}

Usage: \passthrough{\lstinline!(abs x) => num!}

Returns the absolute value of number \passthrough{\lstinline!x.!}

See also: \passthrough{\lstinline!*, -, +, /!}.

\hypertarget{add-hook-procedure2-1}{%
\subsection{\texorpdfstring{\texttt{add-hook} :
procedure/2}{add-hook : procedure/2}}\label{add-hook-procedure2-1}}

Usage: \passthrough{\lstinline!(add-hook hook proc) => id!}

Add hook procedure \passthrough{\lstinline!proc!} which takes a list of
arguments as argument under symbolic or numeric
\passthrough{\lstinline!hook!} and return an integer hook
\passthrough{\lstinline!id!} for this hook. If
\passthrough{\lstinline!hook!} is not known, nil is returned.

See also:
\passthrough{\lstinline!remove-hook, remove-hooks, replace-hook!}.

\hypertarget{add-hook-internal-procedure2-1}{%
\subsection{\texorpdfstring{\texttt{add-hook-internal} :
procedure/2}{add-hook-internal : procedure/2}}\label{add-hook-internal-procedure2-1}}

Usage: \passthrough{\lstinline!(add-hook-internal hook proc) => int!}

Add a procedure \passthrough{\lstinline!proc!} to hook with numeric ID
\passthrough{\lstinline!hook!} and return this procedures hook ID. The
function does not check whether the hook exists.

See also: \passthrough{\lstinline!add-hook!}.

\textbf{Warning: Internal use only.}

\hypertarget{add-hook-once-procedure2-1}{%
\subsection{\texorpdfstring{\texttt{add-hook-once} :
procedure/2}{add-hook-once : procedure/2}}\label{add-hook-once-procedure2-1}}

Usage: \passthrough{\lstinline!(add-hook-once hook proc) => id!}

Add a hook procedure \passthrough{\lstinline!proc!} which takes a list
of arguments under symbolic or numeric \passthrough{\lstinline!hook!}
and return an integer hook \passthrough{\lstinline!id!}. If
\passthrough{\lstinline!hook!} is not known, nil is returned.

See also: \passthrough{\lstinline!add-hook, remove-hook, replace-hook!}.

\hypertarget{add1-procedure1-1}{%
\subsection{\texorpdfstring{\texttt{add1} :
procedure/1}{add1 : procedure/1}}\label{add1-procedure1-1}}

Usage: \passthrough{\lstinline!(add1 n) => num!}

Add 1 to number \passthrough{\lstinline!n.!}

See also: \passthrough{\lstinline!sub1, +, -!}.

\hypertarget{alist-dict-procedure1-1}{%
\subsection{\texorpdfstring{\texttt{alist-\textgreater{}dict} :
procedure/1}{alist-\textgreater dict : procedure/1}}\label{alist-dict-procedure1-1}}

Usage: \passthrough{\lstinline!(alist->dict li) => dict!}

Convert an association list \passthrough{\lstinline!li!} into a
dictionary. Note that the value will be the cdr of each list element,
not the second element, so you need to use an alist with proper pairs
'(a . b) if you want b to be a single value.

See also:
\passthrough{\lstinline!dict->alist, dict, dict->list, list->dict!}.

\hypertarget{alist-procedure1-1}{%
\subsection{\texorpdfstring{\texttt{alist?} :
procedure/1}{alist? : procedure/1}}\label{alist-procedure1-1}}

Usage: \passthrough{\lstinline!(alist? li) => bool!}

Return true if \passthrough{\lstinline!li!} is an association list, nil
otherwise. This also works for a-lists where each element is a pair
rather than a full list.

See also: \passthrough{\lstinline!assoc!}.

\hypertarget{and-macro0-or-more-1}{%
\subsection{\texorpdfstring{\texttt{and} : macro/0 or
more}{and : macro/0 or more}}\label{and-macro0-or-more-1}}

Usage: \passthrough{\lstinline!(and expr1 expr2 ...) => any!}

Evaluate \passthrough{\lstinline!expr1!} and if it is not nil, then
evaluate \passthrough{\lstinline!expr2!} and if it is not nil, evaluate
the next expression, until all expressions have been evaluated. This is
a shortcut logical and.

See also: \passthrough{\lstinline!or!}.

\hypertarget{append-procedure1-or-more-1}{%
\subsection{\texorpdfstring{\texttt{append} : procedure/1 or
more}{append : procedure/1 or more}}\label{append-procedure1-or-more-1}}

Usage: \passthrough{\lstinline!(append li1 li2 ...) => li!}

Concatenate the lists given as arguments.

See also: \passthrough{\lstinline!cons!}.

\hypertarget{apply-procedure2-1}{%
\subsection{\texorpdfstring{\texttt{apply} :
procedure/2}{apply : procedure/2}}\label{apply-procedure2-1}}

Usage: \passthrough{\lstinline!(apply proc arg) => any!}

Apply function \passthrough{\lstinline!proc!} to argument list
\passthrough{\lstinline!arg.!}

See also: \passthrough{\lstinline!functional?!}.

\hypertarget{apropos-procedure1-1}{%
\subsection{\texorpdfstring{\texttt{apropos} :
procedure/1}{apropos : procedure/1}}\label{apropos-procedure1-1}}

Usage: \passthrough{\lstinline!(apropos sym) => \#li!}

Get a list of procedures and symbols related to
\passthrough{\lstinline!sym!} from the help system.

See also: \passthrough{\lstinline!defhelp, help-entry, help, *help*!}.

\hypertarget{array-procedure0-or-more-1}{%
\subsection{\texorpdfstring{\texttt{array} : procedure/0 or
more}{array : procedure/0 or more}}\label{array-procedure0-or-more-1}}

Usage: \passthrough{\lstinline!(array [arg1] ...) => array!}

Create an array containing the arguments given to it.

See also: \passthrough{\lstinline!array?, build-array!}.

\hypertarget{array-list-procedure1-1}{%
\subsection{\texorpdfstring{\texttt{array-\textgreater{}list} :
procedure/1}{array-\textgreater list : procedure/1}}\label{array-list-procedure1-1}}

Usage: \passthrough{\lstinline!(array->list arr) => li!}

Convert array \passthrough{\lstinline!arr!} into a list.

See also: \passthrough{\lstinline!list->array, array!}.

\hypertarget{array-str-procedure1-1}{%
\subsection{\texorpdfstring{\texttt{array-\textgreater{}str} :
procedure/1}{array-\textgreater str : procedure/1}}\label{array-str-procedure1-1}}

Usage: \passthrough{\lstinline!(array-str arr) => s!}

Convert an array of unicode glyphs as integer values into a string. If
the given sequence is not a valid UTF-8 sequence, an error is thrown.

See also: \passthrough{\lstinline!str->array!}.

\hypertarget{array-copy-procedure1-1}{%
\subsection{\texorpdfstring{\texttt{array-copy} :
procedure/1}{array-copy : procedure/1}}\label{array-copy-procedure1-1}}

Usage: \passthrough{\lstinline!(array-copy arr) => array!}

Return a copy of \passthrough{\lstinline!arr.!}

See also:
\passthrough{\lstinline"array, array?, array-map!, array-pmap!"}.

\hypertarget{array-exists-procedure2-1}{%
\subsection{\texorpdfstring{\texttt{array-exists?} :
procedure/2}{array-exists? : procedure/2}}\label{array-exists-procedure2-1}}

Usage: \passthrough{\lstinline!(array-exists? arr pred) => bool!}

Return true if \passthrough{\lstinline!pred!} returns true for at least
one element in array \passthrough{\lstinline!arr!}, nil otherwise.

See also:
\passthrough{\lstinline!exists?, forall?, list-exists?, str-exists?, seq?!}.

\hypertarget{array-forall-procedure2-1}{%
\subsection{\texorpdfstring{\texttt{array-forall?} :
procedure/2}{array-forall? : procedure/2}}\label{array-forall-procedure2-1}}

Usage: \passthrough{\lstinline!(array-forall? arr pred) => bool!}

Return true if predicate \passthrough{\lstinline!pred!} returns true for
all elements of array \passthrough{\lstinline!arr!}, nil otherwise.

See also:
\passthrough{\lstinline!foreach, map, forall?, str-forall?, list-forall?, exists?!}.

\hypertarget{array-foreach-procedure2-1}{%
\subsection{\texorpdfstring{\texttt{array-foreach} :
procedure/2}{array-foreach : procedure/2}}\label{array-foreach-procedure2-1}}

Usage: \passthrough{\lstinline!(array-foreach arr proc)!}

Apply \passthrough{\lstinline!proc!} to each element of array
\passthrough{\lstinline!arr!} in order, for the side effects.

See also: \passthrough{\lstinline!foreach, list-foreach, map!}.

\hypertarget{array-len-procedure1-1}{%
\subsection{\texorpdfstring{\texttt{array-len} :
procedure/1}{array-len : procedure/1}}\label{array-len-procedure1-1}}

Usage: \passthrough{\lstinline!(array-len arr) => int!}

Return the length of array \passthrough{\lstinline!arr.!}

See also: \passthrough{\lstinline!len!}.

\hypertarget{array-map-procedure2-1}{%
\subsection{\texorpdfstring{\texttt{array-map!} :
procedure/2}{array-map! : procedure/2}}\label{array-map-procedure2-1}}

Usage: \passthrough{\lstinline"(array-map! arr proc)"}

Traverse array \passthrough{\lstinline!arr!} in unspecified order and
apply \passthrough{\lstinline!proc!} to each element. This mutates the
array.

See also:
\passthrough{\lstinline"array-walk, array-pmap!, array?, map, seq?"}.

\hypertarget{array-pmap-procedure2-1}{%
\subsection{\texorpdfstring{\texttt{array-pmap!} :
procedure/2}{array-pmap! : procedure/2}}\label{array-pmap-procedure2-1}}

Usage: \passthrough{\lstinline"(array-pmap! arr proc)"}

Apply \passthrough{\lstinline!proc!} in unspecified order in parallel to
array \passthrough{\lstinline!arr!}, mutating the array to contain the
value returned by \passthrough{\lstinline!proc!} each time. Because of
the calling overhead for parallel execution, for many workloads
array-map! might be faster if \passthrough{\lstinline!proc!} is very
fast. If \passthrough{\lstinline!proc!} is slow, then array-pmap! may be
much faster for large arrays on machines with many cores.

See also:
\passthrough{\lstinline"array-map!, array-walk, array?, map, seq?"}.

\hypertarget{array-ref-procedure1-1}{%
\subsection{\texorpdfstring{\texttt{array-ref} :
procedure/1}{array-ref : procedure/1}}\label{array-ref-procedure1-1}}

Usage: \passthrough{\lstinline!(array-ref arr n) => any!}

Return the element of \passthrough{\lstinline!arr!} at index
\passthrough{\lstinline!n!}. Arrays are 0-indexed.

See also: \passthrough{\lstinline!array?, array, nth, seq?!}.

\hypertarget{array-reverse-procedure1-1}{%
\subsection{\texorpdfstring{\texttt{array-reverse} :
procedure/1}{array-reverse : procedure/1}}\label{array-reverse-procedure1-1}}

Usage: \passthrough{\lstinline!(array-reverse arr) => array!}

Create a copy of \passthrough{\lstinline!arr!} that reverses the order
of all of its elements.

See also: \passthrough{\lstinline!reverse, list-reverse, str-reverse!}.

\hypertarget{array-set-procedure3-1}{%
\subsection{\texorpdfstring{\texttt{array-set} :
procedure/3}{array-set : procedure/3}}\label{array-set-procedure3-1}}

Usage: \passthrough{\lstinline!(array-set arr idx value)!}

Set the value at index \passthrough{\lstinline!idx!} in
\passthrough{\lstinline!arr!} to \passthrough{\lstinline!value!}. Arrays
are 0-indexed. This mutates the array.

See also: \passthrough{\lstinline!array?, array!}.

\hypertarget{array-slice-procedure3-1}{%
\subsection{\texorpdfstring{\texttt{array-slice} :
procedure/3}{array-slice : procedure/3}}\label{array-slice-procedure3-1}}

Usage: \passthrough{\lstinline!(array-slice arr low high) => array!}

Slice the array \passthrough{\lstinline!arr!} starting from
\passthrough{\lstinline!low!} (inclusive) and ending at
\passthrough{\lstinline!high!} (exclusive) and return the slice.

See also: \passthrough{\lstinline!array-ref, array-len!}.

\hypertarget{array-sort-procedure2-1}{%
\subsection{\texorpdfstring{\texttt{array-sort} :
procedure/2}{array-sort : procedure/2}}\label{array-sort-procedure2-1}}

Usage: \passthrough{\lstinline!(array-sort arr proc) => arr!}

Destructively sorts array \passthrough{\lstinline!arr!} by using
comparison proc \passthrough{\lstinline!proc!}, which takes two
arguments and returns true if the first argument is smaller than the
second argument, nil otherwise. The array is returned but it is not
copied and modified in place by this procedure. The sorting algorithm is
not guaranteed to be stable.

See also: \passthrough{\lstinline!sort!}.

\hypertarget{array-walk-procedure2-1}{%
\subsection{\texorpdfstring{\texttt{array-walk} :
procedure/2}{array-walk : procedure/2}}\label{array-walk-procedure2-1}}

Usage: \passthrough{\lstinline!(array-walk arr proc)!}

Traverse the array \passthrough{\lstinline!arr!} from first to last
element and apply \passthrough{\lstinline!proc!} to each element for
side-effects. Function \passthrough{\lstinline!proc!} takes the index
and the array element at that index as argument. If
\passthrough{\lstinline!proc!} returns nil, then the traversal stops and
the index is returned. If \passthrough{\lstinline!proc!} returns
non-nil, traversal continues. If \passthrough{\lstinline!proc!} never
returns nil, then the index returned is -1. This function does not
mutate the array.

See also:
\passthrough{\lstinline"array-map!, array-pmap!, array?, map, seq?"}.

\hypertarget{array-procedure1-1}{%
\subsection{\texorpdfstring{\texttt{array?} :
procedure/1}{array? : procedure/1}}\label{array-procedure1-1}}

Usage: \passthrough{\lstinline!(array? obj) => bool!}

Return true of \passthrough{\lstinline!obj!} is an array, nil otherwise.

See also: \passthrough{\lstinline!seq?, array!}.

\hypertarget{ascii85-blob-procedure1-1}{%
\subsection{\texorpdfstring{\texttt{ascii85-\textgreater{}blob} :
procedure/1}{ascii85-\textgreater blob : procedure/1}}\label{ascii85-blob-procedure1-1}}

Usage: \passthrough{\lstinline!(ascii85->blob str) => blob!}

Convert the ascii85 encoded string \passthrough{\lstinline!str!} to a
binary blob. This will raise an error if \passthrough{\lstinline!str!}
is not a valid ascii85 encoded string.

See also:
\passthrough{\lstinline!blob->ascii85, base64->blob, str->blob, hex->blob!}.

\hypertarget{assoc-procedure2-1}{%
\subsection{\texorpdfstring{\texttt{assoc} :
procedure/2}{assoc : procedure/2}}\label{assoc-procedure2-1}}

Usage: \passthrough{\lstinline!(assoc key alist) => li!}

Return the sublist of \passthrough{\lstinline!alist!} that starts with
\passthrough{\lstinline!key!} if there is any, nil otherwise. Testing is
done with equal?. An association list may be of the form ((key1
value1)(key2 value2)\ldots) or ((key1 . value1) (key2 . value2) \ldots)

See also: \passthrough{\lstinline!assoc, assoc1, alist?, eq?, equal?!}.

\hypertarget{assoc1-procedure2-1}{%
\subsection{\texorpdfstring{\texttt{assoc1} :
procedure/2}{assoc1 : procedure/2}}\label{assoc1-procedure2-1}}

Usage: \passthrough{\lstinline!(assoc1 sym li) => any!}

Get the second element in the first sublist in
\passthrough{\lstinline!li!} that starts with
\passthrough{\lstinline!sym!}. This is equivalent to (cadr (assoc sym
li)).

See also: \passthrough{\lstinline!assoc, alist?!}.

\hypertarget{assq-procedure2-1}{%
\subsection{\texorpdfstring{\texttt{assq} :
procedure/2}{assq : procedure/2}}\label{assq-procedure2-1}}

Usage: \passthrough{\lstinline!(assq key alist) => li!}

Return the sublist of \passthrough{\lstinline!alist!} that starts with
\passthrough{\lstinline!key!} if there is any, nil otherwise. Testing is
done with eq?. An association list may be of the form ((key1
value1)(key2 value2)\ldots) or ((key1 . value1) (key2 . value2) \ldots)

See also: \passthrough{\lstinline!assoc, assoc1, eq?, alist?, equal?!}.

\hypertarget{atom-procedure1-1}{%
\subsection{\texorpdfstring{\texttt{atom?} :
procedure/1}{atom? : procedure/1}}\label{atom-procedure1-1}}

Usage: \passthrough{\lstinline!(atom? x) => bool!}

Return true if \passthrough{\lstinline!x!} is an atomic value, nil
otherwise. Atomic values are numbers and symbols.

See also: \passthrough{\lstinline!sym?!}.

\hypertarget{base64-blob-procedure1-1}{%
\subsection{\texorpdfstring{\texttt{base64-\textgreater{}blob} :
procedure/1}{base64-\textgreater blob : procedure/1}}\label{base64-blob-procedure1-1}}

Usage: \passthrough{\lstinline!(base64->blob str) => blob!}

Convert the base64 encoded string \passthrough{\lstinline!str!} to a
binary blob. This will raise an error if \passthrough{\lstinline!str!}
is not a valid base64 encoded string.

See also:
\passthrough{\lstinline!blob->base64, hex->blob, ascii85->blob, str->blob!}.

\hypertarget{beep-procedure1-1}{%
\subsection{\texorpdfstring{\texttt{beep} :
procedure/1}{beep : procedure/1}}\label{beep-procedure1-1}}

Usage: \passthrough{\lstinline!(beep sel)!}

Play a built-in system sound. The argument \passthrough{\lstinline!sel!}
may be one of '(error start ready click okay confirm info).

See also: \passthrough{\lstinline!play-sound, load-sound!}.

\hypertarget{bind-procedure2-1}{%
\subsection{\texorpdfstring{\texttt{bind} :
procedure/2}{bind : procedure/2}}\label{bind-procedure2-1}}

Usage: \passthrough{\lstinline!(bind sym value)!}

Bind \passthrough{\lstinline!value!} to the global symbol
\passthrough{\lstinline!sym!}. In contrast to setq both values need
quoting.

See also: \passthrough{\lstinline!setq!}.

\hypertarget{bitand-procedure2-1}{%
\subsection{\texorpdfstring{\texttt{bitand} :
procedure/2}{bitand : procedure/2}}\label{bitand-procedure2-1}}

Usage: \passthrough{\lstinline!(bitand n m) => int!}

Return the bitwise and of integers \passthrough{\lstinline!n!} and
\passthrough{\lstinline!m.!}

See also:
\passthrough{\lstinline!bitxor, bitor, bitclear, bitshl, bitshr!}.

\hypertarget{bitclear-procedure2-1}{%
\subsection{\texorpdfstring{\texttt{bitclear} :
procedure/2}{bitclear : procedure/2}}\label{bitclear-procedure2-1}}

Usage: \passthrough{\lstinline!(bitclear n m) => int!}

Return the bitwise and-not of integers \passthrough{\lstinline!n!} and
\passthrough{\lstinline!m.!}

See also:
\passthrough{\lstinline!bitxor, bitand, bitor, bitshl, bitshr!}.

\hypertarget{bitor-procedure2-1}{%
\subsection{\texorpdfstring{\texttt{bitor} :
procedure/2}{bitor : procedure/2}}\label{bitor-procedure2-1}}

Usage: \passthrough{\lstinline!(bitor n m) => int!}

Return the bitwise or of integers \passthrough{\lstinline!n!} and
\passthrough{\lstinline!m.!}

See also:
\passthrough{\lstinline!bitxor, bitand, bitclear, bitshl, bitshr!}.

\hypertarget{bitshl-procedure2-1}{%
\subsection{\texorpdfstring{\texttt{bitshl} :
procedure/2}{bitshl : procedure/2}}\label{bitshl-procedure2-1}}

Usage: \passthrough{\lstinline!(bitshl n m) => int!}

Return the bitwise left shift of \passthrough{\lstinline!n!} by
\passthrough{\lstinline!m.!}

See also:
\passthrough{\lstinline!bitxor, bitor, bitand, bitclear, bitshr!}.

\hypertarget{bitshr-procedure2-1}{%
\subsection{\texorpdfstring{\texttt{bitshr} :
procedure/2}{bitshr : procedure/2}}\label{bitshr-procedure2-1}}

Usage: \passthrough{\lstinline!(bitshr n m) => int!}

Return the bitwise right shift of \passthrough{\lstinline!n!} by
\passthrough{\lstinline!m.!}

See also:
\passthrough{\lstinline!bitxor, bitor, bitand, bitclear, bitshl!}.

\hypertarget{bitxor-procedure2-1}{%
\subsection{\texorpdfstring{\texttt{bitxor} :
procedure/2}{bitxor : procedure/2}}\label{bitxor-procedure2-1}}

Usage: \passthrough{\lstinline!(bitxor n m) => int!}

Return the bitwise exclusive or value of integers
\passthrough{\lstinline!n!} and \passthrough{\lstinline!m.!}

See also:
\passthrough{\lstinline!bitand, bitor, bitclear, bitshl, bitshr!}.

\hypertarget{blob-ascii85-procedure1-or-more-1}{%
\subsection{\texorpdfstring{\texttt{blob-\textgreater{}ascii85} :
procedure/1 or
more}{blob-\textgreater ascii85 : procedure/1 or more}}\label{blob-ascii85-procedure1-or-more-1}}

Usage: \passthrough{\lstinline!(blob->ascii85 b [start] [end]) => str!}

Convert the blob \passthrough{\lstinline!b!} to an ascii85 encoded
string. If the optional \passthrough{\lstinline!start!} and
\passthrough{\lstinline!end!} are provided, then only bytes from
\passthrough{\lstinline!start!} (inclusive) to
\passthrough{\lstinline!end!} (exclusive) are converted.

See also:
\passthrough{\lstinline!blob->hex, blob->str, blob->base64, valid?, blob?!}.

\hypertarget{blob-base64-procedure1-or-more-1}{%
\subsection{\texorpdfstring{\texttt{blob-\textgreater{}base64} :
procedure/1 or
more}{blob-\textgreater base64 : procedure/1 or more}}\label{blob-base64-procedure1-or-more-1}}

Usage: \passthrough{\lstinline!(blob->base64 b [start] [end]) => str!}

Convert the blob \passthrough{\lstinline!b!} to a base64 encoded string.
If the optional \passthrough{\lstinline!start!} and
\passthrough{\lstinline!end!} are provided, then only bytes from
\passthrough{\lstinline!start!} (inclusive) to
\passthrough{\lstinline!end!} (exclusive) are converted.

See also:
\passthrough{\lstinline!base64->blob, valid?, blob?, blob->str, blob->hex, blob->ascii85!}.

\hypertarget{blob-hex-procedure1-or-more-1}{%
\subsection{\texorpdfstring{\texttt{blob-\textgreater{}hex} :
procedure/1 or
more}{blob-\textgreater hex : procedure/1 or more}}\label{blob-hex-procedure1-or-more-1}}

Usage: \passthrough{\lstinline!(blob->hex b [start] [end]) => str!}

Convert the blob \passthrough{\lstinline!b!} to a hexadecimal string of
byte values. If the optional \passthrough{\lstinline!start!} and
\passthrough{\lstinline!end!} are provided, then only bytes from
\passthrough{\lstinline!start!} (inclusive) to
\passthrough{\lstinline!end!} (exclusive) are converted.

See also:
\passthrough{\lstinline!hex->blob, str->blob, valid?, blob?, blob->base64, blob->ascii85!}.

\hypertarget{blob-str-procedure1-or-more-1}{%
\subsection{\texorpdfstring{\texttt{blob-\textgreater{}str} :
procedure/1 or
more}{blob-\textgreater str : procedure/1 or more}}\label{blob-str-procedure1-or-more-1}}

Usage: \passthrough{\lstinline!(blob->str b [start] [end]) => str!}

Convert blob \passthrough{\lstinline!b!} into a string. Notice that the
string may contain binary data that is not suitable for displaying and
does not represent valid UTF-8 glyphs. If the optional
\passthrough{\lstinline!start!} and \passthrough{\lstinline!end!} are
provided, then only bytes from \passthrough{\lstinline!start!}
(inclusive) to \passthrough{\lstinline!end!} (exclusive) are converted.

See also: \passthrough{\lstinline!str->blob, valid?, blob?!}.

\hypertarget{blob-chksum-procedure1-or-more-1}{%
\subsection{\texorpdfstring{\texttt{blob-chksum} : procedure/1 or
more}{blob-chksum : procedure/1 or more}}\label{blob-chksum-procedure1-or-more-1}}

Usage: \passthrough{\lstinline!(blob-chksum b [start] [end]) => blob!}

Return the checksum of the contents of blob \passthrough{\lstinline!b!}
as new blob. The checksum is cryptographically secure. If the optional
\passthrough{\lstinline!start!} and \passthrough{\lstinline!end!} are
provided, then only the bytes from \passthrough{\lstinline!start!}
(inclusive) to \passthrough{\lstinline!end!} (exclusive) are
checksummed.

See also: \passthrough{\lstinline!fchksum, blob-free!}.

\hypertarget{blob-equal-procedure2-1}{%
\subsection{\texorpdfstring{\texttt{blob-equal?} :
procedure/2}{blob-equal? : procedure/2}}\label{blob-equal-procedure2-1}}

Usage: \passthrough{\lstinline!(blob-equal? b1 b2) => bool!}

Return true if \passthrough{\lstinline!b1!} and
\passthrough{\lstinline!b2!} are equal, nil otherwise. Two blobs are
equal if they are either both invalid, both contain no valid data, or
their contents contain exactly the same binary data.

See also: \passthrough{\lstinline!str->blob, blob->str, blob-free!}.

\hypertarget{blob-free-procedure1-1}{%
\subsection{\texorpdfstring{\texttt{blob-free} :
procedure/1}{blob-free : procedure/1}}\label{blob-free-procedure1-1}}

Usage: \passthrough{\lstinline!(blob-free b)!}

Frees the binary data stored in blob \passthrough{\lstinline!b!} and
makes the blob invalid.

See also:
\passthrough{\lstinline!make-blob, valid?, str->blob, blob->str, blob-equal?!}.

\hypertarget{bound-macro1-1}{%
\subsection{\texorpdfstring{\texttt{bound?} :
macro/1}{bound? : macro/1}}\label{bound-macro1-1}}

Usage: \passthrough{\lstinline!(bound? sym) => bool!}

Return true if a value is bound to the symbol
\passthrough{\lstinline!sym!}, nil otherwise.

See also: \passthrough{\lstinline!bind, setq!}.

\hypertarget{build-array-procedure2-1}{%
\subsection{\texorpdfstring{\texttt{build-array} :
procedure/2}{build-array : procedure/2}}\label{build-array-procedure2-1}}

Usage: \passthrough{\lstinline!(build-array n init) => array!}

Create an array containing \passthrough{\lstinline!n!} elements with
initial value \passthrough{\lstinline!init.!}

See also: \passthrough{\lstinline!array, array?!}.

\hypertarget{build-list-procedure2-1}{%
\subsection{\texorpdfstring{\texttt{build-list} :
procedure/2}{build-list : procedure/2}}\label{build-list-procedure2-1}}

Usage: \passthrough{\lstinline!(build-list n proc) => list!}

Build a list with \passthrough{\lstinline!n!} elements by applying
\passthrough{\lstinline!proc!} to the counter
\passthrough{\lstinline!n!} each time.

See also: \passthrough{\lstinline!list, list?, map, foreach!}.

\hypertarget{caaar-procedure1-1}{%
\subsection{\texorpdfstring{\texttt{caaar} :
procedure/1}{caaar : procedure/1}}\label{caaar-procedure1-1}}

Usage: \passthrough{\lstinline!(caaar x) => any!}

Equivalent to (car (car (car \passthrough{\lstinline!x!}))).

See also:
\passthrough{\lstinline!car, cdr, caar, cadr, cdar, cddr, caadr, cadar, caddr, cdaar, cdadr, cddar, cdddr, nth, 1st, 2nd, 3rd!}.

\hypertarget{caadr-procedure1-1}{%
\subsection{\texorpdfstring{\texttt{caadr} :
procedure/1}{caadr : procedure/1}}\label{caadr-procedure1-1}}

Usage: \passthrough{\lstinline!(caadr x) => any!}

Equivalent to (car (car (cdr \passthrough{\lstinline!x!}))).

See also:
\passthrough{\lstinline!car, cdr, caar, cadr, cdar, cddr, caaar, cadar, caddr, cdaar, cdadr, cddar, cdddr, nth, 1st, 2nd, 3rd!}.

\hypertarget{caar-procedure1-1}{%
\subsection{\texorpdfstring{\texttt{caar} :
procedure/1}{caar : procedure/1}}\label{caar-procedure1-1}}

Usage: \passthrough{\lstinline!(caar x) => any!}

Equivalent to (car (car \passthrough{\lstinline!x!})).

See also:
\passthrough{\lstinline!car, cdr, cadr, cdar, cddr, caaar, caadr, cadar, caddr, cdaar, cdadr, cddar, cdddr, nth, 1st, 2nd, 3rd!}.

\hypertarget{cadar-procedure1-1}{%
\subsection{\texorpdfstring{\texttt{cadar} :
procedure/1}{cadar : procedure/1}}\label{cadar-procedure1-1}}

Usage: \passthrough{\lstinline!(cadar x) => any!}

Equivalent to (car (cdr (car \passthrough{\lstinline!x!}))).

See also:
\passthrough{\lstinline!car, cdr, caar, cadr, cdar, cddr, caaar, caadr, caddr, cdaar, cdadr, cddar, cdddr, nth, 1st, 2nd, 3rd!}.

\hypertarget{caddr-procedure1-1}{%
\subsection{\texorpdfstring{\texttt{caddr} :
procedure/1}{caddr : procedure/1}}\label{caddr-procedure1-1}}

Usage: \passthrough{\lstinline!(caddr x) => any!}

Equivalent to (car (cdr (cdr \passthrough{\lstinline!x!}))).

See also:
\passthrough{\lstinline!car, cdr, caar, cadr, cdar, cddr, caaar, caadr, cadar, cdaar, cdadr, cddar, cdddr, nth, 1st, 2nd, 3rd!}.

\hypertarget{cadr-procedure1-1}{%
\subsection{\texorpdfstring{\texttt{cadr} :
procedure/1}{cadr : procedure/1}}\label{cadr-procedure1-1}}

Usage: \passthrough{\lstinline!(cadr x) => any!}

Equivalent to (car (cdr \passthrough{\lstinline!x!})).

See also:
\passthrough{\lstinline!car, cdr, caar, cdar, cddr, caaar, caadr, cadar, caddr, cdaar, cdadr, cddar, cdddr, nth, 1st, 2nd, 3rd!}.

\hypertarget{car-procedure1-1}{%
\subsection{\texorpdfstring{\texttt{car} :
procedure/1}{car : procedure/1}}\label{car-procedure1-1}}

Usage: \passthrough{\lstinline!(car li) => any!}

Get the first element of a list or pair \passthrough{\lstinline!li!}, an
error if there is not first element.

See also: \passthrough{\lstinline!list, list?, pair?!}.

\hypertarget{case-macro2-or-more-1}{%
\subsection{\texorpdfstring{\texttt{case} : macro/2 or
more}{case : macro/2 or more}}\label{case-macro2-or-more-1}}

Usage:
\passthrough{\lstinline!(case expr (clause1 ... clausen)) => any!}

Standard case macro, where you should use t for the remaining
alternative. Example: (case (get dict 'key) ((a b) (out ``a or b''))(t
(out ``something else!''))).

See also: \passthrough{\lstinline!cond!}.

\hypertarget{ccmp-macro2-1}{%
\subsection{\texorpdfstring{\texttt{ccmp} :
macro/2}{ccmp : macro/2}}\label{ccmp-macro2-1}}

Usage: \passthrough{\lstinline!(ccmp sym value) => int!}

Compare the integer value of \passthrough{\lstinline!sym!} with the
integer \passthrough{\lstinline!value!}, return 0 if
\passthrough{\lstinline!sym!} = \passthrough{\lstinline!value!}, -1 if
\passthrough{\lstinline!sym!} \textless{}
\passthrough{\lstinline!value!}, and 1 if \passthrough{\lstinline!sym!}
\textgreater{} \passthrough{\lstinline!value!}. This operation is
synchronized between tasks and futures.

See also: \passthrough{\lstinline"cinc!, cdec!, cwait, cst!"}.

\hypertarget{cdaar-procedure1-1}{%
\subsection{\texorpdfstring{\texttt{cdaar} :
procedure/1}{cdaar : procedure/1}}\label{cdaar-procedure1-1}}

Usage: \passthrough{\lstinline!(cdaar x) => any!}

Equivalent to (cdr (car (car \passthrough{\lstinline!x!}))).

See also:
\passthrough{\lstinline!car, cdr, caar, cadr, cdar, cddr, caaar, caadr, cadar, caddr, cdadr, cddar, cdddr, nth, 1st, 2nd, 3rd!}.

\hypertarget{cdadr-procedure1-1}{%
\subsection{\texorpdfstring{\texttt{cdadr} :
procedure/1}{cdadr : procedure/1}}\label{cdadr-procedure1-1}}

Usage: \passthrough{\lstinline!(cdadr x) => any!}

Equivalent to (cdr (car (cdr \passthrough{\lstinline!x!}))).

See also:
\passthrough{\lstinline!car, cdr, caar, cadr, cdar, cddr, caaar, caadr, cadar, caddr, cdaar, cddar, cdddr, nth, 1st, 2nd, 3rd!}.

\hypertarget{cdar-procedure1-1}{%
\subsection{\texorpdfstring{\texttt{cdar} :
procedure/1}{cdar : procedure/1}}\label{cdar-procedure1-1}}

Usage: \passthrough{\lstinline!(cdar x) => any!}

Equivalent to (cdr (car \passthrough{\lstinline!x!})).

See also:
\passthrough{\lstinline!car, cdr, caar, cadr, cddr, caaar, caadr, cadar, caddr, cdaar, cdadr, cddar, cdddr, nth, 1st, 2nd, 3rd!}.

\hypertarget{cddar-procedure1-1}{%
\subsection{\texorpdfstring{\texttt{cddar} :
procedure/1}{cddar : procedure/1}}\label{cddar-procedure1-1}}

Usage: \passthrough{\lstinline!(cddar x) => any!}

Equivalent to (cdr (cdr (car \passthrough{\lstinline!x!}))).

See also:
\passthrough{\lstinline!car, cdr, caar, cadr, cdar, cddr, caaar, caadr, cadar, caddr, cdaar, cdadr, cdddr, nth, 1st, 2nd, 3rd!}.

\hypertarget{cdddr-procedure1-1}{%
\subsection{\texorpdfstring{\texttt{cdddr} :
procedure/1}{cdddr : procedure/1}}\label{cdddr-procedure1-1}}

Usage: \passthrough{\lstinline!(cdddr x) => any!}

Equivalent to (cdr (cdr (cdr \passthrough{\lstinline!x!}))).

See also:
\passthrough{\lstinline!car, cdr, caar, cadr, cdar, cddr, caaar, caadr, cadar, caddr, cdaar, cdadr, cddar, nth, 1st, 2nd, 3rd!}.

\hypertarget{cddr-procedure1-1}{%
\subsection{\texorpdfstring{\texttt{cddr} :
procedure/1}{cddr : procedure/1}}\label{cddr-procedure1-1}}

Usage: \passthrough{\lstinline!(cddr x) => any!}

Equivalent to (cdr (cdr \passthrough{\lstinline!x!})).

See also:
\passthrough{\lstinline!car, cdr, caar, cadr, cdar, caaar, caadr, cadar, caddr, cdaar, cdadr, cddar, cdddr, nth, 1st, 2nd, 3rd!}.

\hypertarget{cdec-macro1-1}{%
\subsection{\texorpdfstring{\texttt{cdec!} :
macro/1}{cdec! : macro/1}}\label{cdec-macro1-1}}

Usage: \passthrough{\lstinline"(cdec! sym) => int"}

Decrease the integer value stored in top-level symbol
\passthrough{\lstinline!sym!} by 1 and return the new value. This
operation is synchronized between tasks and futures.

See also: \passthrough{\lstinline"cinc!, cwait, ccmp, cst!"}.

\hypertarget{cdr-procedure1-1}{%
\subsection{\texorpdfstring{\texttt{cdr} :
procedure/1}{cdr : procedure/1}}\label{cdr-procedure1-1}}

Usage: \passthrough{\lstinline!(cdr li) => any!}

Get the rest of a list \passthrough{\lstinline!li!}. If the list is
proper, the cdr is a list. If it is a pair, then it may be an element.
If the list is empty, nil is returned.

See also: \passthrough{\lstinline!car, list, list?, pair?!}.

\hypertarget{char-str-procedure1-1}{%
\subsection{\texorpdfstring{\texttt{char-\textgreater{}str} :
procedure/1}{char-\textgreater str : procedure/1}}\label{char-str-procedure1-1}}

Usage: \passthrough{\lstinline!(char->str n) => str!}

Return a string containing the unicode char based on integer
\passthrough{\lstinline!n.!}

See also: \passthrough{\lstinline!str->char!}.

\hypertarget{chars-procedure1-1}{%
\subsection{\texorpdfstring{\texttt{chars} :
procedure/1}{chars : procedure/1}}\label{chars-procedure1-1}}

Usage: \passthrough{\lstinline!(chars str) => dict!}

Return a charset based on \passthrough{\lstinline!str!}, i.e., dict with
the chars of \passthrough{\lstinline!str!} as keys and true as value.

See also: \passthrough{\lstinline!dict, get, set, contains!}.

\hypertarget{chars-str-procedure1-1}{%
\subsection{\texorpdfstring{\texttt{chars-\textgreater{}str} :
procedure/1}{chars-\textgreater str : procedure/1}}\label{chars-str-procedure1-1}}

Usage: \passthrough{\lstinline!(chars->str a) => str!}

Convert an array of UTF-8 rune integers \passthrough{\lstinline!a!} into
a UTF-8 encoded string.

See also: \passthrough{\lstinline!str->runes, str->char, char->str!}.

\hypertarget{cinc-macro1-1}{%
\subsection{\texorpdfstring{\texttt{cinc!} :
macro/1}{cinc! : macro/1}}\label{cinc-macro1-1}}

Usage: \passthrough{\lstinline"(cinc! sym) => int"}

Increase the integer value stored in top-level symbol
\passthrough{\lstinline!sym!} by 1 and return the new value. This
operation is synchronized between tasks and futures.

See also: \passthrough{\lstinline"cdec!, cwait, ccmp, cst!"}.

\hypertarget{close-procedure1-1}{%
\subsection{\texorpdfstring{\texttt{close} :
procedure/1}{close : procedure/1}}\label{close-procedure1-1}}

Usage: \passthrough{\lstinline!(close p)!}

Close the port \passthrough{\lstinline!p!}. Calling close twice on the
same port should be avoided.

See also: \passthrough{\lstinline!open, stropen!}.

\hypertarget{closure-procedure1-1}{%
\subsection{\texorpdfstring{\texttt{closure?} :
procedure/1}{closure? : procedure/1}}\label{closure-procedure1-1}}

Usage: \passthrough{\lstinline!(closure? x) => bool!}

Return true if \passthrough{\lstinline!x!} is a closure, nil otherwise.
Use \passthrough{\lstinline!function?!} for texting whether
\passthrough{\lstinline!x!} can be executed.

See also:
\passthrough{\lstinline!functional?, macro?, intrinsic?, functional-arity, functional-has-rest?!}.

\hypertarget{collect-garbage-procedure0-or-more-1}{%
\subsection{\texorpdfstring{\texttt{collect-garbage} : procedure/0 or
more}{collect-garbage : procedure/0 or more}}\label{collect-garbage-procedure0-or-more-1}}

Usage: \passthrough{\lstinline!(collect-garbage [sort])!}

Force a garbage-collection of the system's memory. If
\passthrough{\lstinline!sort!} is 'normal, then only a normal
incremental garbage colllection is performed. If
\passthrough{\lstinline!sort!} is 'total, then the garbage collection is
more thorough and the system attempts to return unused memory to the
host OS. Default is 'normal.

See also: \passthrough{\lstinline!memstats!}.

\textbf{Warning: There should rarely be a use for this. Try to use less
memory-consuming data structures instead.}

\hypertarget{color-procedure1-1}{%
\subsection{\texorpdfstring{\texttt{color} :
procedure/1}{color : procedure/1}}\label{color-procedure1-1}}

Usage: \passthrough{\lstinline!(color sel) => (r g b a)!}

Return the color based on \passthrough{\lstinline!sel!}, which may be
'text for the text color, 'back for the background color, 'textarea for
the color of the text area, 'gfx for the current graphics foreground
color, and 'frame for the frame color.

See also: \passthrough{\lstinline!set-color, the-color, with-colors!}.

\hypertarget{cons-procedure2-1}{%
\subsection{\texorpdfstring{\texttt{cons} :
procedure/2}{cons : procedure/2}}\label{cons-procedure2-1}}

Usage: \passthrough{\lstinline!(cons a b) => pair!}

Cons two values into a pair. If \passthrough{\lstinline!b!} is a list,
the result is a list. Otherwise the result is a pair.

See also: \passthrough{\lstinline!cdr, car, list?, pair?!}.

\hypertarget{cons-procedure1-1}{%
\subsection{\texorpdfstring{\texttt{cons?} :
procedure/1}{cons? : procedure/1}}\label{cons-procedure1-1}}

Usage: \passthrough{\lstinline!(cons? x) => bool!}

return true if \passthrough{\lstinline!x!} is not an atom, nil
otherwise.

See also: \passthrough{\lstinline!atom?!}.

\hypertarget{count-partitions-procedure2-1}{%
\subsection{\texorpdfstring{\texttt{count-partitions} :
procedure/2}{count-partitions : procedure/2}}\label{count-partitions-procedure2-1}}

Usage: \passthrough{\lstinline!(count-partitions m k) => int!}

Return the number of partitions for divding \passthrough{\lstinline!m!}
items into parts of size \passthrough{\lstinline!k!} or less, where the
size of the last partition may be less than \passthrough{\lstinline!k!}
but the remaining ones have size \passthrough{\lstinline!k.!}

See also: \passthrough{\lstinline!nth-partition, get-partitions!}.

\hypertarget{cpunum-procedure0-1}{%
\subsection{\texorpdfstring{\texttt{cpunum} :
procedure/0}{cpunum : procedure/0}}\label{cpunum-procedure0-1}}

Usage: \passthrough{\lstinline!(cpunum)!}

Return the number of cpu cores of this machine.

See also: \passthrough{\lstinline!sys!}.

\textbf{Warning: This function also counts virtual cores on the
emulator. The original Z3S5 machine did not have virtual cpu cores.}

\hypertarget{cst-procedure2-1}{%
\subsection{\texorpdfstring{\texttt{cst!} :
procedure/2}{cst! : procedure/2}}\label{cst-procedure2-1}}

Usage: \passthrough{\lstinline"(cst! sym value)"}

Set the value of \passthrough{\lstinline!sym!} to integer
\passthrough{\lstinline!value!}. This operation is synchronized between
tasks and futures.

See also: \passthrough{\lstinline"cinc!, cdec!, ccmp, cwait"}.

\hypertarget{current-error-handler-procedure0-1}{%
\subsection{\texorpdfstring{\texttt{current-error-handler} :
procedure/0}{current-error-handler : procedure/0}}\label{current-error-handler-procedure0-1}}

Usage: \passthrough{\lstinline!(current-error-handler) => proc!}

Return the current error handler, a default if there is none.

See also:
\passthrough{\lstinline!default-error-handler, push-error-handler, pop-error-handler, *current-error-handler*, *current-error-continuation*!}.

\hypertarget{cwait-procedure3-1}{%
\subsection{\texorpdfstring{\texttt{cwait} :
procedure/3}{cwait : procedure/3}}\label{cwait-procedure3-1}}

Usage: \passthrough{\lstinline!(cwait sym value timeout)!}

Wait until integer counter \passthrough{\lstinline!sym!} has
\passthrough{\lstinline!value!} or \passthrough{\lstinline!timeout!}
milliseconds have passed. If \passthrough{\lstinline!imeout!} is 0, then
this routine might wait indefinitely. This operation is synchronized
between tasks and futures.

See also: \passthrough{\lstinline"cinc!, cdec!, ccmp, cst!"}.

\hypertarget{darken-procedure1-1}{%
\subsection{\texorpdfstring{\texttt{darken} :
procedure/1}{darken : procedure/1}}\label{darken-procedure1-1}}

Usage: \passthrough{\lstinline!(darken color [amount]) => (r g b a)!}

Return a darker version of \passthrough{\lstinline!color!}. The optional
positive \passthrough{\lstinline!amount!} specifies the amount of
darkening (0-255).

See also: \passthrough{\lstinline!the-color, *colors*, lighten!}.

\hypertarget{date-epoch-ns-procedure7-1}{%
\subsection{\texorpdfstring{\texttt{date-\textgreater{}epoch-ns} :
procedure/7}{date-\textgreater epoch-ns : procedure/7}}\label{date-epoch-ns-procedure7-1}}

Usage: \passthrough{\lstinline!(date->epoch-ns Y M D h m s ns) => int!}

Return the Unix epoch nanoseconds based on the given year
\passthrough{\lstinline!Y!}, month \passthrough{\lstinline!M!}, day
\passthrough{\lstinline!D!}, hour \passthrough{\lstinline!h!}, minute
\passthrough{\lstinline!m!}, seconds \passthrough{\lstinline!s!}, and
nanosecond fraction of a second \passthrough{\lstinline!ns!}, as it is
e.g.~returned in a (now) datelist.

See also:
\passthrough{\lstinline!epoch-ns->datelist, datestr->datelist, datestr, datestr*, day-of-week, week-of-date, now!}.

\hypertarget{datelist-epoch-ns-procedure1-1}{%
\subsection{\texorpdfstring{\texttt{datelist-\textgreater{}epoch-ns} :
procedure/1}{datelist-\textgreater epoch-ns : procedure/1}}\label{datelist-epoch-ns-procedure1-1}}

Usage: \passthrough{\lstinline!(datelist->epoch-ns dateli) => int!}

Convert a datelist to Unix epoch nanoseconds. This function uses the
Unix nanoseconds from the 5th value of the second list in the datelist,
as it is provided by functions like (now). However, if the Unix
nanoseconds value is not specified in the list, it uses
\passthrough{\lstinline!date->epoch-ns!} to convert to Unix epoch
nanoseconds. Datelists can be incomplete. If the month is not specified,
January is assumed. If the day is not specified, the 1st is assumed. If
the hour is not specified, 12 is assumed, and corresponding defaults for
minutes, seconds, and nanoseconds are 0.

See also:
\passthrough{\lstinline!date->epoch-ns, datestr, datestr*, datestr->datelist, epoch-ns->datelist, now!}.

\hypertarget{datestr-procedure1-2}{%
\subsection{\texorpdfstring{\texttt{datestr} :
procedure/1}{datestr : procedure/1}}\label{datestr-procedure1-2}}

Usage: \passthrough{\lstinline!(datestr datelist) => str!}

Return datelist, as it is e.g.~returned by (now), as a string in format
YYYY-MM-DD HH:mm.

See also: \passthrough{\lstinline!now, datestr*, datestr->datelist!}.

\hypertarget{datestr-procedure1-3}{%
\subsection{\texorpdfstring{\texttt{datestr*} :
procedure/1}{datestr* : procedure/1}}\label{datestr-procedure1-3}}

Usage: \passthrough{\lstinline!(datestr* datelist) => str!}

Return the datelist, as it is e.g.~returned by (now), as a string in
format YYYY-MM-DD HH:mm:ss.nanoseconds.

See also: \passthrough{\lstinline!now, datestr, datestr->datelist!}.

\hypertarget{datestr-datelist-procedure1-1}{%
\subsection{\texorpdfstring{\texttt{datestr-\textgreater{}datelist} :
procedure/1}{datestr-\textgreater datelist : procedure/1}}\label{datestr-datelist-procedure1-1}}

Usage: \passthrough{\lstinline!(datestr->datelist s) => li!}

Convert a date string in the format of datestr and datestr* into a date
list as it is e.g.~returned by (now).

See also: \passthrough{\lstinline!datestr*, datestr, now!}.

\hypertarget{day-procedure2-1}{%
\subsection{\texorpdfstring{\texttt{day+} :
procedure/2}{day+ : procedure/2}}\label{day-procedure2-1}}

Usage: \passthrough{\lstinline!(day+ dateli n) => dateli!}

Adds \passthrough{\lstinline!n!} days to the given date
\passthrough{\lstinline!dateli!} in datelist format and returns the new
datelist.

See also:
\passthrough{\lstinline!sec+, minute+, hour+, week+, month+, year+, now!}.

\hypertarget{day-of-week-procedure3-1}{%
\subsection{\texorpdfstring{\texttt{day-of-week} :
procedure/3}{day-of-week : procedure/3}}\label{day-of-week-procedure3-1}}

Usage: \passthrough{\lstinline!(day-of-week Y M D) => int!}

Return the day of week based on the date with year
\passthrough{\lstinline!Y!}, month \passthrough{\lstinline!M!}, and day
\passthrough{\lstinline!D!}. The first day number 0 is Sunday, the last
day is Saturday with number 6.

See also:
\passthrough{\lstinline!week-of-date, datestr->datelist, date->epoch-ns, epoch-ns->datelist, datestr, datestr*, now!}.

\hypertarget{def-custom-hook-procedure2-1}{%
\subsection{\texorpdfstring{\texttt{def-custom-hook} :
procedure/2}{def-custom-hook : procedure/2}}\label{def-custom-hook-procedure2-1}}

Usage: \passthrough{\lstinline!(def-custom-hook sym proc)!}

Define a custom hook point, to be called manually from Lisp. These have
IDs starting from 65636.

See also: \passthrough{\lstinline!add-hook!}.

\hypertarget{default-error-handler-procedure0-1}{%
\subsection{\texorpdfstring{\texttt{default-error-handler} :
procedure/0}{default-error-handler : procedure/0}}\label{default-error-handler-procedure0-1}}

Usage: \passthrough{\lstinline!(default-error-handler) => proc!}

Return the default error handler, irrespectively of the
current-error-handler.

See also:
\passthrough{\lstinline!current-error-handler, push-error-handler, pop-error-handler, *current-error-handler*, *current-error-continuation*!}.

\hypertarget{defmacro-macro2-or-more-1}{%
\subsection{\texorpdfstring{\texttt{defmacro} : macro/2 or
more}{defmacro : macro/2 or more}}\label{defmacro-macro2-or-more-1}}

Usage: \passthrough{\lstinline!(defmacro name args body ...)!}

Define a macro \passthrough{\lstinline!name!} with argument list
\passthrough{\lstinline!args!} and \passthrough{\lstinline!body!}.
Macros are expanded at compile-time.

See also: \passthrough{\lstinline!macro!}.

\hypertarget{delete-procedure2-1}{%
\subsection{\texorpdfstring{\texttt{delete} :
procedure/2}{delete : procedure/2}}\label{delete-procedure2-1}}

Usage: \passthrough{\lstinline!(delete d key)!}

Remove the value for \passthrough{\lstinline!key!} in dict
\passthrough{\lstinline!d!}. This also removes the key.

See also: \passthrough{\lstinline!dict?, get, set!}.

\hypertarget{dequeue-macro1-or-more-1}{%
\subsection{\texorpdfstring{\texttt{dequeue!} : macro/1 or
more}{dequeue! : macro/1 or more}}\label{dequeue-macro1-or-more-1}}

Usage: \passthrough{\lstinline"(dequeue! sym [def]) => any"}

Get the next element from queue \passthrough{\lstinline!sym!}, which
must be the unquoted name of a variable, and return it. If a default
\passthrough{\lstinline!def!} is given, then this is returned if the
queue is empty, otherwise nil is returned.

See also:
\passthrough{\lstinline"make-queue, queue?, enqueue!, glance, queue-empty?, queue-len"}.

\hypertarget{dict-procedure0-or-more-1}{%
\subsection{\texorpdfstring{\texttt{dict} : procedure/0 or
more}{dict : procedure/0 or more}}\label{dict-procedure0-or-more-1}}

Usage: \passthrough{\lstinline!(dict [li]) => dict!}

Create a dictionary. The option \passthrough{\lstinline!li!} must be a
list of the form '(key1 value1 key2 value2 \ldots). Dictionaries are
unordered, hence also not sequences. Dictionaries are safe for
concurrent access.

See also: \passthrough{\lstinline!array, list!}.

\hypertarget{dict-alist-procedure1-1}{%
\subsection{\texorpdfstring{\texttt{dict-\textgreater{}alist} :
procedure/1}{dict-\textgreater alist : procedure/1}}\label{dict-alist-procedure1-1}}

Usage: \passthrough{\lstinline!(dict->alist d) => li!}

Convert a dictionary into an association list. Note that the resulting
alist will be a set of proper pairs of the form '(a . b) if the values
in the dictionary are not lists.

See also: \passthrough{\lstinline!dict, dict-map, dict->list!}.

\hypertarget{dict-array-procedure1-1}{%
\subsection{\texorpdfstring{\texttt{dict-\textgreater{}array} :
procedure/1}{dict-\textgreater array : procedure/1}}\label{dict-array-procedure1-1}}

Usage: \passthrough{\lstinline!(dict-array d) => array!}

Return an array that contains all key, value pairs of
\passthrough{\lstinline!d!}. A key comes directly before its value, but
otherwise the order is unspecified.

See also: \passthrough{\lstinline!dict->list, dict!}.

\hypertarget{dict-keys-procedure1-1}{%
\subsection{\texorpdfstring{\texttt{dict-\textgreater{}keys} :
procedure/1}{dict-\textgreater keys : procedure/1}}\label{dict-keys-procedure1-1}}

Usage: \passthrough{\lstinline!(dict->keys d) => li!}

Return the keys of dictionary \passthrough{\lstinline!d!} in arbitrary
order.

See also:
\passthrough{\lstinline!dict, dict->values, dict->alist, dict->list!}.

\hypertarget{dict-list-procedure1-1}{%
\subsection{\texorpdfstring{\texttt{dict-\textgreater{}list} :
procedure/1}{dict-\textgreater list : procedure/1}}\label{dict-list-procedure1-1}}

Usage: \passthrough{\lstinline!(dict->list d) => li!}

Return a list of the form '(key1 value1 key2 value2 \ldots), where the
order of key, value pairs is unspecified.

See also: \passthrough{\lstinline!dict->array, dict!}.

\hypertarget{dict-values-procedure1-1}{%
\subsection{\texorpdfstring{\texttt{dict-\textgreater{}values} :
procedure/1}{dict-\textgreater values : procedure/1}}\label{dict-values-procedure1-1}}

Usage: \passthrough{\lstinline!(dict->values d) => li!}

Return the values of dictionary \passthrough{\lstinline!d!} in arbitrary
order.

See also:
\passthrough{\lstinline!dict, dict->keys, dict->alist, dict->list!}.

\hypertarget{dict-copy-procedure1-1}{%
\subsection{\texorpdfstring{\texttt{dict-copy} :
procedure/1}{dict-copy : procedure/1}}\label{dict-copy-procedure1-1}}

Usage: \passthrough{\lstinline!(dict-copy d) => dict!}

Return a copy of dict \passthrough{\lstinline!d.!}

See also: \passthrough{\lstinline!dict, dict?!}.

\hypertarget{dict-empty-procedure1-1}{%
\subsection{\texorpdfstring{\texttt{dict-empty?} :
procedure/1}{dict-empty? : procedure/1}}\label{dict-empty-procedure1-1}}

Usage: \passthrough{\lstinline!(dict-empty? d) => bool!}

Return true if dict \passthrough{\lstinline!d!} is empty, nil otherwise.
As crazy as this may sound, this can have O(n) complexity if the dict is
not empty, but it is still going to be more efficient than any other
method.

See also: \passthrough{\lstinline!dict!}.

\hypertarget{dict-foreach-procedure2-1}{%
\subsection{\texorpdfstring{\texttt{dict-foreach} :
procedure/2}{dict-foreach : procedure/2}}\label{dict-foreach-procedure2-1}}

Usage: \passthrough{\lstinline!(dict-foreach d proc)!}

Call \passthrough{\lstinline!proc!} for side-effects with the key and
value for each key, value pair in dict \passthrough{\lstinline!d.!}

See also: \passthrough{\lstinline"dict-map!, dict?, dict"}.

\hypertarget{dict-map-procedure2-2}{%
\subsection{\texorpdfstring{\texttt{dict-map} :
procedure/2}{dict-map : procedure/2}}\label{dict-map-procedure2-2}}

Usage: \passthrough{\lstinline!(dict-map dict proc) => dict!}

Returns a copy of \passthrough{\lstinline!dict!} with
\passthrough{\lstinline!proc!} applies to each key value pair as
aruments. Keys are immutable, so \passthrough{\lstinline!proc!} must
take two arguments and return the new value.

See also: \passthrough{\lstinline"dict-map!, map"}.

\hypertarget{dict-map-procedure2-3}{%
\subsection{\texorpdfstring{\texttt{dict-map!} :
procedure/2}{dict-map! : procedure/2}}\label{dict-map-procedure2-3}}

Usage: \passthrough{\lstinline"(dict-map! d proc)"}

Apply procedure \passthrough{\lstinline!proc!} which takes the key and
value as arguments to each key, value pair in dict
\passthrough{\lstinline!d!} and set the respective value in
\passthrough{\lstinline!d!} to the result of
\passthrough{\lstinline!proc!}. Keys are not changed.

See also: \passthrough{\lstinline!dict, dict?, dict-foreach!}.

\hypertarget{dict-merge-procedure2-1}{%
\subsection{\texorpdfstring{\texttt{dict-merge} :
procedure/2}{dict-merge : procedure/2}}\label{dict-merge-procedure2-1}}

Usage: \passthrough{\lstinline!(dict-merge a b) => dict!}

Create a new dict that contains all key-value pairs from dicts
\passthrough{\lstinline!a!} and \passthrough{\lstinline!b!}. Note that
this function is not symmetric. If a key is in both
\passthrough{\lstinline!a!} and \passthrough{\lstinline!b!}, then the
key value pair in \passthrough{\lstinline!a!} is retained for this key.

See also:
\passthrough{\lstinline"dict, dict-map, dict-map!, dict-foreach"}.

\hypertarget{dict-protect-procedure1-1}{%
\subsection{\texorpdfstring{\texttt{dict-protect} :
procedure/1}{dict-protect : procedure/1}}\label{dict-protect-procedure1-1}}

Usage: \passthrough{\lstinline!(dict-protect d)!}

Protect dict \passthrough{\lstinline!d!} against changes. Attempting to
set values in a protected dict will cause an error, but all values can
be read and the dict can be copied. This function requires permission
'allow-protect.

See also:
\passthrough{\lstinline!dict-unprotect, dict-protected?, protect, unprotect, protected?, permissions, permission?!}.

\textbf{Warning: Protected dicts are full readable and can be copied, so
you may need to use protect to also prevent changes to the toplevel
symbol storing the dict!}

\hypertarget{dict-protected-procedure1-1}{%
\subsection{\texorpdfstring{\texttt{dict-protected?} :
procedure/1}{dict-protected? : procedure/1}}\label{dict-protected-procedure1-1}}

Usage: \passthrough{\lstinline!(dict-protected? d)!}

Return true if the dict \passthrough{\lstinline!d!} is protected against
mutation, nil otherwise.

See also:
\passthrough{\lstinline!dict-protect, dict-unprotect, protect, unprotect, protected?, permissions, permission?!}.

\hypertarget{dict-unprotect-procedure1-1}{%
\subsection{\texorpdfstring{\texttt{dict-unprotect} :
procedure/1}{dict-unprotect : procedure/1}}\label{dict-unprotect-procedure1-1}}

Usage: \passthrough{\lstinline!(dict-unprotect d)!}

Unprotect the dict \passthrough{\lstinline!d!} so it can be mutated
again. This function requires permission 'allow-unprotect.

See also:
\passthrough{\lstinline!dict-protect, dict-protected?, protect, unprotect, protected?, permissions, permission?!}.

\hypertarget{dict-procedure1-1}{%
\subsection{\texorpdfstring{\texttt{dict?} :
procedure/1}{dict? : procedure/1}}\label{dict-procedure1-1}}

Usage: \passthrough{\lstinline!(dict? obj) => bool!}

Return true if \passthrough{\lstinline!obj!} is a dict, nil otherwise.

See also: \passthrough{\lstinline!dict!}.

\hypertarget{dir-procedure1-2}{%
\subsection{\texorpdfstring{\texttt{dir} :
procedure/1}{dir : procedure/1}}\label{dir-procedure1-2}}

Usage: \passthrough{\lstinline!(dir [path]) => li!}

Obtain a directory list for \passthrough{\lstinline!path!}. If
\passthrough{\lstinline!path!} is not specified, the current working
directory is listed.

See also: \passthrough{\lstinline!dir?, open, close, read, write!}.

\hypertarget{dir-procedure1-3}{%
\subsection{\texorpdfstring{\texttt{dir?} :
procedure/1}{dir? : procedure/1}}\label{dir-procedure1-3}}

Usage: \passthrough{\lstinline!(dir? path) => bool!}

Check if the file at \passthrough{\lstinline!path!} is a directory and
return true, nil if the file does not exist or is not a directory.

See also:
\passthrough{\lstinline!file-exists?, dir, open, close, read, write!}.

\hypertarget{div-procedure2-1}{%
\subsection{\texorpdfstring{\texttt{div} :
procedure/2}{div : procedure/2}}\label{div-procedure2-1}}

Usage: \passthrough{\lstinline!(div n k) => int!}

Integer division of \passthrough{\lstinline!n!} by
\passthrough{\lstinline!k.!}

See also: \passthrough{\lstinline!truncate, /, int!}.

\hypertarget{dolist-macro1-or-more-1}{%
\subsection{\texorpdfstring{\texttt{dolist} : macro/1 or
more}{dolist : macro/1 or more}}\label{dolist-macro1-or-more-1}}

Usage:
\passthrough{\lstinline!(dolist (name list [result]) body ...) => li!}

Traverse the list \passthrough{\lstinline!list!} in order, binding
\passthrough{\lstinline!name!} to each element subsequently and evaluate
the \passthrough{\lstinline!body!} expressions with this binding. The
optional \passthrough{\lstinline!result!} is the result of the
traversal, nil if it is not provided.

See also: \passthrough{\lstinline!letrec, foreach, map!}.

\hypertarget{dotimes-macro1-or-more-1}{%
\subsection{\texorpdfstring{\texttt{dotimes} : macro/1 or
more}{dotimes : macro/1 or more}}\label{dotimes-macro1-or-more-1}}

Usage:
\passthrough{\lstinline!(dotimes (name count [result]) body ...) => any!}

Iterate \passthrough{\lstinline!count!} times, binding
\passthrough{\lstinline!name!} to the counter starting from 0 until the
counter has reached count-1, and evaluate the
\passthrough{\lstinline!body!} expressions each time with this binding.
The optional \passthrough{\lstinline!result!} is the result of the
iteration, nil if it is not provided.

See also: \passthrough{\lstinline!letrec, dolist, while!}.

\hypertarget{dump-procedure0-or-more-1}{%
\subsection{\texorpdfstring{\texttt{dump} : procedure/0 or
more}{dump : procedure/0 or more}}\label{dump-procedure0-or-more-1}}

Usage: \passthrough{\lstinline!(dump [sym] [all?]) => li!}

Return a list of symbols starting with the characters of
\passthrough{\lstinline!sym!} or starting with any characters if
\passthrough{\lstinline!sym!} is omitted, sorted alphabetically. When
\passthrough{\lstinline!all?!} is true, then all symbols are listed,
otherwise only symbols that do not contain "\_" are listed. By
convention, the underscore is used for auxiliary functions.

See also:
\passthrough{\lstinline!dump-bindings, save-zimage, load-zimage!}.

\hypertarget{dump-bindings-procedure0-1}{%
\subsection{\texorpdfstring{\texttt{dump-bindings} :
procedure/0}{dump-bindings : procedure/0}}\label{dump-bindings-procedure0-1}}

Usage: \passthrough{\lstinline!(dump-bindings) => li!}

Return a list of all top-level symbols with bound values, including
those intended for internal use.

See also: \passthrough{\lstinline!dump!}.

\hypertarget{enq-procedure1-1}{%
\subsection{\texorpdfstring{\texttt{enq} :
procedure/1}{enq : procedure/1}}\label{enq-procedure1-1}}

Usage: \passthrough{\lstinline!(enq proc)!}

Put \passthrough{\lstinline!proc!} on a special internal queue for
sequential execution and execute it when able.
\passthrough{\lstinline!proc!} must be a prodedure that takes no
arguments. The queue can be used to synchronizing i/o commands but
special care must be taken that \passthrough{\lstinline!proc!}
terminates, or else the system might be damaged.

See also: \passthrough{\lstinline!task, future, synout, synouty!}.

\textbf{Warning: Calls to enq can never be nested, neither explicitly or
implicitly by calling enq anywhere else in the call chain!}

\hypertarget{enqueue-macro2-1}{%
\subsection{\texorpdfstring{\texttt{enqueue!} :
macro/2}{enqueue! : macro/2}}\label{enqueue-macro2-1}}

Usage: \passthrough{\lstinline"(enqueue! sym elem)"}

Put \passthrough{\lstinline!elem!} in queue
\passthrough{\lstinline!sym!}, where \passthrough{\lstinline!sym!} is
the unquoted name of a variable.

See also:
\passthrough{\lstinline"make-queue, queue?, dequeue!, glance, queue-empty?, queue-len"}.

\hypertarget{epoch-ns-datelist-procedure1-1}{%
\subsection{\texorpdfstring{\texttt{epoch-ns-\textgreater{}datelist} :
procedure/1}{epoch-ns-\textgreater datelist : procedure/1}}\label{epoch-ns-datelist-procedure1-1}}

Usage: \passthrough{\lstinline!(epoch-ns->datelist ns) => li!}

Return the date list in UTC time corresponding to the Unix epoch
nanoseconds \passthrough{\lstinline!ns.!}

See also:
\passthrough{\lstinline!date->epoch-ns, datestr->datelist, datestr, datestr*, day-of-week, week-of-date, now!}.

\hypertarget{eq-procedure2-1}{%
\subsection{\texorpdfstring{\texttt{eq?} :
procedure/2}{eq? : procedure/2}}\label{eq-procedure2-1}}

Usage: \passthrough{\lstinline!(eq? x y) => bool!}

Return true if \passthrough{\lstinline!x!} and
\passthrough{\lstinline!y!} are equal, nil otherwise. In contrast to
other LISPs, eq? checks for deep equality of arrays and dicts. However,
lists are compared by checking whether they are the same cell in memory.
Use \passthrough{\lstinline!equal?!} to check for deep equality of lists
and other objects.

See also: \passthrough{\lstinline!equal?!}.

\hypertarget{eql-procedure2-1}{%
\subsection{\texorpdfstring{\texttt{eql?} :
procedure/2}{eql? : procedure/2}}\label{eql-procedure2-1}}

Usage: \passthrough{\lstinline!(eql? x y) => bool!}

Returns true if \passthrough{\lstinline!x!} is equal to
\passthrough{\lstinline!y!}, nil otherwise. This is currently the same
as equal? but the behavior might change.

See also: \passthrough{\lstinline!equal?!}.

\textbf{Warning: Deprecated.}

\hypertarget{equal-procedure2-1}{%
\subsection{\texorpdfstring{\texttt{equal?} :
procedure/2}{equal? : procedure/2}}\label{equal-procedure2-1}}

Usage: \passthrough{\lstinline!(equal? x y) => bool!}

Return true if \passthrough{\lstinline!x!} and
\passthrough{\lstinline!y!} are equal, nil otherwise. The equality is
tested recursively for containers like lists and arrays.

See also: \passthrough{\lstinline!eq?, eql?!}.

\hypertarget{error-procedure0-or-more-1}{%
\subsection{\texorpdfstring{\texttt{error} : procedure/0 or
more}{error : procedure/0 or more}}\label{error-procedure0-or-more-1}}

Usage: \passthrough{\lstinline!(error [msgstr] [expr] ...)!}

Raise an error, where \passthrough{\lstinline!msgstr!} and the optional
expressions \passthrough{\lstinline!expr!}\ldots{} work as in a call to
fmt.

See also: \passthrough{\lstinline!fmt, with-final!}.

\hypertarget{eval-procedure1-1}{%
\subsection{\texorpdfstring{\texttt{eval} :
procedure/1}{eval : procedure/1}}\label{eval-procedure1-1}}

Usage: \passthrough{\lstinline!(eval expr) => any!}

Evaluate the expression \passthrough{\lstinline!expr!} in the Z3S5
Machine Lisp interpreter and return the result. The evaluation
environment is the system's environment at the time of the call.

See also: \passthrough{\lstinline!break, apply!}.

\hypertarget{even-procedure1-1}{%
\subsection{\texorpdfstring{\texttt{even?} :
procedure/1}{even? : procedure/1}}\label{even-procedure1-1}}

Usage: \passthrough{\lstinline!(even? n) => bool!}

Returns true if the integer \passthrough{\lstinline!n!} is even, nil if
it is not even.

See also: \passthrough{\lstinline!odd?!}.

\hypertarget{exists-procedure2-1}{%
\subsection{\texorpdfstring{\texttt{exists?} :
procedure/2}{exists? : procedure/2}}\label{exists-procedure2-1}}

Usage: \passthrough{\lstinline!(exists? seq pred) => bool!}

Return true if \passthrough{\lstinline!pred!} returns true for at least
one element in sequence \passthrough{\lstinline!seq!}, nil otherwise.

See also:
\passthrough{\lstinline!forall?, list-exists?, array-exists?, str-exists?, seq?!}.

\hypertarget{exit-procedure0-or-more-1}{%
\subsection{\texorpdfstring{\texttt{exit} : procedure/0 or
more}{exit : procedure/0 or more}}\label{exit-procedure0-or-more-1}}

Usage: \passthrough{\lstinline!(exit [n])!}

Immediately shut down the system and return OS host error code
\passthrough{\lstinline!n!}. The shutdown is performed gracefully and
exit hooks are executed.

See also: \passthrough{\lstinline!n/a!}.

\hypertarget{expect-macro2-1}{%
\subsection{\texorpdfstring{\texttt{expect} :
macro/2}{expect : macro/2}}\label{expect-macro2-1}}

Usage: \passthrough{\lstinline!(expect value given)!}

Registers a test under the current test name that checks that
\passthrough{\lstinline!value!} is returned by
\passthrough{\lstinline!given!}. The test is only executed when
(run-selftest) is executed.

See also:
\passthrough{\lstinline!expect-err, expect-ok, run-selftest, testing!}.

\hypertarget{expect-err-macro1-or-more-1}{%
\subsection{\texorpdfstring{\texttt{expect-err} : macro/1 or
more}{expect-err : macro/1 or more}}\label{expect-err-macro1-or-more-1}}

Usage: \passthrough{\lstinline!(expect-err expr ...)!}

Registers a test under the current test name that checks that
\passthrough{\lstinline!expr!} produces an error.

See also:
\passthrough{\lstinline!expect, expect-ok, run-selftest, testing!}.

\hypertarget{expect-false-macro1-or-more-1}{%
\subsection{\texorpdfstring{\texttt{expect-false} : macro/1 or
more}{expect-false : macro/1 or more}}\label{expect-false-macro1-or-more-1}}

Usage: \passthrough{\lstinline!(expect-false expr ...)!}

Registers a test under the current test name that checks that
\passthrough{\lstinline!expr!} is nil.

See also:
\passthrough{\lstinline!expect, expect-ok, run-selftest, testing!}.

\hypertarget{expect-ok-macro1-or-more-1}{%
\subsection{\texorpdfstring{\texttt{expect-ok} : macro/1 or
more}{expect-ok : macro/1 or more}}\label{expect-ok-macro1-or-more-1}}

Usage: \passthrough{\lstinline!(expect-err expr ...)!}

Registers a test under the current test name that checks that
\passthrough{\lstinline!expr!} does not produce an error.

See also:
\passthrough{\lstinline!expect, expect-ok, run-selftest, testing!}.

\hypertarget{expect-true-macro1-or-more-1}{%
\subsection{\texorpdfstring{\texttt{expect-true} : macro/1 or
more}{expect-true : macro/1 or more}}\label{expect-true-macro1-or-more-1}}

Usage: \passthrough{\lstinline!(expect-true expr ...)!}

Registers a test under the current test name that checks that
\passthrough{\lstinline!expr!} is true (not nil).

See also:
\passthrough{\lstinline!expect, expect-ok, run-selftest, testing!}.

\hypertarget{expr-str-procedure1-1}{%
\subsection{\texorpdfstring{\texttt{expr-\textgreater{}str} :
procedure/1}{expr-\textgreater str : procedure/1}}\label{expr-str-procedure1-1}}

Usage: \passthrough{\lstinline!(expr->str expr) => str!}

Convert a Lisp expression \passthrough{\lstinline!expr!} into a string.
Does not use a stream port.

See also:
\passthrough{\lstinline!str->expr, str->expr*, openstr, internalize, externalize!}.

\hypertarget{fdelete-procedure1-1}{%
\subsection{\texorpdfstring{\texttt{fdelete} :
procedure/1}{fdelete : procedure/1}}\label{fdelete-procedure1-1}}

Usage: \passthrough{\lstinline!(fdelete path)!}

Removes the file or directory at \passthrough{\lstinline!path.!}

See also: \passthrough{\lstinline!file-exists?, dir?, dir!}.

\textbf{Warning: This function also deletes directories containing files
and all of their subdirectories!}

\hypertarget{feature-procedure1-1}{%
\subsection{\texorpdfstring{\texttt{feature?} :
procedure/1}{feature? : procedure/1}}\label{feature-procedure1-1}}

Usage: \passthrough{\lstinline!(feature? sym) => bool!}

Return true if the Lisp feature identified by symbol
\passthrough{\lstinline!sym!} is available, nil otherwise.

See also: \passthrough{\lstinline!*reflect*, on-feature!}.

\hypertarget{file-port-procedure1-1}{%
\subsection{\texorpdfstring{\texttt{file-port?} :
procedure/1}{file-port? : procedure/1}}\label{file-port-procedure1-1}}

Usage: \passthrough{\lstinline!(file-port? p) => bool!}

Return true if \passthrough{\lstinline!p!} is a file port, nil
otherwise.

See also: \passthrough{\lstinline!port?, str-port?, open, stropen!}.

\hypertarget{filter-procedure2-1}{%
\subsection{\texorpdfstring{\texttt{filter} :
procedure/2}{filter : procedure/2}}\label{filter-procedure2-1}}

Usage: \passthrough{\lstinline!(filter li pred) => li!}

Return the list based on \passthrough{\lstinline!li!} with each element
removed for which \passthrough{\lstinline!pred!} returns nil.

See also: \passthrough{\lstinline!list!}.

\hypertarget{find-missing-help-entries-procedure0-1}{%
\subsection{\texorpdfstring{\texttt{find-missing-help-entries} :
procedure/0}{find-missing-help-entries : procedure/0}}\label{find-missing-help-entries-procedure0-1}}

Usage: \passthrough{\lstinline!(find-missing-help-entries) => li!}

Return a list of global symbols for which help entries are missing.

See also:
\passthrough{\lstinline!dump, dump-bindings, find-unneeded-help-entries!}.

\hypertarget{find-unneeded-help-entries-procedure0-1}{%
\subsection{\texorpdfstring{\texttt{find-unneeded-help-entries} :
procedure/0}{find-unneeded-help-entries : procedure/0}}\label{find-unneeded-help-entries-procedure0-1}}

Usage: \passthrough{\lstinline!(find-unneeded-help-entries) => li!}

Return a list of help entries for which no symbols are defined.

See also:
\passthrough{\lstinline!dump, dump-bindings, find-missing-help-entries!}.

\hypertarget{fl.abs-procedure1-1}{%
\subsection{\texorpdfstring{\texttt{fl.abs} :
procedure/1}{fl.abs : procedure/1}}\label{fl.abs-procedure1-1}}

Usage: \passthrough{\lstinline!(fl.abs x) => fl!}

Return the absolute value of \passthrough{\lstinline!x.!}

See also: \passthrough{\lstinline!float, *!}.

\hypertarget{fl.acos-procedure1-1}{%
\subsection{\texorpdfstring{\texttt{fl.acos} :
procedure/1}{fl.acos : procedure/1}}\label{fl.acos-procedure1-1}}

Usage: \passthrough{\lstinline!(fl.acos x) => fl!}

Return the arc cosine of \passthrough{\lstinline!x.!}

See also: \passthrough{\lstinline!fl.cos!}.

\hypertarget{fl.asin-procedure1-1}{%
\subsection{\texorpdfstring{\texttt{fl.asin} :
procedure/1}{fl.asin : procedure/1}}\label{fl.asin-procedure1-1}}

Usage: \passthrough{\lstinline!(fl.asin x) => fl!}

Return the arc sine of \passthrough{\lstinline!x.!}

See also: \passthrough{\lstinline!fl.acos!}.

\hypertarget{fl.asinh-procedure1-1}{%
\subsection{\texorpdfstring{\texttt{fl.asinh} :
procedure/1}{fl.asinh : procedure/1}}\label{fl.asinh-procedure1-1}}

Usage: \passthrough{\lstinline!(fl.asinh x) => fl!}

Return the inverse hyperbolic sine of \passthrough{\lstinline!x.!}

See also: \passthrough{\lstinline!fl.cosh!}.

\hypertarget{fl.atan-procedure1-1}{%
\subsection{\texorpdfstring{\texttt{fl.atan} :
procedure/1}{fl.atan : procedure/1}}\label{fl.atan-procedure1-1}}

Usage: \passthrough{\lstinline!(fl.atan x) => fl!}

Return the arctangent of \passthrough{\lstinline!x!} in radians.

See also: \passthrough{\lstinline!fl.atanh, fl.tan!}.

\hypertarget{fl.atan2-procedure2-1}{%
\subsection{\texorpdfstring{\texttt{fl.atan2} :
procedure/2}{fl.atan2 : procedure/2}}\label{fl.atan2-procedure2-1}}

Usage: \passthrough{\lstinline!(fl.atan2 x y) => fl!}

Atan2 returns the arc tangent of \passthrough{\lstinline!y!} /
\passthrough{\lstinline!x!}, using the signs of the two to determine the
quadrant of the return value.

See also: \passthrough{\lstinline!fl.atan!}.

\hypertarget{fl.atanh-procedure1-1}{%
\subsection{\texorpdfstring{\texttt{fl.atanh} :
procedure/1}{fl.atanh : procedure/1}}\label{fl.atanh-procedure1-1}}

Usage: \passthrough{\lstinline!(fl.atanh x) => fl!}

Return the inverse hyperbolic tangent of \passthrough{\lstinline!x.!}

See also: \passthrough{\lstinline!fl.atan!}.

\hypertarget{fl.cbrt-procedure1-1}{%
\subsection{\texorpdfstring{\texttt{fl.cbrt} :
procedure/1}{fl.cbrt : procedure/1}}\label{fl.cbrt-procedure1-1}}

Usage: \passthrough{\lstinline!(fl.cbrt x) => fl!}

Return the cube root of \passthrough{\lstinline!x.!}

See also: \passthrough{\lstinline!fl.sqrt!}.

\hypertarget{fl.ceil-procedure1-1}{%
\subsection{\texorpdfstring{\texttt{fl.ceil} :
procedure/1}{fl.ceil : procedure/1}}\label{fl.ceil-procedure1-1}}

Usage: \passthrough{\lstinline!(fl.ceil x) => fl!}

Round \passthrough{\lstinline!x!} up to the nearest integer, return it
as a floating point number.

See also:
\passthrough{\lstinline!fl.floor, truncate, int, fl.round, fl.trunc!}.

\hypertarget{fl.cos-procedure1-1}{%
\subsection{\texorpdfstring{\texttt{fl.cos} :
procedure/1}{fl.cos : procedure/1}}\label{fl.cos-procedure1-1}}

Usage: \passthrough{\lstinline!(fl.cos x) => fl!}

Return the cosine of \passthrough{\lstinline!x.!}

See also: \passthrough{\lstinline!fl.sin!}.

\hypertarget{fl.cosh-procedure1-1}{%
\subsection{\texorpdfstring{\texttt{fl.cosh} :
procedure/1}{fl.cosh : procedure/1}}\label{fl.cosh-procedure1-1}}

Usage: \passthrough{\lstinline!(fl.cosh x) => fl!}

Return the hyperbolic cosine of \passthrough{\lstinline!x.!}

See also: \passthrough{\lstinline!fl.cos!}.

\hypertarget{fl.dim-procedure2-1}{%
\subsection{\texorpdfstring{\texttt{fl.dim} :
procedure/2}{fl.dim : procedure/2}}\label{fl.dim-procedure2-1}}

Usage: \passthrough{\lstinline!(fl.dim x y) => fl!}

Return the maximum of x, y or 0.

See also: \passthrough{\lstinline!max!}.

\hypertarget{fl.erf-procedure1-1}{%
\subsection{\texorpdfstring{\texttt{fl.erf} :
procedure/1}{fl.erf : procedure/1}}\label{fl.erf-procedure1-1}}

Usage: \passthrough{\lstinline!(fl.erf x) => fl!}

Return the result of the error function of \passthrough{\lstinline!x.!}

See also: \passthrough{\lstinline!fl.erfc, fl.dim!}.

\hypertarget{fl.erfc-procedure1-1}{%
\subsection{\texorpdfstring{\texttt{fl.erfc} :
procedure/1}{fl.erfc : procedure/1}}\label{fl.erfc-procedure1-1}}

Usage: \passthrough{\lstinline!(fl.erfc x) => fl!}

Return the result of the complementary error function of
\passthrough{\lstinline!x.!}

See also: \passthrough{\lstinline!fl.erfcinv, fl.erf!}.

\hypertarget{fl.erfcinv-procedure1-1}{%
\subsection{\texorpdfstring{\texttt{fl.erfcinv} :
procedure/1}{fl.erfcinv : procedure/1}}\label{fl.erfcinv-procedure1-1}}

Usage: \passthrough{\lstinline!(fl.erfcinv x) => fl!}

Return the inverse of (fl.erfc \passthrough{\lstinline!x!}).

See also: \passthrough{\lstinline!fl.erfc!}.

\hypertarget{fl.erfinv-procedure1-1}{%
\subsection{\texorpdfstring{\texttt{fl.erfinv} :
procedure/1}{fl.erfinv : procedure/1}}\label{fl.erfinv-procedure1-1}}

Usage: \passthrough{\lstinline!(fl.erfinv x) => fl!}

Return the inverse of (fl.erf \passthrough{\lstinline!x!}).

See also: \passthrough{\lstinline!fl.erf!}.

\hypertarget{fl.exp-procedure1-1}{%
\subsection{\texorpdfstring{\texttt{fl.exp} :
procedure/1}{fl.exp : procedure/1}}\label{fl.exp-procedure1-1}}

Usage: \passthrough{\lstinline!(fl.exp x) => fl!}

Return e\^{}\passthrough{\lstinline!x!}, the base-e exponential of
\passthrough{\lstinline!x.!}

See also: \passthrough{\lstinline!fl.exp!}.

\hypertarget{fl.exp2-procedure2-1}{%
\subsection{\texorpdfstring{\texttt{fl.exp2} :
procedure/2}{fl.exp2 : procedure/2}}\label{fl.exp2-procedure2-1}}

Usage: \passthrough{\lstinline!(fl.exp2 x) => fl!}

Return 2\^{}\passthrough{\lstinline!x!}, the base-2 exponential of
\passthrough{\lstinline!x.!}

See also: \passthrough{\lstinline!fl.exp!}.

\hypertarget{fl.expm1-procedure1-1}{%
\subsection{\texorpdfstring{\texttt{fl.expm1} :
procedure/1}{fl.expm1 : procedure/1}}\label{fl.expm1-procedure1-1}}

Usage: \passthrough{\lstinline!(fl.expm1 x) => fl!}

Return e\^{}\passthrough{\lstinline!x-1!}, the base-e exponential of
(sub1 \passthrough{\lstinline!x!}). This is more accurate than (sub1
(fl.exp \passthrough{\lstinline!x!})) when \passthrough{\lstinline!x!}
is very small.

See also: \passthrough{\lstinline!fl.exp!}.

\hypertarget{fl.floor-procedure1-1}{%
\subsection{\texorpdfstring{\texttt{fl.floor} :
procedure/1}{fl.floor : procedure/1}}\label{fl.floor-procedure1-1}}

Usage: \passthrough{\lstinline!(fl.floor x) => fl!}

Return \passthrough{\lstinline!x!} rounded to the nearest integer below
as floating point number.

See also: \passthrough{\lstinline!fl.ceil, truncate, int!}.

\hypertarget{fl.fma-procedure3-1}{%
\subsection{\texorpdfstring{\texttt{fl.fma} :
procedure/3}{fl.fma : procedure/3}}\label{fl.fma-procedure3-1}}

Usage: \passthrough{\lstinline!(fl.fma x y z) => fl!}

Return the fused multiply-add of \passthrough{\lstinline!x!},
\passthrough{\lstinline!y!}, \passthrough{\lstinline!z!}, which is
\passthrough{\lstinline!x!} * \passthrough{\lstinline!y!} +
\passthrough{\lstinline!z.!}

See also: \passthrough{\lstinline!*, +!}.

\hypertarget{fl.frexp-procedure1-1}{%
\subsection{\texorpdfstring{\texttt{fl.frexp} :
procedure/1}{fl.frexp : procedure/1}}\label{fl.frexp-procedure1-1}}

Usage: \passthrough{\lstinline!(fl.frexp x) => li!}

Break \passthrough{\lstinline!x!} into a normalized fraction and an
integral power of two. It returns a list of (frac exp) containing a
float and an integer satisfying \passthrough{\lstinline!x!} ==
\passthrough{\lstinline!frac!} × 2\^{}\passthrough{\lstinline!exp!}
where the absolute value of \passthrough{\lstinline!frac!} is in the
interval {[}0.5, 1).

See also: \passthrough{\lstinline!fl.exp!}.

\hypertarget{fl.gamma-procedure1-1}{%
\subsection{\texorpdfstring{\texttt{fl.gamma} :
procedure/1}{fl.gamma : procedure/1}}\label{fl.gamma-procedure1-1}}

Usage: \passthrough{\lstinline!(fl.gamma x) => fl!}

Compute the Gamma function of \passthrough{\lstinline!x.!}

See also: \passthrough{\lstinline!fl.lgamma!}.

\hypertarget{fl.hypot-procedure2-1}{%
\subsection{\texorpdfstring{\texttt{fl.hypot} :
procedure/2}{fl.hypot : procedure/2}}\label{fl.hypot-procedure2-1}}

Usage: \passthrough{\lstinline!(fl.hypot x y) => fl!}

Compute the square root of x\^{}2 and y\^{}2.

See also: \passthrough{\lstinline!fl.sqrt!}.

\hypertarget{fl.ilogb-procedure1-1}{%
\subsection{\texorpdfstring{\texttt{fl.ilogb} :
procedure/1}{fl.ilogb : procedure/1}}\label{fl.ilogb-procedure1-1}}

Usage: \passthrough{\lstinline!(fl.ilogb x) => fl!}

Return the binary exponent of \passthrough{\lstinline!x!} as a floating
point number.

See also: \passthrough{\lstinline!fl.exp2!}.

\hypertarget{fl.inf-procedure1-1}{%
\subsection{\texorpdfstring{\texttt{fl.inf} :
procedure/1}{fl.inf : procedure/1}}\label{fl.inf-procedure1-1}}

Usage: \passthrough{\lstinline!(fl.inf x) => fl!}

Return positive 64 bit floating point infinity +INF if
\passthrough{\lstinline!x!} \textgreater= 0 and negative 64 bit floating
point finfinity -INF if \passthrough{\lstinline!x!} \textless{} 0.

See also: \passthrough{\lstinline!fl.is-nan?!}.

\hypertarget{fl.is-nan-procedure1-1}{%
\subsection{\texorpdfstring{\texttt{fl.is-nan?} :
procedure/1}{fl.is-nan? : procedure/1}}\label{fl.is-nan-procedure1-1}}

Usage: \passthrough{\lstinline!(fl.is-nan? x) => bool!}

Return true if \passthrough{\lstinline!x!} is not a number according to
IEEE 754 floating point arithmetics, nil otherwise.

See also: \passthrough{\lstinline!fl.inf!}.

\hypertarget{fl.j0-procedure1-1}{%
\subsection{\texorpdfstring{\texttt{fl.j0} :
procedure/1}{fl.j0 : procedure/1}}\label{fl.j0-procedure1-1}}

Usage: \passthrough{\lstinline!(fl.j0 x) => fl!}

Apply the order-zero Bessel function of the first kind to
\passthrough{\lstinline!x.!}

See also: \passthrough{\lstinline!fl.j1, fl.jn, fl.y0, fl.y1, fl.yn!}.

\hypertarget{fl.j1-procedure1-1}{%
\subsection{\texorpdfstring{\texttt{fl.j1} :
procedure/1}{fl.j1 : procedure/1}}\label{fl.j1-procedure1-1}}

Usage: \passthrough{\lstinline!(fl.j1 x) => fl!}

Apply the the order-one Bessel function of the first kind
\passthrough{\lstinline!x.!}

See also: \passthrough{\lstinline!fl.j0, fl.jn, fl.y0, fl.y1, fl.yn!}.

\hypertarget{fl.jn-procedure1-1}{%
\subsection{\texorpdfstring{\texttt{fl.jn} :
procedure/1}{fl.jn : procedure/1}}\label{fl.jn-procedure1-1}}

Usage: \passthrough{\lstinline!(fl.jn n x) => fl!}

Apply the Bessel function of order \passthrough{\lstinline!n!} to
\passthrough{\lstinline!x!}. The number \passthrough{\lstinline!n!} must
be an integer.

See also: \passthrough{\lstinline!fl.j1, fl.j0, fl.y0, fl.y1, fl.yn!}.

\hypertarget{fl.ldexp-procedure2-1}{%
\subsection{\texorpdfstring{\texttt{fl.ldexp} :
procedure/2}{fl.ldexp : procedure/2}}\label{fl.ldexp-procedure2-1}}

Usage: \passthrough{\lstinline!(fl.ldexp x n) => fl!}

Return the inverse of fl.frexp, \passthrough{\lstinline!x!} *
2\^{}\passthrough{\lstinline!n.!}

See also: \passthrough{\lstinline!fl.frexp!}.

\hypertarget{fl.lgamma-procedure1-1}{%
\subsection{\texorpdfstring{\texttt{fl.lgamma} :
procedure/1}{fl.lgamma : procedure/1}}\label{fl.lgamma-procedure1-1}}

Usage: \passthrough{\lstinline!(fl.lgamma x) => li!}

Return a list containing the natural logarithm and sign (-1 or +1) of
the Gamma function applied to \passthrough{\lstinline!x.!}

See also: \passthrough{\lstinline!fl.gamma!}.

\hypertarget{fl.log-procedure1-1}{%
\subsection{\texorpdfstring{\texttt{fl.log} :
procedure/1}{fl.log : procedure/1}}\label{fl.log-procedure1-1}}

Usage: \passthrough{\lstinline!(fl.log x) => fl!}

Return the natural logarithm of \passthrough{\lstinline!x.!}

See also:
\passthrough{\lstinline!fl.log10, fl.log2, fl.logb, fl.log1p!}.

\hypertarget{fl.log10-procedure1-1}{%
\subsection{\texorpdfstring{\texttt{fl.log10} :
procedure/1}{fl.log10 : procedure/1}}\label{fl.log10-procedure1-1}}

Usage: \passthrough{\lstinline!(fl.log10 x) => fl!}

Return the decimal logarithm of \passthrough{\lstinline!x.!}

See also: \passthrough{\lstinline!fl.log, fl.log2, fl.logb, fl.log1p!}.

\hypertarget{fl.log1p-procedure1-1}{%
\subsection{\texorpdfstring{\texttt{fl.log1p} :
procedure/1}{fl.log1p : procedure/1}}\label{fl.log1p-procedure1-1}}

Usage: \passthrough{\lstinline!(fl.log1p x) => fl!}

Return the natural logarithm of \passthrough{\lstinline!x!} + 1. This
function is more accurate than (fl.log (add1 x)) if
\passthrough{\lstinline!x!} is close to 0.

See also: \passthrough{\lstinline!fl.log, fl.log2, fl.logb, fl.log10!}.

\hypertarget{fl.log2-procedure1-1}{%
\subsection{\texorpdfstring{\texttt{fl.log2} :
procedure/1}{fl.log2 : procedure/1}}\label{fl.log2-procedure1-1}}

Usage: \passthrough{\lstinline!(fl.log2 x) => fl!}

Return the binary logarithm of \passthrough{\lstinline!x!}. This is
important for calculating entropy, for example.

See also: \passthrough{\lstinline!fl.log, fl.log10, fl.log1p, fl.logb!}.

\hypertarget{fl.logb-procedure1-1}{%
\subsection{\texorpdfstring{\texttt{fl.logb} :
procedure/1}{fl.logb : procedure/1}}\label{fl.logb-procedure1-1}}

Usage: \passthrough{\lstinline!(fl.logb x) => fl!}

Return the binary exponent of \passthrough{\lstinline!x.!}

See also:
\passthrough{\lstinline!fl.log, fl.log10, fl.log1p, fl.logb, fl.log2!}.

\hypertarget{fl.max-procedure2-1}{%
\subsection{\texorpdfstring{\texttt{fl.max} :
procedure/2}{fl.max : procedure/2}}\label{fl.max-procedure2-1}}

Usage: \passthrough{\lstinline!(fl.max x y) => fl!}

Return the larger value of two floating point arguments
\passthrough{\lstinline!x!} and \passthrough{\lstinline!y.!}

See also: \passthrough{\lstinline!fl.min, max, min!}.

\hypertarget{fl.min-procedure2-1}{%
\subsection{\texorpdfstring{\texttt{fl.min} :
procedure/2}{fl.min : procedure/2}}\label{fl.min-procedure2-1}}

Usage: \passthrough{\lstinline!(fl.min x y) => fl!}

Return the smaller value of two floating point arguments
\passthrough{\lstinline!x!} and \passthrough{\lstinline!y.!}

See also: \passthrough{\lstinline!fl.min, max, min!}.

\hypertarget{fl.mod-procedure2-1}{%
\subsection{\texorpdfstring{\texttt{fl.mod} :
procedure/2}{fl.mod : procedure/2}}\label{fl.mod-procedure2-1}}

Usage: \passthrough{\lstinline!(fl.mod x y) => fl!}

Return the floating point remainder of \passthrough{\lstinline!x!} /
\passthrough{\lstinline!y.!}

See also: \passthrough{\lstinline!fl.remainder!}.

\hypertarget{fl.modf-procedure1-1}{%
\subsection{\texorpdfstring{\texttt{fl.modf} :
procedure/1}{fl.modf : procedure/1}}\label{fl.modf-procedure1-1}}

Usage: \passthrough{\lstinline!(fl.modf x) => li!}

Return integer and fractional floating-point numbers that sum to
\passthrough{\lstinline!x!}. Both values have the same sign as
\passthrough{\lstinline!x.!}

See also: \passthrough{\lstinline!fl.mod!}.

\hypertarget{fl.nan-procedure1-1}{%
\subsection{\texorpdfstring{\texttt{fl.nan} :
procedure/1}{fl.nan : procedure/1}}\label{fl.nan-procedure1-1}}

Usage: \passthrough{\lstinline!(fl.nan) => fl!}

Return the IEEE 754 not-a-number value.

See also: \passthrough{\lstinline!fl.is-nan?, fl.inf!}.

\hypertarget{fl.next-after-procedure1-1}{%
\subsection{\texorpdfstring{\texttt{fl.next-after} :
procedure/1}{fl.next-after : procedure/1}}\label{fl.next-after-procedure1-1}}

Usage: \passthrough{\lstinline!(fl.next-after x) => fl!}

Return the next representable floating point number after
\passthrough{\lstinline!x.!}

See also: \passthrough{\lstinline!fl.is-nan?, fl.nan, fl.inf!}.

\hypertarget{fl.pow-procedure2-1}{%
\subsection{\texorpdfstring{\texttt{fl.pow} :
procedure/2}{fl.pow : procedure/2}}\label{fl.pow-procedure2-1}}

Usage: \passthrough{\lstinline!(fl.pow x y) => fl!}

Return \passthrough{\lstinline!x!} to the power of
\passthrough{\lstinline!y!} according to 64 bit floating point
arithmetics.

See also: \passthrough{\lstinline!fl.pow10!}.

\hypertarget{fl.pow10-procedure1-1}{%
\subsection{\texorpdfstring{\texttt{fl.pow10} :
procedure/1}{fl.pow10 : procedure/1}}\label{fl.pow10-procedure1-1}}

Usage: \passthrough{\lstinline!(fl.pow10 n) => fl!}

Return 10 to the power of integer \passthrough{\lstinline!n!} as a 64
bit floating point number.

See also: \passthrough{\lstinline!fl.pow!}.

\hypertarget{fl.remainder-procedure2-1}{%
\subsection{\texorpdfstring{\texttt{fl.remainder} :
procedure/2}{fl.remainder : procedure/2}}\label{fl.remainder-procedure2-1}}

Usage: \passthrough{\lstinline!(fl.remainder x y) => fl!}

Return the IEEE 754 floating-point remainder of
\passthrough{\lstinline!x!} / \passthrough{\lstinline!y.!}

See also: \passthrough{\lstinline!fl.mod!}.

\hypertarget{fl.round-procedure1-1}{%
\subsection{\texorpdfstring{\texttt{fl.round} :
procedure/1}{fl.round : procedure/1}}\label{fl.round-procedure1-1}}

Usage: \passthrough{\lstinline!(fl.round x) => fl!}

Round \passthrough{\lstinline!x!} to the nearest integer floating point
number according to floating point arithmetics.

See also:
\passthrough{\lstinline!fl.round-to-even, fl.truncate, int, float!}.

\hypertarget{fl.round-to-even-procedure1-1}{%
\subsection{\texorpdfstring{\texttt{fl.round-to-even} :
procedure/1}{fl.round-to-even : procedure/1}}\label{fl.round-to-even-procedure1-1}}

Usage: \passthrough{\lstinline!(fl.round-to-even x) => fl!}

Round \passthrough{\lstinline!x!} to the nearest even integer floating
point number according to floating point arithmetics.

See also: \passthrough{\lstinline!fl.round, fl.truncate, int, float!}.

\hypertarget{fl.signbit-procedure1-1}{%
\subsection{\texorpdfstring{\texttt{fl.signbit} :
procedure/1}{fl.signbit : procedure/1}}\label{fl.signbit-procedure1-1}}

Usage: \passthrough{\lstinline!(fl.signbit x) => bool!}

Return true if \passthrough{\lstinline!x!} is negative, nil otherwise.

See also: \passthrough{\lstinline!fl.abs!}.

\hypertarget{fl.sin-procedure1-1}{%
\subsection{\texorpdfstring{\texttt{fl.sin} :
procedure/1}{fl.sin : procedure/1}}\label{fl.sin-procedure1-1}}

Usage: \passthrough{\lstinline!(fl.sin x) => fl!}

Return the sine of \passthrough{\lstinline!x.!}

See also: \passthrough{\lstinline!fl.cos!}.

\hypertarget{fl.sinh-procedure1-1}{%
\subsection{\texorpdfstring{\texttt{fl.sinh} :
procedure/1}{fl.sinh : procedure/1}}\label{fl.sinh-procedure1-1}}

Usage: \passthrough{\lstinline!(fl.sinh x) => fl!}

Return the hyperbolic sine of \passthrough{\lstinline!x.!}

See also: \passthrough{\lstinline!fl.sin!}.

\hypertarget{fl.sqrt-procedure1-1}{%
\subsection{\texorpdfstring{\texttt{fl.sqrt} :
procedure/1}{fl.sqrt : procedure/1}}\label{fl.sqrt-procedure1-1}}

Usage: \passthrough{\lstinline!(fl.sqrt x) => fl!}

Return the square root of \passthrough{\lstinline!x.!}

See also: \passthrough{\lstinline!fl.pow!}.

\hypertarget{fl.tan-procedure1-1}{%
\subsection{\texorpdfstring{\texttt{fl.tan} :
procedure/1}{fl.tan : procedure/1}}\label{fl.tan-procedure1-1}}

Usage: \passthrough{\lstinline!(fl.tan x) => fl!}

Return the tangent of \passthrough{\lstinline!x!} in radian.

See also: \passthrough{\lstinline!fl.tanh, fl.sin, fl.cos!}.

\hypertarget{fl.tanh-procedure1-1}{%
\subsection{\texorpdfstring{\texttt{fl.tanh} :
procedure/1}{fl.tanh : procedure/1}}\label{fl.tanh-procedure1-1}}

Usage: \passthrough{\lstinline!(fl.tanh x) => fl!}

Return the hyperbolic tangent of \passthrough{\lstinline!x.!}

See also: \passthrough{\lstinline!fl.tan, flsinh, fl.cosh!}.

\hypertarget{fl.trunc-procedure1-1}{%
\subsection{\texorpdfstring{\texttt{fl.trunc} :
procedure/1}{fl.trunc : procedure/1}}\label{fl.trunc-procedure1-1}}

Usage: \passthrough{\lstinline!(fl.trunc x) => fl!}

Return the integer value of \passthrough{\lstinline!x!} as floating
point number.

See also: \passthrough{\lstinline!truncate, int, fl.floor!}.

\hypertarget{fl.y0-procedure1-1}{%
\subsection{\texorpdfstring{\texttt{fl.y0} :
procedure/1}{fl.y0 : procedure/1}}\label{fl.y0-procedure1-1}}

Usage: \passthrough{\lstinline!(fl.y0 x) => fl!}

Return the order-zero Bessel function of the second kind applied to
\passthrough{\lstinline!x.!}

See also: \passthrough{\lstinline!fl.y1, fl.yn, fl.j0, fl.j1, fl.jn!}.

\hypertarget{fl.y1-procedure1-1}{%
\subsection{\texorpdfstring{\texttt{fl.y1} :
procedure/1}{fl.y1 : procedure/1}}\label{fl.y1-procedure1-1}}

Usage: \passthrough{\lstinline!(fl.y1 x) => fl!}

Return the order-one Bessel function of the second kind applied to
\passthrough{\lstinline!x.!}

See also: \passthrough{\lstinline!fl.y0, fl.yn, fl.j0, fl.j1, fl.jn!}.

\hypertarget{fl.yn-procedure1-1}{%
\subsection{\texorpdfstring{\texttt{fl.yn} :
procedure/1}{fl.yn : procedure/1}}\label{fl.yn-procedure1-1}}

Usage: \passthrough{\lstinline!(fl.yn n x) => fl!}

Return the Bessel function of the second kind of order
\passthrough{\lstinline!n!} applied to \passthrough{\lstinline!x!}.
Argument \passthrough{\lstinline!n!} must be an integer value.

See also: \passthrough{\lstinline!fl.y0, fl.y1, fl.j0, fl.j1, fl.jn!}.

\hypertarget{flatten-procedure1-1}{%
\subsection{\texorpdfstring{\texttt{flatten} :
procedure/1}{flatten : procedure/1}}\label{flatten-procedure1-1}}

Usage: \passthrough{\lstinline!(flatten lst) => list!}

Flatten \passthrough{\lstinline!lst!}, making all elements of sublists
elements of the flattened list.

See also: \passthrough{\lstinline!car, cdr, remove-duplicates!}.

\hypertarget{float-procedure1-1}{%
\subsection{\texorpdfstring{\texttt{float} :
procedure/1}{float : procedure/1}}\label{float-procedure1-1}}

Usage: \passthrough{\lstinline!(float n) => float!}

Convert \passthrough{\lstinline!n!} to a floating point value.

See also: \passthrough{\lstinline!int!}.

\hypertarget{fmt-procedure1-or-more-1}{%
\subsection{\texorpdfstring{\texttt{fmt} : procedure/1 or
more}{fmt : procedure/1 or more}}\label{fmt-procedure1-or-more-1}}

Usage: \passthrough{\lstinline!(fmt s [args] ...) => str!}

Format string \passthrough{\lstinline!s!} that contains format
directives with arbitrary many \passthrough{\lstinline!args!} as
arguments. The number of format directives must match the number of
arguments. The format directives are the same as those for the esoteric
and arcane programming language ``Go'', which was used on Earth for some
time.

See also: \passthrough{\lstinline!out!}.

\hypertarget{forall-procedure2-1}{%
\subsection{\texorpdfstring{\texttt{forall?} :
procedure/2}{forall? : procedure/2}}\label{forall-procedure2-1}}

Usage: \passthrough{\lstinline!(forall? seq pred) => bool!}

Return true if predicate \passthrough{\lstinline!pred!} returns true for
all elements of sequence \passthrough{\lstinline!seq!}, nil otherwise.

See also:
\passthrough{\lstinline!foreach, map, list-forall?, array-forall?, str-forall?, exists?, str-exists?, array-exists?, list-exists?!}.

\hypertarget{force-procedure1-1}{%
\subsection{\texorpdfstring{\texttt{force} :
procedure/1}{force : procedure/1}}\label{force-procedure1-1}}

Usage: \passthrough{\lstinline!(force fut) => any!}

Obtain the value of the computation encapsulated by future
\passthrough{\lstinline!fut!}, halting the current task until it has
been obtained. If the future never ends computation, e.g.~in an infinite
loop, the program may halt indefinitely.

See also: \passthrough{\lstinline!future, task, make-mutex!}.

\hypertarget{foreach-procedure2-1}{%
\subsection{\texorpdfstring{\texttt{foreach} :
procedure/2}{foreach : procedure/2}}\label{foreach-procedure2-1}}

Usage: \passthrough{\lstinline!(foreach seq proc)!}

Apply \passthrough{\lstinline!proc!} to each element of sequence
\passthrough{\lstinline!seq!} in order, for the side effects.

See also: \passthrough{\lstinline!seq?, map!}.

\hypertarget{functional-arity-procedure1-1}{%
\subsection{\texorpdfstring{\texttt{functional-arity} :
procedure/1}{functional-arity : procedure/1}}\label{functional-arity-procedure1-1}}

Usage: \passthrough{\lstinline!(functional-arity proc) => int!}

Return the arity of a functional \passthrough{\lstinline!proc.!}

See also: \passthrough{\lstinline!functional?, functional-has-rest?!}.

\hypertarget{functional-has-rest-procedure1-1}{%
\subsection{\texorpdfstring{\texttt{functional-has-rest?} :
procedure/1}{functional-has-rest? : procedure/1}}\label{functional-has-rest-procedure1-1}}

Usage: \passthrough{\lstinline!(functional-has-rest? proc) => bool!}

Return true if the functional \passthrough{\lstinline!proc!} has a
\&rest argument, nil otherwise.

See also: \passthrough{\lstinline!functional?, functional-arity!}.

\hypertarget{functional-macro1-1}{%
\subsection{\texorpdfstring{\texttt{functional?} :
macro/1}{functional? : macro/1}}\label{functional-macro1-1}}

Usage: \passthrough{\lstinline!(functional? arg) => bool!}

Return true if \passthrough{\lstinline!arg!} is either a builtin
function, a closure, or a macro, nil otherwise. This is the right
predicate for testing whether the argument is applicable and has an
arity.

See also:
\passthrough{\lstinline!closure?, proc?, functional-arity, functional-has-rest?!}.

\hypertarget{gensym-procedure0-1}{%
\subsection{\texorpdfstring{\texttt{gensym} :
procedure/0}{gensym : procedure/0}}\label{gensym-procedure0-1}}

Usage: \passthrough{\lstinline!(gensym) => sym!}

Return a new symbol guaranteed to be unique during runtime.

See also: \passthrough{\lstinline!nonce!}.

\hypertarget{get-procedure2-or-more-1}{%
\subsection{\texorpdfstring{\texttt{get} : procedure/2 or
more}{get : procedure/2 or more}}\label{get-procedure2-or-more-1}}

Usage: \passthrough{\lstinline!(get dict key [default]) => any!}

Get the value for \passthrough{\lstinline!key!} in
\passthrough{\lstinline!dict!}, return \passthrough{\lstinline!default!}
if there is no value for \passthrough{\lstinline!key!}. If
\passthrough{\lstinline!default!} is omitted, then nil is returned.
Provide your own default if you want to store nil.

See also: \passthrough{\lstinline!dict, dict?, set!}.

\hypertarget{get-or-set-procedure3-1}{%
\subsection{\texorpdfstring{\texttt{get-or-set} :
procedure/3}{get-or-set : procedure/3}}\label{get-or-set-procedure3-1}}

Usage: \passthrough{\lstinline!(get-or-set d key value)!}

Get the value for \passthrough{\lstinline!key!} in dict
\passthrough{\lstinline!d!} if it already exists, otherwise set it to
\passthrough{\lstinline!value.!}

See also: \passthrough{\lstinline!dict?, get, set!}.

\hypertarget{get-partitions-procedure2-1}{%
\subsection{\texorpdfstring{\texttt{get-partitions} :
procedure/2}{get-partitions : procedure/2}}\label{get-partitions-procedure2-1}}

Usage: \passthrough{\lstinline!(get-partitions x n) => proc/1*!}

Return an iterator procedure that returns lists of the form
(start-offset end-offset bytes) with 0-index offsets for a given index
\passthrough{\lstinline!k!}, or nil if there is no corresponding part,
such that the sizes of the partitions returned in
\passthrough{\lstinline!bytes!} summed up are
\passthrough{\lstinline!x!} and and each partition is
\passthrough{\lstinline!n!} or lower in size. The last partition will be
the smallest partition with a \passthrough{\lstinline!bytes!} value
smaller than \passthrough{\lstinline!n!} if \passthrough{\lstinline!x!}
is not dividable without rest by \passthrough{\lstinline!n!}. If no
argument is provided for the returned iterator, then it returns the
number of partitions.

See also:
\passthrough{\lstinline!nth-partition, count-partitions, get-file-partitions, iterate!}.

\hypertarget{getstacked-procedure3-1}{%
\subsection{\texorpdfstring{\texttt{getstacked} :
procedure/3}{getstacked : procedure/3}}\label{getstacked-procedure3-1}}

Usage: \passthrough{\lstinline!(getstacked dict key default)!}

Get the topmost element from the stack stored at
\passthrough{\lstinline!key!} in \passthrough{\lstinline!dict!}. If the
stack is empty or no stack is stored at key, then
\passthrough{\lstinline!default!} is returned.

See also: \passthrough{\lstinline!pushstacked, popstacked!}.

\hypertarget{glance-procedure1-1}{%
\subsection{\texorpdfstring{\texttt{glance} :
procedure/1}{glance : procedure/1}}\label{glance-procedure1-1}}

Usage: \passthrough{\lstinline!(glance s [def]) => any!}

Peek the next element in a stack or queue without changing the data
structure. If default \passthrough{\lstinline!def!} is provided, this is
returned in case the stack or queue is empty; otherwise nil is returned.

See also:
\passthrough{\lstinline"make-queue, make-stack, queue?, enqueue?, dequeue?, queue-len, stack-len, pop!, push!"}.

\hypertarget{has-procedure2-1}{%
\subsection{\texorpdfstring{\texttt{has} :
procedure/2}{has : procedure/2}}\label{has-procedure2-1}}

Usage: \passthrough{\lstinline!(has dict key) => bool!}

Return true if the dict \passthrough{\lstinline!dict!} contains an entry
for \passthrough{\lstinline!key!}, nil otherwise.

See also: \passthrough{\lstinline!dict, get, set!}.

\hypertarget{has-key-procedure2-1}{%
\subsection{\texorpdfstring{\texttt{has-key?} :
procedure/2}{has-key? : procedure/2}}\label{has-key-procedure2-1}}

Usage: \passthrough{\lstinline!(has-key? d key) => bool!}

Return true if \passthrough{\lstinline!d!} has key
\passthrough{\lstinline!key!}, nil otherwise.

See also: \passthrough{\lstinline!dict?, get, set, delete!}.

\hypertarget{help-macro1-1}{%
\subsection{\texorpdfstring{\texttt{help} :
macro/1}{help : macro/1}}\label{help-macro1-1}}

Usage: \passthrough{\lstinline!(help sym)!}

Display help information about \passthrough{\lstinline!sym!} (unquoted).

See also:
\passthrough{\lstinline!defhelp, help-entry, *help*, apropos!}.

\hypertarget{help-manual-entry-nil-1}{%
\subsection{help-\textgreater manual-entry :
nil}\label{help-manual-entry-nil-1}}

Usage: \passthrough{\lstinline!(help->manual-entry key [level]) => str!}

Looks up help for \passthrough{\lstinline!key!} and converts it to a
manual section as markdown string. If there is no entry for
\passthrough{\lstinline!key!}, then nil is returned. The optional
\passthrough{\lstinline!level!} integer indicates the heading nesting.

See also: \passthrough{\lstinline!help!}.

\hypertarget{help-about-procedure1-or-more-1}{%
\subsection{\texorpdfstring{\texttt{help-about} : procedure/1 or
more}{help-about : procedure/1 or more}}\label{help-about-procedure1-or-more-1}}

Usage: \passthrough{\lstinline!(help-about topic [sel]) => li!}

Obtain a list of symbols for which help about
\passthrough{\lstinline!topic!} is available. If optional
\passthrough{\lstinline!sel!} argument is left out or
\passthrough{\lstinline!any!}, then any symbols with which the topic is
associated are listed. If the optional \passthrough{\lstinline!sel!}
argument is \passthrough{\lstinline!first!}, then a symbol is only
listed if it has \passthrough{\lstinline!topic!} as first topic entry.
This restricts the number of entries returned to a more essential
selection.

See also: \passthrough{\lstinline!help-topics, help, apropos!}.

\hypertarget{help-entry-procedure1-1}{%
\subsection{\texorpdfstring{\texttt{help-entry} :
procedure/1}{help-entry : procedure/1}}\label{help-entry-procedure1-1}}

Usage: \passthrough{\lstinline!(help-entry sym) => list!}

Get usage and help information for \passthrough{\lstinline!sym.!}

See also: \passthrough{\lstinline!defhelp, help, apropos, *help*!}.

\hypertarget{help-topic-info-procedure1-1}{%
\subsection{\texorpdfstring{\texttt{help-topic-info} :
procedure/1}{help-topic-info : procedure/1}}\label{help-topic-info-procedure1-1}}

Usage: \passthrough{\lstinline!(help-topic-info topic) => li!}

Return a list containing a heading and an info string for help
\passthrough{\lstinline!topic!}, or nil if no info is available.

See also: \passthrough{\lstinline!set-help-topic-info, defhelp, help!}.

\hypertarget{help-topics-procedure0-1}{%
\subsection{\texorpdfstring{\texttt{help-topics} :
procedure/0}{help-topics : procedure/0}}\label{help-topics-procedure0-1}}

Usage: \passthrough{\lstinline!(help-topics) => li!}

Obtain a list of help topics for commands.

See also: \passthrough{\lstinline!help, help-topic, apropos!}.

\hypertarget{hex-blob-procedure1-1}{%
\subsection{\texorpdfstring{\texttt{hex-\textgreater{}blob} :
procedure/1}{hex-\textgreater blob : procedure/1}}\label{hex-blob-procedure1-1}}

Usage: \passthrough{\lstinline!(hex->blob str) => blob!}

Convert hex string \passthrough{\lstinline!str!} to a blob. This will
raise an error if \passthrough{\lstinline!str!} is not a valid hex
string.

See also:
\passthrough{\lstinline!blob->hex, base64->blob, ascii85->blob, str->blob!}.

\hypertarget{hook-procedure1-1}{%
\subsection{\texorpdfstring{\texttt{hook} :
procedure/1}{hook : procedure/1}}\label{hook-procedure1-1}}

Usage: \passthrough{\lstinline!(hook symbol)!}

Lookup the internal hook number from a symbolic name.

See also:
\passthrough{\lstinline!*hooks*, add-hook, remove-hook, remove-hooks!}.

\hypertarget{hour-procedure2-1}{%
\subsection{\texorpdfstring{\texttt{hour+} :
procedure/2}{hour+ : procedure/2}}\label{hour-procedure2-1}}

Usage: \passthrough{\lstinline!(hour+ dateli n) => dateli!}

Adds \passthrough{\lstinline!n!} hours to the given date
\passthrough{\lstinline!dateli!} in datelist format and returns the new
datelist.

See also:
\passthrough{\lstinline!sec+, minute+, day+, week+, month+, year+, now!}.

\hypertarget{identity-procedure1-1}{%
\subsection{\texorpdfstring{\texttt{identity} :
procedure/1}{identity : procedure/1}}\label{identity-procedure1-1}}

Usage: \passthrough{\lstinline!(identity x)!}

Return \passthrough{\lstinline!x.!}

See also: \passthrough{\lstinline!apply, equal?!}.

\hypertarget{if-macro3-1}{%
\subsection{\texorpdfstring{\texttt{if} :
macro/3}{if : macro/3}}\label{if-macro3-1}}

Usage: \passthrough{\lstinline!(if cond expr1 expr2) => any!}

Evaluate \passthrough{\lstinline!expr1!} if
\passthrough{\lstinline!cond!} is true, otherwise evaluate
\passthrough{\lstinline!expr2.!}

See also: \passthrough{\lstinline!cond, when, unless!}.

\hypertarget{inchars-procedure2-1}{%
\subsection{\texorpdfstring{\texttt{inchars} :
procedure/2}{inchars : procedure/2}}\label{inchars-procedure2-1}}

Usage: \passthrough{\lstinline!(inchars char chars) => bool!}

Return true if char is in the charset chars, nil otherwise.

See also: \passthrough{\lstinline!chars, dict, get, set, has!}.

\hypertarget{include-procedure1-1}{%
\subsection{\texorpdfstring{\texttt{include} :
procedure/1}{include : procedure/1}}\label{include-procedure1-1}}

Usage: \passthrough{\lstinline!(include fi) => any!}

Evaluate the lisp file \passthrough{\lstinline!fi!} one expression after
the other in the current environment.

See also: \passthrough{\lstinline!read, write, open, close!}.

\hypertarget{index-procedure2-or-more-1}{%
\subsection{\texorpdfstring{\texttt{index} : procedure/2 or
more}{index : procedure/2 or more}}\label{index-procedure2-or-more-1}}

Usage: \passthrough{\lstinline!(index seq elem [pred]) => int!}

Return the first index of \passthrough{\lstinline!elem!} in
\passthrough{\lstinline!seq!} going from left to right, using equality
predicate \passthrough{\lstinline!pred!} for comparisons (default is
eq?). If \passthrough{\lstinline!elem!} is not in
\passthrough{\lstinline!seq!}, -1 is returned.

See also: \passthrough{\lstinline!nth, seq?!}.

\hypertarget{instr-procedure2-1}{%
\subsection{\texorpdfstring{\texttt{instr} :
procedure/2}{instr : procedure/2}}\label{instr-procedure2-1}}

Usage: \passthrough{\lstinline!(instr s1 s2) => int!}

Return the index of the first occurrence of \passthrough{\lstinline!s2!}
in \passthrough{\lstinline!s1!} (from left), or -1 if
\passthrough{\lstinline!s1!} does not contain
\passthrough{\lstinline!s2.!}

See also: \passthrough{\lstinline!str?, index!}.

\hypertarget{int-procedure1-1}{%
\subsection{\texorpdfstring{\texttt{int} :
procedure/1}{int : procedure/1}}\label{int-procedure1-1}}

Usage: \passthrough{\lstinline!(int n) => int!}

Return \passthrough{\lstinline!n!} as an integer, rounding down to the
nearest integer if necessary.

See also: \passthrough{\lstinline!float!}.

\textbf{Warning: If the number is very large this may result in
returning the maximum supported integer number rather than the number as
integer.}

\hypertarget{intern-procedure1-1}{%
\subsection{\texorpdfstring{\texttt{intern} :
procedure/1}{intern : procedure/1}}\label{intern-procedure1-1}}

Usage: \passthrough{\lstinline!(intern s) => sym!}

Create a new interned symbol based on string
\passthrough{\lstinline!s.!}

See also: \passthrough{\lstinline!gensym, str->sym, make-symbol!}.

\hypertarget{intrinsic-procedure1-2}{%
\subsection{\texorpdfstring{\texttt{intrinsic} :
procedure/1}{intrinsic : procedure/1}}\label{intrinsic-procedure1-2}}

Usage: \passthrough{\lstinline!(intrinsic sym) => any!}

Attempt to obtain the value that is intrinsically bound to
\passthrough{\lstinline!sym!}. Use this function to express the
intention to use the pre-defined builtin value of a symbol in the base
language.

See also: \passthrough{\lstinline!bind, unbind!}.

\textbf{Warning: This function currently only returns the binding but
this behavior might change in future.}

\hypertarget{intrinsic-procedure1-3}{%
\subsection{\texorpdfstring{\texttt{intrinsic?} :
procedure/1}{intrinsic? : procedure/1}}\label{intrinsic-procedure1-3}}

Usage: \passthrough{\lstinline!(intrinsic? x) => bool!}

Return true if \passthrough{\lstinline!x!} is an intrinsic built-in
function, nil otherwise. Notice that this function tests the value and
not that a symbol has been bound to the intrinsic.

See also: \passthrough{\lstinline!functional?, macro?, closure?!}.

\textbf{Warning: What counts as an intrinsic or not may change from
version to version. This is for internal use only.}

\hypertarget{iterate-procedure2-1}{%
\subsection{\texorpdfstring{\texttt{iterate} :
procedure/2}{iterate : procedure/2}}\label{iterate-procedure2-1}}

Usage: \passthrough{\lstinline!(iterate it proc)!}

Apply \passthrough{\lstinline!proc!} to each argument returned by
iterator \passthrough{\lstinline!it!} in sequence, similar to the way
foreach works. An iterator is a procedure that takes one integer as
argument or no argument at all. If no argument is provided, the iterator
returns the number of iterations. If an integer is provided, the
iterator returns a non-nil value for the given index.

See also: \passthrough{\lstinline!foreach, get-partitions!}.

\hypertarget{last-procedure1-or-more-1}{%
\subsection{\texorpdfstring{\texttt{last} : procedure/1 or
more}{last : procedure/1 or more}}\label{last-procedure1-or-more-1}}

Usage: \passthrough{\lstinline!(last seq [default]) => any!}

Get the last element of sequence \passthrough{\lstinline!seq!} or return
\passthrough{\lstinline!default!} if the sequence is empty. If
\passthrough{\lstinline!default!} is not given and the sequence is
empty, an error is raised.

See also:
\passthrough{\lstinline!nth, nthdef, car, list-ref, array-ref, string, ref, 1st, 2nd, 3rd, 4th, 5th, 6th, 7th, 8th, 9th, 10th!}.

\hypertarget{lcons-procedure2-1}{%
\subsection{\texorpdfstring{\texttt{lcons} :
procedure/2}{lcons : procedure/2}}\label{lcons-procedure2-1}}

Usage: \passthrough{\lstinline!(lcons datum li) => list!}

Insert \passthrough{\lstinline!datum!} at the end of the list
\passthrough{\lstinline!li!}. There may be a more efficient
implementation of this in the future. Or, maybe not. Who knows?

See also: \passthrough{\lstinline!cons, list, append, nreverse!}.

\hypertarget{len-procedure1-1}{%
\subsection{\texorpdfstring{\texttt{len} :
procedure/1}{len : procedure/1}}\label{len-procedure1-1}}

Usage: \passthrough{\lstinline!(len seq) => int!}

Return the length of \passthrough{\lstinline!seq!}. Works for lists,
strings, arrays, and dicts.

See also: \passthrough{\lstinline!seq?!}.

\hypertarget{let-macro1-or-more-1}{%
\subsection{\texorpdfstring{\texttt{let} : macro/1 or
more}{let : macro/1 or more}}\label{let-macro1-or-more-1}}

Usage: \passthrough{\lstinline!(let args body ...) => any!}

Bind each pair of symbol and expression in
\passthrough{\lstinline!args!} and evaluate the expressions in
\passthrough{\lstinline!body!} with these local bindings. Return the
value of the last expression in \passthrough{\lstinline!body.!}

See also: \passthrough{\lstinline!letrec!}.

\hypertarget{letrec-macro1-or-more-1}{%
\subsection{\texorpdfstring{\texttt{letrec} : macro/1 or
more}{letrec : macro/1 or more}}\label{letrec-macro1-or-more-1}}

Usage: \passthrough{\lstinline!(letrec args body ...) => any!}

Recursive let binds the symbol, expression pairs in
\passthrough{\lstinline!args!} in a way that makes prior bindings
available to later bindings and allows for recursive definitions in
\passthrough{\lstinline!args!}, then evaluates the
\passthrough{\lstinline!body!} expressions with these bindings.

See also: \passthrough{\lstinline!let!}.

\hypertarget{lighten-procedure1-1}{%
\subsection{\texorpdfstring{\texttt{lighten} :
procedure/1}{lighten : procedure/1}}\label{lighten-procedure1-1}}

Usage: \passthrough{\lstinline!(lighten color [amount]) => (r g b a)!}

Return a lighter version of \passthrough{\lstinline!color!}. The
optional positive \passthrough{\lstinline!amount!} specifies the amount
of lightening (0-255).

See also: \passthrough{\lstinline!the-color, *colors*, darken!}.

\hypertarget{ling.damerau-levenshtein-procedure2-1}{%
\subsection{\texorpdfstring{\texttt{ling.damerau-levenshtein} :
procedure/2}{ling.damerau-levenshtein : procedure/2}}\label{ling.damerau-levenshtein-procedure2-1}}

Usage: \passthrough{\lstinline!(ling.damerau-levenshtein s1 s2) => num!}

Compute the Damerau-Levenshtein distance between
\passthrough{\lstinline!s1!} and \passthrough{\lstinline!s2.!}

See also:
\passthrough{\lstinline!ling.match-rating-compare, ling.levenshtein, ling.jaro-winkler, ling.jaro, ling.hamming, ling.match-rating-codex, ling.porter, ling.nysiis, ling.metaphone, ling.soundex!}.

\hypertarget{ling.hamming-procedure2-1}{%
\subsection{\texorpdfstring{\texttt{ling.hamming} :
procedure/2}{ling.hamming : procedure/2}}\label{ling.hamming-procedure2-1}}

Usage: \passthrough{\lstinline!(ling-hamming s1 s2) => num!}

Compute the Hamming distance between \passthrough{\lstinline!s1!} and
\passthrough{\lstinline!s2.!}

See also:
\passthrough{\lstinline!ling.match-rating-compare, ling.levenshtein, ling.jaro-winkler, ling.jaro, ling.damerau-levenshtein, ling.match-rating-codex, ling.porter, ling.nysiis, ling.metaphone, ling.soundex!}.

\hypertarget{ling.jaro-procedure2-1}{%
\subsection{\texorpdfstring{\texttt{ling.jaro} :
procedure/2}{ling.jaro : procedure/2}}\label{ling.jaro-procedure2-1}}

Usage: \passthrough{\lstinline!(ling.jaro s1 s2) => num!}

Compute the Jaro distance between \passthrough{\lstinline!s1!} and
\passthrough{\lstinline!s2.!}

See also:
\passthrough{\lstinline!ling.match-rating-compare, ling.levenshtein, ling.jaro-winkler, ling.hamming, ling.damerau-levenshtein, ling.match-rating-codex, ling.porter, ling.nysiis, ling.metaphone, ling.soundex!}.

\hypertarget{ling.jaro-winkler-procedure2-1}{%
\subsection{\texorpdfstring{\texttt{ling.jaro-winkler} :
procedure/2}{ling.jaro-winkler : procedure/2}}\label{ling.jaro-winkler-procedure2-1}}

Usage: \passthrough{\lstinline!(ling.jaro-winkler s1 s2) => num!}

Compute the Jaro-Winkler distance between \passthrough{\lstinline!s1!}
and \passthrough{\lstinline!s2.!}

See also:
\passthrough{\lstinline!ling.match-rating-compare, ling.levenshtein, ling.jaro, ling.hamming, ling.damerau-levenshtein, ling.match-rating-codex, ling.porter, ling.nysiis, ling.metaphone, ling.soundex!}.

\hypertarget{ling.levenshtein-procedure2-1}{%
\subsection{\texorpdfstring{\texttt{ling.levenshtein} :
procedure/2}{ling.levenshtein : procedure/2}}\label{ling.levenshtein-procedure2-1}}

Usage: \passthrough{\lstinline!(ling.levenshtein s1 s2) => num!}

Compute the Levenshtein distance between \passthrough{\lstinline!s1!}
and \passthrough{\lstinline!s2.!}

See also:
\passthrough{\lstinline!ling.match-rating-compare, ling.jaro-winkler, ling.jaro, ling.hamming, ling.damerau-levenshtein, ling.match-rating-codex, ling.porter, ling.nysiis, ling.metaphone, ling.soundex!}.

\hypertarget{ling.match-rating-codex-procedure1-1}{%
\subsection{\texorpdfstring{\texttt{ling.match-rating-codex} :
procedure/1}{ling.match-rating-codex : procedure/1}}\label{ling.match-rating-codex-procedure1-1}}

Usage: \passthrough{\lstinline!(ling.match-rating-codex s) => str!}

Compute the Match-Rating-Codex of string \passthrough{\lstinline!s.!}

See also:
\passthrough{\lstinline!ling.match-rating-compare, ling.levenshtein, ling.jaro-winkler, ling.jaro, ling.hamming, ling.damerau-levenshtein, ling.porter, ling.nysiis, ling.metaphone, ling.soundex!}.

\hypertarget{ling.match-rating-compare-procedure2-1}{%
\subsection{\texorpdfstring{\texttt{ling.match-rating-compare} :
procedure/2}{ling.match-rating-compare : procedure/2}}\label{ling.match-rating-compare-procedure2-1}}

Usage:
\passthrough{\lstinline!(ling.match-rating-compare s1 s2) => bool!}

Returns true if \passthrough{\lstinline!s1!} and
\passthrough{\lstinline!s2!} are equal according to the Match-rating
Comparison algorithm, nil otherwise.

See also:
\passthrough{\lstinline!ling.match-rating-compare, ling.levenshtein, ling.jaro-winkler, ling.jaro, ling.hamming, ling.damerau-levenshtein, ling.match-rating-codex, ling.porter, ling.nysiis, ling.metaphone, ling.soundex!}.

\hypertarget{ling.metaphone-procedure1-1}{%
\subsection{\texorpdfstring{\texttt{ling.metaphone} :
procedure/1}{ling.metaphone : procedure/1}}\label{ling.metaphone-procedure1-1}}

Usage: \passthrough{\lstinline!(ling.metaphone s) => str!}

Compute the Metaphone representation of string
\passthrough{\lstinline!s.!}

See also:
\passthrough{\lstinline!ling.match-rating-compare, ling.levenshtein, ling.jaro-winkler, ling.jaro, ling.hamming, ling.damerau-levenshtein, ling.match-rating-codex, ling.porter, ling.nysiis, ling.soundex!}.

\hypertarget{ling.nysiis-procedure1-1}{%
\subsection{\texorpdfstring{\texttt{ling.nysiis} :
procedure/1}{ling.nysiis : procedure/1}}\label{ling.nysiis-procedure1-1}}

Usage: \passthrough{\lstinline!(ling.nysiis s) => str!}

Compute the Nysiis representation of string \passthrough{\lstinline!s.!}

See also:
\passthrough{\lstinline!ling.match-rating-compare, ling.levenshtein, ling.jaro-winkler, ling.jaro, ling.hamming, ling.damerau-levenshtein, ling.match-rating-codex, ling.porter, ling.metaphone, ling.soundex!}.

\hypertarget{ling.porter-procedure1-1}{%
\subsection{\texorpdfstring{\texttt{ling.porter} :
procedure/1}{ling.porter : procedure/1}}\label{ling.porter-procedure1-1}}

Usage: \passthrough{\lstinline!(ling.porter s) => str!}

Compute the stem of word string \passthrough{\lstinline!s!} using the
Porter stemming algorithm.

See also:
\passthrough{\lstinline!ling.match-rating-compare, ling.levenshtein, ling.jaro-winkler, ling.jaro, ling.hamming, ling.damerau-levenshtein, ling.match-rating-codex, ling.nysiis, ling.metaphone, ling.soundex!}.

\hypertarget{ling.soundex-procedure1-1}{%
\subsection{\texorpdfstring{\texttt{ling.soundex} :
procedure/1}{ling.soundex : procedure/1}}\label{ling.soundex-procedure1-1}}

Usage: \passthrough{\lstinline!(ling.soundex s) => str!}

Compute the Soundex representation of string
\passthrough{\lstinline!s.!}

See also:
\passthrough{\lstinline!ling.match-rating-compare, ling.levenshtein, ling.jaro-winkler, ling.jaro, ling.hamming, ling.damerau-levenshtein, ling.match-rating-codex, ling.porter, ling.nysiis, ling.metaphone, ling.soundex!}.

\hypertarget{list-procedure0-or-more-1}{%
\subsection{\texorpdfstring{\texttt{list} : procedure/0 or
more}{list : procedure/0 or more}}\label{list-procedure0-or-more-1}}

Usage: \passthrough{\lstinline!(list [args] ...) => li!}

Create a list from all \passthrough{\lstinline!args!}. The arguments
must be quoted.

See also: \passthrough{\lstinline!cons!}.

\hypertarget{list-array-procedure1-1}{%
\subsection{\texorpdfstring{\texttt{list-\textgreater{}array} :
procedure/1}{list-\textgreater array : procedure/1}}\label{list-array-procedure1-1}}

Usage: \passthrough{\lstinline!(list->array li) => array!}

Convert the list \passthrough{\lstinline!li!} to an array.

See also: \passthrough{\lstinline!list, array, string, nth, seq?!}.

\hypertarget{list-set-procedure1-1}{%
\subsection{\texorpdfstring{\texttt{list-\textgreater{}set} :
procedure/1}{list-\textgreater set : procedure/1}}\label{list-set-procedure1-1}}

Usage: \passthrough{\lstinline!(list->set li) => dict!}

Create a dict containing true for each element of list
\passthrough{\lstinline!li.!}

See also:
\passthrough{\lstinline!make-set, set-element?, set-union, set-intersection, set-complement, set-difference, set?, set-empty!}.

\hypertarget{list-str-procedure1-1}{%
\subsection{\texorpdfstring{\texttt{list-\textgreater{}str} :
procedure/1}{list-\textgreater str : procedure/1}}\label{list-str-procedure1-1}}

Usage: \passthrough{\lstinline!(list->str li) => string!}

Return the string that is composed out of the chars in list
\passthrough{\lstinline!li.!}

See also: \passthrough{\lstinline!array->str, str->list, chars!}.

\hypertarget{list-exists-procedure2-1}{%
\subsection{\texorpdfstring{\texttt{list-exists?} :
procedure/2}{list-exists? : procedure/2}}\label{list-exists-procedure2-1}}

Usage: \passthrough{\lstinline!(list-exists? li pred) => bool!}

Return true if \passthrough{\lstinline!pred!} returns true for at least
one element in list \passthrough{\lstinline!li!}, nil otherwise.

See also:
\passthrough{\lstinline!exists?, forall?, array-exists?, str-exists?, seq?!}.

\hypertarget{list-forall-procedure2-1}{%
\subsection{\texorpdfstring{\texttt{list-forall?} :
procedure/2}{list-forall? : procedure/2}}\label{list-forall-procedure2-1}}

Usage: \passthrough{\lstinline!(list-all? li pred) => bool!}

Return true if predicate \passthrough{\lstinline!pred!} returns true for
all elements of list \passthrough{\lstinline!li!}, nil otherwise.

See also:
\passthrough{\lstinline!foreach, map, forall?, array-forall?, str-forall?, exists?!}.

\hypertarget{list-foreach-procedure2-1}{%
\subsection{\texorpdfstring{\texttt{list-foreach} :
procedure/2}{list-foreach : procedure/2}}\label{list-foreach-procedure2-1}}

Usage: \passthrough{\lstinline!(list-foreach li proc)!}

Apply \passthrough{\lstinline!proc!} to each element of list
\passthrough{\lstinline!li!} in order, for the side effects.

See also: \passthrough{\lstinline!mapcar, map, foreach!}.

\hypertarget{list-last-procedure1-1}{%
\subsection{\texorpdfstring{\texttt{list-last} :
procedure/1}{list-last : procedure/1}}\label{list-last-procedure1-1}}

Usage: \passthrough{\lstinline!(list-last li) => any!}

Return the last element of \passthrough{\lstinline!li.!}

See also: \passthrough{\lstinline!reverse, nreverse, car, 1st, last!}.

\hypertarget{list-ref-procedure2-1}{%
\subsection{\texorpdfstring{\texttt{list-ref} :
procedure/2}{list-ref : procedure/2}}\label{list-ref-procedure2-1}}

Usage: \passthrough{\lstinline!(list-ref li n) => any!}

Return the element with index \passthrough{\lstinline!n!} of list
\passthrough{\lstinline!li!}. Lists are 0-indexed.

See also: \passthrough{\lstinline!array-ref, nth!}.

\hypertarget{list-reverse-procedure1-1}{%
\subsection{\texorpdfstring{\texttt{list-reverse} :
procedure/1}{list-reverse : procedure/1}}\label{list-reverse-procedure1-1}}

Usage: \passthrough{\lstinline!(list-reverse li) => li!}

Create a reversed copy of \passthrough{\lstinline!li.!}

See also: \passthrough{\lstinline!reverse, array-reverse, str-reverse!}.

\hypertarget{list-slice-procedure3-1}{%
\subsection{\texorpdfstring{\texttt{list-slice} :
procedure/3}{list-slice : procedure/3}}\label{list-slice-procedure3-1}}

Usage: \passthrough{\lstinline!(list-slice li low high) => li!}

Return the slice of the list \passthrough{\lstinline!li!} starting at
index \passthrough{\lstinline!low!} (inclusive) and ending at index
\passthrough{\lstinline!high!} (exclusive).

See also: \passthrough{\lstinline!slice, array-slice!}.

\hypertarget{list-procedure1-1}{%
\subsection{\texorpdfstring{\texttt{list?} :
procedure/1}{list? : procedure/1}}\label{list-procedure1-1}}

Usage: \passthrough{\lstinline!(list? obj) => bool!}

Return true if \passthrough{\lstinline!obj!} is a list, nil otherwise.

See also: \passthrough{\lstinline!cons?, atom?, null?!}.

\hypertarget{macro-procedure1-1}{%
\subsection{\texorpdfstring{\texttt{macro?} :
procedure/1}{macro? : procedure/1}}\label{macro-procedure1-1}}

Usage: \passthrough{\lstinline!(macro? x) => bool!}

Return true if \passthrough{\lstinline!x!} is a macro, nil otherwise.

See also:
\passthrough{\lstinline!functional?, intrinsic?, closure?, functional-arity, functional-has-rest?!}.

\hypertarget{make-blob-procedure1-1}{%
\subsection{\texorpdfstring{\texttt{make-blob} :
procedure/1}{make-blob : procedure/1}}\label{make-blob-procedure1-1}}

Usage: \passthrough{\lstinline!(make-blob n) => blob!}

Make a binary blob of size \passthrough{\lstinline!n!} initialized to
zeroes.

See also: \passthrough{\lstinline!blob-free, valid?, blob-equal?!}.

\hypertarget{make-mutex-procedure1-1}{%
\subsection{\texorpdfstring{\texttt{make-mutex} :
procedure/1}{make-mutex : procedure/1}}\label{make-mutex-procedure1-1}}

Usage: \passthrough{\lstinline!(make-mutex) => mutex!}

Create a new mutex.

See also:
\passthrough{\lstinline!mutex-lock, mutex-unlock, mutex-rlock, mutex-runlock!}.

\hypertarget{make-queue-procedure0-1}{%
\subsection{\texorpdfstring{\texttt{make-queue} :
procedure/0}{make-queue : procedure/0}}\label{make-queue-procedure0-1}}

Usage: \passthrough{\lstinline!(make-queue) => array!}

Make a synchronized queue.

See also:
\passthrough{\lstinline"queue?, enqueue!, dequeue!, glance, queue-empty?, queue-len"}.

\textbf{Warning: Never change the array of a synchronized data structure
directly, or your warranty is void!}

\hypertarget{make-set-procedure0-or-more-1}{%
\subsection{\texorpdfstring{\texttt{make-set} : procedure/0 or
more}{make-set : procedure/0 or more}}\label{make-set-procedure0-or-more-1}}

Usage: \passthrough{\lstinline!(make-set [arg1] ... [argn]) => dict!}

Create a dictionary out of arguments \passthrough{\lstinline!arg1!} to
\passthrough{\lstinline!argn!} that stores true for very argument.

See also:
\passthrough{\lstinline!list->set, set->list, set-element?, set-union, set-intersection, set-complement, set-difference, set?, set-empty?!}.

\hypertarget{make-stack-procedure0-1}{%
\subsection{\texorpdfstring{\texttt{make-stack} :
procedure/0}{make-stack : procedure/0}}\label{make-stack-procedure0-1}}

Usage: \passthrough{\lstinline!(make-stack) => array!}

Make a synchronized stack.

See also:
\passthrough{\lstinline"stack?, push!, pop!, stack-empty?, stack-len, glance"}.

\textbf{Warning: Never change the array of a synchronized data structure
directly, or your warranty is void!}

\hypertarget{make-symbol-procedure1-1}{%
\subsection{\texorpdfstring{\texttt{make-symbol} :
procedure/1}{make-symbol : procedure/1}}\label{make-symbol-procedure1-1}}

Usage: \passthrough{\lstinline!(make-symbol s) => sym!}

Create a new symbol based on string \passthrough{\lstinline!s.!}

See also: \passthrough{\lstinline!str->sym!}.

\hypertarget{map-procedure2-1}{%
\subsection{\texorpdfstring{\texttt{map} :
procedure/2}{map : procedure/2}}\label{map-procedure2-1}}

Usage: \passthrough{\lstinline!(map seq proc) => seq!}

Return the copy of \passthrough{\lstinline!seq!} that is the result of
applying \passthrough{\lstinline!proc!} to each element of
\passthrough{\lstinline!seq.!}

See also: \passthrough{\lstinline!seq?, mapcar, strmap!}.

\hypertarget{map-pairwise-procedure2-1}{%
\subsection{\texorpdfstring{\texttt{map-pairwise} :
procedure/2}{map-pairwise : procedure/2}}\label{map-pairwise-procedure2-1}}

Usage: \passthrough{\lstinline!(map-pairwise seq proc) => seq!}

Applies \passthrough{\lstinline!proc!} in order to subsequent pairs in
\passthrough{\lstinline!seq!}, assembling the sequence that results from
the results of \passthrough{\lstinline!proc!}. Function
\passthrough{\lstinline!proc!} takes two arguments and must return a
proper list containing two elements. If the number of elements in
\passthrough{\lstinline!seq!} is odd, an error is raised.

See also: \passthrough{\lstinline!map!}.

\hypertarget{mapcar-procedure2-1}{%
\subsection{\texorpdfstring{\texttt{mapcar} :
procedure/2}{mapcar : procedure/2}}\label{mapcar-procedure2-1}}

Usage: \passthrough{\lstinline!(mapcar li proc) => li!}

Return the list obtained from applying \passthrough{\lstinline!proc!} to
each elements in \passthrough{\lstinline!li.!}

See also: \passthrough{\lstinline!map, foreach!}.

\hypertarget{max-procedure1-or-more-1}{%
\subsection{\texorpdfstring{\texttt{max} : procedure/1 or
more}{max : procedure/1 or more}}\label{max-procedure1-or-more-1}}

Usage: \passthrough{\lstinline!(max x1 x2 ...) => num!}

Return the maximum of the given numbers.

See also: \passthrough{\lstinline!min, minmax!}.

\hypertarget{member-procedure2-1}{%
\subsection{\texorpdfstring{\texttt{member} :
procedure/2}{member : procedure/2}}\label{member-procedure2-1}}

Usage: \passthrough{\lstinline!(member key li) => li!}

Return the cdr of \passthrough{\lstinline!li!} starting with
\passthrough{\lstinline!key!} if \passthrough{\lstinline!li!} contains
an element equal? to \passthrough{\lstinline!key!}, nil otherwise.

See also: \passthrough{\lstinline!assoc, equal?!}.

\hypertarget{memq-procedure2-1}{%
\subsection{\texorpdfstring{\texttt{memq} :
procedure/2}{memq : procedure/2}}\label{memq-procedure2-1}}

Usage: \passthrough{\lstinline!(memq key li)!}

Return the cdr of \passthrough{\lstinline!li!} starting with
\passthrough{\lstinline!key!} if \passthrough{\lstinline!li!} contains
an element eq? to \passthrough{\lstinline!key!}, nil otherwise.

See also: \passthrough{\lstinline!member, eq?!}.

\hypertarget{memstats-procedure0-1}{%
\subsection{\texorpdfstring{\texttt{memstats} :
procedure/0}{memstats : procedure/0}}\label{memstats-procedure0-1}}

Usage: \passthrough{\lstinline!(memstats) => dict!}

Return a dict with detailed memory statistics for the system.

See also: \passthrough{\lstinline!collect-garbage!}.

\hypertarget{min-procedure1-or-more-1}{%
\subsection{\texorpdfstring{\texttt{min} : procedure/1 or
more}{min : procedure/1 or more}}\label{min-procedure1-or-more-1}}

Usage: \passthrough{\lstinline!(min x1 x2 ...) => num!}

Return the minimum of the given numbers.

See also: \passthrough{\lstinline!max, minmax!}.

\hypertarget{minmax-procedure3-1}{%
\subsection{\texorpdfstring{\texttt{minmax} :
procedure/3}{minmax : procedure/3}}\label{minmax-procedure3-1}}

Usage: \passthrough{\lstinline!(minmax pred li acc) => any!}

Go through \passthrough{\lstinline!li!} and test whether for each
\passthrough{\lstinline!elem!} the comparison (pred elem acc) is true.
If so, \passthrough{\lstinline!elem!} becomes
\passthrough{\lstinline!acc!}. Once all elements of the list have been
compared, \passthrough{\lstinline!acc!} is returned. This procedure can
be used to implement generalized minimum or maximum procedures.

See also: \passthrough{\lstinline!min, max!}.

\hypertarget{minute-procedure2-1}{%
\subsection{\texorpdfstring{\texttt{minute+} :
procedure/2}{minute+ : procedure/2}}\label{minute-procedure2-1}}

Usage: \passthrough{\lstinline!(minute+ dateli n) => dateli!}

Adds \passthrough{\lstinline!n!} minutes to the given date
\passthrough{\lstinline!dateli!} in datelist format and returns the new
datelist.

See also:
\passthrough{\lstinline!sec+, hour+, day+, week+, month+, year+, now!}.

\hypertarget{mod-procedure2-1}{%
\subsection{\texorpdfstring{\texttt{mod} :
procedure/2}{mod : procedure/2}}\label{mod-procedure2-1}}

Usage: \passthrough{\lstinline!(mod x y) => num!}

Compute \passthrough{\lstinline!x!} modulo \passthrough{\lstinline!y.!}

See also: \passthrough{\lstinline!\%, /!}.

\hypertarget{month-procedure2-1}{%
\subsection{\texorpdfstring{\texttt{month+} :
procedure/2}{month+ : procedure/2}}\label{month-procedure2-1}}

Usage: \passthrough{\lstinline!(month+ dateli n) => dateli!}

Adds \passthrough{\lstinline!n!} months to the given date
\passthrough{\lstinline!dateli!} in datelist format and returns the new
datelist.

See also:
\passthrough{\lstinline!sec+, minute+, hour+, day+, week+, year+, now!}.

\hypertarget{mutex-lock-procedure1-1}{%
\subsection{\texorpdfstring{\texttt{mutex-lock} :
procedure/1}{mutex-lock : procedure/1}}\label{mutex-lock-procedure1-1}}

Usage: \passthrough{\lstinline!(mutex-lock m)!}

Lock the mutex \passthrough{\lstinline!m!} for writing. This may halt
the current task until the mutex has been unlocked by another task.

See also:
\passthrough{\lstinline!mutex-unlock, make-mutex, mutex-rlock, mutex-runlock!}.

\hypertarget{mutex-rlock-procedure1-1}{%
\subsection{\texorpdfstring{\texttt{mutex-rlock} :
procedure/1}{mutex-rlock : procedure/1}}\label{mutex-rlock-procedure1-1}}

Usage: \passthrough{\lstinline!(mutex-rlock m)!}

Lock the mutex \passthrough{\lstinline!m!} for reading. This will allow
other tasks to read from it, too, but may block if another task is
currently locking it for writing.

See also:
\passthrough{\lstinline!mutex-runlock, mutex-lock, mutex-unlock, make-mutex!}.

\hypertarget{mutex-runlock-procedure1-1}{%
\subsection{\texorpdfstring{\texttt{mutex-runlock} :
procedure/1}{mutex-runlock : procedure/1}}\label{mutex-runlock-procedure1-1}}

Usage: \passthrough{\lstinline!(mutex-runlock m)!}

Unlock the mutex \passthrough{\lstinline!m!} from reading.

See also:
\passthrough{\lstinline!mutex-lock, mutex-unlock, mutex-rlock, make-mutex!}.

\hypertarget{mutex-unlock-procedure1-1}{%
\subsection{\texorpdfstring{\texttt{mutex-unlock} :
procedure/1}{mutex-unlock : procedure/1}}\label{mutex-unlock-procedure1-1}}

Usage: \passthrough{\lstinline!(mutex-unlock m)!}

Unlock the mutex \passthrough{\lstinline!m!} for writing. This releases
ownership of the mutex and allows other tasks to lock it for writing.

See also:
\passthrough{\lstinline!mutex-lock, make-mutex, mutex-rlock, mutex-runlock!}.

\hypertarget{nconc-procedure0-or-more-1}{%
\subsection{\texorpdfstring{\texttt{nconc} : procedure/0 or
more}{nconc : procedure/0 or more}}\label{nconc-procedure0-or-more-1}}

Usage: \passthrough{\lstinline!(nconc li1 li2 ...) => li!}

Concatenate \passthrough{\lstinline!li1!},
\passthrough{\lstinline!li2!}, and so forth, like with append, but
destructively modifies \passthrough{\lstinline!li1.!}

See also: \passthrough{\lstinline!append!}.

\hypertarget{nl-procedure0-1}{%
\subsection{\texorpdfstring{\texttt{nl} :
procedure/0}{nl : procedure/0}}\label{nl-procedure0-1}}

Usage: \passthrough{\lstinline!(nl)!}

Display a newline, advancing the cursor to the next line.

See also: \passthrough{\lstinline!out, outy, output-at!}.

\hypertarget{nonce-procedure0-1}{%
\subsection{\texorpdfstring{\texttt{nonce} :
procedure/0}{nonce : procedure/0}}\label{nonce-procedure0-1}}

Usage: \passthrough{\lstinline!(nonce) => str!}

Return a unique random string. This is not cryptographically secure but
the string satisfies reasonable GUID requirements.

See also: \passthrough{\lstinline!externalize, internalize!}.

\hypertarget{not-procedure1-1}{%
\subsection{\texorpdfstring{\texttt{not} :
procedure/1}{not : procedure/1}}\label{not-procedure1-1}}

Usage: \passthrough{\lstinline!(not x) => bool!}

Return true if \passthrough{\lstinline!x!} is nil, nil otherwise.

See also: \passthrough{\lstinline!and, or!}.

\hypertarget{now-procedure0-1}{%
\subsection{\texorpdfstring{\texttt{now} :
procedure/0}{now : procedure/0}}\label{now-procedure0-1}}

Usage: \passthrough{\lstinline!(now) => li!}

Return the current datetime in UTC format as a list of values in the
form '((year month day weekday iso-week) (hour minute second nanosecond
unix-nano-second)).

See also:
\passthrough{\lstinline!now-ns, datestr, time, date->epoch-ns, epoch-ns->datelist!}.

\hypertarget{now-ms-procedure0-1}{%
\subsection{\texorpdfstring{\texttt{now-ms} :
procedure/0}{now-ms : procedure/0}}\label{now-ms-procedure0-1}}

Usage: \passthrough{\lstinline!(now-ms) => num!}

Return the relative system time as a call to (now-ns) but in
milliseconds.

See also: \passthrough{\lstinline!now-ns, now!}.

\hypertarget{now-ns-procedure0-1}{%
\subsection{\texorpdfstring{\texttt{now-ns} :
procedure/0}{now-ns : procedure/0}}\label{now-ns-procedure0-1}}

Usage: \passthrough{\lstinline!(now-ns) => int!}

Return the current time in Unix nanoseconds.

See also: \passthrough{\lstinline!now, time!}.

\hypertarget{nreverse-procedure1-1}{%
\subsection{\texorpdfstring{\texttt{nreverse} :
procedure/1}{nreverse : procedure/1}}\label{nreverse-procedure1-1}}

Usage: \passthrough{\lstinline!(nreverse li) => li!}

Destructively reverse \passthrough{\lstinline!li.!}

See also: \passthrough{\lstinline!reverse!}.

\hypertarget{nth-procedure2-1}{%
\subsection{\texorpdfstring{\texttt{nth} :
procedure/2}{nth : procedure/2}}\label{nth-procedure2-1}}

Usage: \passthrough{\lstinline!(nth seq n) => any!}

Get the \passthrough{\lstinline!n-th!} element of sequence
\passthrough{\lstinline!seq!}. Sequences are 0-indexed.

See also:
\passthrough{\lstinline!nthdef, list, array, string, 1st, 2nd, 3rd, 4th, 5th, 6th, 7th, 8th, 9th, 10th!}.

\hypertarget{nth-partition-procedure3-1}{%
\subsection{\texorpdfstring{\texttt{nth-partition} :
procedure/3}{nth-partition : procedure/3}}\label{nth-partition-procedure3-1}}

Usage: \passthrough{\lstinline!(nth-partition m k idx) => li!}

Return a list of the form (start-offset end-offset bytes) for the
partition with index \passthrough{\lstinline!idx!} of
\passthrough{\lstinline!m!} into parts of size
\passthrough{\lstinline!k!}. The index \passthrough{\lstinline!idx!} as
well as the start- and end-offsets are 0-based.

See also: \passthrough{\lstinline!count-partitions, get-partitions!}.

\hypertarget{nthdef-procedure3-1}{%
\subsection{\texorpdfstring{\texttt{nthdef} :
procedure/3}{nthdef : procedure/3}}\label{nthdef-procedure3-1}}

Usage: \passthrough{\lstinline!(nthdef seq n default) => any!}

Return the \passthrough{\lstinline!n-th!} element of sequence
\passthrough{\lstinline!seq!} (0-indexed) if
\passthrough{\lstinline!seq!} is a sequence and has at least
\passthrough{\lstinline!n+1!} elements, default otherwise.

See also:
\passthrough{\lstinline!nth, seq?, 1st, 2nd, 3rd, 4th, 5th, 6th, 7th, 8th, 9th, 10th!}.

\hypertarget{null-procedure1-1}{%
\subsection{\texorpdfstring{\texttt{null?} :
procedure/1}{null? : procedure/1}}\label{null-procedure1-1}}

Usage: \passthrough{\lstinline!(null? li) => bool!}

Return true if \passthrough{\lstinline!li!} is nil, nil otherwise.

See also: \passthrough{\lstinline!not, list?, cons?!}.

\hypertarget{num-procedure1-1}{%
\subsection{\texorpdfstring{\texttt{num?} :
procedure/1}{num? : procedure/1}}\label{num-procedure1-1}}

Usage: \passthrough{\lstinline!(num? n) => bool!}

Return true if \passthrough{\lstinline!n!} is a number (exact or
inexact), nil otherwise.

See also:
\passthrough{\lstinline!str?, atom?, sym?, closure?, intrinsic?, macro?!}.

\hypertarget{odd-procedure1-1}{%
\subsection{\texorpdfstring{\texttt{odd?} :
procedure/1}{odd? : procedure/1}}\label{odd-procedure1-1}}

Usage: \passthrough{\lstinline!(odd? n) => bool!}

Returns true if the integer \passthrough{\lstinline!n!} is odd, nil
otherwise.

See also: \passthrough{\lstinline!even?!}.

\hypertarget{on-feature-macro1-or-more-1}{%
\subsection{\texorpdfstring{\texttt{on-feature} : macro/1 or
more}{on-feature : macro/1 or more}}\label{on-feature-macro1-or-more-1}}

Usage: \passthrough{\lstinline!(on-feature sym body ...) => any!}

Evaluate the expressions of \passthrough{\lstinline!body!} if the Lisp
feature \passthrough{\lstinline!sym!} is supported by this
implementation, do nothing otherwise.

See also: \passthrough{\lstinline!feature?, *reflect*!}.

\hypertarget{open-procedure1-or-more-1}{%
\subsection{\texorpdfstring{\texttt{open} : procedure/1 or
more}{open : procedure/1 or more}}\label{open-procedure1-or-more-1}}

Usage:
\passthrough{\lstinline!(open file-path [modes] [permissions]) => int!}

Open the file at \passthrough{\lstinline!file-path!} for reading and
writing, and return the stream ID. The optional
\passthrough{\lstinline!modes!} argument must be a list containing one
of `(read write read-write) for read, write, or read-write access
respectively, and may contain any of the following symbols: 'append to
append to an existing file, 'create for creating the file if it doesn't
exist, 'exclusive for exclusive file access, 'truncate for truncating
the file if it exists, and 'sync for attempting to sync file access. The
optional \passthrough{\lstinline!permissions!} argument must be a
numeric value specifying the Unix file permissions of the file. If these
are omitted, then default values'(read-write append create) and 0640 are
used.

See also: \passthrough{\lstinline!stropen, close, read, write!}.

\hypertarget{or-macro0-or-more-1}{%
\subsection{\texorpdfstring{\texttt{or} : macro/0 or
more}{or : macro/0 or more}}\label{or-macro0-or-more-1}}

Usage: \passthrough{\lstinline!(or expr1 expr2 ...) => any!}

Evaluate the expressions until one of them is not nil. This is a logical
shortcut or.

See also: \passthrough{\lstinline!and!}.

\hypertarget{out-procedure1-1}{%
\subsection{\texorpdfstring{\texttt{out} :
procedure/1}{out : procedure/1}}\label{out-procedure1-1}}

Usage: \passthrough{\lstinline!(out expr)!}

Output \passthrough{\lstinline!expr!} on the console with current
default background and foreground color.

See also: \passthrough{\lstinline!outy, synout, synouty, output-at!}.

\hypertarget{outy-procedure1-1}{%
\subsection{\texorpdfstring{\texttt{outy} :
procedure/1}{outy : procedure/1}}\label{outy-procedure1-1}}

Usage: \passthrough{\lstinline!(outy spec)!}

Output styled text specified in \passthrough{\lstinline!spec!}. A
specification is a list of lists starting with 'fg for foreground, 'bg
for background, or 'text for unstyled text. If the list starts with 'fg
or 'bg then the next element must be a color suitable for (the-color
spec). Following may be a string to print or another color
specification. If a list starts with 'text then one or more strings may
follow.

See also:
\passthrough{\lstinline!*colors*, the-color, set-color, color, gfx.color, output-at, out!}.

\hypertarget{peek-procedure4-1}{%
\subsection{\texorpdfstring{\texttt{peek} :
procedure/4}{peek : procedure/4}}\label{peek-procedure4-1}}

Usage: \passthrough{\lstinline!(peek b pos end sel) => num!}

Read a numeric value determined by selector
\passthrough{\lstinline!sel!} from binary blob
\passthrough{\lstinline!b!} at position \passthrough{\lstinline!pos!}
with endianness \passthrough{\lstinline!end!}. Possible values for
endianness are `little and 'big, and possible values for
\passthrough{\lstinline!sel!} must be one of'(bool int8 uint8 int16
uint16 int32 uint32 int64 uint64 float32 float64).

See also: \passthrough{\lstinline!poke, read-binary!}.

\hypertarget{permission-procedure1-1}{%
\subsection{\texorpdfstring{\texttt{permission?} :
procedure/1}{permission? : procedure/1}}\label{permission-procedure1-1}}

Usage: \passthrough{\lstinline!(permission? sym [default]) => bool!}

Return true if the permission for \passthrough{\lstinline!sym!} is set,
nil otherwise. If the permission flag is unknown, then
\passthrough{\lstinline!default!} is returned. The default for
\passthrough{\lstinline!default!} is nil.

See also:
\passthrough{\lstinline!permissions, set-permissions, when-permission, sys!}.

\hypertarget{permissions-procedure0-1}{%
\subsection{\texorpdfstring{\texttt{permissions} :
procedure/0}{permissions : procedure/0}}\label{permissions-procedure0-1}}

Usage: \passthrough{\lstinline!(permissions)!}

Return a list of all active permissions of the current interpreter.
Permissions are: \passthrough{\lstinline!load-prelude!} - load the init
file on start; \passthrough{\lstinline!load-user-init!} - load the local
user init on startup, file if present;
\passthrough{\lstinline!allow-unprotect!} - allow the user to unprotect
protected symbols (for redefining them);
\passthrough{\lstinline!allow-protect!} - allow the user to protect
symbols from redefinition or unbinding;
\passthrough{\lstinline!interactive!} - make the session interactive,
this is particularly used during startup to determine whether hooks are
installed and feedback is given. Permissions have to generally be set or
removed in careful combination with
\passthrough{\lstinline!revoke-permissions!}, which redefines symbols
and functions.

See also:
\passthrough{\lstinline!set-permissions, permission?, when-permission, sys!}.

\hypertarget{poke-procedure5-1}{%
\subsection{\texorpdfstring{\texttt{poke} :
procedure/5}{poke : procedure/5}}\label{poke-procedure5-1}}

Usage: \passthrough{\lstinline!(poke b pos end sel n)!}

Write numeric value \passthrough{\lstinline!n!} as type
\passthrough{\lstinline!sel!} with endianness
\passthrough{\lstinline!end!} into the binary blob
\passthrough{\lstinline!b!} at position \passthrough{\lstinline!pos!}.
Possible values for endianness are `little and 'big, and possible values
for \passthrough{\lstinline!sel!} must be one of'(bool int8 uint8 int16
uint16 int32 uint32 int64 uint64 float32 float64).

See also: \passthrough{\lstinline!peek, write-binary!}.

\hypertarget{pop-macro1-or-more-1}{%
\subsection{\texorpdfstring{\texttt{pop!} : macro/1 or
more}{pop! : macro/1 or more}}\label{pop-macro1-or-more-1}}

Usage: \passthrough{\lstinline"(pop! sym [def]) => any"}

Get the next element from stack \passthrough{\lstinline!sym!}, which
must be the unquoted name of a variable, and return it. If a default
\passthrough{\lstinline!def!} is given, then this is returned if the
queue is empty, otherwise nil is returned.

See also:
\passthrough{\lstinline"make-stack, stack?, push!, stack-len, stack-empty?, glance"}.

\hypertarget{pop-error-handler-procedure0-1}{%
\subsection{\texorpdfstring{\texttt{pop-error-handler} :
procedure/0}{pop-error-handler : procedure/0}}\label{pop-error-handler-procedure0-1}}

Usage: \passthrough{\lstinline!(pop-error-handler) => proc!}

Remove the topmost error handler from the error handler stack and return
it. For internal use only.

See also: \passthrough{\lstinline!with-error-handler!}.

\hypertarget{pop-finalizer-procedure0-1}{%
\subsection{\texorpdfstring{\texttt{pop-finalizer} :
procedure/0}{pop-finalizer : procedure/0}}\label{pop-finalizer-procedure0-1}}

Usage: \passthrough{\lstinline!(pop-finalizer) => proc!}

Remove a finalizer from the finalizer stack and return it. For internal
use only.

See also: \passthrough{\lstinline!push-finalizer, with-final!}.

\hypertarget{popstacked-procedure3-1}{%
\subsection{\texorpdfstring{\texttt{popstacked} :
procedure/3}{popstacked : procedure/3}}\label{popstacked-procedure3-1}}

Usage: \passthrough{\lstinline!(popstacked dict key default)!}

Get the topmost element from the stack stored at
\passthrough{\lstinline!key!} in \passthrough{\lstinline!dict!} and
remove it from the stack. If the stack is empty or no stack is stored at
key, then \passthrough{\lstinline!default!} is returned.

See also: \passthrough{\lstinline!pushstacked, getstacked!}.

\hypertarget{prin1-procedure1-1}{%
\subsection{\texorpdfstring{\texttt{prin1} :
procedure/1}{prin1 : procedure/1}}\label{prin1-procedure1-1}}

Usage: \passthrough{\lstinline!(prin1 s)!}

Print \passthrough{\lstinline!s!} to the host OS terminal, where strings
are quoted.

See also: \passthrough{\lstinline!princ, terpri, out, outy!}.

\hypertarget{princ-procedure1-1}{%
\subsection{\texorpdfstring{\texttt{princ} :
procedure/1}{princ : procedure/1}}\label{princ-procedure1-1}}

Usage: \passthrough{\lstinline!(princ s)!}

Print \passthrough{\lstinline!s!} to the host OS terminal without
quoting strings.

See also: \passthrough{\lstinline!prin1, terpri, out, outy!}.

\hypertarget{print-procedure1-1}{%
\subsection{\texorpdfstring{\texttt{print} :
procedure/1}{print : procedure/1}}\label{print-procedure1-1}}

Usage: \passthrough{\lstinline!(print x)!}

Output \passthrough{\lstinline!x!} on the host OS console and end it
with a newline.

See also: \passthrough{\lstinline!prin1, princ!}.

\hypertarget{proc-macro1-1}{%
\subsection{\texorpdfstring{\texttt{proc?} :
macro/1}{proc? : macro/1}}\label{proc-macro1-1}}

Usage: \passthrough{\lstinline!(proc? arg) => bool!}

Return true if \passthrough{\lstinline!arg!} is a procedure, nil
otherwise.

See also:
\passthrough{\lstinline!functional?, closure?, functional-arity, functional-has-rest?!}.

\hypertarget{protect-procedure0-or-more-1}{%
\subsection{\texorpdfstring{\texttt{protect} : procedure/0 or
more}{protect : procedure/0 or more}}\label{protect-procedure0-or-more-1}}

Usage: \passthrough{\lstinline!(protect [sym] ...)!}

Protect symbols \passthrough{\lstinline!sym!} \ldots{} against changes
or rebinding. The symbols need to be quoted. This operation requires the
permission 'allow-protect to be set.

See also:
\passthrough{\lstinline!protected?, unprotect, dict-protect, dict-unprotect, dict-protected?, permissions, permission?, setq, bind, interpret!}.

\hypertarget{protect-toplevel-symbols-procedure0-1}{%
\subsection{\texorpdfstring{\texttt{protect-toplevel-symbols} :
procedure/0}{protect-toplevel-symbols : procedure/0}}\label{protect-toplevel-symbols-procedure0-1}}

Usage: \passthrough{\lstinline!(protect-toplevel-symbols)!}

Protect all toplevel symbols that are not yet protected and aren't in
the \emph{mutable-toplevel-symbols} dict.

See also:
\passthrough{\lstinline!protected?, protect, unprotect, declare-unprotected, when-permission?, dict-protect, dict-protected?, dict-unprotect!}.

\hypertarget{protected-procedure1-1}{%
\subsection{\texorpdfstring{\texttt{protected?} :
procedure/1}{protected? : procedure/1}}\label{protected-procedure1-1}}

Usage: \passthrough{\lstinline!(protected? sym)!}

Return true if \passthrough{\lstinline!sym!} is protected, nil
otherwise.

See also:
\passthrough{\lstinline!protect, unprotect, dict-unprotect, dict-protected?, permission, permission?, setq, bind, interpret!}.

\hypertarget{prune-task-table-procedure0-1}{%
\subsection{\texorpdfstring{\texttt{prune-task-table} :
procedure/0}{prune-task-table : procedure/0}}\label{prune-task-table-procedure0-1}}

Usage: \passthrough{\lstinline!(prune-task-table)!}

Remove tasks that are finished from the task table. This includes tasks
for which an error has occurred.

See also: \passthrough{\lstinline!task-remove, task, task?, task-run!}.

\hypertarget{push-macro2-1}{%
\subsection{\texorpdfstring{\texttt{push!} :
macro/2}{push! : macro/2}}\label{push-macro2-1}}

Usage: \passthrough{\lstinline"(push! sym elem)"}

Put \passthrough{\lstinline!elem!} in stack
\passthrough{\lstinline!sym!}, where \passthrough{\lstinline!sym!} is
the unquoted name of a variable.

See also:
\passthrough{\lstinline"make-stack, stack?, pop!, stack-len, stack-empty?, glance"}.

\hypertarget{push-error-handler-procedure1-1}{%
\subsection{\texorpdfstring{\texttt{push-error-handler} :
procedure/1}{push-error-handler : procedure/1}}\label{push-error-handler-procedure1-1}}

Usage: \passthrough{\lstinline!(push-error-handler proc)!}

Push an error handler \passthrough{\lstinline!proc!} on the error
handler stack. For internal use only.

See also: \passthrough{\lstinline!with-error-handler!}.

\hypertarget{push-finalizer-procedure1-1}{%
\subsection{\texorpdfstring{\texttt{push-finalizer} :
procedure/1}{push-finalizer : procedure/1}}\label{push-finalizer-procedure1-1}}

Usage: \passthrough{\lstinline!(push-finalizer proc)!}

Push a finalizer procedure \passthrough{\lstinline!proc!} on the
finalizer stack. For internal use only.

See also: \passthrough{\lstinline!with-final, pop-finalizer!}.

\hypertarget{pushstacked-procedure3-1}{%
\subsection{\texorpdfstring{\texttt{pushstacked} :
procedure/3}{pushstacked : procedure/3}}\label{pushstacked-procedure3-1}}

Usage: \passthrough{\lstinline!(pushstacked dict key datum)!}

Push \passthrough{\lstinline!datum!} onto the stack maintained under
\passthrough{\lstinline!key!} in the \passthrough{\lstinline!dict.!}

See also: \passthrough{\lstinline!getstacked, popstacked!}.

\hypertarget{queue-empty-procedure1-1}{%
\subsection{\texorpdfstring{\texttt{queue-empty?} :
procedure/1}{queue-empty? : procedure/1}}\label{queue-empty-procedure1-1}}

Usage: \passthrough{\lstinline!(queue-empty? q) => bool!}

Return true if the queue \passthrough{\lstinline!q!} is empty, nil
otherwise.

See also:
\passthrough{\lstinline"make-queue, queue?, enqueue!, dequeue!, glance, queue-len"}.

\hypertarget{queue-len-procedure1-1}{%
\subsection{\texorpdfstring{\texttt{queue-len} :
procedure/1}{queue-len : procedure/1}}\label{queue-len-procedure1-1}}

Usage: \passthrough{\lstinline!(queue-len q) => int!}

Return the length of the queue \passthrough{\lstinline!q.!}

See also:
\passthrough{\lstinline"make-queue, queue?, enqueue!, dequeue!, glance, queue-len"}.

\textbf{Warning: Be advised that this is of limited use in some
concurrent contexts, since the length of the queue might have changed
already once you've obtained it!}

\hypertarget{queue-procedure1-1}{%
\subsection{\texorpdfstring{\texttt{queue?} :
procedure/1}{queue? : procedure/1}}\label{queue-procedure1-1}}

Usage: \passthrough{\lstinline!(queue? q) => bool!}

Return true if \passthrough{\lstinline!q!} is a queue, nil otherwise.

See also:
\passthrough{\lstinline"make-queue, enqueue!, dequeue, glance, queue-empty?, queue-len"}.

\hypertarget{rand-procedure2-1}{%
\subsection{\texorpdfstring{\texttt{rand} :
procedure/2}{rand : procedure/2}}\label{rand-procedure2-1}}

Usage: \passthrough{\lstinline!(rand prng lower upper) => int!}

Return a random integer in the interval
{[}\passthrough{\lstinline!lower`` upper!}{]}, both inclusive, from
pseudo-random number generator \passthrough{\lstinline!prng!}. The
\passthrough{\lstinline!prng!} argument must be an integer from 0 to 9
(inclusive).

See also: \passthrough{\lstinline!rnd, rndseed!}.

\hypertarget{random-color-procedure0-or-more-1}{%
\subsection{\texorpdfstring{\texttt{random-color} : procedure/0 or
more}{random-color : procedure/0 or more}}\label{random-color-procedure0-or-more-1}}

Usage: \passthrough{\lstinline!(random-color [alpha])!}

Return a random color with optional \passthrough{\lstinline!alpha!}
value. If \passthrough{\lstinline!alpha!} is not specified, it is 255.

See also:
\passthrough{\lstinline!the-color, *colors*, darken, lighten!}.

\hypertarget{read-procedure1-1}{%
\subsection{\texorpdfstring{\texttt{read} :
procedure/1}{read : procedure/1}}\label{read-procedure1-1}}

Usage: \passthrough{\lstinline!(read p) => any!}

Read an expression from input port \passthrough{\lstinline!p.!}

See also: \passthrough{\lstinline!input, write!}.

\hypertarget{read-binary-procedure3-1}{%
\subsection{\texorpdfstring{\texttt{read-binary} :
procedure/3}{read-binary : procedure/3}}\label{read-binary-procedure3-1}}

Usage: \passthrough{\lstinline!(read-binary p buff n) => int!}

Read \passthrough{\lstinline!n!} or less bytes from input port
\passthrough{\lstinline!p!} into binary blob
\passthrough{\lstinline!buff!}. If \passthrough{\lstinline!buff!} is
smaller than \passthrough{\lstinline!n!}, then an error is raised. If
less than \passthrough{\lstinline!n!} bytes are available before the end
of file is reached, then the amount k of bytes is read into
\passthrough{\lstinline!buff!} and k is returned. If the end of file is
reached and no byte has been read, then 0 is returned. So to loop
through this, read into the buffer and do something with it while the
amount of bytes returned is larger than 0.

See also: \passthrough{\lstinline!write-binary, read, close, open!}.

\hypertarget{read-string-procedure2-1}{%
\subsection{\texorpdfstring{\texttt{read-string} :
procedure/2}{read-string : procedure/2}}\label{read-string-procedure2-1}}

Usage: \passthrough{\lstinline!(read-string p delstr) => str!}

Reads a string from port \passthrough{\lstinline!p!} until the
single-byte delimiter character in \passthrough{\lstinline!delstr!} is
encountered, and returns the string including the delimiter. If the
input ends before the delimiter is encountered, it returns the string up
until EOF. Notice that if the empty string is returned then the end of
file must have been encountered, since otherwise the string would
contain the delimiter.

See also:
\passthrough{\lstinline!read, read-binary, write-string, write, read, close, open!}.

\hypertarget{remove-duplicates-procedure1-1}{%
\subsection{\texorpdfstring{\texttt{remove-duplicates} :
procedure/1}{remove-duplicates : procedure/1}}\label{remove-duplicates-procedure1-1}}

Usage: \passthrough{\lstinline!(remove-duplicates seq) => seq!}

Remove all duplicates in sequence \passthrough{\lstinline!seq!}, return
a new sequence with the duplicates removed.

See also: \passthrough{\lstinline!seq?, map, foreach, nth!}.

\hypertarget{remove-hook-procedure2-1}{%
\subsection{\texorpdfstring{\texttt{remove-hook} :
procedure/2}{remove-hook : procedure/2}}\label{remove-hook-procedure2-1}}

Usage: \passthrough{\lstinline!(remove-hook hook id) => bool!}

Remove the symbolic or numberic \passthrough{\lstinline!hook!} with
\passthrough{\lstinline!id!} and return true if the hook was removed,
nil otherwise.

See also:
\passthrough{\lstinline!add-hook, remove-hooks, replace-hook!}.

\hypertarget{remove-hook-internal-procedure2-1}{%
\subsection{\texorpdfstring{\texttt{remove-hook-internal} :
procedure/2}{remove-hook-internal : procedure/2}}\label{remove-hook-internal-procedure2-1}}

Usage: \passthrough{\lstinline!(remove-hook-internal hook id)!}

Remove the hook with ID \passthrough{\lstinline!id!} from numeric
\passthrough{\lstinline!hook.!}

See also: \passthrough{\lstinline!remove-hook!}.

\textbf{Warning: Internal use only.}

\hypertarget{remove-hooks-procedure1-1}{%
\subsection{\texorpdfstring{\texttt{remove-hooks} :
procedure/1}{remove-hooks : procedure/1}}\label{remove-hooks-procedure1-1}}

Usage: \passthrough{\lstinline!(remove-hooks hook) => bool!}

Remove all hooks for symbolic or numeric \passthrough{\lstinline!hook!},
return true if the hook exists and the associated procedures were
removed, nil otherwise.

See also: \passthrough{\lstinline!add-hook, remove-hook, replace-hook!}.

\hypertarget{replace-hook-procedure2-1}{%
\subsection{\texorpdfstring{\texttt{replace-hook} :
procedure/2}{replace-hook : procedure/2}}\label{replace-hook-procedure2-1}}

Usage: \passthrough{\lstinline!(replace-hook hook proc)!}

Remove all hooks for symbolic or numeric \passthrough{\lstinline!hook!}
and install the given \passthrough{\lstinline!proc!} as the only hook
procedure.

See also: \passthrough{\lstinline!add-hook, remove-hook, remove-hooks!}.

\hypertarget{reverse-procedure1-1}{%
\subsection{\texorpdfstring{\texttt{reverse} :
procedure/1}{reverse : procedure/1}}\label{reverse-procedure1-1}}

Usage: \passthrough{\lstinline!(reverse seq) => sequence!}

Reverse a sequence non-destructively, i.e., return a copy of the
reversed sequence.

See also:
\passthrough{\lstinline!nth, seq?, 1st, 2nd, 3rd, 4th, 6th, 7th, 8th, 9th, 10th, last!}.

\hypertarget{rnd-procedure0-1}{%
\subsection{\texorpdfstring{\texttt{rnd} :
procedure/0}{rnd : procedure/0}}\label{rnd-procedure0-1}}

Usage: \passthrough{\lstinline!(rnd prng) => num!}

Return a random value in the interval {[}0, 1{]} from pseudo-random
number generator \passthrough{\lstinline!prng!}. The
\passthrough{\lstinline!prng!} argument must be an integer from 0 to 9
(inclusive).

See also: \passthrough{\lstinline!rand, rndseed!}.

\hypertarget{rndseed-procedure1-1}{%
\subsection{\texorpdfstring{\texttt{rndseed} :
procedure/1}{rndseed : procedure/1}}\label{rndseed-procedure1-1}}

Usage: \passthrough{\lstinline!(rndseed prng n)!}

Seed the pseudo-random number generator \passthrough{\lstinline!prng!}
(0 to 9) with 64 bit integer value \passthrough{\lstinline!n!}. Larger
values will be truncated. Seeding affects both the rnd and the rand
function for the given \passthrough{\lstinline!prng.!}

See also: \passthrough{\lstinline!rnd, rand!}.

\hypertarget{rplaca-procedure2-1}{%
\subsection{\texorpdfstring{\texttt{rplaca} :
procedure/2}{rplaca : procedure/2}}\label{rplaca-procedure2-1}}

Usage: \passthrough{\lstinline!(rplaca li a) => li!}

Destructively mutate \passthrough{\lstinline!li!} such that its car is
\passthrough{\lstinline!a!}, return the list afterwards.

See also: \passthrough{\lstinline!rplacd!}.

\hypertarget{run-at-procedure2-1}{%
\subsection{\texorpdfstring{\texttt{run-at} :
procedure/2}{run-at : procedure/2}}\label{run-at-procedure2-1}}

Usage: \passthrough{\lstinline!(run-at date repeater proc) => int!}

Run procedure \passthrough{\lstinline!proc!} with no arguments as task
periodically according to the specification in
\passthrough{\lstinline!spec!} and return the task ID for the periodic
task. Herbey, \passthrough{\lstinline!date!} is either a datetime
specification or one of `(now skip next-minute next-quarter
next-halfhour next-hour in-2-hours in-3-hours tomorrow next-week
next-month next-year), and \passthrough{\lstinline!repeater!} is nil or
a procedure that takes a task ID and unix-epoch-nanoseconds and yields a
new unix-epoch-nanoseconds value for the next time the procedure shall
be run. While the other names are self-explanatory, the 'skip
specification means that the task is not run immediately but rather that
it is first run at (repeater -1 (now)). Timing resolution for the
scheduler is about 1 minute. Consider using interrupts for periodic
events with smaller time resolutions. The scheduler uses relative
intervals and has 'drift'.

See also: \passthrough{\lstinline!task, task-send!}.

\textbf{Warning: Tasks scheduled by run-at are not persistent! They are
only run until the system is shutdown.}

\hypertarget{run-hook-procedure1-1}{%
\subsection{\texorpdfstring{\texttt{run-hook} :
procedure/1}{run-hook : procedure/1}}\label{run-hook-procedure1-1}}

Usage: \passthrough{\lstinline!(run-hook hook)!}

Manually run the hook, executing all procedures for the hook.

See also: \passthrough{\lstinline!add-hook, remove-hook!}.

\hypertarget{run-hook-internal-procedure1-or-more-1}{%
\subsection{\texorpdfstring{\texttt{run-hook-internal} : procedure/1 or
more}{run-hook-internal : procedure/1 or more}}\label{run-hook-internal-procedure1-or-more-1}}

Usage: \passthrough{\lstinline!(run-hook-internal hook [args] ...)!}

Run all hooks for numeric hook ID \passthrough{\lstinline!hook!} with
\passthrough{\lstinline!args!}\ldots{} as arguments.

See also: \passthrough{\lstinline!run-hook!}.

\textbf{Warning: Internal use only.}

\hypertarget{run-selftest-procedure1-or-more-1}{%
\subsection{\texorpdfstring{\texttt{run-selftest} : procedure/1 or
more}{run-selftest : procedure/1 or more}}\label{run-selftest-procedure1-or-more-1}}

Usage: \passthrough{\lstinline!(run-selftest [silent?]) => any!}

Run a diagnostic self-test of the Z3S5 Machine. If
\passthrough{\lstinline!silent?!} is true, then the self-test returns a
list containing a boolean for success, the number of tests performed,
the number of successes, the number of errors, and the number of
failures. If \passthrough{\lstinline!silent?!} is not provided or nil,
then the test progress and results are displayed. An error indicates a
problem with the testing, whereas a failure means that an expected value
was not returned.

See also: \passthrough{\lstinline!expect, testing!}.

\hypertarget{sec-procedure2-1}{%
\subsection{\texorpdfstring{\texttt{sec+} :
procedure/2}{sec+ : procedure/2}}\label{sec-procedure2-1}}

Usage: \passthrough{\lstinline!(sec+ dateli n) => dateli!}

Adds \passthrough{\lstinline!n!} seconds to the given date
\passthrough{\lstinline!dateli!} in datelist format and returns the new
datelist.

See also:
\passthrough{\lstinline!minute+, hour+, day+, week+, month+, year+, now!}.

\hypertarget{semver.build-procedure1-1}{%
\subsection{\texorpdfstring{\texttt{semver.build} :
procedure/1}{semver.build : procedure/1}}\label{semver.build-procedure1-1}}

Usage: \passthrough{\lstinline!(semver.build s) => str!}

Return the build part of a semantic versioning string.

See also:
\passthrough{\lstinline!semver.canonical, semver.major, semver.major-minor!}.

\hypertarget{semver.canonical-procedure1-1}{%
\subsection{\texorpdfstring{\texttt{semver.canonical} :
procedure/1}{semver.canonical : procedure/1}}\label{semver.canonical-procedure1-1}}

Usage: \passthrough{\lstinline!(semver.canonical s) => str!}

Return a canonical semver string based on a valid, yet possibly not
canonical version string \passthrough{\lstinline!s.!}

See also: \passthrough{\lstinline!semver.major!}.

\hypertarget{semver.compare-procedure2-1}{%
\subsection{\texorpdfstring{\texttt{semver.compare} :
procedure/2}{semver.compare : procedure/2}}\label{semver.compare-procedure2-1}}

Usage: \passthrough{\lstinline!(semver.compare s1 s2) => int!}

Compare two semantic version strings \passthrough{\lstinline!s1!} and
\passthrough{\lstinline!s2!}. The result is 0 if
\passthrough{\lstinline!s1!} and \passthrough{\lstinline!s2!} are the
same version, -1 if \passthrough{\lstinline!s1!} \textless{}
\passthrough{\lstinline!s2!} and 1 if \passthrough{\lstinline!s1!}
\textgreater{} \passthrough{\lstinline!s2.!}

See also: \passthrough{\lstinline!semver.major, semver.major-minor!}.

\hypertarget{semver.is-valid-procedure1-1}{%
\subsection{\texorpdfstring{\texttt{semver.is-valid?} :
procedure/1}{semver.is-valid? : procedure/1}}\label{semver.is-valid-procedure1-1}}

Usage: \passthrough{\lstinline!(semver.is-valid? s) => bool!}

Return true if \passthrough{\lstinline!s!} is a valid semantic
versioning string, nil otherwise.

See also:
\passthrough{\lstinline!semver.major, semver.major-minor, semver.compare!}.

\hypertarget{semver.major-procedure1-1}{%
\subsection{\texorpdfstring{\texttt{semver.major} :
procedure/1}{semver.major : procedure/1}}\label{semver.major-procedure1-1}}

Usage: \passthrough{\lstinline!(semver.major s) => str!}

Return the major part of the semantic versioning string.

See also: \passthrough{\lstinline!semver.major-minor, semver.build!}.

\hypertarget{semver.major-minor-procedure1-1}{%
\subsection{\texorpdfstring{\texttt{semver.major-minor} :
procedure/1}{semver.major-minor : procedure/1}}\label{semver.major-minor-procedure1-1}}

Usage: \passthrough{\lstinline!(semver.major-minor s) => str!}

Return the major.minor prefix of a semantic versioning string. For
example, (semver.major-minor ``v2.1.4'') returns ``v2.1''.

See also: \passthrough{\lstinline!semver.major, semver.build!}.

\hypertarget{semver.max-procedure2-1}{%
\subsection{\texorpdfstring{\texttt{semver.max} :
procedure/2}{semver.max : procedure/2}}\label{semver.max-procedure2-1}}

Usage: \passthrough{\lstinline!(semver.max s1 s2) => str!}

Canonicalize \passthrough{\lstinline!s1!} and
\passthrough{\lstinline!s2!} and return the larger version of them.

See also: \passthrough{\lstinline!semver.compare!}.

\hypertarget{semver.prerelease-procedure1-1}{%
\subsection{\texorpdfstring{\texttt{semver.prerelease} :
procedure/1}{semver.prerelease : procedure/1}}\label{semver.prerelease-procedure1-1}}

Usage: \passthrough{\lstinline!(semver.prerelease s) => str!}

Return the prerelease part of a version string, or the empty string if
there is none. For example, (semver.prerelease ``v2.1.0-pre+build'')
returns ``-pre''.

See also:
\passthrough{\lstinline!semver.build, semver.major, semver.major-minor!}.

\hypertarget{seq-procedure1-1}{%
\subsection{\texorpdfstring{\texttt{seq?} :
procedure/1}{seq? : procedure/1}}\label{seq-procedure1-1}}

Usage: \passthrough{\lstinline!(seq? seq) => bool!}

Return true if \passthrough{\lstinline!seq!} is a sequence, nil
otherwise.

See also: \passthrough{\lstinline!list, array, string, slice, nth!}.

\hypertarget{set-procedure3-1}{%
\subsection{\texorpdfstring{\texttt{set} :
procedure/3}{set : procedure/3}}\label{set-procedure3-1}}

Usage: \passthrough{\lstinline!(set d key value)!}

Set \passthrough{\lstinline!value!} for \passthrough{\lstinline!key!} in
dict \passthrough{\lstinline!d.!}

See also: \passthrough{\lstinline!dict, get, get-or-set!}.

\hypertarget{set-procedure2-1}{%
\subsection{\texorpdfstring{\texttt{set*} :
procedure/2}{set* : procedure/2}}\label{set-procedure2-1}}

Usage: \passthrough{\lstinline!(set* d li)!}

Set in dict \passthrough{\lstinline!d!} the keys and values in list
\passthrough{\lstinline!li!}. The list \passthrough{\lstinline!li!} must
be of the form (key-1 value-1 key-2 value-2 \ldots{} key-n value-n).
This function may be slightly faster than using individual
\passthrough{\lstinline!set!} operations.

See also: \passthrough{\lstinline!dict, set!}.

\hypertarget{set-list-procedure1-1}{%
\subsection{\texorpdfstring{\texttt{set-\textgreater{}list} :
procedure/1}{set-\textgreater list : procedure/1}}\label{set-list-procedure1-1}}

Usage: \passthrough{\lstinline!(set->list s) => li!}

Convert set \passthrough{\lstinline!s!} to a list of set elements.

See also:
\passthrough{\lstinline!list->set, make-set, set-element?, set-union, set-intersection, set-complement, set-difference, set?, set-empty!}.

\hypertarget{set-color-procedure1-1}{%
\subsection{\texorpdfstring{\texttt{set-color} :
procedure/1}{set-color : procedure/1}}\label{set-color-procedure1-1}}

Usage: \passthrough{\lstinline!(set-color sel colorlist)!}

Set the color according to \passthrough{\lstinline!sel!} to the color
\passthrough{\lstinline!colorlist!} of the form '(r g b a). See
\passthrough{\lstinline!color!} for information about
\passthrough{\lstinline!sel.!}

See also: \passthrough{\lstinline!color, the-color, with-colors!}.

\hypertarget{set-complement-procedure2-1}{%
\subsection{\texorpdfstring{\texttt{set-complement} :
procedure/2}{set-complement : procedure/2}}\label{set-complement-procedure2-1}}

Usage: \passthrough{\lstinline!(set-complement a domain) => set!}

Return all elements in \passthrough{\lstinline!domain!} that are not
elements of \passthrough{\lstinline!a.!}

See also:
\passthrough{\lstinline!list->set, set->list, make-set, set-element?, set-union, set-difference, set-intersection, set?, set-empty?, set-subset?, set-equal?!}.

\hypertarget{set-difference-procedure2-1}{%
\subsection{\texorpdfstring{\texttt{set-difference} :
procedure/2}{set-difference : procedure/2}}\label{set-difference-procedure2-1}}

Usage: \passthrough{\lstinline!(set-difference a b) => set!}

Return the set-theoretic difference of set \passthrough{\lstinline!a!}
minus set \passthrough{\lstinline!b!}, i.e., all elements in
\passthrough{\lstinline!a!} that are not in \passthrough{\lstinline!b.!}

See also:
\passthrough{\lstinline!list->set, set->list, make-set, set-element?, set-union, set-intersection, set-complement, set?, set-empty?, set-subset?, set-equal?!}.

\hypertarget{set-element-procedure2-1}{%
\subsection{\texorpdfstring{\texttt{set-element?} :
procedure/2}{set-element? : procedure/2}}\label{set-element-procedure2-1}}

Usage: \passthrough{\lstinline!(set-element? s elem) => bool!}

Return true if set \passthrough{\lstinline!s!} has element
\passthrough{\lstinline!elem!}, nil otherwise.

See also:
\passthrough{\lstinline!make-set, list->set, set->list, set-union, set-intersection, set-complement, set-difference, set?, set-empty?!}.

\hypertarget{set-empty-procedure1-1}{%
\subsection{\texorpdfstring{\texttt{set-empty?} :
procedure/1}{set-empty? : procedure/1}}\label{set-empty-procedure1-1}}

Usage: \passthrough{\lstinline!(set-empty? s) => bool!}

Return true if set \passthrough{\lstinline!s!} is empty, nil otherwise.

See also:
\passthrough{\lstinline!make-set, list->set, set->list, set-union, set-intersection, set-complement, set-difference, set?!}.

\hypertarget{set-equal-procedure2-1}{%
\subsection{\texorpdfstring{\texttt{set-equal?} :
procedure/2}{set-equal? : procedure/2}}\label{set-equal-procedure2-1}}

Usage: \passthrough{\lstinline!(set-equal? a b) => bool!}

Return true if \passthrough{\lstinline!a!} and
\passthrough{\lstinline!b!} contain the same elements.

See also:
\passthrough{\lstinline!set-subset?, list->set, set-element?, set->list, set-union, set-difference, set-intersection, set-complement, set?, set-empty?!}.

\hypertarget{set-help-topic-info-procedure3-1}{%
\subsection{\texorpdfstring{\texttt{set-help-topic-info} :
procedure/3}{set-help-topic-info : procedure/3}}\label{set-help-topic-info-procedure3-1}}

Usage: \passthrough{\lstinline!(set-help-topic-info topic header info)!}

Set a human-readable information entry for help
\passthrough{\lstinline!topic!} with human-readable
\passthrough{\lstinline!header!} and \passthrough{\lstinline!info!}
strings.

See also: \passthrough{\lstinline!defhelp, help-topic-info!}.

\hypertarget{set-intersection-procedure2-1}{%
\subsection{\texorpdfstring{\texttt{set-intersection} :
procedure/2}{set-intersection : procedure/2}}\label{set-intersection-procedure2-1}}

Usage: \passthrough{\lstinline!(set-intersection a b) => set!}

Return the intersection of sets \passthrough{\lstinline!a!} and
\passthrough{\lstinline!b!}, i.e., the set of elements that are both in
\passthrough{\lstinline!a!} and in \passthrough{\lstinline!b.!}

See also:
\passthrough{\lstinline!list->set, set->list, make-set, set-element?, set-union, set-complement, set-difference, set?, set-empty?, set-subset?, set-equal?!}.

\hypertarget{set-permissions-nil-1}{%
\subsection{set-permissions : nil}\label{set-permissions-nil-1}}

Usage: \passthrough{\lstinline!(set-permissions li)!}

Set the permissions for the current interpreter. This will trigger an
error when the permission cannot be set due to a security violation.
Generally, permissions can only be downgraded (made more stringent) and
never relaxed. See the information for
\passthrough{\lstinline!permissions!} for an overview of symbolic flags.

See also:
\passthrough{\lstinline!permissions, permission?, when-permission, sys!}.

\hypertarget{set-subset-procedure2-1}{%
\subsection{\texorpdfstring{\texttt{set-subset?} :
procedure/2}{set-subset? : procedure/2}}\label{set-subset-procedure2-1}}

Usage: \passthrough{\lstinline!(set-subset? a b) => bool!}

Return true if \passthrough{\lstinline!a!} is a subset of
\passthrough{\lstinline!b!}, nil otherwise.

See also:
\passthrough{\lstinline!set-equal?, list->set, set->list, make-set, set-element?, set-union, set-difference, set-intersection, set-complement, set?, set-empty?!}.

\hypertarget{set-union-procedure2-1}{%
\subsection{\texorpdfstring{\texttt{set-union} :
procedure/2}{set-union : procedure/2}}\label{set-union-procedure2-1}}

Usage: \passthrough{\lstinline!(set-union a b) => set!}

Return the union of sets \passthrough{\lstinline!a!} and
\passthrough{\lstinline!b!} containing all elements that are in
\passthrough{\lstinline!a!} or in \passthrough{\lstinline!b!} (or both).

See also:
\passthrough{\lstinline!list->set, set->list, make-set, set-element?, set-intersection, set-complement, set-difference, set?, set-empty?!}.

\hypertarget{set-volume-procedure1-1}{%
\subsection{\texorpdfstring{\texttt{set-volume} :
procedure/1}{set-volume : procedure/1}}\label{set-volume-procedure1-1}}

Usage: \passthrough{\lstinline!(set-volume fl)!}

Set the master volume for all sound to \passthrough{\lstinline!fl!}, a
value between 0.0 and 1.0.

See also: \passthrough{\lstinline!play-sound, play-music!}.

\hypertarget{set-procedure1-1}{%
\subsection{\texorpdfstring{\texttt{set?} :
procedure/1}{set? : procedure/1}}\label{set-procedure1-1}}

Usage: \passthrough{\lstinline!(set? x) => bool!}

Return true if \passthrough{\lstinline!x!} can be used as a set, nil
otherwise.

See also:
\passthrough{\lstinline!list->set, make-set, set->list, set-element?, set-union, set-intersection, set-complement, set-difference, set-empty?!}.

\hypertarget{setcar-procedure1-1}{%
\subsection{\texorpdfstring{\texttt{setcar} :
procedure/1}{setcar : procedure/1}}\label{setcar-procedure1-1}}

Usage: \passthrough{\lstinline!(setcar li elem) => li!}

Mutate \passthrough{\lstinline!li!} such that its car is
\passthrough{\lstinline!elem!}. Same as rplaca.

See also: \passthrough{\lstinline!rplaca, rplacd, setcdr!}.

\hypertarget{setcdr-procedure1-1}{%
\subsection{\texorpdfstring{\texttt{setcdr} :
procedure/1}{setcdr : procedure/1}}\label{setcdr-procedure1-1}}

Usage: \passthrough{\lstinline!(setcdr li1 li2) => li!}

Mutate \passthrough{\lstinline!li1!} such that its cdr is
\passthrough{\lstinline!li2!}. Same as rplacd.

See also: \passthrough{\lstinline!rplacd, rplaca, setcar!}.

\hypertarget{shorten-procedure2-1}{%
\subsection{\texorpdfstring{\texttt{shorten} :
procedure/2}{shorten : procedure/2}}\label{shorten-procedure2-1}}

Usage: \passthrough{\lstinline!(shorten s n) => str!}

Shorten string \passthrough{\lstinline!s!} to length
\passthrough{\lstinline!n!} in a smart way if possible, leave it
untouched if the length of \passthrough{\lstinline!s!} is smaller than
\passthrough{\lstinline!n.!}

See also: \passthrough{\lstinline!substr!}.

\hypertarget{sleep-procedure1-1}{%
\subsection{\texorpdfstring{\texttt{sleep} :
procedure/1}{sleep : procedure/1}}\label{sleep-procedure1-1}}

Usage: \passthrough{\lstinline!(sleep ms)!}

Halt the current task execution for \passthrough{\lstinline!ms!}
milliseconds.

See also: \passthrough{\lstinline!sleep-ns, time, now, now-ns!}.

\hypertarget{sleep-ns-procedure1-1}{%
\subsection{\texorpdfstring{\texttt{sleep-ns} :
procedure/1}{sleep-ns : procedure/1}}\label{sleep-ns-procedure1-1}}

Usage: \passthrough{\lstinline!(sleep-ns n!}

Halt the current task execution for \passthrough{\lstinline!n!}
nanoseconds.

See also: \passthrough{\lstinline!sleep, time, now, now-ns!}.

\hypertarget{slice-procedure3-1}{%
\subsection{\texorpdfstring{\texttt{slice} :
procedure/3}{slice : procedure/3}}\label{slice-procedure3-1}}

Usage: \passthrough{\lstinline!(slice seq low high) => seq!}

Return the subsequence of \passthrough{\lstinline!seq!} starting from
\passthrough{\lstinline!low!} inclusive and ending at
\passthrough{\lstinline!high!} exclusive. Sequences are 0-indexed.

See also: \passthrough{\lstinline!list, array, string, nth, seq?!}.

\hypertarget{sort-procedure2-1}{%
\subsection{\texorpdfstring{\texttt{sort} :
procedure/2}{sort : procedure/2}}\label{sort-procedure2-1}}

Usage: \passthrough{\lstinline!(sort li proc) => li!}

Sort the list \passthrough{\lstinline!li!} by the given less-than
procedure \passthrough{\lstinline!proc!}, which takes two arguments and
returns true if the first one is less than the second, nil otheriwse.

See also: \passthrough{\lstinline!array-sort!}.

\hypertarget{sort-symbols-nil-1}{%
\subsection{sort-symbols : nil}\label{sort-symbols-nil-1}}

Usage: \passthrough{\lstinline!(sort-symbols li) => list!}

Sort the list of symbols \passthrough{\lstinline!li!} alphabetically.

See also: \passthrough{\lstinline!out, dp, du, dump!}.

\hypertarget{spaces-procedure1-1}{%
\subsection{\texorpdfstring{\texttt{spaces} :
procedure/1}{spaces : procedure/1}}\label{spaces-procedure1-1}}

Usage: \passthrough{\lstinline!(spaces n) => str!}

Create a string consisting of \passthrough{\lstinline!n!} spaces.

See also: \passthrough{\lstinline!strbuild, strleft, strright!}.

\hypertarget{stack-empty-procedure1-1}{%
\subsection{\texorpdfstring{\texttt{stack-empty?} :
procedure/1}{stack-empty? : procedure/1}}\label{stack-empty-procedure1-1}}

Usage: \passthrough{\lstinline!(queue-empty? s) => bool!}

Return true if the stack \passthrough{\lstinline!s!} is empty, nil
otherwise.

See also:
\passthrough{\lstinline"make-stack, stack?, push!, pop!, stack-len, glance"}.

\hypertarget{stack-len-procedure1-1}{%
\subsection{\texorpdfstring{\texttt{stack-len} :
procedure/1}{stack-len : procedure/1}}\label{stack-len-procedure1-1}}

Usage: \passthrough{\lstinline!(stack-len s) => int!}

Return the length of the stack \passthrough{\lstinline!s.!}

See also:
\passthrough{\lstinline"make-queue, queue?, enqueue!, dequeue!, glance, queue-len"}.

\textbf{Warning: Be advised that this is of limited use in some
concurrent contexts, since the length of the queue might have changed
already once you've obtained it!}

\hypertarget{stack-procedure1-1}{%
\subsection{\texorpdfstring{\texttt{stack?} :
procedure/1}{stack? : procedure/1}}\label{stack-procedure1-1}}

Usage: \passthrough{\lstinline!(stack? q) => bool!}

Return true if \passthrough{\lstinline!q!} is a stack, nil otherwise.

See also:
\passthrough{\lstinline"make-stack, push!, pop!, stack-empty?, stack-len, glance"}.

\hypertarget{str-procedure0-or-more-1}{%
\subsection{\texorpdfstring{\texttt{str+} : procedure/0 or
more}{str+ : procedure/0 or more}}\label{str-procedure0-or-more-1}}

Usage: \passthrough{\lstinline!(str+ [s] ...) => str!}

Append all strings given to the function.

See also: \passthrough{\lstinline!str?!}.

\hypertarget{str-array-procedure1-1}{%
\subsection{\texorpdfstring{\texttt{str-\textgreater{}array} :
procedure/1}{str-\textgreater array : procedure/1}}\label{str-array-procedure1-1}}

Usage: \passthrough{\lstinline!(str->array s) => array!}

Return the string \passthrough{\lstinline!s!} as an array of unicode
glyph integer values.

See also: \passthrough{\lstinline!array->str!}.

\hypertarget{str-blob-procedure1-1}{%
\subsection{\texorpdfstring{\texttt{str-\textgreater{}blob} :
procedure/1}{str-\textgreater blob : procedure/1}}\label{str-blob-procedure1-1}}

Usage: \passthrough{\lstinline!(str->blob s) => blob!}

Convert string \passthrough{\lstinline!s!} into a blob.

See also: \passthrough{\lstinline!blob->str!}.

\hypertarget{str-char-procedure1-1}{%
\subsection{\texorpdfstring{\texttt{str-\textgreater{}char} :
procedure/1}{str-\textgreater char : procedure/1}}\label{str-char-procedure1-1}}

Usage: \passthrough{\lstinline!(str->char s)!}

Return the first character of \passthrough{\lstinline!s!} as unicode
integer.

See also: \passthrough{\lstinline!char->str!}.

\hypertarget{str-chars-procedure1-1}{%
\subsection{\texorpdfstring{\texttt{str-\textgreater{}chars} :
procedure/1}{str-\textgreater chars : procedure/1}}\label{str-chars-procedure1-1}}

Usage: \passthrough{\lstinline!(str->chars s) => array!}

Convert the UTF-8 string \passthrough{\lstinline!s!} into an array of
UTF-8 rune integers. An error may occur if the string is not a valid
UTF-8 string.

See also: \passthrough{\lstinline!runes->str, str->char, char->str!}.

\hypertarget{str-expr-procedure0-or-more-2}{%
\subsection{\texorpdfstring{\texttt{str-\textgreater{}expr} :
procedure/0 or
more}{str-\textgreater expr : procedure/0 or more}}\label{str-expr-procedure0-or-more-2}}

Usage: \passthrough{\lstinline!(str->expr s [default]) => any!}

Convert a string \passthrough{\lstinline!s!} into a Lisp expression. If
\passthrough{\lstinline!default!} is provided, it is returned if an
error occurs, otherwise an error is raised.

See also:
\passthrough{\lstinline!expr->str, str->expr*, openstr, externalize, internalize!}.

\hypertarget{str-expr-procedure0-or-more-3}{%
\subsection{\texorpdfstring{\texttt{str-\textgreater{}expr*} :
procedure/0 or
more}{str-\textgreater expr* : procedure/0 or more}}\label{str-expr-procedure0-or-more-3}}

Usage: \passthrough{\lstinline!(str->expr* s [default]) => li!}

Convert a string \passthrough{\lstinline!s!} into a list consisting of
the Lisp expressions in \passthrough{\lstinline!s!}. If
\passthrough{\lstinline!default!} is provided, then this value is put in
the result list whenever an error occurs. Otherwise an error is raised.
Notice that it might not always be obvious what expression in
\passthrough{\lstinline!s!} triggers an error, since this hinges on the
way the internal expession parser works.

See also:
\passthrough{\lstinline!str->expr, expr->str, openstr, internalize, externalize!}.

\hypertarget{str-list-procedure1-1}{%
\subsection{\texorpdfstring{\texttt{str-\textgreater{}list} :
procedure/1}{str-\textgreater list : procedure/1}}\label{str-list-procedure1-1}}

Usage: \passthrough{\lstinline!(str->list s) => list!}

Return the sequence of numeric chars that make up string
\passthrough{\lstinline!s.!}

See also:
\passthrough{\lstinline!str->array, list->str, array->str, chars!}.

\hypertarget{str-sym-procedure1-1}{%
\subsection{\texorpdfstring{\texttt{str-\textgreater{}sym} :
procedure/1}{str-\textgreater sym : procedure/1}}\label{str-sym-procedure1-1}}

Usage: \passthrough{\lstinline!(str->sym s) => sym!}

Convert a string into a symbol.

See also: \passthrough{\lstinline!sym->str, intern, make-symbol!}.

\hypertarget{str-count-substr-procedure2-1}{%
\subsection{\texorpdfstring{\texttt{str-count-substr} :
procedure/2}{str-count-substr : procedure/2}}\label{str-count-substr-procedure2-1}}

Usage: \passthrough{\lstinline!(str-count-substr s1 s2) => int!}

Count the number of non-overlapping occurrences of substring
\passthrough{\lstinline!s2!} in string \passthrough{\lstinline!s1.!}

See also: \passthrough{\lstinline!str-replace, str-replace*, instr!}.

\hypertarget{str-empty-procedure1-1}{%
\subsection{\texorpdfstring{\texttt{str-empty?} :
procedure/1}{str-empty? : procedure/1}}\label{str-empty-procedure1-1}}

Usage: \passthrough{\lstinline!(str-empty? s) => bool!}

Return true if the string \passthrough{\lstinline!s!} is empty, nil
otherwise.

See also: \passthrough{\lstinline!strlen!}.

\hypertarget{str-exists-procedure2-1}{%
\subsection{\texorpdfstring{\texttt{str-exists?} :
procedure/2}{str-exists? : procedure/2}}\label{str-exists-procedure2-1}}

Usage: \passthrough{\lstinline!(str-exists? s pred) => bool!}

Return true if \passthrough{\lstinline!pred!} returns true for at least
one character in string \passthrough{\lstinline!s!}, nil otherwise.

See also:
\passthrough{\lstinline!exists?, forall?, list-exists?, array-exists?, seq?!}.

\hypertarget{str-forall-procedure2-1}{%
\subsection{\texorpdfstring{\texttt{str-forall?} :
procedure/2}{str-forall? : procedure/2}}\label{str-forall-procedure2-1}}

Usage: \passthrough{\lstinline!(str-forall? s pred) => bool!}

Return true if predicate \passthrough{\lstinline!pred!} returns true for
all characters in string \passthrough{\lstinline!s!}, nil otherwise.

See also:
\passthrough{\lstinline!foreach, map, forall?, array-forall?, list-forall, exists?!}.

\hypertarget{str-foreach-procedure2-1}{%
\subsection{\texorpdfstring{\texttt{str-foreach} :
procedure/2}{str-foreach : procedure/2}}\label{str-foreach-procedure2-1}}

Usage: \passthrough{\lstinline!(str-foreach s proc)!}

Apply \passthrough{\lstinline!proc!} to each element of string
\passthrough{\lstinline!s!} in order, for the side effects.

See also:
\passthrough{\lstinline!foreach, list-foreach, array-foreach, map!}.

\hypertarget{str-index-procedure2-or-more-1}{%
\subsection{\texorpdfstring{\texttt{str-index} : procedure/2 or
more}{str-index : procedure/2 or more}}\label{str-index-procedure2-or-more-1}}

Usage: \passthrough{\lstinline!(str-index s chars [pos]) => int!}

Find the first char in \passthrough{\lstinline!s!} that is in the
charset \passthrough{\lstinline!chars!}, starting from the optional
\passthrough{\lstinline!pos!} in \passthrough{\lstinline!s!}, and return
its index in the string. If no macthing char is found, nil is returned.

See also: \passthrough{\lstinline!strsplit, chars, inchars!}.

\hypertarget{str-join-procedure2-1}{%
\subsection{\texorpdfstring{\texttt{str-join} :
procedure/2}{str-join : procedure/2}}\label{str-join-procedure2-1}}

Usage: \passthrough{\lstinline!(str-join li del) => str!}

Join a list of strings \passthrough{\lstinline!li!} where each of the
strings is separated by string \passthrough{\lstinline!del!}, and return
the result string.

See also: \passthrough{\lstinline!strlen, strsplit, str-slice!}.

\hypertarget{str-port-procedure1-1}{%
\subsection{\texorpdfstring{\texttt{str-port?} :
procedure/1}{str-port? : procedure/1}}\label{str-port-procedure1-1}}

Usage: \passthrough{\lstinline!(str-port? p) => bool!}

Return true if \passthrough{\lstinline!p!} is a string port, nil
otherwise.

See also: \passthrough{\lstinline!port?, file-port?, stropen, open!}.

\hypertarget{str-ref-procedure2-1}{%
\subsection{\texorpdfstring{\texttt{str-ref} :
procedure/2}{str-ref : procedure/2}}\label{str-ref-procedure2-1}}

Usage: \passthrough{\lstinline!(str-ref s n) => n!}

Return the unicode char as integer at position
\passthrough{\lstinline!n!} in \passthrough{\lstinline!s!}. Strings are
0-indexed.

See also: \passthrough{\lstinline!nth!}.

\hypertarget{str-remove-number-procedure1-1}{%
\subsection{\texorpdfstring{\texttt{str-remove-number} :
procedure/1}{str-remove-number : procedure/1}}\label{str-remove-number-procedure1-1}}

Usage: \passthrough{\lstinline!(str-remove-number s [del]) => str!}

Remove the suffix number in \passthrough{\lstinline!s!}, provided there
is one and it is separated from the rest of the string by
\passthrough{\lstinline!del!}, where the default is a space character.
For instance, ``Test 29'' will be converted to ``Test'',
``User-Name1-23-99'' with delimiter ``-'' will be converted to
``User-Name1-23''. This function will remove intermediate delimiters in
the middle of the string, since it disassembles and reassembles the
string, so be aware that this is not preserving inputs in that respect.

See also: \passthrough{\lstinline!strsplit!}.

\hypertarget{str-remove-prefix-procedure1-1}{%
\subsection{\texorpdfstring{\texttt{str-remove-prefix} :
procedure/1}{str-remove-prefix : procedure/1}}\label{str-remove-prefix-procedure1-1}}

Usage: \passthrough{\lstinline!(str-remove-prefix s prefix) => str!}

Remove the prefix \passthrough{\lstinline!prefix!} from string
\passthrough{\lstinline!s!}, return the string without the prefix. If
the prefix does not match, \passthrough{\lstinline!s!} is returned. If
\passthrough{\lstinline!prefix!} is longer than
\passthrough{\lstinline!s!} and matches, the empty string is returned.

See also: \passthrough{\lstinline!str-remove-suffix!}.

\hypertarget{str-remove-suffix-procedure1-1}{%
\subsection{\texorpdfstring{\texttt{str-remove-suffix} :
procedure/1}{str-remove-suffix : procedure/1}}\label{str-remove-suffix-procedure1-1}}

Usage: \passthrough{\lstinline!(str-remove-suffix s suffix) => str!}

remove the suffix \passthrough{\lstinline!suffix!} from string
\passthrough{\lstinline!s!}, return the string without the suffix. If
the suffix does not match, \passthrough{\lstinline!s!} is returned. If
\passthrough{\lstinline!suffix!} is longer than
\passthrough{\lstinline!s!} and matches, the empty string is returned.

See also: \passthrough{\lstinline!str-remove-prefix!}.

\hypertarget{str-replace-procedure4-1}{%
\subsection{\texorpdfstring{\texttt{str-replace} :
procedure/4}{str-replace : procedure/4}}\label{str-replace-procedure4-1}}

Usage: \passthrough{\lstinline!(str-replace s t1 t2 n) => str!}

Replace the first \passthrough{\lstinline!n!} instances of substring
\passthrough{\lstinline!t1!} in \passthrough{\lstinline!s!} by
\passthrough{\lstinline!t2.!}

See also: \passthrough{\lstinline!str-replace*, str-count-substr!}.

\hypertarget{str-replace-procedure3-1}{%
\subsection{\texorpdfstring{\texttt{str-replace*} :
procedure/3}{str-replace* : procedure/3}}\label{str-replace-procedure3-1}}

Usage: \passthrough{\lstinline!(str-replace* s t1 t2) => str!}

Replace all non-overlapping substrings \passthrough{\lstinline!t1!} in
\passthrough{\lstinline!s!} by \passthrough{\lstinline!t2.!}

See also: \passthrough{\lstinline!str-replace, str-count-substr!}.

\hypertarget{str-reverse-procedure1-1}{%
\subsection{\texorpdfstring{\texttt{str-reverse} :
procedure/1}{str-reverse : procedure/1}}\label{str-reverse-procedure1-1}}

Usage: \passthrough{\lstinline!(str-reverse s) => str!}

Reverse string \passthrough{\lstinline!s.!}

See also:
\passthrough{\lstinline!reverse, array-reverse, list-reverse!}.

\hypertarget{str-segment-procedure3-1}{%
\subsection{\texorpdfstring{\texttt{str-segment} :
procedure/3}{str-segment : procedure/3}}\label{str-segment-procedure3-1}}

Usage: \passthrough{\lstinline!(str-segment str start end) => list!}

Parse a string \passthrough{\lstinline!str!} into words that start with
one of the characters in string \passthrough{\lstinline!start!} and end
in one of the characters in string \passthrough{\lstinline!end!} and
return a list consisting of lists of the form (bool s) where bool is
true if the string starts with a character in
\passthrough{\lstinline!start!}, nil otherwise, and
\passthrough{\lstinline!s!} is the extracted string including start and
end characters.

See also: \passthrough{\lstinline!str+, strsplit, fmt, strbuild!}.

\hypertarget{str-slice-procedure3-1}{%
\subsection{\texorpdfstring{\texttt{str-slice} :
procedure/3}{str-slice : procedure/3}}\label{str-slice-procedure3-1}}

Usage: \passthrough{\lstinline!(str-slice s low high) => s!}

Return a slice of string \passthrough{\lstinline!s!} starting at
character with index \passthrough{\lstinline!low!} (inclusive) and
ending at character with index \passthrough{\lstinline!high!}
(exclusive).

See also: \passthrough{\lstinline!slice!}.

\hypertarget{str-procedure1}{%
\subsection{\texorpdfstring{\texttt{str?} :
procedure/1}{str? : procedure/1}}\label{str-procedure1}}

Usage: \passthrough{\lstinline!(str? s) => bool!}

Return true if \passthrough{\lstinline!s!} is a string, nil otherwise.

See also:
\passthrough{\lstinline!num?, atom?, sym?, closure?, intrinsic?, macro?!}.

\hypertarget{strbuild-procedure2-1}{%
\subsection{\texorpdfstring{\texttt{strbuild} :
procedure/2}{strbuild : procedure/2}}\label{strbuild-procedure2-1}}

Usage: \passthrough{\lstinline!(strbuild s n) => str!}

Build a string by repeating string \passthrough{\lstinline!s`` n!}
times.

See also: \passthrough{\lstinline!str+!}.

\hypertarget{strcase-procedure2-1}{%
\subsection{\texorpdfstring{\texttt{strcase} :
procedure/2}{strcase : procedure/2}}\label{strcase-procedure2-1}}

Usage: \passthrough{\lstinline!(strcase s sel) => str!}

Change the case of the string \passthrough{\lstinline!s!} according to
selector \passthrough{\lstinline!sel!} and return a copy. Valid values
for \passthrough{\lstinline!sel!} are 'lower for conversion to
lower-case, 'upper for uppercase, 'title for title case and 'utf-8 for
utf-8 normalization (which replaces unprintable characters with ``?'').

See also: \passthrough{\lstinline!strmap!}.

\hypertarget{strcenter-procedure2-1}{%
\subsection{\texorpdfstring{\texttt{strcenter} :
procedure/2}{strcenter : procedure/2}}\label{strcenter-procedure2-1}}

Usage: \passthrough{\lstinline!(strcenter s n) => str!}

Center string \passthrough{\lstinline!s!} by wrapping space characters
around it, such that the total length the result string is
\passthrough{\lstinline!n.!}

See also: \passthrough{\lstinline!strleft, strright, strlimit!}.

\hypertarget{strcnt-procedure2-1}{%
\subsection{\texorpdfstring{\texttt{strcnt} :
procedure/2}{strcnt : procedure/2}}\label{strcnt-procedure2-1}}

Usage: \passthrough{\lstinline!(strcnt s del) => int!}

Returnt the number of non-overlapping substrings
\passthrough{\lstinline!del!} in \passthrough{\lstinline!s.!}

See also: \passthrough{\lstinline!strsplit, str-index!}.

\hypertarget{strleft-procedure2-1}{%
\subsection{\texorpdfstring{\texttt{strleft} :
procedure/2}{strleft : procedure/2}}\label{strleft-procedure2-1}}

Usage: \passthrough{\lstinline!(strleft s n) => str!}

Align string \passthrough{\lstinline!s!} left by adding space characters
to the right of it, such that the total length the result string is
\passthrough{\lstinline!n.!}

See also: \passthrough{\lstinline!strcenter, strright, strlimit!}.

\hypertarget{strlen-procedure1-1}{%
\subsection{\texorpdfstring{\texttt{strlen} :
procedure/1}{strlen : procedure/1}}\label{strlen-procedure1-1}}

Usage: \passthrough{\lstinline!(strlen s) => int!}

Return the length of \passthrough{\lstinline!s.!}

See also: \passthrough{\lstinline!len, seq?, str?!}.

\hypertarget{strless-procedure2-1}{%
\subsection{\texorpdfstring{\texttt{strless} :
procedure/2}{strless : procedure/2}}\label{strless-procedure2-1}}

Usage: \passthrough{\lstinline!(strless s1 s2) => bool!}

Return true if string \passthrough{\lstinline!s1!} \textless{}
\passthrough{\lstinline!s2!} in lexicographic comparison, nil otherwise.

See also: \passthrough{\lstinline!sort, array-sort, strcase!}.

\hypertarget{strlimit-procedure2-1}{%
\subsection{\texorpdfstring{\texttt{strlimit} :
procedure/2}{strlimit : procedure/2}}\label{strlimit-procedure2-1}}

Usage: \passthrough{\lstinline!(strlimit s n) => str!}

Return a string based on \passthrough{\lstinline!s!} cropped to a
maximal length of \passthrough{\lstinline!n!} (or less if
\passthrough{\lstinline!s!} is shorter).

See also: \passthrough{\lstinline!strcenter, strleft, strright!}.

\hypertarget{strmap-procedure2-1}{%
\subsection{\texorpdfstring{\texttt{strmap} :
procedure/2}{strmap : procedure/2}}\label{strmap-procedure2-1}}

Usage: \passthrough{\lstinline!(strmap s proc) => str!}

Map function \passthrough{\lstinline!proc!}, which takes a number and
returns a number, over all unicode characters in
\passthrough{\lstinline!s!} and return the result as new string.

See also: \passthrough{\lstinline!map!}.

\hypertarget{stropen-procedure1-1}{%
\subsection{\texorpdfstring{\texttt{stropen} :
procedure/1}{stropen : procedure/1}}\label{stropen-procedure1-1}}

Usage: \passthrough{\lstinline!(stropen s) => streamport!}

Open the string \passthrough{\lstinline!s!} as input stream.

See also: \passthrough{\lstinline!open, close!}.

\hypertarget{strright-procedure2-1}{%
\subsection{\texorpdfstring{\texttt{strright} :
procedure/2}{strright : procedure/2}}\label{strright-procedure2-1}}

Usage: \passthrough{\lstinline!(strright s n) => str!}

Align string \passthrough{\lstinline!s!} right by adding space
characters in front of it, such that the total length the result string
is \passthrough{\lstinline!n.!}

See also: \passthrough{\lstinline!strcenter, strleft, strlimit!}.

\hypertarget{strsplit-procedure2-1}{%
\subsection{\texorpdfstring{\texttt{strsplit} :
procedure/2}{strsplit : procedure/2}}\label{strsplit-procedure2-1}}

Usage: \passthrough{\lstinline!(strsplit s del) => array!}

Return an array of strings obtained from \passthrough{\lstinline!s!} by
splitting \passthrough{\lstinline!s!} at each occurrence of string
\passthrough{\lstinline!del.!}

See also: \passthrough{\lstinline!str?!}.

\hypertarget{sub1-procedure1-1}{%
\subsection{\texorpdfstring{\texttt{sub1} :
procedure/1}{sub1 : procedure/1}}\label{sub1-procedure1-1}}

Usage: \passthrough{\lstinline!(sub1 n) => num!}

Subtract 1 from \passthrough{\lstinline!n.!}

See also: \passthrough{\lstinline!add1, +, -!}.

\hypertarget{sym-str-procedure1-1}{%
\subsection{\texorpdfstring{\texttt{sym-\textgreater{}str} :
procedure/1}{sym-\textgreater str : procedure/1}}\label{sym-str-procedure1-1}}

Usage: \passthrough{\lstinline!(sym->str sym) => str!}

Convert a symbol into a string.

See also: \passthrough{\lstinline!str->sym, intern, make-symbol!}.

\hypertarget{sym-procedure1-1}{%
\subsection{\texorpdfstring{\texttt{sym?} :
procedure/1}{sym? : procedure/1}}\label{sym-procedure1-1}}

Usage: \passthrough{\lstinline!(sym? sym) => bool!}

Return true if \passthrough{\lstinline!sym!} is a symbol, nil otherwise.

See also: \passthrough{\lstinline!str?, atom?!}.

\hypertarget{synout-procedure1-1}{%
\subsection{\texorpdfstring{\texttt{synout} :
procedure/1}{synout : procedure/1}}\label{synout-procedure1-1}}

Usage: \passthrough{\lstinline!(synout arg)!}

Like out, but enforcing a new input line afterwards. This needs to be
used when outputing concurrently in a future or task.

See also: \passthrough{\lstinline!out, outy, synouty!}.

\textbf{Warning: Concurrent display output can lead to unexpected visual
results and ought to be avoided.}

\hypertarget{synouty-procedure1}{%
\subsection{\texorpdfstring{\texttt{synouty} :
procedure/1}{synouty : procedure/1}}\label{synouty-procedure1}}

Usage: \passthrough{\lstinline!(synouty li)!}

Like outy, but enforcing a new input line afterwards. This needs to be
used when outputing concurrently in a future or task.

See also: \passthrough{\lstinline!synout, out, outy!}.

\textbf{Warning: Concurrent display output can lead to unexpected visual
results and ought to be avoided.}

\hypertarget{sys-key-procedure1-1}{%
\subsection{\texorpdfstring{\texttt{sys-key?} :
procedure/1}{sys-key? : procedure/1}}\label{sys-key-procedure1-1}}

Usage: \passthrough{\lstinline!(sys-key? key) => bool!}

Return true if the given sys key \passthrough{\lstinline!key!} exists,
nil otherwise.

See also: \passthrough{\lstinline!sys, setsys!}.

\hypertarget{sysmsg-procedure1-2}{%
\subsection{\texorpdfstring{\texttt{sysmsg} :
procedure/1}{sysmsg : procedure/1}}\label{sysmsg-procedure1-2}}

Usage: \passthrough{\lstinline!(sysmsg msg)!}

Asynchronously display a system message string
\passthrough{\lstinline!msg!} if in console or page mode, otherwise the
message is logged.

See also: \passthrough{\lstinline!sysmsg*, synout, synouty, out, outy!}.

\hypertarget{sysmsg-procedure1-3}{%
\subsection{\texorpdfstring{\texttt{sysmsg*} :
procedure/1}{sysmsg* : procedure/1}}\label{sysmsg-procedure1-3}}

Usage: \passthrough{\lstinline!(sysmsg* msg)!}

Display a system message string \passthrough{\lstinline!msg!} if in
console or page mode, otherwise the message is logged.

See also: \passthrough{\lstinline!sysmsg, synout, synouty, out, outy!}.

\hypertarget{take-procedure3-1}{%
\subsection{\texorpdfstring{\texttt{take} :
procedure/3}{take : procedure/3}}\label{take-procedure3-1}}

Usage: \passthrough{\lstinline!(take seq n) => seq!}

Return the sequence consisting of the \passthrough{\lstinline!n!} first
elements of \passthrough{\lstinline!seq.!}

See also: \passthrough{\lstinline!list, array, string, nth, seq?!}.

\hypertarget{task-procedure1-2}{%
\subsection{\texorpdfstring{\texttt{task} :
procedure/1}{task : procedure/1}}\label{task-procedure1-2}}

Usage: \passthrough{\lstinline!(task sel proc) => int!}

Create a new task for concurrently running
\passthrough{\lstinline!proc!}, a procedure that takes its own ID as
argument. The \passthrough{\lstinline!sel!} argument must be a symbol in
'(auto manual remove). If \passthrough{\lstinline!sel!} is 'remove, then
the task is always removed from the task table after it has finished,
even if an error has occurred. If sel is 'auto, then the task is removed
from the task table if it ends without producing an error. If
\passthrough{\lstinline!sel!} is 'manual then the task is not removed
from the task table, its state is either 'canceled, 'finished, or
'error, and it and must be removed manually with
\passthrough{\lstinline!task-remove!} or
\passthrough{\lstinline!prune-task-table!}. Broadcast messages are never
removed. Tasks are more heavy-weight than futures and allow for
message-passing.

See also:
\passthrough{\lstinline!task?, task-run, task-state, task-broadcast, task-send, task-recv, task-remove, prune-task-table!}.

\hypertarget{task-broadcast-procedure2-1}{%
\subsection{\texorpdfstring{\texttt{task-broadcast} :
procedure/2}{task-broadcast : procedure/2}}\label{task-broadcast-procedure2-1}}

Usage: \passthrough{\lstinline!(task-broadcast id msg)!}

Send a message from task \passthrough{\lstinline!id!} to the blackboard.
Tasks automatically send the message 'finished to the blackboard when
they are finished.

See also:
\passthrough{\lstinline!task, task?, task-run, task-state, task-send, task-recv!}.

\hypertarget{task-recv-procedure1-1}{%
\subsection{\texorpdfstring{\texttt{task-recv} :
procedure/1}{task-recv : procedure/1}}\label{task-recv-procedure1-1}}

Usage: \passthrough{\lstinline!(task-recv id) => any!}

Receive a message for task \passthrough{\lstinline!id!}, or nil if there
is no message. This is typically used by the task with
\passthrough{\lstinline!id!} itself to periodically check for new
messages while doing other work. By convention, if a task receives the
message 'end it ought to terminate at the next convenient occasion,
whereas upon receiving 'cancel it ought to terminate in an expedited
manner.

See also:
\passthrough{\lstinline!task-send, task, task?, task-run, task-state, task-broadcast!}.

\textbf{Warning: Busy polling for new messages in a tight loop is
inefficient and ought to be avoided.}

\hypertarget{task-remove-procedure1-1}{%
\subsection{\texorpdfstring{\texttt{task-remove} :
procedure/1}{task-remove : procedure/1}}\label{task-remove-procedure1-1}}

Usage: \passthrough{\lstinline!(task-remove id)!}

Remove task \passthrough{\lstinline!id!} from the task table. The task
can no longer be interacted with.

See also: \passthrough{\lstinline!task, task?, task-state!}.

\hypertarget{task-run-procedure1-1}{%
\subsection{\texorpdfstring{\texttt{task-run} :
procedure/1}{task-run : procedure/1}}\label{task-run-procedure1-1}}

Usage: \passthrough{\lstinline!(task-run id)!}

Run task \passthrough{\lstinline!id!}, which must have been previously
created with task. Attempting to run a task that is already running
results in an error unless \passthrough{\lstinline!silent?!} is true. If
silent? is true, the function does never produce an error.

See also:
\passthrough{\lstinline!task, task?, task-state, task-send, task-recv, task-broadcast-!}.

\hypertarget{task-schedule-procedure1-1}{%
\subsection{\texorpdfstring{\texttt{task-schedule} :
procedure/1}{task-schedule : procedure/1}}\label{task-schedule-procedure1-1}}

Usage: \passthrough{\lstinline!(task-schedule sel id)!}

Schedule task \passthrough{\lstinline!id!} for running, starting it as
soon as other tasks have finished. The scheduler attempts to avoid
running more than (cpunum) tasks at once.

See also: \passthrough{\lstinline!task, task-run!}.

\hypertarget{task-send-procedure2-1}{%
\subsection{\texorpdfstring{\texttt{task-send} :
procedure/2}{task-send : procedure/2}}\label{task-send-procedure2-1}}

Usage: \passthrough{\lstinline!(task-send id msg)!}

Send a message \passthrough{\lstinline!msg!} to task
\passthrough{\lstinline!id!}. The task needs to cooperatively use
task-recv to reply to the message. It is up to the receiving task what
to do with the message once it has been received, or how often to check
for new messages.

See also:
\passthrough{\lstinline!task-broadcast, task-recv, task, task?, task-run, task-state!}.

\hypertarget{task-state-procedure1-1}{%
\subsection{\texorpdfstring{\texttt{task-state} :
procedure/1}{task-state : procedure/1}}\label{task-state-procedure1-1}}

Usage: \passthrough{\lstinline!(task-state id) => sym!}

Return the state of the task, which is a symbol in '(finished error
stopped new waiting running).

See also:
\passthrough{\lstinline!task, task?, task-run, task-broadcast, task-recv, task-send!}.

\hypertarget{task-procedure1-3}{%
\subsection{\texorpdfstring{\texttt{task?} :
procedure/1}{task? : procedure/1}}\label{task-procedure1-3}}

Usage: \passthrough{\lstinline!(task? id) => bool!}

Check whether the given \passthrough{\lstinline!id!} is for a valid
task, return true if it is valid, nil otherwise.

See also:
\passthrough{\lstinline!task, task-run, task-state, task-broadcast, task-send, task-recv!}.

\hypertarget{terpri-procedure0-1}{%
\subsection{\texorpdfstring{\texttt{terpri} :
procedure/0}{terpri : procedure/0}}\label{terpri-procedure0-1}}

Usage: \passthrough{\lstinline!(terpri)!}

Advance the host OS terminal to the next line.

See also: \passthrough{\lstinline!princ, out, outy!}.

\hypertarget{testing-macro1-1}{%
\subsection{\texorpdfstring{\texttt{testing} :
macro/1}{testing : macro/1}}\label{testing-macro1-1}}

Usage: \passthrough{\lstinline!(testing name)!}

Registers the string \passthrough{\lstinline!name!} as the name of the
tests that are next registered with expect.

See also:
\passthrough{\lstinline!expect, expect-err, expect-ok, run-selftest!}.

\hypertarget{the-color-procedure1-1}{%
\subsection{\texorpdfstring{\texttt{the-color} :
procedure/1}{the-color : procedure/1}}\label{the-color-procedure1-1}}

Usage: \passthrough{\lstinline!(the-color colors-spec) => (r g b a)!}

Return the color list (r g b a) based on a color specification, which
may be a color list (r g b), a color selector for (color selector) or a
color name such as 'dark-blue.

See also: \passthrough{\lstinline!*colors*, color, set-color, outy!}.

\hypertarget{the-color-names-procedure0-1}{%
\subsection{\texorpdfstring{\texttt{the-color-names} :
procedure/0}{the-color-names : procedure/0}}\label{the-color-names-procedure0-1}}

Usage: \passthrough{\lstinline!(the-color-names) => li!}

Return the list of color names in \emph{colors}.

See also: \passthrough{\lstinline!*colors*, the-color!}.

\hypertarget{time-procedure1-1}{%
\subsection{\texorpdfstring{\texttt{time} :
procedure/1}{time : procedure/1}}\label{time-procedure1-1}}

Usage: \passthrough{\lstinline!(time proc) => int!}

Return the time in nanoseconds that it takes to execute the procedure
with no arguments \passthrough{\lstinline!proc.!}

See also: \passthrough{\lstinline!now-ns, now!}.

\hypertarget{truncate-procedure1-or-more-1}{%
\subsection{\texorpdfstring{\texttt{truncate} : procedure/1 or
more}{truncate : procedure/1 or more}}\label{truncate-procedure1-or-more-1}}

Usage: \passthrough{\lstinline!(truncate x [y]) => int!}

Round down to nearest integer of \passthrough{\lstinline!x!}. If
\passthrough{\lstinline!y!} is present, divide
\passthrough{\lstinline!x!} by \passthrough{\lstinline!y!} and round
down to the nearest integer.

See also: \passthrough{\lstinline!div, /, int!}.

\hypertarget{try-macro2-or-more-1}{%
\subsection{\texorpdfstring{\texttt{try} : macro/2 or
more}{try : macro/2 or more}}\label{try-macro2-or-more-1}}

Usage: \passthrough{\lstinline!(try (finals ...) body ...)!}

Evaluate the forms of the \passthrough{\lstinline!body!} and afterwards
the forms in \passthrough{\lstinline!finals!}. If during the execution
of \passthrough{\lstinline!body!} an error occurs, first all
\passthrough{\lstinline!finals!} are executed and then the error is
printed by the default error printer.

See also: \passthrough{\lstinline!with-final, with-error-handler!}.

\hypertarget{unless-macro1-or-more-1}{%
\subsection{\texorpdfstring{\texttt{unless} : macro/1 or
more}{unless : macro/1 or more}}\label{unless-macro1-or-more-1}}

Usage: \passthrough{\lstinline!(unless cond expr ...) => any!}

Evaluate expressions \passthrough{\lstinline!expr!} if
\passthrough{\lstinline!cond!} is not true, returns void otherwise.

See also: \passthrough{\lstinline!if, when, cond!}.

\hypertarget{unprotect-procedure0-or-more-1}{%
\subsection{\texorpdfstring{\texttt{unprotect} : procedure/0 or
more}{unprotect : procedure/0 or more}}\label{unprotect-procedure0-or-more-1}}

Usage: \passthrough{\lstinline!(unprotect [sym] ...)!}

Unprotect symbols \passthrough{\lstinline!sym!} \ldots, allowing
mutation or rebinding them. The symbols need to be quoted. This
operation requires the permission 'allow-unprotect to be set, or else an
error is caused.

See also:
\passthrough{\lstinline!protect, protected?, dict-unprotect, dict-protected?, permissions, permission?, setq, bind, interpret!}.

\hypertarget{valid-procedure1-1}{%
\subsection{\texorpdfstring{\texttt{valid?} :
procedure/1}{valid? : procedure/1}}\label{valid-procedure1-1}}

Usage: \passthrough{\lstinline!(valid? obj) => bool!}

Return true if \passthrough{\lstinline!obj!} is a valid object, nil
otherwise. What exactly object validity means is undefined, but certain
kind of objects such as graphics objects may be marked invalid when they
can no longer be used because they have been disposed off by a subsystem
and cannot be automatically garbage collected. Generally, invalid
objects ought no longer be used and need to be discarded.

See also: \passthrough{\lstinline!gfx.reset!}.

\hypertarget{void-procedure0-or-more-1}{%
\subsection{\texorpdfstring{\texttt{void} : procedure/0 or
more}{void : procedure/0 or more}}\label{void-procedure0-or-more-1}}

Usage: \passthrough{\lstinline!(void [any] ...)!}

Always returns void, no matter what values are given to it. Void is a
special value that is not printed in the console.

See also: \passthrough{\lstinline!void?!}.

\hypertarget{wait-for-procedure2-1}{%
\subsection{\texorpdfstring{\texttt{wait-for} :
procedure/2}{wait-for : procedure/2}}\label{wait-for-procedure2-1}}

Usage: \passthrough{\lstinline!(wait-for dict key)!}

Block execution until the value for \passthrough{\lstinline!key!} in
\passthrough{\lstinline!dict!} is not-nil. This function may wait
indefinitely if no other thread sets the value for
\passthrough{\lstinline!key!} to not-nil.

See also:
\passthrough{\lstinline!wait-for*, future, force, wait-until, wait-until*!}.

\textbf{Warning: This cannot be used for synchronization of multiple
tasks due to potential race-conditions.}

\hypertarget{wait-for-procedure3-1}{%
\subsection{\texorpdfstring{\texttt{wait-for*} :
procedure/3}{wait-for* : procedure/3}}\label{wait-for-procedure3-1}}

Usage: \passthrough{\lstinline!(wait-for* dict key timeout)!}

Blocks execution until the value for \passthrough{\lstinline!key!} in
\passthrough{\lstinline!dict!} is not-nil or
\passthrough{\lstinline!timeout!} nanoseconds have passed, and returns
that value or nil if waiting timed out. If
\passthrough{\lstinline!timeout!} is negative, then the function waits
potentially indefinitely without any timeout. If a non-nil key is not
found, the function sleeps at least \emph{sync-wait-lower-bound}
nanoseconds and up to \emph{sync-wait-upper-bound} nanoseconds until it
looks for the key again.

See also:
\passthrough{\lstinline!future, force, wait-for, wait-until, wait-until*!}.

\textbf{Warning: This cannot be used for synchronization of multiple
tasks due to potential race-conditions.}

\hypertarget{wait-for-empty-procedure3-1}{%
\subsection{\texorpdfstring{\texttt{wait-for-empty*} :
procedure/3}{wait-for-empty* : procedure/3}}\label{wait-for-empty-procedure3-1}}

Usage: \passthrough{\lstinline!(wait-for-empty* dict key timeout)!}

Blocks execution until the \passthrough{\lstinline!key!} is no longer
present in \passthrough{\lstinline!dict!} or
\passthrough{\lstinline!timeout!} nanoseconds have passed. If
\passthrough{\lstinline!timeout!} is negative, then the function waits
potentially indefinitely without any timeout.

See also:
\passthrough{\lstinline!future, force, wait-for, wait-until, wait-until*!}.

\textbf{Warning: This cannot be used for synchronization of multiple
tasks due to potential race-conditions.}

\hypertarget{wait-until-procedure2-1}{%
\subsection{\texorpdfstring{\texttt{wait-until} :
procedure/2}{wait-until : procedure/2}}\label{wait-until-procedure2-1}}

Usage: \passthrough{\lstinline!(wait-until dict key pred)!}

Blocks execution until the unary predicate
\passthrough{\lstinline!pred!} returns true for the value at
\passthrough{\lstinline!key!} in \passthrough{\lstinline!dict!}. This
function may wait indefinitely if no other thread sets the value in such
a way that \passthrough{\lstinline!pred!} returns true when applied to
it.

See also:
\passthrough{\lstinline!wait-for, future, force, wait-until*!}.

\textbf{Warning: This cannot be used for synchronization of multiple
tasks due to potential race-conditions.}

\hypertarget{wait-until-procedure4-1}{%
\subsection{\texorpdfstring{\texttt{wait-until*} :
procedure/4}{wait-until* : procedure/4}}\label{wait-until-procedure4-1}}

Usage: \passthrough{\lstinline!(wait-until* dict key pred timeout)!}

Blocks execution until the unary predicate
\passthrough{\lstinline!pred!} returns true for the value at
\passthrough{\lstinline!key!} in \passthrough{\lstinline!dict!}, or
\passthrough{\lstinline!timeout!} nanoseconds have passed, and returns
the value or nil if waiting timed out. If
\passthrough{\lstinline!timeout!} is negative, then the function waits
potentially indefinitely without any timeout. If a non-nil key is not
found, the function sleeps at least \emph{sync-wait-lower-bound}
nanoseconds and up to \emph{sync-wait-upper-bound} nanoseconds until it
looks for the key again.

See also:
\passthrough{\lstinline!future, force, wait-for, wait-until*, wait-until!}.

\textbf{Warning: This cannot be used for synchronization of multiple
tasks due to potential race-conditions.}

\hypertarget{warn-procedure1-or-more-1}{%
\subsection{\texorpdfstring{\texttt{warn} : procedure/1 or
more}{warn : procedure/1 or more}}\label{warn-procedure1-or-more-1}}

Usage: \passthrough{\lstinline!(warn msg [args...])!}

Output the warning message \passthrough{\lstinline!msg!} in error
colors. The optional \passthrough{\lstinline!args!} are applied to the
message as in fmt. The message should not end with a newline.

See also: \passthrough{\lstinline!error!}.

\hypertarget{week-procedure2-1}{%
\subsection{\texorpdfstring{\texttt{week+} :
procedure/2}{week+ : procedure/2}}\label{week-procedure2-1}}

Usage: \passthrough{\lstinline!(week+ dateli n) => dateli!}

Adds \passthrough{\lstinline!n!} weeks to the given date
\passthrough{\lstinline!dateli!} in datelist format and returns the new
datelist.

See also:
\passthrough{\lstinline!sec+, minute+, hour+, day+, month+, year+, now!}.

\hypertarget{week-of-date-procedure3-1}{%
\subsection{\texorpdfstring{\texttt{week-of-date} :
procedure/3}{week-of-date : procedure/3}}\label{week-of-date-procedure3-1}}

Usage: \passthrough{\lstinline!(week-of-date Y M D) => int!}

Return the week of the date in the year given by year
\passthrough{\lstinline!Y!}, month \passthrough{\lstinline!M!}, and day
\passthrough{\lstinline!D.!}

See also:
\passthrough{\lstinline!day-of-week, datestr->datelist, date->epoch-ns, epoch-ns->datelist, datestr, datestr*, now!}.

\hypertarget{when-macro1-or-more-1}{%
\subsection{\texorpdfstring{\texttt{when} : macro/1 or
more}{when : macro/1 or more}}\label{when-macro1-or-more-1}}

Usage: \passthrough{\lstinline!(when cond expr ...) => any!}

Evaluate the expressions \passthrough{\lstinline!expr!} if
\passthrough{\lstinline!cond!} is true, returns void otherwise.

See also: \passthrough{\lstinline!if, cond, unless!}.

\hypertarget{when-permission-macro1-or-more-1}{%
\subsection{\texorpdfstring{\texttt{when-permission} : macro/1 or
more}{when-permission : macro/1 or more}}\label{when-permission-macro1-or-more-1}}

Usage: \passthrough{\lstinline!(when-permission perm body ...) => any!}

Execute the expressions in \passthrough{\lstinline!body!} if and only if
the symbolic permission \passthrough{\lstinline!perm!} is available.

See also: \passthrough{\lstinline!permission?!}.

\hypertarget{while-macro1-or-more-1}{%
\subsection{\texorpdfstring{\texttt{while} : macro/1 or
more}{while : macro/1 or more}}\label{while-macro1-or-more-1}}

Usage: \passthrough{\lstinline!(while test body ...) => any!}

Evaluate the expressions in \passthrough{\lstinline!body!} while
\passthrough{\lstinline!test!} is not nil.

See also: \passthrough{\lstinline!letrec, dotimes, dolist!}.

\hypertarget{with-colors-procedure3-1}{%
\subsection{\texorpdfstring{\texttt{with-colors} :
procedure/3}{with-colors : procedure/3}}\label{with-colors-procedure3-1}}

Usage: \passthrough{\lstinline!(with-colors textcolor backcolor proc)!}

Execute \passthrough{\lstinline!proc!} for display side effects, where
the default colors are set to \passthrough{\lstinline!textcolor!} and
\passthrough{\lstinline!backcolor!}. These are color specifications like
in the-color. After \passthrough{\lstinline!proc!} has finished or if an
error occurs, the default colors are restored to their original state.

See also:
\passthrough{\lstinline!the-color, color, set-color, with-final!}.

\hypertarget{with-error-handler-macro2-or-more-1}{%
\subsection{\texorpdfstring{\texttt{with-error-handler} : macro/2 or
more}{with-error-handler : macro/2 or more}}\label{with-error-handler-macro2-or-more-1}}

Usage: \passthrough{\lstinline!(with-error-handler handler body ...)!}

Evaluate the forms of the \passthrough{\lstinline!body!} with error
handler \passthrough{\lstinline!handler!} in place. The handler is a
procedure that takes the error as argument and handles it. If an error
occurs in \passthrough{\lstinline!handler!}, a default error handler is
used. Handlers are only active within the same thread.

See also: \passthrough{\lstinline!with-final!}.

\hypertarget{with-final-macro2-or-more-1}{%
\subsection{\texorpdfstring{\texttt{with-final} : macro/2 or
more}{with-final : macro/2 or more}}\label{with-final-macro2-or-more-1}}

Usage: \passthrough{\lstinline!(with-final finalizer body ...)!}

Evaluate the forms of the \passthrough{\lstinline!body!} with the given
finalizer as error handler. If an error occurs, then
\passthrough{\lstinline!finalizer!} is called with that error and nil.
If no error occurs, \passthrough{\lstinline!finalizer!} is called with
nil as first argument and the result of evaluating all forms of
\passthrough{\lstinline!body!} as second argument.

See also: \passthrough{\lstinline!with-error-handler!}.

\hypertarget{with-mutex-lock-macro1-or-more}{%
\subsection{\texorpdfstring{\texttt{with-mutex-lock} : macro/1 or
more}{with-mutex-lock : macro/1 or more}}\label{with-mutex-lock-macro1-or-more}}

Usage: \passthrough{\lstinline!(with-mutex-lock m ...) => any!}

Execute the body with mutex \passthrough{\lstinline!m!} locked for
writing and unlock the mutex afterwards.

See also:
\passthrough{\lstinline!with-mutex-rlock, make-mutex, mutex-lock, mutex-rlock, mutex-unlock, mutex-runlock!}.

\textbf{Warning: Make sure to never lock the same mutex twice from the
same task, otherwise a deadlock will occur!}

\hypertarget{with-mutex-rlock-macro1-or-more-1}{%
\subsection{\texorpdfstring{\texttt{with-mutex-rlock} : macro/1 or
more}{with-mutex-rlock : macro/1 or more}}\label{with-mutex-rlock-macro1-or-more-1}}

Usage: \passthrough{\lstinline!(with-mutex-rlock m ...) => any!}

Execute the body with mutex \passthrough{\lstinline!m!} locked for
reading and unlock the mutex afterwards.

See also:
\passthrough{\lstinline!with-mutex-lock, make-mutex, mutex-lock, mutex-rlock, mutex-unlock, mutex-runlock!}.

\hypertarget{write-procedure2-1}{%
\subsection{\texorpdfstring{\texttt{write} :
procedure/2}{write : procedure/2}}\label{write-procedure2-1}}

Usage: \passthrough{\lstinline!(write p datum) => int!}

Write \passthrough{\lstinline!datum!} to output port
\passthrough{\lstinline!p!} and return the number of bytes written.

See also:
\passthrough{\lstinline!write-binary, write-binary-at, read, close, open!}.

\hypertarget{write-binary-procedure4-1}{%
\subsection{\texorpdfstring{\texttt{write-binary} :
procedure/4}{write-binary : procedure/4}}\label{write-binary-procedure4-1}}

Usage: \passthrough{\lstinline!(write-binary p buff n offset) => int!}

Write \passthrough{\lstinline!n!} bytes starting at
\passthrough{\lstinline!offset!} in binary blob
\passthrough{\lstinline!buff!} to the stream port
\passthrough{\lstinline!p!}. This function returns the number of bytes
actually written.

See also:
\passthrough{\lstinline!write-binary-at, read-binary, write, close, open!}.

\hypertarget{write-binary-at-procedure5-1}{%
\subsection{\texorpdfstring{\texttt{write-binary-at} :
procedure/5}{write-binary-at : procedure/5}}\label{write-binary-at-procedure5-1}}

Usage:
\passthrough{\lstinline!(write-binary-at p buff n offset fpos) => int!}

Write \passthrough{\lstinline!n!} bytes starting at
\passthrough{\lstinline!offset!} in binary blob
\passthrough{\lstinline!buff!} to the seekable stream port
\passthrough{\lstinline!p!} at the stream position
\passthrough{\lstinline!fpos!}. If there is not enough data in
\passthrough{\lstinline!p!} to overwrite at position
\passthrough{\lstinline!fpos!}, then an error is caused and only part of
the data might be written. The function returns the number of bytes
actually written.

See also:
\passthrough{\lstinline!read-binary, write-binary, write, close, open!}.

\hypertarget{write-string-procedure2-1}{%
\subsection{\texorpdfstring{\texttt{write-string} :
procedure/2}{write-string : procedure/2}}\label{write-string-procedure2-1}}

Usage: \passthrough{\lstinline!(write-string p s) => int!}

Write string \passthrough{\lstinline!s!} to output port
\passthrough{\lstinline!p!} and return the number of bytes written. LF
are \emph{not} automatically converted to CR LF sequences on windows.

See also:
\passthrough{\lstinline!write, write-binary, write-binary-at, read, close, open!}.

\hypertarget{year-procedure2-1}{%
\subsection{\texorpdfstring{\texttt{year+} :
procedure/2}{year+ : procedure/2}}\label{year-procedure2-1}}

Usage: \passthrough{\lstinline!(month+ dateli n) => dateli!}

Adds \passthrough{\lstinline!n!} years to the given date
\passthrough{\lstinline!dateli!} in datelist format and returns the new
datelist.

See also:
\passthrough{\lstinline!sec+, minute+, hour+, day+, week+, month+, now!}.

\end{document}
