% Options for packages loaded elsewhere
\PassOptionsToPackage{unicode}{hyperref}
\PassOptionsToPackage{hyphens}{url}
%
\documentclass[
]{article}
\usepackage{lmodern}
\usepackage{amssymb,amsmath}
\usepackage{ifxetex,ifluatex}
\ifnum 0\ifxetex 1\fi\ifluatex 1\fi=0 % if pdftex
  \usepackage[T1]{fontenc}
  \usepackage[utf8]{inputenc}
  \usepackage{textcomp} % provide euro and other symbols
\else % if luatex or xetex
  \usepackage{unicode-math}
  \defaultfontfeatures{Scale=MatchLowercase}
  \defaultfontfeatures[\rmfamily]{Ligatures=TeX,Scale=1}
\fi
% Use upquote if available, for straight quotes in verbatim environments
\IfFileExists{upquote.sty}{\usepackage{upquote}}{}
\IfFileExists{microtype.sty}{% use microtype if available
  \usepackage[]{microtype}
  \UseMicrotypeSet[protrusion]{basicmath} % disable protrusion for tt fonts
}{}
\makeatletter
\@ifundefined{KOMAClassName}{% if non-KOMA class
  \IfFileExists{parskip.sty}{%
    \usepackage{parskip}
  }{% else
    \setlength{\parindent}{0pt}
    \setlength{\parskip}{6pt plus 2pt minus 1pt}}
}{% if KOMA class
  \KOMAoptions{parskip=half}}
\makeatother
\usepackage{xcolor}
\IfFileExists{xurl.sty}{\usepackage{xurl}}{} % add URL line breaks if available
\IfFileExists{bookmark.sty}{\usepackage{bookmark}}{\usepackage{hyperref}}
\hypersetup{
  pdftitle={Z3S5 Lisp User Manual},
  pdfauthor={by Erich H. Rast, Ph.D.},
  hidelinks,
  pdfcreator={LaTeX via pandoc}}
\urlstyle{same} % disable monospaced font for URLs
\usepackage{listings}
\newcommand{\passthrough}[1]{#1}
\lstset{defaultdialect=[5.3]Lua}
\lstset{defaultdialect=[x86masm]Assembler}
\setlength{\emergencystretch}{3em} % prevent overfull lines
\providecommand{\tightlist}{%
  \setlength{\itemsep}{0pt}\setlength{\parskip}{0pt}}
\setcounter{secnumdepth}{5}
\lstset{% for listings
  basicstyle=\footnotesize\ttfamily,
  breaklines=true,
}
\usepackage{xcolor}

\title{Z3S5 Lisp User Manual}
\author{by Erich H. Rast, Ph.D.}
\date{July 22, 2022}

\begin{document}
\maketitle

\emph{for Z3S5 Lisp Version 2.3+}

\hypertarget{introduction}{%
\section{Introduction}\label{introduction}}

Z3S5 Lisp can be used as a standalone interpreter or as an extension
language embedded into Go programs. It is a traditional Lisp-1 dialect,
where the suffix 1 means that symbols hold one value and this value
might either represent functions and closures or data. This is in
contrast to Lisp-2 language like CommonLisp in which symbols may hold
functions in a separate slot. Scheme dialects are also Lisp-1 and Z3S5
Lisp shares many similarities with Scheme while also having features of
traditional Lisp systems.

\hypertarget{invoking-lisp}{%
\subsection{Invoking Lisp}\label{invoking-lisp}}

\hypertarget{the-standalone-interpreter}{%
\subsubsection{The Standalone
Interpreter}\label{the-standalone-interpreter}}

In the directory \passthrough{\lstinline!cmd/z3!} there is an example
standalone version \passthrough{\lstinline!z3.go!} that you can build on
your system using \passthrough{\lstinline!go build z3.go!}. The
interpreter is started in a terminal using
\passthrough{\lstinline!./z3!}.

The \passthrough{\lstinline!z3!} interpreter is fairly limited. It
starts a read-eval-reply loop until it is quit with the command
\passthrough{\lstinline!(exit [n])!} where the optional number
\passthrough{\lstinline!n!} is an integer for the Unix return code of
the program. It reads one line of input from the command line and
returns the result of evaluating it. Better editing capabilities and
parenthesis matching are planned for the future; for now, the
interpreter is more of a testing tool and a proof of concept and can be
used as an example of how to implement your own Z3S5 standalone
executable.

When an interpreter starts either in a standalone executable or when the
\passthrough{\lstinline!interp.Boot()!} function is called in a Go
program, then it first loads the standard prelude and help files in
directory \passthrough{\lstinline!embed!}. These are embedded into the
executable and so the directory is not needed to run the interpreter.
After this start sequence, the interpreter checks whether there is a
file named \passthrough{\lstinline!init.lisp!} in the executable
directory -- that is, the \passthrough{\lstinline!z3!} directory for
standalone or the directory of the program that includes Z3S5 Lisp as a
package. If there is such a file, then it is loaded and executed.

\hypertarget{using-the-interpreter-in-go}{%
\subsubsection{Using the Interpreter in
Go}\label{using-the-interpreter-in-go}}

See \passthrough{\lstinline!cmd/z3/z3.go!} for an example of how to use
Z3S5 Lisp in Go. It is best to include the package with
\passthrough{\lstinline!z3 "github.com/rasteric/z3s5-lisp"!} so
\passthrough{\lstinline!z3!} can be used as the package shortcut, since
this has to be called often when writing extensions. The following
snippet from \passthrough{\lstinline!z3.go!} imports the Lisp
interpreter, boots the standard prelude, and runs an interactive
read-eval-reply loop:

\begin{lstlisting}[language=Go]
import (
    "fmt"
    "os"

    z3 "github.com/rasteric/z3s5-lisp"
)

func main() {
    interp, err := z3.NewInterp(z3.NewBasicRuntime(z3.FullPermissions))
    if err != nil {
        fmt.Fprintf(os.Stderr, "Z3S5 Lisp failed to start: %v\n", err)
        os.Exit(1)
    }
    err = interp.Boot()
    if err != nil {
        fmt.Fprintf(os.Stderr, "Z3S5 Lisp failed to boot the standard prelude: %v\n", err)
        os.Exit(2)
    }
    interp.SafeEval(&z3.Cell{Car: z3.NewSym("protect-toplevel-symbols"), Cdr: z3.Nil}, z3.Nil)
    interp.Run(nil)
}
\end{lstlisting}

What deserves explanation is the line starting with
\passthrough{\lstinline!interp.SafeEval!}. This creates a list with the
symbol \passthrough{\lstinline!protect-toplevel-symbols!} as car and an
empty cdr and executes it within an empty environment. So it runs the
equivalent of \passthrough{\lstinline!(protect-toplevel-symbols)!},
which is interpreted as a function call that applies
\passthrough{\lstinline!protect!} to all toplevel symbols defined thus
far. It is not put in the standard prelude because a common way to
extend Z3S5 Lisp is to simply redefine some of its primitives. Once a
symbol is protected, attempts of redefining or mutating it result in a
security violation error. As long as the
\passthrough{\lstinline!Permissions!} given to
\passthrough{\lstinline!NewBasicRuntime!} have
\passthrough{\lstinline!AllowUnprotect!} set to true, you may use
\passthrough{\lstinline!unprotect!} to remove this safeguard again and
redefine existing functions. Symbols are protected in the
\passthrough{\lstinline!z3!} interpreter because it is easy to mess up
the system with dynamic redefinitions. For example, you could redefine
\passthrough{\lstinline!(setq + -)!} and \passthrough{\lstinline!+!} is
suddenly interpreted as subtraction! However, redefining toplevel
symbols is a very important feature. By using
\passthrough{\lstinline!unbind!} on certain functions or setting them to
something else like
\passthrough{\lstinline'(setq func (lambda () (error "not allowed to use func!")))'}
it is possible to create fine-grained access control to functions even
when they are intrinsic or defined in the standard prelude. It is not
possible to access an intrinsic function after it has been defined away
in this manner. After you added all your toplevel symbols and removed or
redefined the ones you do not like, you may thus protect all symbols and
use \passthrough{\lstinline!set-permissions!} to revoke the privilege
\passthrough{\lstinline!AllowUnprotect!}. This fixes the base vocabulary
and any attempt to redefine it in the future will result in a security
violation, since permissions can only be changed from less secure to
more secure and never vice versa.

See the Chapter \protect\hyperlink{extending-z3s5-lisp}{Extending Z3S5
Lisp} for information about how to define your own functions in Z3S5
Lisp.

\hypertarget{data-types}{%
\subsection{Data Types}\label{data-types}}

Basic data types of Z3S5 Lisp are:

\begin{itemize}
\tightlist
\item
  Bools: \passthrough{\lstinline!t!} and \passthrough{\lstinline!nil!}.
  The empty list \passthrough{\lstinline!nil!} is interpreted as false
  and any non-nil value is true; the symbol \passthrough{\lstinline!t!}
  is predefined as non-nil and usually used for true.
\item
  Symbols: \passthrough{\lstinline!abracadabra!},
  \passthrough{\lstinline!foo!}, \passthrough{\lstinline!bar!},
  \passthrough{\lstinline!...hel\%lo*\_!}. These have few restrictions,
  as the last example illustrates.
\item
  Integers: \passthrough{\lstinline!3!}, \passthrough{\lstinline!1!},
  \passthrough{\lstinline!-28!}. These are bignums with functions such
  as \passthrough{\lstinline!+!}, \passthrough{\lstinline!-!},
  \passthrough{\lstinline!sub!}, \passthrough{\lstinline!div!},\ldots{}
\item
  Floats: \passthrough{\lstinline!3.14!},
  \passthrough{\lstinline!1.92829!} with a large range of functions with
  prefix \passthrough{\lstinline!fl.!}. Get a list with
  \passthrough{\lstinline!(dump 'fl.)!}
\item
  UTF-8 Strings: \passthrough{\lstinline!"This is a test."!} with
  functions such as \passthrough{\lstinline!str+!},
  \passthrough{\lstinline!str-empty?!},
  \passthrough{\lstinline!str-forall?!},
  \passthrough{\lstinline!str-join!},\ldots{} Get an overview with
  \passthrough{\lstinline!(dump 'str)!}.
\item
  Lists: \passthrough{\lstinline!'(a b c d e f)!} with functions such as
  \passthrough{\lstinline!car!}, \passthrough{\lstinline!cdr!},
  \passthrough{\lstinline!mapcar!}, \passthrough{\lstinline!1st!},
  \passthrough{\lstinline!2nd!}, \passthrough{\lstinline!member!},
  \passthrough{\lstinline!filter!},\ldots{}
\item
  Arrays: \passthrough{\lstinline!\#(a b c d e f)!} with functions such
  as \passthrough{\lstinline!array-ref!},
  \passthrough{\lstinline!array-set!},
  \passthrough{\lstinline!array-len!}, \ldots{}
\item
  Dictionaries: \passthrough{\lstinline!(dict)!} with functions such as
  \passthrough{\lstinline!get!}, \passthrough{\lstinline!set!},
  \passthrough{\lstinline!dict->alist!},
  \passthrough{\lstinline!dict->array!},
  \passthrough{\lstinline!dict->values!},
  \passthrough{\lstinline!dict-map!},\ldots{}
\item
  Blobs: These contain binary data with functions like
  \passthrough{\lstinline!peek!}, \passthrough{\lstinline!poke!},
  \passthrough{\lstinline!blob->base64!},\ldots{}
\item
  Boxed values with functions like \passthrough{\lstinline!valid?!}.
  Boxed values are used to embed foreign types like Go structures into
  the runtime system which cannot be automatically garbage collected.
  They typically require manual destruction to free memory. See the
  source code for \passthrough{\lstinline!lisp\_decimal.go!} for an
  example of such an embedding.
\item
  Futures: Futures encapsulate the result of a future computation and
  are returned by the special form
  \passthrough{\lstinline!(future ...)!} whose body evaluates to a
  future. The result of a future computation may be obtained with
  \passthrough{\lstinline!(force <future>)!}.
\item
  Tasks: Tasks are heavyweight concurrency constructs similar to threads
  in other programming languages. See
  \passthrough{\lstinline!(dump 'task)!} for a list of functions.
\end{itemize}

Strings, lists, and arrays are also sequences testable with the
\passthrough{\lstinline!seq?!} predicate. These support convenience
successor functions like \passthrough{\lstinline!map!} and
\passthrough{\lstinline!foreach!}. Check the
\passthrough{\lstinline!init.lisp!} preamble in directory
\passthrough{\lstinline!embed/!} for how these work.

Numbers are generally one bignum type for which
\passthrough{\lstinline!num?!} is true and they will extend and shrink
in their representation as necessary. However, numbers larger than 64
bit only support basic arithmetics and floating point arithmetics is
limited to values that fit into a Float64. There is also a built-in
decimal arithmetics extension with prefix \passthrough{\lstinline!dec!}
for correct accounting and banking arithmetics and rounding -- see
\passthrough{\lstinline!(dump 'dec.)!} for a list of available decimal
arithmetics functions and \passthrough{\lstinline!lisp\_decimal.go!} for
implementation details.

\hypertarget{writing-programs}{%
\section{Writing Programs}\label{writing-programs}}

Currently, the only way to load programs into the
\passthrough{\lstinline!z3!} interpreter instead of typing them by hand
is to edit the \passthrough{\lstinline!init.lisp!} file in the same
directory as the executable. This file is loaded and executed at
startup.

\hypertarget{finding-functions-and-online-help}{%
\subsection{Finding Functions and Online
Help}\label{finding-functions-and-online-help}}

Built-in help on functions can be obtained with
\passthrough{\lstinline!(help func)!} where the function name is not
quoted. For example, \passthrough{\lstinline!(help help)!} returns the
help entry for the \passthrough{\lstinline!help!} function. A list of
bound toplevel symbols can be obtained with
\passthrough{\lstinline!(dump [sym])!} where the optional symbol
\passthrough{\lstinline!sym!} must be a quoted prefix. For example,
\passthrough{\lstinline!(dump)!} returns all toplevel bindings and
\passthrough{\lstinline!(dump 'str)!} returns a list of all bound
symbols starting with ``str''. By convention, internal helper functions
without help entry are prefixed with an underscore and omitted by
\passthrough{\lstinline!dump!}. To get a list of all bindings, including
those starting with an underscore, use
\passthrough{\lstinline!(dump-bindings)!}.

\hypertarget{peculiarities-of-z3s5-lisp}{%
\subsection{Peculiarities of Z3S5
Lisp}\label{peculiarities-of-z3s5-lisp}}

\begin{itemize}
\item
  Iterators over sequences use the order
  \passthrough{\lstinline!(iterator sequence function)!}, not sometimes
  one and sometimes the other. Exception: \passthrough{\lstinline!memq!}
  and \passthrough{\lstinline!member!} ask whether an element is a
  member of a list and have order
  \passthrough{\lstinline!(member element list)!}.
\item
  There is no meaningful \passthrough{\lstinline!eq!}, equality is
  generally tested with \passthrough{\lstinline!equal?!}.
\item
  Predicate names usually end in a question mark like in Scheme
  dialects, with a few exceptions like \passthrough{\lstinline!=!} for
  numeric equality. Example: \passthrough{\lstinline!equal?!}. There is
  no p-suffix like in other Lisps.
\item
  \passthrough{\lstinline!dict!} data structures are multi-threading
  safe.
\item
  There is support for futures and concurrent tasks.
\end{itemize}

\hypertarget{basic-control-flow-and-examples}{%
\subsection{Basic Control Flow and
Examples}\label{basic-control-flow-and-examples}}

When Z3S5 Lisp encounters an unquoted list, it attempts to interpret the
first element of the list as a function call. Thus, if the symbol is
bound to a function or macro, it calls the function with the remainder
of the list as argument:

\begin{lstlisting}[language=Lisp]
> (+ 10 20 30 40 50)
150
\end{lstlisting}

Symbols evaluate to their values and need to be quoted otherwise. They
can be defined with \passthrough{\lstinline!setq!}:

\begin{lstlisting}[language=Lisp]
> hello
EvalError: void variable: hello
    hello
> (setq hello 'world)
world
> hello
world
\end{lstlisting}

To define a function, the macro \passthrough{\lstinline!defun!} can be
used as in other Lisp dialects. The syntax is the traditional one, not
in Scheme dialects:

\begin{lstlisting}[language=Lisp]
> (defun fib (n)
    (if (or (= n 0) (= n 1))
        1
        (+ (fib (- n 1))
           (fib (- n 2)))))
fib
> (fib 10)
89
> (fib 30)
1346269
\end{lstlisting}

\passthrough{\lstinline!(defun foo (bar) ...)!} is really just a macro
shortcut for \passthrough{\lstinline!(setq foo (lambda (bar) ...))!}
like in most other Lisp dialects. Macros are expanded before a program
is executed, which unfortunately makes error reporting in Z3S5 Lisp
sometimes a bit obtuse. The system neither keeps track of source
locations nor of original definitions and will present the expanded
macros when an error occurs. This will not change in the future, so
better get used to it! Changing this and turning Z3S5 Lisp into a real,
full-fledged Lisp would be a gigantic task and not worth the effort, as
there are already far more capable Lisp systems like
\passthrough{\lstinline!CommonLisp!} and
\passthrough{\lstinline!Racket!} out there.

The above function demonstrates recursion, a feature that is often used
in Lisp. Z3S5 Lisp eliminates tail recursion. The example also
illustrates the use of \passthrough{\lstinline!=!} for numeric equality,
\passthrough{\lstinline!+!} and \passthrough{\lstinline!-!} functions on
(potential) bignums and the macro
\passthrough{\lstinline!(if <condition> <then-clause> <else-clause>)!}.
The more general \passthrough{\lstinline!cond!} construct is also
available and in fact the primitive operation:

\begin{lstlisting}[language=Lisp]
>  (defun day (n)
     (cond 
       ((= n 0) 'sunday)
       ((= n 1) 'monday)
       ((= n 2) 'tuesday)
       (t 'late-in-the-week)))
day
> (day 1)
monday
> (day 0)
sunday
> (day -1)
late-in-the-week
> (day 5)
late-in-the-week
\end{lstlisting}

Looks like some programmer was lazy there.

In Lisp languages it is common to iterate over lists and other data
structures either using recursive functions or by using some of the
standard iteration constructs such as \passthrough{\lstinline!map!},
\passthrough{\lstinline!foreach!}, \passthrough{\lstinline!mapcar!}, and
so on. Nesting calls of macros or functions like
\passthrough{\lstinline!map!} and \passthrough{\lstinline!filter!} to
achieve the desired data transformation is very common functional
programming style and often desirable for performance. Here is an
example:

\begin{lstlisting}[language=Lisp]
> (map '(1 2 3 4 5 6) add1)
(2 3 4 5 6 7)
\end{lstlisting}

Local variables are bound with \passthrough{\lstinline!let!} or with
\passthrough{\lstinline!letrec!} in Z3S5 Lisp. The latter needs to be
used whenever a variable in the form depends on the other variable,
which is not possible with \passthrough{\lstinline!let!} since it makes
the bindings only available in the body. Here is the definition of
\passthrough{\lstinline!apropos!} from the preamble:

\begin{lstlisting}[language=Lisp]
(defun apropos (arg)
  (let ((info (get *help* arg nil)))
    (if info
      (cadr (assoc 'see info))
      nil)))
\end{lstlisting}

The variable \passthrough{\lstinline!info!} is bound to the result of
\passthrough{\lstinline!(get *help* arg nil)!}, which evaluates to
default \passthrough{\lstinline!nil!} if there is no help entry for
\passthrough{\lstinline!arg!} in the global dictionary
\passthrough{\lstinline!*help*!}. That avoids calling
\passthrough{\lstinline!(get *help* arg nil)!} twice, once for the check
to nil and once in the first \passthrough{\lstinline!if!} branch. But it
can often be better to avoid uses of \passthrough{\lstinline!let!} and
re-use accessors; it depends a bit on clarity and intent, and whether
the initial computation is costly or not. Here is an example where
\passthrough{\lstinline!let!} would not do and instead
\passthrough{\lstinline!letrec!} has to be used:

\begin{lstlisting}[language=Lisp]
> (letrec ((is-even? (lambda (n)
                       (or (= n 0)
                           (is-odd? (sub1 n)))))
           (is-odd? (lambda (n)
                      (and (not (= n 0))
                           (is-even? (sub1 n))))))
    (is-odd? 11))
t
\end{lstlisting}

The reason this doesn't work with \passthrough{\lstinline!let!} is that
\passthrough{\lstinline!is-even?!} is not bound in the definition of
\passthrough{\lstinline!is-odd?!} and vice versa when
\passthrough{\lstinline!let!} is used, whereas
\passthrough{\lstinline!letrec!} makes sure that these bindings are
mutually available. Although the current implementation always uses
\passthrough{\lstinline!letrec!} under the hood, it is best to use
\passthrough{\lstinline!let!} whenever it is possible since at least in
theory it could be implemented more efficiently. Notice that
\passthrough{\lstinline!setq!} can be used to mutate local bindings, of
course, and is not just intended for toplevel symbols. Although
beginners should avoid it, sometimes mutating variables with
\passthrough{\lstinline!setq!} can make definitions simpler at the cost
of some elegance.

\hypertarget{advanced-topics}{%
\section{Advanced Topics}\label{advanced-topics}}

\hypertarget{error-handling}{%
\subsection{Error Handling}\label{error-handling}}

Error handling is very rudimentary in Z3S5 Lisp. There are no
continuations and there is no fancy stuff like dynamic wind. Instead,
there are primitives like \passthrough{\lstinline!push-error-handler!}
and \passthrough{\lstinline!pop-error-handler!} and macros such as
\passthrough{\lstinline!with-final!} and
\passthrough{\lstinline!with-error-handler!}.

\hypertarget{debugging}{%
\subsection{Debugging}\label{debugging}}

Not only are error messages fairly rudimentary, they also use the
expanded macro definitions and are therefore hard to read. Local
bindings are implemented with lambda-terms and displayed as such. This
makes debugging a challenge, as it should be. Z3S5 programmers have the
habit of writing bug-free code from the start and thereby avoid
debugging entirely. But you could write your own trace or stepper
functions for enhanced debugging features by redefining all toplevel
symbols appropriately. Then again, you could also just write your own
Lisp with better debugging capabilities. The choice is up to you!

\hypertarget{concurrency}{%
\subsection{Concurrency}\label{concurrency}}

Dicts use Go's \passthrough{\lstinline!sync.Map!} under the hood and are
therefore concurrency-safe. The global symbol table is also
cuncurrency-safe. This means that dicts can be accessed safely from
futures and tasks and can even be used for synchronization purposes, as
the (admittedly horrible) current implementation of tasks in
\passthrough{\lstinline!embed/init.lisp!} illustrates. Generally,
futures should be used and can be spawned in large quantities without
much of a performance penalty. Tasks need some work to become efficient
and there are plans to include a more direct interface to Go's
goroutines with cancelable contexts in the future.

\hypertarget{file-and-network-access}{%
\subsection{File and Network Access}\label{file-and-network-access}}

These are coming soon. The reason they haven't been included yet in this
release is that Z3S5 Lisp has been extracted from a much larger project
called \emph{Z3S5 Machine}, a currently proprietary virtual retro Lisp
machine with thousands of graphics and sound commands, and in this
machine network and file access is heavily abstracted from and guarded.
So the existing file and network functions could not be used for a more
general-purpose Lisp and need to be re-written. You can implement your
own, of course. Simply define functions like
\passthrough{\lstinline!open!}, \passthrough{\lstinline!close!},
\passthrough{\lstinline!port?!}, \passthrough{\lstinline!file-port?!},
\passthrough{\lstinline!network-port?!}, \passthrough{\lstinline!read!},
\passthrough{\lstinline!write!}, \passthrough{\lstinline!read-bbinary!},
\passthrough{\lstinline!write-binary!}, etc. It's not that much work,
just a bit tedious. Or, wait until these are available officially. You
can expect a fairly standard system of streams and ports.

\hypertarget{language-stability}{%
\subsection{Language Stability}\label{language-stability}}

At this stage, built-in commands may still change. The good news is that
any change introduced in Z3S5 Lisp needs to be tracked and checked in
Z3S5 Machine, and so no larger changes are planned. However, no
guarantees can be made and you should vendor the repository or even fork
it if you want to make sure no breaking changes occur. Once the language
is stable, it will be marked on the home page.

\hypertarget{roadmap}{%
\subsection{Roadmap}\label{roadmap}}

File and network access will be added soon. An interface to sqlite3 will
also be ported from Z3S5 Machine, although it might involve some
abstraction and only indirect access. A simple persistence layer might
also be ported from Z3S5 Machine.

\hypertarget{extending-z3s5-lisp}{%
\section{Extending Z3S5 Lisp}\label{extending-z3s5-lisp}}

The system is extended by calling \passthrough{\lstinline!interp.Def!}.
As an example, consider the following function from
\passthrough{\lstinline!lisp\_base.go!}:

\begin{lstlisting}[language=Go]
    // (str->chars s) => array of int convert a UTF-8 string into an array of runes
    interp.Def("str->chars", 1, func(a []any) any {
        runes := []rune(norm.NFC.String(a[0].(string)))
        arr := make([]any, len(runes), len(runes))
        for i := range runes {
            arr[i] = goarith.AsNumber(runes[i])
        }
        return arr
    })
\end{lstlisting}

The \passthrough{\lstinline!Def!} function takes the function symbol as
string, the number \emph{n} of arguments, and a function that takes an
array of \passthrough{\lstinline!any!} and returns a value
\passthrough{\lstinline!any!} (aka
\passthrough{\lstinline!interface\{\}!}). The function may explicitly
check the type of the arguments for correctness but doesn't need to. It
is normal for functions to panic and even deliberately throw an error.
These are caught by the intepreter and displayed to the user. So how
much you check depends primarily on what errors you want to provide. By
convenience, custom errors thrown in function definitions start with the
name of the function such as
\passthrough{\lstinline!panic(error.New("foobar: the foobar function has failed"))!}
for a function \passthrough{\lstinline!foobar!}. If a function returns
no value, then the definition should return
\passthrough{\lstinline!Void!}, a special Lisp value that is not printed
in the read-eval-print loop.

Care must be taken with numbers. Pure Go numbers will not do and may
lead to bizzare and unexpected runtime behavior. Every number needs to
be converted using \passthrough{\lstinline!goarith.AsNumber!} and other
conversion functions from \passthrough{\lstinline!z3s5-lisp!} and the
package \passthrough{\lstinline!github.com/nukata/goarith!}. Check the
source code of some of the implementation files for examples. Again,
this is really important: Always convert Go numbers to bigint numbers
with \passthrough{\lstinline!goarith.AsNumber!}!

There are a few helper functions such as
\passthrough{\lstinline!ExpectInts!} that can be used for checking
arguments. Other useful conversion functions are
\passthrough{\lstinline!AsBool!}, \passthrough{\lstinline!ToLispBool!},
\passthrough{\lstinline!ArrayToList!},
\passthrough{\lstinline!ListToArray!}, etc.

Notice that lists are structures composed by
\passthrough{\lstinline!\&Cell\{Car: a, Cdr: b\}!}, where
\passthrough{\lstinline!Cdr!} might be another Cell and the final Cdr is
Nil, just like in any Lisp. If you construct these without the member
names \passthrough{\lstinline!Car!} and \passthrough{\lstinline!Cdr!} Go
will complain about this and refuse to compile, even though in this case
it is perfectly fine to use anonymous access. If you want to create a
lot of lists by hand in your function definitions without creating
runtime performance penalties by using functions like
\passthrough{\lstinline!ArrayToList!}, then it might make sense to
switch off this behavior of the \passthrough{\lstinline!go vet!} command
by running \passthrough{\lstinline!go vet -composites=false!} instead.
Of course, this will also disable such checks for other parts of your
application where they might be useful. The only other way is to make
the list construction fully explicit, as in:
\passthrough{\lstinline!\&z3.Cell\{z3.NewSym("hello"), \&z3.Cell\{z3.NewSym("world"), z3.Nil\}\}!}
which yields the list \passthrough{\lstinline!(hello world)!}.

Custom data structures: Since there is currently no way to modify the
printer for custom structures directly in the Lisp system, it is best to
put them into a box using \passthrough{\lstinline!interp.DefBoxed!},
which takes a symbol for a boxed value and creates a number of auxiliary
functions. See \passthrough{\lstinline!lisp\_decimal.go!} for an example
of how to use boxed values.

\hypertarget{license-and-contact}{%
\section{License and Contact}\label{license-and-contact}}

Z3S5 Lisp was written by Erich H. Rast and based on Nukata Lisp by
SUZUKI Hisao. It is licensed under the MIT License that allows free use
and modification as long as the LICENSE and copyright notices remain.
Please read the accompanying LICENSE file for more information.

Please send inquiries and bug reports to
\href{mailto:erich@snafu.de}{\nolinkurl{erich@snafu.de}} or open an
issue on Github. Happy hacking!

\end{document}
